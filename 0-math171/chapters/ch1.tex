\documentclass[../m171main.tex]{subfiles}
\graphicspath{{\subfix{../figures/}}}

\begin{document}

\chapter{Groups and Subgroups}
\section{Axioms and Properties}
For most of this course, the central object of study will be the group.

\begin{definition}[Binary operation]
    A binary operation on a set $G$ is a function $\star : G \times G \to G$.
    \begin{itemize}
        \item If $a \star (b \star c) = (a \star b) \star c$ for all $a,b,c \in G$, then we say $\star$ is associative.
        \item If $a \star b = b \star a$ for all $a,b \in G$, then we say $\star$ is commutative.
    \end{itemize}
    For $a,b \in G$ we'll typically write $a \star b$ for $\star (a,b)$.
\end{definition}

\begin{definition}[Group]
    A group is an ordered pair $(G, \star)$ where $G$ is a set and $\star$ is a binary operation on $G$ such that
    \begin{itemize}
        \item $\star$ is associative,
        \item there exists an $e \in G$, called an identity, such that $a \star e = e \star a = a$ for all $a \in G$, and
        \item for each $a \in G$ there is an $a^{-1} \in G$, called an inverse of $a$, such that $a \star a^{-1} = a^{-1} \star a = e$.
    \end{itemize}
    We say the group $(G, \star)$ is commutative (or abelian) if $\star$ is commutative.
\end{definition}

We've already encountered many groups in our previous studies!
For example, under addition we have $\Z$, $\Q$, $\R$, $\C$, and $\Z / n\Z$ (the integers modulo $n$), and under multiplication we have $\Q^\times$, $\R^\times$, $\C^\times$, and $Z / n\Z^\times$ (where the $\times$ denotes zero-exclusion).
These examples help make the following properties a bit more concrete.

\begin{theorem}[]
    If $(G, \star)$ is a group then
    \begin{enumerate}[label=(\alph*)]
        \item the identity of $G$ is unique,
        \item $a^{-1}$ is unique for each $a \in G$,
        \item $(a^{-1})^{-1} = a$,
        \item $(a \star b)^{-1} = b^{-1} \star a^{-1}$, and
        \item for all $a_1, \cdots, a_n \in G$, the value of $a_1 \star \cdots \star a_n$is independent of how the expression is bracketed.
    \end{enumerate}
\end{theorem}

\begin{proof}
    We prove the first two parts.
    
    (a) Suppose $e$ and $e'$ are both identities.
    Then we have the chain of equalities
    \[ e = e \star e' = e'. \]

    (b) Suppose some $a \in G$ has two inverses $a', a''$.
    Then we have the chain of equalities
    \[ a' = a' \star e = a' \star (a \star a'') = (a' \star a) \star a'' = a''. \]

    Other parts are left as exercises.
\end{proof}

With these properties in mind, we'll make a few notes on notation.
\begin{itemize}
    \item We read $(G, \star)$ aloud as ``G is a group under $\star$''.
    In practice, if the binary operation is self-evident, we'll simply write $G$ to mean $(G, \star)$.

    \item For a group $(G, \star)$ we'll usually write $ab$ to mean $a \star b$.
    In the same spirit, we can write a length-$n$ product $x \star \cdots \star x$ as $x^{n}$, and $x^{-n} = \left( x^{-1} \right)^{n}$.
    (This is called multiplicative notation.)

    \item When multiplicative notation is being used, we will usually denote the identity of $G$ by 1 and set $x^{0} = 1$.
\end{itemize}
We'll finish off here with a few definitions which will be useful in future discussions.

\begin{definition}[Order of an element]
    Let $G$ be a group and let $x \in G$.
    The order $|x|$ of $x$ is the smallest $n \in \Z^+$ such that $x^{n} = 1$.
\end{definition}

\begin{definition}[Generator]
    Let $G$ be a group, and let $S$ be a subset of $G$.
    We say that $S$ generates $G$ if every element of $G$ can be written as a finite product of elements in $S$ and their inverses.

    In this case, $S$ is a set of generators for $G$ and we write $G \left< S \right>$.
\end{definition}

\begin{definition}[Presentation]
    Let $G$ be a group that is generated by $S$ with a set of relations $R$.
    $G$ has presentation $\left< S \mid R \right>$.
\end{definition}

\section{Some Important Groups}
Now we'll look at a few different kinds of well-known groups.

\begin{definition}[Dihedral group]
    The dihedral group of order $2n$ is the group $D_{2n}$ of symmetries of a regular $n$-gon.
\end{definition}

In general we'll use $r \in D_{2n}$ to denote clockwise rotation by $2\pi / n$ and $s \in D_{2n}$ for reflection through a fixed line of symmetry.
We can get a few nice results from this!
\begin{itemize}
    \item $D_{2n} = \left\{ 1, \, r, \, r^2, \, \ldots, \, r^{n-1} \right\} \cup \left\{ s, \, sr, \, sr^2, \, \ldots, \, sr^{n-1} \right\}$ (where the two sets are disjoint).
    \item $|r| = n$ and $|s| = 2$.
    \item $r^{i} s = s r^{-i}$ for all $i \in \Z$.
\end{itemize}
Thinking of the elements of $D_{2n}$ as physical transformations is useful.
But they're perhaps better understood as \textit{equivalence classes} of physical moves since, for example, $r^{n}$ is equivalent to $r^{2n}$.

\begin{definition}[Symmetric group]
    Let $\Omega$ be a non-empty set, and let $S_\Omega$ be the set of all bijections from $\Omega$ to $\Omega$.
    The set $S_\Omega$ forms a group under function composition, and it is called the symmetric group on $\Omega$.
\end{definition}

Note that if $\Omega = \left\{ 1, \, 2, \, \ldots, \, n \right\}$ then we write $S_\Omega = S_n$.
Permutations on such $\Omega$ can be communicated in several different ways---for example, if we had the bijection
\[ 1 \mapsto 3 \qquad 2 \mapsto 5 \qquad 3 \mapsto 1 \qquad 4 \mapsto 2 \qquad 5 \mapsto 4, \]
then we have the two-line, one-line, and cycle notations, respectively:
\[ \begin{bmatrix} 1 & 2 & 3 & 4 & 5 \\ 3 & 5 & 1 & 2 & 4 \end{bmatrix}, \qquad \begin{bmatrix} 3 & 5 & 1 & 2 & 4 \end{bmatrix}, \qquad (1 \; 3)(2 \; 5 \; 4). \]

% s = (1 3)(2 5 4)
% t = (1 4)(2 3)

% tst^-1 = ( t(1) t(3) )( t(2) t(5) t(4) )
% this is a conjugation of s by t.
% same idea as similar matrices. same transformation, different coordinates!

We again have a few important facts about symmetric groups.
\begin{itemize}
    \item The order of $S_n$ is $n!$ ($|S_n| = n!$).
    \item $S_n$ is non-abelian for $n \geq 3$, but disjoint cycles always commute.
    \item The order of a permutation is the least common multiple of the cycle lengths in its decomposition.
    \item $S_n$ is generated by the adjacent transpositions (the 2-cycles comprised of adjacent elements).
    The group can also be generated by $\left\{ (1 \; 2), \, (1 \; 2 \; \cdots \; n) \right\}$.
\end{itemize}
As a fun fact, $S_3$ can be used to create a permutation representation of $D_6$---if we label the vertices of a triangle with 1, 2, and 3, the movements of the vertices are represented by
\begin{align*}
    e' &\mapsto e & s &\mapsto (2 \; 3) \\
    r &\mapsto (1 \; 2 \; 3) & sr &\mapsto (1 \; 3) \\
    r^2 &\mapsto (1 \; 3 \; 2) & sr^2 &\mapsto (1 \; 2)
\end{align*}
Thus $S_3$ is isomorphic to $D_6$ ($S_3 \cong D_6$).

\begin{definition}[General linear group]
    For each $n \in \Z^+$ let $GL_n(F)$ be the set of all invertible $n \times n$ matrices whose entries come from a field $F$.
    $GL_n(F)$ is a group under matrix multiplication and is called the general linear group of degree $n$.
\end{definition}

\begin{definition}[Quaternion group]
    The quaternion group $Q_8$ has elements $Q_8 = \left\{ 1, -1, \, i, -i, \, j, -j, \, k, -k \right\}$, where 1 is the identity.
    For any $a \in Q_8$, the elements multiply as follows.
    \begin{gather*}
        (-1)^2 = 1, \quad (-1) \cdot a = a \cdot (-1) = -a \\
        i^2 = j^2 = k^2 = -1 \\
        ij = k, \quad jk = i, \quad ki = j \\
        ji = -k, \quad kj = -i, \quad ik = -k
    \end{gather*}
\end{definition}

Finally, given two groups $G$ and $H$, their Cartesian product $G \times H$ is also a group with the binary operation $(a, b)(a', b') = (aa', bb')$.

\section{Homomorphisms}
Now we'll look at different kinds of maps between groups, starting with the simplest one possible.

\begin{definition}[Homomorphism]
    Let $G$ and $H$ be groups.
    A homomorphism from $G$ to $H$ is a function $\varphi : G \to H$ such that, for all $x,y \in G$, $\varphi(xy) = \varphi(x) \varphi(y)$.
    The kernel and image of $\varphi$ are, respectively,
    \[ \ker (\varphi) = \left\{ x \in G \; | \; \varphi(x) = 1 \right\}, \quad \pfn{image} (\varphi) = \left\{ \varphi(x) \; | \; x \in G \right\}. \]
\end{definition}

% in the homework we prove that ker(f) <= G and image(f) <= H.

\begin{theorem}[]
    Let $\varphi : G \to H$ be a homomorphism.
    Then
    \begin{enumerate}[label=(\alph*)]
        \item $\varphi(1)$ is the identity of $H$.
        \item $\varphi(x^{-1}) = \varphi(x)^{-1}$ for all $x \in G$.
        \item $\varphi(x^{n}) = \varphi(x)^{n}$ for all $x \in G$, $n \in \Z$.
    \end{enumerate}
\end{theorem}

If we want to do a better job at preserving structure in our map, we can go a step further.

\begin{definition}[Isomorphism]
    A homomorphism $\varphi : G \to H$ is an isomorphism if it is bijective.
    In this case we say $G$ and $H$ are isomorphic, and we write $G \cong H$.
\end{definition}

The existence of the identity map on $G$ is enough to show that $G \cong G$, but other isomorphisms may exist.
For example, we may fix $g$ and define $\varphi_g : G \to G$ by setting $\varphi_g(x) = g x g^{-1}$ for all $x \in G$.
(This is a particular kind of isomorphism called an inner automorphism.)

\begin{definition}[Automorphism]
    An automorphism of a group $G$ is an isomorphism from G to $G$.
\end{definition}

Notably, the set $\pfn{Aut}(G)$ of automorphisms of $G$ forms a group under function composition!

\section{Group Actions}
We'll finish off this preliminary discussion by looking at what might happen when a group acts on a set.

\begin{definition}[Group action]
    A (left) group action of a group $G$ on a set $X$ is a map from $G \times X$ to $X$, where the image of $(g,x)$ is written as $g \cdot x$ or simply $gx$, such that
    \begin{itemize}
        \item $g(hx) = (gh)x$ for all $g,h \in G$ and $x \in X$.
        \item $1x = x$ for all $x \in X$.
    \end{itemize}
\end{definition}

There are many easily accessible examples of group actions---here's the most glaring one.

\begin{definition}[Left regular action]
    Every group acts on itself by left multiplication.
    This is called the left regular action of $G$.
\end{definition}

As for some others: $\R^\times$ acts on $\R^n$ by scaling, $S_\Omega$ acts on $\Omega$ by permuting, and $D_{2n}$ acts on the vertices of a regular $n$-gon.

\begin{theorem}[]
    Suppose $G$ acts on $X$.
    For each $g \in G$, $\sigma_g(x) = g \cdot x$ defines a permutation of $X$.
    Moreover, the map from $G$ to $S_X$ defined by $g \mapsto \sigma_g$ is a homomorphism.
\end{theorem}

\begin{proof}
    Let $g \in G$.
    Since $\sigma_g \circ \sigma_{g^{-1}}$ and $\sigma_{g^{-1}} \circ \sigma_g$ are both the identity map on $X$, $\sigma_g$ has a two-sided inverse and is therefore a bijection from $X$ to $X$.
    In other words, $\sigma_g$ is a permutation of $X$.

    Now define a map $\varphi : G \to S_X$ such that $\varphi(g) = \sigma_g$.
    We have
    \begin{align*}
        \varphi(gh) (x) &= \sigma_{gh}(x) \\
        &= (gh) \cdot x \\
        &= g \cdot (h \cdot x) \\
        &= \sigma_g (\sigma_h (x)) \\
        &= \left( \varphi(g) \circ \varphi(h) \right)(x)
    \end{align*}
    So we have $\varphi(gh) = \varphi(g) \circ \varphi(h)$ for all $g,h \in G$ and $\varphi$ is a homomorphism.
\end{proof}

All this motivates the following.

\begin{definition}[Representation]
    Let $G$ be a group, and let $n \in \Z^+$.
    \begin{itemize}
        \item A homomorphism $\varphi : G \to S_n$ is called a permutation representation.
        \item A homomorphism $\rho : G \to GL_n(\C)$ is called a linear representation.
    \end{itemize}
\end{definition}

\section{Subgroups}
So far we've engaged in a direct study of groups and some of their most important properties.
Another way we can discern the structure of these objects, though, is to look at any smaller groups embedded inside.

\begin{definition}[Subgroup]
    Let $G$ be a group.
    A subset $H$ of $G$ is a subgroup of $G$ if
    \begin{itemize}
        \item $H$ is nonempty,
        \item $x,y \in H$ implies $xy \in H$, and
        \item $x \in H$ implies $x^{-1} \in H$.
    \end{itemize}
    If $H$ is a subgroup of $G$ then we will write $H \leq G$.
    If we also have $H \neq G$ then we write $H < G$ and call $H$ a proper subgroup of $G$.
\end{definition}

Notice that every group has $\left\{ 1 \right\}$ and itself as subgroups; the latter is called the trivial subgroup of $G$.
If we're working with a finite group, the criteria for subgroups become slightly less strict.

\begin{theorem}[]
    Let $G$ be a finite group.
    A subset $H$ of $G$ is a subgroup of $G$ if $H$ is both nonempty and closed under multiplication.
\end{theorem}

\begin{proof}
    Let $x \in H$ where $x \neq 1$.
    If $H$ is closed under multiplication then $\left\{ x, x^2, x^3, \ldots \right\} \subseteq H$.
    But $G$ is finite, meaning $H$ is also finite and by the pigeonhole principle there exist $a,b \in \Z^+$ such that $x^{a} = x^{b}$ with $b-a \geq 2$.
    Thus $x^{-1} = x^{b-a-1}$, so $x^{-1} \in H$ and $H \leq G$.
\end{proof}

As a side note, we have a handy result about the subgroups of $\Z$.

\begin{theorem}[Subgroups of the integers]
    Every subgroup of $\Z$ has the form $n\Z$ for some $n \in \Z$.
\end{theorem}

\begin{proof}
    Let $H \leq \Z$.
    If $H$ is the trivial subgroup then $H = 0\Z$.
    Otherwise, let $a$ be the smallest positive integer in $H$.
    We claim that $H = a\Z$.

    Suppose, for contradiction, that there is a $b \in H$ such that $b \not\in a\Z$.
    Then let $d = \gcd(a,b)$; since there exist $x,y \in \Z$ such that $xa + yb = d$, we have $d \in H$.
    Since $0 < d < a$, we have a contradiction and $H = a\Z$.
\end{proof}

Now we'll look at a few kinds of subgroups which are fundamental to understanding the broader structure of whatever group we're looking at.

\pagebreak

\begin{definition}[Centralizer]
    The centralizer of $A$ in $G$ is
    \begin{align*}
        C_G(A) &= \left\{ g \in G \mid ga = ag \;\text{for all $a \in A$} \right\} \\
        &= \left\{ g \in G \mid gag^{-1} = a \;\text{for all $a \in A$} \right\}.
    \end{align*}
    That is, it is the set of all elements of $G$ that commute with every element of $A$.
\end{definition}

\begin{definition}[Center]
    The center of $G$ is
    \[ Z(G) = C_G(G). \]
    That is, it is the set of all elements of $G$ that commute with every other element of $G$.
\end{definition}

We may also view these definitions through the lens of conjugations.
Consider, for example, the element $r^{-1} s r$ with $r,s \in D_{2n}$.
We call this the conjugation of $s$ by $r$, and we can understand it as application of $s$ from the ``perspective'' of $r$.
The result is a reflection about line of symmetry running through the vertex just above the horizontal axis.
(The interpretation of $rs r^{-1}$ is very similar, in this case just viewed from the perspective of $r^{-1}$ rather than $r$.)

So conjugation, in a way, entails viewing a group element from the perspective of some other element in the group.
The centralizer of $A$ in $G$, then, is the set of all elements in $G$ that we may go to if we'd like to ``preserve'' $A$ under a shift in perspective, and the center of $G$ is the set of elements that preserve all of $G$ under such a shift.

We may generalize this notion of preservation slightly by requiring only that conjugacy permutes $A$ rather than preserving it.

\begin{definition}[Normalizer]
    The normalizer of $A$ in $G$ is
    \[ N_G(A) = \left\{ g \in G \mid gAg^{-1} = A \right\}, \]
    where $gAg^{-1} = \left\{ gag^{-1} \mid a \in A \right\}$.
\end{definition}

Notice that we've developed a kind of subgroup hierarchy here.
For any group $G$ we have
\[ Z(G) \leq C_G(A) \leq N_G(A) \leq G. \]
Also notice that if $G$ is abelian then $Z(G) = G$ and all of these are simply equal to $G$.
Two more subgroups!

\begin{definition}[Stabilizer]
    If $G$ acts on $X$ and $x \in X$, then the stabilizer of $x$ in $G$ is
    \[ G_x = \left\{ g \in G \mid g \cdot x = x \right\}. \]
\end{definition}

\begin{definition}[Kernel]
    If $G$ acts on $X$ then the kernel of the action is
    \[ \left\{ g \in G \mid g \cdot x = x \;\;\forall\, x \in X \right\} = \bigcap_{x \in X} G_x. \]
\end{definition}

A nice fact that ties all this together: if $N_G(A)$ acts on $A$ by conjugation, then the kernel of the action is the centralizer of $A$ in $G$.

\section{Cyclic Groups}
Now we'll look at a very simple kind of group which often appears as a subgroup.
The first few results about these groups are pretty straightforward, so we'll fly through them quickly.

\begin{definition}[Cyclic group]
    A group is cyclic if it can be generated by one element.
\end{definition}

\begin{theorem}[]
    Any two cyclic groups of the same order are isomorphic.
\end{theorem}

\begin{theorem}[Subgroups of a cyclic group]
    Every subgroup of a cyclic group is cyclic.
\end{theorem}

\begin{proof}
    (Sketch) It suffices to show that if $K$ is a nontrivial subgroup of $\left< x \right>$, then $K = \left< x^{d} \right>$ where $d$ is the smallest positive integer such that $x^{d} \in K$.
    This involves showing that if $x^{n} \in K$ then, by the division algorithm, $d$ divides $a$.
\end{proof}

\begin{theorem}[]
    Let $G$ be a group, let $x \in G$, and suppose $|x| = n < \infty$.
    if $m \in \Z^+$ and $x^{m} = 1$, then $n$ divides $m$.
\end{theorem}

\begin{proof}
    If $x^{m} = 1$ then, by the definition of order, $m \geq n$.
    So $m = kn + r$ for some integers $k \geq 1$ and $0 \leq r < n$ and \vspace{-6pt}
    \[ 1 = x^{m} = x^{kn + r} = \left( x^{n} \right)^{k} x^{r} = x^{r} \]
    and we must have $r = 0$, meaning $m = kn$.
    Thus $n$ divides $m$.
\end{proof}

Now we have a neat connection to an idea from number theory.
Let $(m,n) = \gcd(m,n)$, and let $\varphi$ denote the Euler totient function.

\begin{theorem}[]
    If $G = \left< x \right>$ is a finite cyclic group of order $n$, then for every positive integer $d$ that divides $n$, $\left< x^{n / d} \right>$ is the unique order-$d$ subgroup of $G$.
    Furthermore, $\left< x^{m} \right> = \left< x^{(m,n)} \right>$ for every $m \in \Z$, so the subgroups of $G$ correspond to the divisors of $n$ and $G$ has $\varphi(n)$ generators.
\end{theorem}

\begin{corollary}[]
    If $n$ is a positive integer then $n = \sum_{d \mid n} \varphi(d)$.
\end{corollary}

Finally, one more pair of facts about the orders of elements.

\begin{theorem}[]
    Let $G$ be a group, let $x \in G$, and let $a$ be a nonzero integer.
    If $|x| = n < \infty$, then $|x^{a}| = n / (n,a)$.
\end{theorem}

\begin{corollary}[]
    If $|x| = n < \infty$ and $a$ is a positive integer that divides $n$, then $|x^{a}| = n / a$.
\end{corollary}

\end{document}
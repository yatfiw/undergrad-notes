\documentclass[../m171main.tex]{subfiles}
\graphicspath{{\subfix{../figures/}}}

\begin{document}

\chapter{Group Actions}
\section{Action by Left Multiplication}
We've already seen how groups can act on themselves by left multiplication, but they can also act on their own left cosets by left multiplication.
If $H \leq G$ then $G$ acts on the left cosets of $H$ in $G$ by setting
\[ a \cdot bH = abH \]
for all $a,b \in G$.
We might view this as a generalization of $G$ acting on itself---if $H = \left\{ 1 \right\}$ then there is a bijection between its cosets and the elements of $G$.

\begin{theorem}[]
    Consider the above action of $G$ on $G / H$, and let $\pi_H : G \to S_{G / H}$ denote the permutation representation of $G$ on $G / H$.
    \begin{enumerate}[label=(\alph*),topsep=0pt]
        \item The action is transitive---that is, for any $x,y \in G / H$ there is a $g \in G$ such that $g \cdot x = y$.
        \item The stabilizer $G_{1H} = H$.
        \item $\ker \pi_G = \bigcap_{x \in G} xHx^{-1}$, which is the largest normal subgroup of $G$ contained in $H$.
    \end{enumerate}
\end{theorem}

\begin{proof}
    We'll just prove (c).
    By definition,
    \begin{align*}
        \ker \pi_H &= \left\{ g \in G \;|\; g \cdot xH = xH \textrm{ for all $x \in G$} \right\} \\
        &= \left\{ g \in G \;|\; x^{-1} gx \textrm{ for all $x \in G$} \right\} \\
        &= \left\{ g \in G \;|\; g \in xHx^{-1} \textrm{ for all $x \in G$} \right\} \\
        &= \bigcap_{x \in G} xHx^{-1}.
    \end{align*}
    Now, if $N \trianglelefteq G$ and $N \leq H$ then $N = xNx^{-1} \leq xHx^{-1}$ for all $x \in G$.
    Thus
    \[ N \leq \bigcap_{x \in G} xHx^{-1} = \ker \pi_H \leq H, \]
    so $\ker \pi_H$ is the largest normal subgroup of $G$ contained in $H$.
\end{proof}

We can use this to formalize an observation we made much earlier on, about being able to view some groups through the lens of permuting their elements.

\begin{corollary}[Cayley's theorem]
    Every group is isomorphic to a subgroup of some symmetric group.
\end{corollary}

\begin{proof}
    Let $G$ act on itself by left multiplication.
    This is equivalent to having $G$ act on the left cosets of $H = \left\{ 1 \right\}$, and the associated permutation representation $\pi_H : G \to S_{G / H}$ has a trivial kernel.
    Thus by the first isomorphism theorem
    \[ G \cong G / \left\{ 1 \right\} \cong \pi_H(G) \leq S_{G / H} \]
    and $G$ is isomorphic to a subgroup of $S_{G / H}$.
\end{proof}

We can also generalize our result about subgroups of index 2 being normal.

\begin{corollary}[]
    If $|G| = n$ and $p$ is the smallest prime factor of $n$, then any subgroup $H$ of index $p$ is normal in $G$.
\end{corollary}

\begin{proof}
    Let $K = \ker \pi_H$ and let $k = |H : K|$.
    Then
    \[ |G : K| = |G : H| |H : K| = pk. \]
    By the first isomorphism theorem $G / K$ is isomorphic to $\pi_H(G)$, a subgroup of $S_{G / H} = S_p$.
    So by Lagrange's theorem $|G / K| = pk$ must divide $|S_p| = p!$, and $k \,|\, (p-1)!$.

    Now, the prime divisors of $(p-1)!$ are all less than $p$.
    But the prime divisors of $k$ are all greater than or equal to $p$ by the minimality of $p$.
    Thus $k$ has no prime divisors, $k = 1$, and $H = K \trianglelefteq G$.
\end{proof}

\section{Action by Conjugation}
Groups can also act on themselves by conjugation.

\begin{definition}[Conjugacy class]
    The elements $a,b \in G$ are conjugate in $G$ if there is some $g \in G$ such that $gag^{-1} = b$.
    In other words, $a,b$ are conjugate if they are in the same orbit under conjugation; these orbits are called conjugacy classes.
\end{definition}

% in symmetric groups the conjugacy classes are determined by "cycle types"

We'll make a general statement about group actions and apply it to conjugation.
If we have a group acting on a set $X$, then in order to determine how many elements can be ``reached'' by acting on an $x \in X$, in a rough sense we can divide out the symmetries that leave $x$ fixed.

\begin{theorem}[Orbit-stabilizer theorem]
    Let $G$ be a group acting on a nonempty set $X$.
    For each $x \in X$, the orbit containing $x$ has size $|G : G_x|$.
\end{theorem}

\begin{proof}
    Consider the map from the orbit of $x$ to the left cosets of $G_x$, where
    \[ g \cdot x \mapsto g G_x. \]
    This map is well-defined: if $g \cdot x = h \cdot x$ then $h^{-1}g$ stabilizes $x$, meaning $h^{-1} g \in G_x$ and $gG_x = hG_x$.
    We can trace this argument in reverse to show that the map is injective.
    It is clearly also surjective, meaning the map is a bijection and
    \[ \big| \left\{ g \cdot x \;|\; g \in G \right\} \big| = |G / G_x| = |G : G_x|, \]
    as desired.
\end{proof}

\begin{corollary}[]
    If $a \in G$ then $a$ has $|G : C_G(a)|$ conjugates.
\end{corollary}

\begin{proof}
    Under conjugation, $G_a = \left\{ g \in G \;|\; gag^{-1} = a \right\} = C_G(a)$.
\end{proof}

Now, if an element commutes with every element in $G$, it's in its own conjugacy class.
Considering how the singleton classes together comprise $Z(G)$, and that the conjugacy classes partition $G$, we ge the following.

\pagebreak

\begin{theorem}[Class equation]
    Let $g$ be a finite group and $g_1, \ldots, g_r$ be representatives of the conjugacy classes that are not in the center $Z(G)$.
    Then
    \[ |G| = |Z(G)| + \sum_{j=1}^{r} | G : C_G(g_j) |. \]
\end{theorem}

Notice how each of the summands here must divide $|G|$ by Lagrange's theorem.
This can put some useful constraints on the structure of the group we're working with.

\begin{theorem}[]
    If $P$ is a group of prime-power order $p^{a}$ for some $a \geq 1$, then $Z(P) \neq \left\{ 1 \right\}$.
\end{theorem}

\begin{proof}
    We have the class equation
    \[ |P| = |Z(P)| + \sum_{j=1}^{r} | P : C_P(g_j) |, \]
    where the $g_i$ are representatives of distinct non-central conjugacy classes.
    So $C_P(g_i) \neq P$ by definition, and $p$ divides $|P : C_P(g_i)|$.
    It follows that $p$ also divides $|Z(P)|$, meaning the center is nontrivial.
\end{proof}

Finally, we have a nice application of this theory to combinatorics---in plain terms, if a group $G$ acts on a set $X$ then the associated number of orbits is equal to the average number of $x \in X$ fixed by the $g \in G$.

\begin{theorem}[Cauchy-Frobenius lemma]   % "the lemma that's not burnside's"
    Let $G$ be a finite group acting on a nonempty finite set $X$.
    For each $g \in G$, let
    \[ \pfn{fix}(g) = \big| \left\{ x \in X \;|\; g \cdot x = x \right\} \big|. \]
    Then the number of orbits of the action of $G$ on $X$ is
    \[ \frac{1}{|G|} \sum_{g \in G}^{} \pfn{fix}(g). \]
\end{theorem}

\begin{proof}
    First note that
    \[ \sum_{g \in G}^{} \pfn{fix}(g) = \big| \left\{ (g,x) \in G \times X \;|\; g \cdot x = x \right\} \big| = \sum_{x \in X}^{} |G_x|. \]
    Any $x \in X$ has an orbit of size $|G : G_X|$, meaning the total number of orbits is
    \[ \sum_{x \in X}^{} \frac{1}{|G : G_x|} = \frac{1}{|G|} \sum_{x \in X}^{} |G_x| = \sum_{g \in G}^{} \pfn{fix}(g), \]     % the 1/size thing is prof. o's favorite trick in this class!
    as desired.
\end{proof}

Suppose we want to count the number of ways we can color the vertices of a 3-cycle using two colors.
We can see that there are eight ways to color the vertices, but there are two triplets that are the same up to rigid symmetry.
To formalize this observation, let $D_6$ act on the eight colorings---if we go through each $g \in D_6$, sum the $\pfn{fix}(g)$, and divide by 6 we get 4, the number of orbits for this action.

\section{The Sylow Theorems}
As mentioned before, the first of the Sylow theorems offer a ``sharp'' partial converse to Lagrange's theorem.
It turns out to be useful in a wide variety of contexts, including in classifying groups and, in particular, in showing when a group is not simple---that is, when a group is not trivial and has no nontrivial normal subgroups.
(The simple groups can be thought of as the ``prime'' groups, the unique building blocks of all others.)
We'll define some things before getting into the rest of the theorems.

\begin{definition}[$p$-group]
    Let $G$ be a finite group and let $p$ be prime.
    \begin{itemize}[topsep=0pt]
        \item A group of order $p^{a}$ with $a \geq 0$ is called a $p$-group.
        Subgroups of $G$ that are $p$-groups are called $p$-subgroups.

        \item If $|G| = p^{a} m$ where $p \nmid m$ then a subgroup of $G$ of order $p^{a}$ is a Sylow $p$-subgroup of $G$.
        
        \item The set of Sylow $p$-subgroups of $G$ is denoted by $Syl_p(G)$, and the number of these subgroups is denoted by $n_p(G)$.
    \end{itemize}
\end{definition}

\begin{theorem}[Sylow theorems]
    Let $G$ be a finite group of order $p^{a}m$ where $p$ is prime and $p \nmid m$.
    \begin{enumerate}[label=(\alph*),topsep=0pt]
        \item $Syl_p(G)$ is nonempty.
        \item If $P \in Syl_p(G)$ and $Q$ is a $p$-subgroup of $G$ then $Q$ is contained in a conjugate of $P$.
        \item $n_p(G) \equiv 1 \bmod p$.
        \item If $P \in Syl_p(G)$ then $n_p(G) = |G : N_G(P)|$, implying that $n_p(G)$ divides $m$.
    \end{enumerate}
\end{theorem}

\begin{proof}
    We'll prove part (a) using strong induction.
    If $|G| = 1$ then we're done; otherwise, suppose every group with order less than $|G|$ has a Sylow $p$-subgroup.
    We have two cases.

    If $p$ divides $|Z(G)|$ then by Cauchy's theorem $Z(G)$ contains a subgroup $N$ of order $p$.
    So $|G / N| = p^{a-1} m$ and, by the inductive hypothesis, $G / N$ has a Sylow $p$-subgroup $\overline P$ of order $p^{a-1}$.
    By the fourth isomorphism theorem there is a subgroup $P$ of $G$ such that $P / N = \overline P$ and $|P| = p^{a}$.
    Thus $P$ is a Sylow $p$-subgroup of $G$.

    If $p$ does not divide $|Z(G)|$ then consideration of the class equation
    \[ |G| = |Z(G)| + \sum_{i=1}^{r} |G : C_G(g_i)| \]
    reveals that there must be some $g_i$ such that $p$ does not divide $|G : C_G(g_)i| = p^{a} k$ where $p$ does not divide $k < m$.
    By the inductive hypothesis $C_G(g_i)$ has a Sylow $p$-subgroup $P$ of order $p^{a}$, and $P$ is also a Sylow $p$-subgroup of $G$.
\end{proof}

\begin{corollary}[]
    $P$ is the unique Sylow $p$-subgroup of $G$ if and only if $P \trianglelefteq G$.
\end{corollary}

\begin{corollary}[]
    If $P,P' \in Syl_p(G)$ then $P$ is conjugate to $P'$.
\end{corollary}

\end{document}
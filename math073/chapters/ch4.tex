\documentclass[../m073main.tex]{subfiles}
\graphicspath{{\subfix{../figures/}}}

\begin{document}

\chapter{Eigenvalues and Eigenvectors}

\section{Determinants}
We have already encountered determinants in our study of inverse matrices, albeit not by name.
Here, we generalize the determinant to square matrices of any size.
These first two definitions together define the determinant recursively.

\begin{definition}[Cofactor]
	Let $A$ be a square matrix, and let $A_{ij}$ be the submatrix of $A$ with its $i$th row and $j$th column deleted.
	Then the quantity
	\[ C_{ij} = (-1)^{i+j} \det A_{ij} \]
	is called the $(i,j)$-cofactor of $A$.
	(The quantity $\det A_{ij}$ is called the $(i,j)$-minor of $A$.)
\end{definition}

\begin{definition}[Determinant]
	Let $A$ be an $n \times n$ matrix, where $n \geq 2$.
	Then the determinant of $A$ is the scalar
	\[ \det A = |A| = \sum_{j = 1}^{n} a_{1j} C_{1j}. \]
\end{definition}

Though this definition gives us a way to compute determinants via a ``cofactor expansion'' along the top row of a matrix, we can actually apply the algorithm to any row or column we'd like.
This gives us a lot of flexibility and can possibly save us some work.

\begin{theorem}[Computing determinants]
	The determinant of an $n \times n$ matrix $A$, where $n \geq 2$, can be computed  in two ways:
	\begin{align*}
		\det A &= \sum_{j = 1}^{n} a_{ij} C_{ij} \\
			   &= \sum_{i = 1}^{n} a_{ij} C_{ij}
	\end{align*}
\end{theorem}

There are other cases involving special types of matrices in which finding determinants is especially easy.

\begin{theorem}[Determinant of a triangular matrix]
	The determinant of a triangular matrix is the product of the entries on its main diagonal.
\end{theorem}

Sometimes it will be useful to use row reduction as an alternative to cofactor expansion.
We do this by taking advantage of the following rules.

\begin{theorem}[Determinants via row reduction]
	Let $A$ be a square matrix.
	\begin{enumerate}[label=(\alph*)]
		\item If $A$ has a zero row, then $\det A = 0$.
		\item If $B$ is obtained by interchanging two rows of $A$, then $\det B = -\det A$.
		\item If $A$ has two identical rows, then $\det A = 0$.
		\item If $B$ is obtained by multiplying a row of $A$ by $k$, then $\det B = k \det A$.
		\item If $A$, $B$, and $C$ are identical except the $i$th row of $C$ is the sum of the $i$th rows of $A$ and $B$, then $\det C = \det A + \det B$.
		\item If $B$ is obtained by adding a multiple of one row of $A$ to another row, then $\det B = \det A$.
	\end{enumerate}
\end{theorem}

We'll use these properties to draw a connection between determinants and invertibility.
Elementary matrices will serve as our bridge.

\begin{theorem}[Determinant of an elementary matrix]
	Let $E$ be an $n \times n$ elementary matrix,
	\begin{enumerate}[label=(\alph*)]
		\item If $E$ results from interchanging two rows of $I_n$, then $\det E = -1$.
		\item If $E$ results from multiplying one row of $I_n$ by $k$, then $\det E = k$.
		\item If $E$ results from adding a multiple of one row of $I_n$ to another row, then $\det E = 1$.
	\end{enumerate}
\end{theorem}

This allows us to succinctly summarize parts (b), (d), and (f) of Theorem 4.3 in another theorem.

\begin{lemma}[Determinant after an ERO]
	Let $B$ be a square matrix and let $E$ be an elementary matrix of the same size.
	Then
	\[ \det (EB) = (\det E)(\det B). \]
\end{lemma}

Finally, with the power of the fundamental theorem, we get the following.

\begin{theorem}[Condition for invertibility]
	A square matrix $A$ is invertible if and only if $\det A \neq 0$.
\end{theorem}

Many matrix operations come with their own ways to calculate the determinants of their resulting matrices.
We'll consider scalar multiplication, matrix multiplication, inverses, and transposes; we exclude addition because there is no clear relationship between $\det (A + B)$ and the individual determinants of $A$ and $B$.

\begin{theorem}[Determinant after scalar multiplication]
	If $A$ is an $n \times n$ matrix, then
	\[ \det kA = k^n \det A. \]
\end{theorem}

\begin{theorem}[Determinant after matrix multiplication]
	If $A$ and $B$ are square matrices of the same size, then
	\[ \det (AB) = (\det A)(\det B). \]
\end{theorem}

\begin{theorem}[Determinant after inversion]
	If $A$ is invertible, then
	\[ \det (A^{-1}) = \frac{1}{\det A}. \]
\end{theorem}

\begin{theorem}[Determinant after transposition]
	For any square matrix $A$,
	\[ \det A = \det A^T. \]
\end{theorem}

\section{Eigenvalues and Eigenvectors}
We now begin investigating one of the most central problems to linear algebra: the eigenvalue problem.

\begin{definition}[Eigen-stuff]
	Let $A$ be a square matrix.
	A scalar $\lambda$ is called an eigenvalue of $A$ if there is a nonzero vector $\mbf{x}$ such that $A \mbf{x} = \lambda \mbf{x}$.
	Such a vector $\mbf{x}$ is called an eigenvector of $A$ corresponding to $\lambda$.
	The set of all eigenvectors corersponding to $\lambda$, including the zero vector, is called the eigenspace of $\lambda$ and is denoted by $E_\lambda$.
\end{definition}

The study of eigenvectors is the study of vectors whose directions are invariant under a matrix transformation.
We now discuss how to compute them.

From the above definition, $(\lambda, \mbf{x})$ is an ``eigenpair'' of $A$ if they satisfy the equation
\[ (A - \lambda I) \mbf{x} = \mbf{0}. \]
This gives us an easy way to immediately compute the eigenvalues of $A$.

\begin{theorem}[Computing eigenvalues]
	The eigenvalues of a square matrix $A$ are precisely the solutions $\lambda$ of the equation
	\[ \det (A - \lambda I) = 0. \]
\end{theorem}

This determinant gives us a polynomial in $\lambda$,
Computing the eigenvalues of a matrix, therefore, is equivalent to finding the roots of this polynomial.

\begin{definition}[Characteristic polynomial]
	The polynomial in $\lambda$ resulting from the expansion of $\det (A - \lambda I)$ is called the characteristic polynomial $c_A (\lambda)$ of $A$.
	The equation $\det (A - \lambda I) = 0$ is called the characteristic equation of $A$.
\end{definition}

Once we have our eigenvalues, we use them in the equation $(A - \lambda I) \mbf{x} = 0$ to determine the eigenvectors and, thus, eigenspaces.

\begin{theorem}[Computing eigenvectors]
	The eigenspace of a matrix $A$ corresponding to the eigenvalue $\lambda$ is precisely $\pfn{null} (A - \lambda I)$.
\end{theorem}

The remainder of this chapter will be dedicated to investigating this eigenstuff, both in their own right and in applications.
Firstly, just like with determinants, when our matrix is triangular, our job turns out to be easy.

\begin{theorem}[Eigenvalues of a triangular matrix]
	The eigenvalues of a triangular matrix are the entries on its main diagonal.
\end{theorem}

Similarly, the eigenvalues of a matirix raised to a power relate to those of the original matrix.

\begin{theorem}[Eigenvalues after exponentiation]
	Let $A$ be a square matrix with an eigen-pair $(\lambda, \mbf{x})$.
	\begin{enumerate}[label=(\alph*)]
		\item For any positive $n$, $(\lambda^n, \mbf{x})$ is an eigen-pair of $A^n$.
		\item Is $A$ is invertible, then $(1/\lambda, \mbf{x})$ is an eigen-pair of $A^{-1}$.
		\item If $A$ is invertible, then for any integer $n$, $(\lambda^n, \mbf{x})$ is an eigen-pair of $A^n$.
	\end{enumerate}
\end{theorem}

We can now draw a connection to invertible matrices.
Just like with determinants, for a matrix to be invertible, its eigenvalues must not nonzero.

\begin{theorem}[Condition for invertibility]
	A square matrix is invertible if and only if it does not have 0 as an eigenvalue.
\end{theorem}

With this, we can add two more statements to our fundamental theorem.

\begin{theorem}[Additions to the FTLA]
	Let $A$ be an $n \times n$ matrix.
	The following statements are equivalent.
	\begin{enumerate}[label=(\alph*)]
		\item $A$ is invertible. \\
		\phantom{~}\hspace{-19.5pt} \vdots
		\setcounter{enumi}{13}
		\item $\det A \neq 0$.
		\item 0 is not an eigenvalue of $A$.
	\end{enumerate}
	For the full theorem, see Appendix A.
\end{theorem}

We can use the properties discussed to far to show what happens when we multiply a vector by the same matrix multiple time in succession.
We take advantage of the case in which the vector can be represented as a linear combination of eigenvectors.

\begin{theorem}[Repeated matrix transformations]
	Suppose the $n \times n$ matrix $A$ has eigenpairs $(\lambda_1, \mbf{x}_1), (\lambda_2, \mbf{x}_2), \ldots, (\lambda_m, \mbf{x}_m)$.
	If $\mbf{x}$ is a vector in $\R^n$ that can be expressed as a linear combination of these eigenvectors, then for every integer $k$,
	\[ A^k \mbf{x} = c_1 \lambda_1^k \mbf{v}_1 + c_2 \lambda_2^k \mbf{v}_2 + \cdots + c_m \lambda_m^k \mbf{v}_m. \]
\end{theorem}

We won't always be able to apply this, though, since it's possible that $x$ is not a linear combination of eigenvectors.
As a step toward seeing when we can apply the theorem, we have the following.

\begin{theorem}[Condition for linearly independent eigenvectors]
	Let $A$ be an $n \times n$ matrix and let $\lambda_1, \lambda_2, \ldots, \lambda_m$ be distinct eigenvalues of $A$ with corresponding eigenvectors $\mbf{v}_1, \mbf{v}_2, \ldots, \mbf{v}_m$.
	Then these eigenvectors are linearly independent.
\end{theorem}

\section{Similarity and Diagonalization}
We've already considered a couple of different ways to transform a matrix, namely Gaussian and Gauss-Jordan elimination.
Here, we consider another type of transformation that preserves eigenvalues and other key properties.

\begin{definition}[Similar matrices]
	Let $A$ and $B$ be square matrices of the same size.
	We say that $A$ is similar to $B$ if there is an invertible matrix $P$ such that $P^{-1}AP = B$.
	If $A$ is similar to $B$, we write $A \sim B$.
\end{definition}

Similarity can be interpreted as a kind of equivalence between matrices.
We thus call similarity an equivalence relation, meaning it has the following famiiar properties.

\begin{theorem}[Similarity as an equivalence relation]
	Let $A$, $B$, and $C$ be square matrices of the same size.
	Then:
	\begin{enumerate}[label=(\alph*)]
		\item $A \sim B$.
		\item If $A \sim B$, then $B \sim A$.
		\item If $A \sim B$ and $B \sim C$, then $A \sim C$.
	\end{enumerate}
\end{theorem}

Objects that are related via an equivalence relation usually share important properties.
This is true of similar matrices, and some of these properties are below.

\begin{theorem}[Shared properties of similar matrices]
	Let $A$ and $B$ be square matrices of the same size with $A \sim B$.
	Then:
	\begin{enumerate}[label=(\alph*)]
		\item $\det A = \det B$.
		\item $A$ is invertible if and only of $B$ is invertible.
		\item $A$ and $B$ have the same rank.
		\item $A$ and $B$ have the same characteristic polynomial.
		\item $A$ and $B$ have the same eigenvalues.
		\item $A^m \sim B^m$ for all integers $m \geq 0$.
		\item If $A$ is invertible, then $A^m \sim B^m$ or all integers $m$.
	\end{enumerate}
\end{theorem}

The best possible situation is when a square matrix is similar to a diagonal matrix, because these are really nice to work with.

\begin{definition}[Diagonalizable matrix]
	A square matrix $A$ is diagonalizible if there is a diagonal matrix $D$ such that $A$ is similar to $D$---that is, if there is an invertible matrix $P$ such that $P^{-1}AP = D$.
\end{definition}

The idea of diagonalization is intimately related to eigenvalues and eigenvectors.
Not only do they give us a condition for diagonalizability, but they help us do the actual diagonalization.

\begin{theorem}[Condition and method for diagonalization]
	Let $A$ be an $n \times n$ matrix.
	Then $A$ is diagonalizible if and only if $A$ has $n$ linearly independent eigenvectors.

	More precisely, there exist an invertible matrix $P$ and a diagonal matrix $D$ such that $P^{-1}AP = D$ if and only if the columns of $P$ are $n$ linearly independent eigenvectors of $A$ and the diagonal entries of $D$ are the eigenvalues of $A$ corresponding to the eigenvectors in $P$ in the same order.
\end{theorem}

Usually, we would have to verify that $P$ is invertible before using it to find a matrix similar to $A$.
The next theorem shows that this is not necessary for diagonalization problems.

\begin{theorem}[Basis eigenvectors are linearly independent]
	Let $A$ be a square matrix and let $\lambda_1, \lambda_2, \ldots, \lambda_k$ be distinct eigenvalues of $A$.
	If $\mathcal{B}_i$ is a basis for the eigenspace $E_\lambda$, then $\displaystyle\bigcup_{i=1}^k\mathcal{B}_i$ is linearly independent.
\end{theorem}

This shows that linear independence is preserved when the bases of different eigenspaces are combined.
So when we have $n$ different eigenspaces, each of dimension one, we automatically have a diagonalizable matrix.

\begin{theorem}[Distinct eigenvalues imply diagonalizibility]
	If $A$ is an $n \times n$ matrix with $n$ distinct eigenvalues, then $A$ is diagonalizible.
\end{theorem}

When this theorem isn't applicable, there's still a way we can use eigenvalues to determine diagonalizibility.
It requires a preliminary definition.

\begin{definition}[Algebraic and geometric multiplicity]
	The algebraic multiplicity of an eigenvalue $\lambda$ is its multiplicity as a root of the characteristic polynomial.
	The geometric multiplicity of the eigenvalue is $\pfn{dim} (E_\lambda)$
\end{definition}

\begin{lemma}[Diagonalizibility from multiplicities]
	If $A$ is a square matrix, then the geometric mutiplicity of each eigenvalue is less than or equal to its algebraic multiplicity.
\end{lemma}

We summarize the different ways to check for diagonalizibility in one final theorem.

\begin{theorem}[Conditions for diagonalizibility]
	Let $A$ be an $n \times n$ matrix.
	The following statements are equivalent.
	\begin{enumerate}[label=(\alph*)]
		\item $A$ is diagonalizible.
		\item The union $\mathcal{B}$ of the bases of the eigenspaces of $A$ contains $n$ vectors.
		\item The algebraic multiplicity of each eigenvalue equals its geometric multiplicity.
	\end{enumerate}
\end{theorem}

\end{document}
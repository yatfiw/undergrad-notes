\documentclass[../m73main.tex]{subfiles}
\graphicspath{{\subfix{../figures/}}}

\begin{document}

\chapter{Matrices}

\section{Matrix Operations}
Now we turn our attention to matrices in their own right, rather than as a means to understand vectors and linear equations.

\begin{definition}[Matrix]
	A matrix is a rectangular array of numbers called the entries, or elements, of the matrix.
	A matrix all of whose entries are zero is called the zero matrix $O$.
\end{definition}

There are several different kinds of matrices, all of which have their own special properties.
Here are some of the simpler forms.

\begin{definition}[Diagonal matrices]
	Let $A$ be an $m \times n$ matrix.
	\begin{itemize}
		\item If $ m =n$, then $A$ is called a square matrix.
		\item A square matrix whose nondiagonal entries are all zero is called a diagonal matrix.
		\item A diagonal matrix all of whose diagonal entries are the same is called a scalar matrix.
		\item If the scalar on the diagonal is 1, the scalar matrix is called an identity matrix.
	\end{itemize}
\end{definition}

Like vectors, matrices can be added and scaled.

\begin{definition}[Matrix addition]
	Let $A = [a_{ij}]$ and $B_{ij}$ be $m \times n$ matrices.
	Then their sum is defined componentwise:
	\[ A + B = [a_{ij} + b_{ij}]. \]
\end{definition}

\begin{definition}[Scalar multiplication of matrices]
	Let $A$ be an $m\times n$ matrix and let $c$ be a scalar.
	Then the scalar multiple is defined componentwise:
	\[ cA = [ca_{ij}]. \]
\end{definition}

Matrices can also be multiplied with each other, albeit in an at-first unintuitive way.
The idea is that a matrix can be thought of as a function whose chief purpose is to transform vectors.

Suppose we want to multiply $A \mbf{x}$, where $A$ is a matrix and $\mbf{x}$ is a vector.
We can treat $A$ as some kind of function that maps one vector to another.
The first column of $A$ tells us what happens to the first component $\mbf{x}$, the second column determines the second component, and so on.
If $\mbf{a}_i$ is the $i$th column of $A$, then the resulting vector is the linear combination
\[ A \mbf{x} = x_1 \mbf{a}_1 + x_2 \mbf{a}_2 + \cdots + x_n \mbf{a}_n. \]
When we multiply matrices, we're essentially determining the overall effect of multiple transformations.
This motivates the below definition of the matrix product.

\begin{definition}[Matrix multiplication]
	If $A$ is an $m\times n$ matrix and $B$ is an $n\times r$ matrix, then the product $C = AB$ is an $m\times r$ matrix.
	The $(i,j)$ entry of the product is computed as follows:
	\[ c_{ij} = \sum_{k=1}^n a_{ik} b_{kj}. \]
	The matrix-column and row-matrix representations of the product are, respectively,
	\[ AB = \begin{bmatrix} A \mbf{b}_1 & A \mbf{b}_2 & \cdots & A \mbf{b}_r \end{bmatrix} \;\text{ and }\; AB = \begin{bmatrix} \mathbf{A}_1 B \\ \mathbf{A}_2 B\\ \vdots \\ \mathbf{A}_m B \end{bmatrix}. \]
\end{definition}

Sometimes it will be useful to use matrix multiplication to extract a particular row or column from a matrix.
It's pretty easy to do using standard unit vectors.

\begin{theorem}[Extracting a row or column]
	Let $A$ be an $m\times n$ matrix, and let $e_i$ and $e_j$ be row and column versions of standard unit vectors.
	Then
	\begin{itemize}
		\item $\mbf{e}_iA$ is the $i$th row of $A$ and
		\item $A\mbf{e}_j$ is the $j$th column of $A$.
	\end{itemize}
\end{theorem}

We'll define two more operations we can perform on matrices.
The first is repeated multiplication, which is written as an exponent.

\begin{definition}[Matrix exponentiation]
	If $A$ is a square matrices, then $A^n = AA\cdots A$, where there are $n$ factors in this product.
\end{definition}

We'll also find it helpful to define something completely new and different: flipping a matrix over its main diagonal.

\begin{definition}[Transpose of a matrix]
	The transpose of an $m\times n$ matrix $A$ is the $n\times m$ matrix $A^T$ obtained by interchanging the rows and columns of $A$.
	That is, the $i$th column of $A^T$ is the $i$th row of $A$ for all $i$, or $(A^T)_{ij} = A_{ji}$ for all $i, j$.
\end{definition}

This allows us to define another special, usuallly non-diagonal type of matrix that exhibit symmetry with respect to transposition.

\begin{definition}[Symmetric matrix]
	A square matrix $A$ is symmetric if $A^T = A$.
	That is, $A$ is symmetric if and only if $A_{ij} = A_{ji}$ for all $i$ and $j$.
\end{definition}

Note that all diagonal matrices are symmetric.

Using these matrix operations, we can define one last operation between vectors.
It's minor in the context of this course, but still sometimes useful.

\begin{definition}[Outer product]
	Let $\mbf{u}$ and $\mbf{v}$ be vectors in $\R^m$ and $\R^n$, respectively.
	Then their outer product is defined as the matrix product $\mbf{u}\mbf{v}^T$.
\end{definition}

\section{Matrix Algebra}
We use the definitions from the previous section to derive some important properties for matrix algebra.
We begin with the basic operations of addition and scalar multiplication.

\begin{theorem}[Properties of matrix addition and scalar multiplication]
	Let $A$, $B$, and $C$ be matrices of the same size and let $c$ and $d$ be scalars.
	Then
	\begin{enumerate}[label=(\alph*)]
		\item $A + B = B + A$
		\item $(A + B) + C = A + (B + C)$
		\item $A + O = A$
		\item $A + (-A) = O$
		\item $c(A + B) = cA + cB$
		\item $(c+d)A = cA + dA$
		\item $c(dA) = (cd)A$
		\item $1A = A$
	\end{enumerate}
\end{theorem}

(Note the similarities to the properties of vector addition and scalar multiplication!)
Next, we analyze matrix multiplication.

\begin{theorem}[Properties of matrix multiplication]
	Let $A$, $B$, and $C$ be matrices (such that all indicated operations are defined) and let $k$ be a scalar.
	Then
	\begin{enumerate}[label=(\alph*)]
		\item $A(BC) = (AB)C$
		\item $A(B+C) = AB+AC$
		\item $(A+B)C = AC+BC$
		\item $k(AB) = (kA)B = A(kB)$
		\item $I_mA = A = AI_n$
	\end{enumerate}
\end{theorem}

Most of these properties are familiar, but notice that we didn't include the commutative property anywhere.
That's because matrix multiplication isn't commutative!
It isn't generally true that $AB = BA$.

Finally, let's talk about matrix transposition.

\begin{theorem}[Properties of matrix transposition]
	Let $A$ and $B$ be matrices (such that all indicated operations are defined) and let $k$ be a scalar.
	Then
	\begin{enumerate}[label=(\alph*)]
		\item $(A^T)^T = A$
		\item $(A+B)^T = A^T + B^T$
		\item $(kA)^T = k(A^T)$
		\item $(AB)^T = B^TA^T$
		\item $(A^r)^T = (A^T)^r$
	\end{enumerate}
\end{theorem}

We can use some of these to show what happens when we add or multiply a matrix by its own transpose.

\begin{theorem}[Adding or multiplying by a transpose]
	We have two relationships.
	\begin{enumerate}[label=(\alph*)]
		\item If $A$ is a square matrix, then $A + A^T$ is a symmetric matrix.
		\item For any matrix $A$, $AA^T$ and $A^TA$ are symmetric matrices.
	\end{enumerate}
\end{theorem}

\section{The Inverse of a Matrix}
The additive inverse of a matrix is obvious.
Our attention now turns to deriving the multiplicative inverse of a matrix.

\begin{definition}[Inverse of a matrix]
	If $A$ is a square matrix, an inverse of $A$ is a matrix $A^{-1}$ of the same size with the property that
	\[ A A^{-1} = I = A^{-1} A. \]
	If such an $A^{-1}$ exists, then $A$ is called invertible.
\end{definition}

Just like with scalars, each invertible matrix can only have one inverse.

\begin{theorem}[Uniqueness of the matirx inverse]
	If $A$ is an invertible matrix, then its inverse is unique.
\end{theorem}

Since we can use matrix inverses to ``undo'' the effects of matrix multiplication, we can use them to solve certain equations involving matrices.

\begin{theorem}[Unique solution of a linear system]
	If $A$ is an invertible $n\times n$ matrix, then the system of linear equations given by $A\mbf{x} = \mbf{b}$ has the unique solution $\mbf{x} = A^{-1}\mbf{b}$ for any $\mbf{b}\in\R^n$.
\end{theorem}

This is great, but how do we actually compute these matrix inverses?
We'll arrive at a general method shortly, but for now, the two-by-two case is relatively straightforward.

\begin{theorem}[Inverse of a two-by-two matrix]
	If $A = \begin{bmatrix} a & b \\ c & d \end{bmatrix}$, then $A$ is invertible if $ad - bc \neq 0$, in which case
	\[ A^{-1} = \frac{1}{ad-bc} \begin{bmatrix} d & -b \\ -c & a \end{bmatrix}. \]
	If $ad-bc = 0$, then $A$ is not invertible.
\end{theorem}

To begin our search for an inverse matrix algorithm, we'll list out some significant properties of inverses.

\begin{theorem}[Properties of matrix inverses]
	If $A$ and $B$ are invertible matrices of the same size, then the following are true:
	\begin{enumerate}[label=(\alph*)]
		\item $A^{-1}$ is invertible and $(A^{-1})^{-1} = A$.
		\item If $c$ is a nonzero scalar, then $cA$ is invertible and $(cA)^{-1} = \frac{1}{c}A^{-1}$.
		\item $AB$ is invertible and $(AB)^{-1} = B^{-1}A^{-1}$.
		\item $A^T$ is invertible and $(A^T)^{-1} = (A^{-1})^T$.
		\item $A^n$ is invertible for all nonnegative integers $n$ and $(A^n)^{-1} = (A^{-1})^n$.
	\end{enumerate}
\end{theorem}

As a side note, property (e) allows us to define negative integer powers of a matrix.

\begin{definition}[Negative powers of a matrix]
	If $a$ is an invertible matrix and $n$ is a positive integer, then $A^{-n}$ is defined by
	\[ A^{-n} = (A^{-1})^n = (A^n)^{-1}. \]
\end{definition}

We're now in a positon to introduce a new type of matrix that will lead us to the inverse algorithm.

\begin{definition}[Elementary matrix]
	An elementary matrix is any matrix that can be obtained by performing an elementary row operation on an identity matrix.
\end{definition}

Elementary matrices are useful because they are representations of the EROs performed on them.
In this way, performing an ERO on a matrix is equivalent to multiplying by the corresponding elementary matrix.

\begin{theorem}[Multiplication by an elementary matrix]
	Let $E$ be the elementary matrix obtained by performing an elementary row operation on $I_n$.
	If the same elementary row operation is performed on an $n\times r$ matrix $A$, the result is the same as the product $EA$.
\end{theorem}

For any ERO we might apply to a matrix, we can also undo it with the opposite ERO.
The elementary-matrix analog of this fact is described in terms of inverses.

\begin{theorem}[Inverse of an elementary matrix]
	Each elementary matrix is invertible, and its inverse is an elementary matrix of the same type.
\end{theorem}

This allows us to begin formulating what is arguably the most important theorem of this course: the fundamental theorem of linear algebra (FTLA).

\begin{theorem}[Additions to the FTLA]
	Let $A$ be an $n\times n$ matrix.
	The following statements are equivalent:
	\begin{enumerate}[label=(\alph*)]
		\item $A$ is invertible.
		\item $A\mbf{x}= \mbf{b}$ has a unique solution for every $\mbf{b}\in\R^n$.
		\item $A\mbf{x}=\mbf{0}$ has only the trivial solution.
		\item The reduced row echelon form of $A$ is $I_n$.
		\item $A$ is a product of elementary matrices.
	\end{enumerate}
	For the full theorem, see Appendix A.
\end{theorem}

This is a powerful theorem!
It gives us everything we need to prove these last two very important theorems, the first a stepping stone for the next.

\begin{theorem}[Condition for matrix invertibility]
	Let $A$ be a square matrix.
	If $B$ is a square matrix such that either $AB = I$ or $BA = I$, then $A$ is invertible and $B = A^{-1}$.
\end{theorem}

\begin{theorem}[Computing the inverse of a matrix]
	Let $A$ be a square matrix.
	If a series of elementary row operations reduces $A$ to $I$, then the same series of elementary row operations transforms $I$ into $A^{-1}$.
\end{theorem}

Therefore, in order to invert a matrix, we simply reduce it to the identity matrx while simultaneously performing the same sequence of row operations on the identity matrix.

\section{Subspaces, Basis, Dimension, and Rank}
Now, we'll introduce the ideas of spaces and subspaces, and we'll see how they relate to matrices.

\begin{definition}[Subspace]
	A subspace of $\R^n$ is any collection $S$ of vectors in $\R^n$ such that:
	\begin{itemize}
		\item The zero vector $\mbf{0}$ is in $S$.
		\item $S$ is closed under addition.
		\item $S$ is closed under scalar multiplication.
	\end{itemize}
\end{definition}

A very simple example of a subspace is the set of all vectors that can be reached by just a few vectors.

\begin{theorem}[Span as a subspace]
	Let $S$ be a set of vectors in $\R^n$.
	Then $\pfn{span}(S)$ is a subspace of $\R^n$.
\end{theorem}

In this section, we will focus on three particular examples of subspaces, two of which are constructed by taking the span of some vectors.

\begin{definition}[Row space and column space]
	Let $A$ be an $m\times n$ matrix.
	\begin{itemize}
		\item The row space $\pfn{row}(A)$ of $A$ is the subspace of $\R^n$ spanned by the rows of $A$.
		\item The column space $\pfn{col}(A)$ of $A$ is the subspace of $\R^m$ spanned by the columns of $A$.
	\end{itemize}
\end{definition}

Drawing a connection to something we already know, we can note that a matrix's row space is invariant under elementary row operations.
(The column space is also invariant under ``column'' operations, but that's less relevant to us).

\begin{theorem}[Row spaces of row equivalent matrices]
	Let $B$ be any matrix that is row equivalent to a matrix $A$.
	Then $\pfn{row}(B) = \pfn{row}(A)$.
\end{theorem}

As for the third space of interest, we should also know that the solution set of a homogeneous linear system is a subspace.
This subspace, like the row and column spaces, has a special name.

\begin{theorem}[Solution space of a homogeneous linear system]
	Let $A$ be an $m\times n$ matrix and let $S$ be the set of solutions of the homogeneous linear system $A\mbf{x} = \mbf{0}$.
	Then $S$ is a subspace of $\R^n$.
\end{theorem}

\begin{definition}[Null space]
	Let $A$ be an $m\times n$ matrix.
	The null space $\pfn{null}(A)$ of $A$ is the subspaace of $A$ consisting of solutions of the homogeneous linear system $A\mbf{x} = \mbf{0}$.
\end{definition}

Now, given any two solutions $\mbf{x}_1$ and $\mbf{x}_2$ to the linear system $A\mbf{x} = \mbf{b}$, their difference $\mbf{x}_0 = \mbf{x}_2 - \mbf{x}_1$ solves the associated homogeneous linear system $A\mbf{x} = \mbf{0}$.
If $\mbf{x}_1$ and $\mbf{x}_2$ are distinct, then $\mbf{x}_0 \neq \mbf{0}$ and $\pfn{null} (A)$ is nontrivial.
Since $\mbf{x}_1 + c\mbf{x}_0$ solves the linear system for all $c \in \R$, the system must have infinitely many solutions.
This discussion is summarized in the following theorem.

\begin{theorem}[Number of solutions to a linear system]
	Let $A$ be a matrix whose entries are real numbers.
	For any system of linear equations $A\mbf{x} = \mbf{b}$, exactly one of the following is true:
	\begin{enumerate}[label=(\alph*)]
		\item There is no solution.
		\item There is a unique solution.
		\item There are infinitely many solutions.
	\end{enumerate}
\end{theorem}

Let's move our discussion along with a new definition.

\begin{definition}[Basis of a subspace]
	A basis for a subspace $S$ of $\R^n$ is a set of vectors in $S$ that
	\begin{itemize}
		\item spans $S$ and
		\item is linearly independent.
	\end{itemize}
\end{definition}

A basis is the minimum information required to completely describe a subspace, so there's a lot of value in knowing it.
Thankfully, for each subspace we've discussed, there is a straightforward way to find a basis.

\begin{definition}[Determining bases for row, column, and null spaces]
	To find bases for the row, column, and null spaces of a matrix $A$:
	\begin{enumerate}
		\item Find the reduced row echelon form $R$ of $A$.
		\item $\pfn{row}(A)$: Use the nonzero row vectors of $R$.
		\item $\pfn{col}(A)$: Use the column vectors of $A$ that correspond to the columns of $R$ with the leading 1s.
		\item $\pfn{null}(A)$: Solve the linear system $R\mbf{x} = \mbf{0}$ and write the solution as a linear combination of vectors times free variables.
	\end{enumerate}
\end{definition}

No subspace (except for the trivial space $\{ \mbf{0} \}$) has a unique basis.
However, every possible basis for a subspace will have the same number of vectors, making this number an important intrinsic property of subspaces.

\begin{theorem}[Sizes of distinct bases]
	Let $S$ be a subspace of $\R^n$.
	Then any two bases for $S$ have the same number of vectors.
\end{theorem}

\begin{definition}[Dimension]
	If $S$ is a subspace of $\R^n$, then the number of vectors in a basis for $S$ is called the dimension of $S$, denoted by $\dim(S)$.
\end{definition}

The dimensions of the row and column space do not change under the influence of elementary row operations.
Therefore, since in RREF the dimensions of the row and column space are the same, this is also true for the non-RREF matrix.

\begin{theorem}[Row and column space dimensions]
	The row and column spaces of a matrix $A$ have the same dimension.
\end{theorem}

This quantity, the dimension of the row and column space, also gets a name.

\begin{definition}[Rank]
	The rank $\pfn{rank}(A)$ of a matrix $A$ is the dimension of its row and column space.
\end{definition}

When we take the transpose of a matrix, its row and column spaces simply get swapped.
The following, then, should not be surprising.

\begin{theorem}[Rank of a transpose]
	For any matrix $A$, $\pfn{rank}(A^T) = \pfn{rank}(A)$.
\end{theorem}

Just as the dimension of the row and column space has a name, so too does the dimension of the null space.

\begin{definition}[Nullity]
	The nullity $\pfn{nullity}(A)$ of a matrix $A$ is the dimension of its null space.
\end{definition}

Roughly, the nullity quantifies how many vectors are mapped to $\mbf{0}$ by the matrix, and the rank quantifies how many vectors are not mapped to $\mbf{0}$.
It should not be surprising, then, that these quantities add up to the total ``amount'' of vectors being acted upon.

\begin{theorem}[Rank theorem]
	If $A$ is an $m\times n$ matrix, then
	\[ \pfn{rank}(A) + \pfn{nullity}(A) = n. \]
\end{theorem}

At this point, we can add a bit more to the fundamental theorem.

\begin{theorem}[Additions to the FTLA]
	Let $A$ be an $n\times n$ matrix.
	The following statements are equivalent:
	\begin{enumerate}[label=(\alph*)]
		\item $A$ is invertible. \\
		\phantom{~}\hspace{-19.5pt} \vdots
		\setcounter{enumi}{5}		
		\item $\pfn{rank}(A) = n$
		\item $\pfn{nullity}(A) = 0$
		\item The column vectors of $A$ are linearly independent.
		\item The column vectors of $A$ span $\R^n$.
		\item The column vectors of $A$ form a basis for $\R^n$.
		\item The row vectors of $A$ are linearly independent.
		\item The row vectors of $A$ span $\R^n$.
		\item The row vectors of $A$ form a basis for $\R^n$.
	\end{enumerate}
	For the full theorem, see Appendix A.
\end{theorem}

The following is a small application of both the FTLA and the rank theorem.

\begin{theorem}[Rank and invertibility of $A^TA$]
	Let $A$ be an $m\times n$ matrix.
	Then:
	\begin{enumerate}[label=(\alph*)]
		\item $\pfn{rank}(A^TA) = \pfn{rank}(A)$
		\item The $n\times n$ matrix $A^TA$ is invertible if and only if $\pfn{rank}(A) = n$
	\end{enumerate}
\end{theorem}

There's one more very important reason why we're concerned with bases.
They allow us to uniquely represent vectors in a particular subspace.

\begin{theorem}[Unique representation of a vector]
	Let $S$ be a subspace of $\R^n$ and let $\mathcal{B}$ be a basis for $S$.
	For every vector $\mbf{v}$ in $S$, there is exactly one way to write $\mbf{v}$ as a linear combination of the basis vectors in $\mathcal{B}$.
\end{theorem}

Eventually, we'll use this idea to construct arbitrary coordinate systems for subspaces, all based around the bases of those subspaces.
More on that later.

\end{document}
\documentclass[../m055main.tex]{subfiles}
\graphicspath{{\subfix{../figures/}}}

\begin{document}

\chapter{Number Theory}

\section{Introduction}
In this chapter we'll extend our discussion to the set of integers $\Z$.
We don't know anything about primes, GCDs, or modular arithmetic yet---only the basic properties of integer arithmetic.

We begin by formalizing the idea of division and remainders, as we learned back in fourth grade.
Remember that, when we divide two integers $a$ and $b$, we're left with a unique quotient $q$ and remainder $r$.

\begin{theorem}[Division algorithm]
    For any integers $a$ and $b > 0$, there exist unique integers $q$ and $r$ such that
    \[ a = bq + r, \quad 0 \leq r < b. \]
\end{theorem}

Oftentimes, we're only interested in the remainder that comes from this division.
In this case, we write $r = a \bmod b$.

The division algorithm is a powerful tool in number theory!
We can use it to define and prove a variety of important things, the first of these being the notion of divisibility.

\begin{definition}[Divisibility]
    We say that $b$ divides $a$ (or $b$ is a divisor of $a$) if there exists an integer $q$ such that $a = bq$.
    When $b$ divides $a$, we write $b \mid a$.
\end{definition}

One intuitive consequence of this theorem is that the ``divisibility relation'' is transitive.

\begin{theorem}[Transitive divisibility]
    If $c \mid b$ and $b \mid a$ then $c \mid a$.
\end{theorem}

\begin{proof}
    If $c \mid b$ and $b \mid a$, then there exist integers $p,q$ such that $b = cp$ and $a = bq$.
    So $a = (cp)q = (pq)c$, and since $pq$ is an integer, $c \mid a$.
\end{proof}

We may use this to prove a much more general and useful theorem about integer combinations.
(These are like linear combinations, but with strictly integer coefficients.)

\begin{theorem}[Integer combination theorem]
    If $d \mid a$ and $d \mid b$ then $d \mid ax + by$ for all $x,y \in \Z$.
\end{theorem}

\begin{proof}
    If $d \mid a$ and $d \mid b$, then there exist integers $p,q$ such that
    \[ a = dp \;\text{ and }\; b = dq. \]
    Multiply the equations by the integers $x$ and $y$, respectively, to get
    \[ ax = dpx \;\text{ and }\; by = dqy. \]
    Adding these equations together gives $ax + by = dpx + dqy$ or, alternatively,
    \[ ax + by = d(px + qy). \]
    Since $px + qy$ is an integer, $d \mid ax + by$.
\end{proof}


With this groundwork laid, we can begin talking about GCDs and related topics.

\section{Greatest Common Divisors}
The greatest common divisor between two numbers $a$ and $b$ is the largest divisor that the numbers have in common.
This GCD is represented by $\gcd (a,b)$ or, more compactly, $(a,b)$.
(We define $(a,0) = a$, and $(0,0)$ is undefined.)
If $(a,b) = 1$, then we say that $a$ and $b$ are relatively prime.

In this section are two significant theorems regarding the GCD.
In preparation, we have another intuitive fact, which we state as a lemma.

\begin{lemma}
    For positive integers $d$ and $a$, if $d \mid a$, then $d \leq a$.
\end{lemma}

\begin{proof}
    Suppose $d \mid a$.
    Then $a = dq$ for some integer $q \geq 1$, meaning
    \begin{align*}
        a-d &= dq - d \\
        &= d (q-1)
    \end{align*}
    Since $q-1 \geq 0$, $a-d \geq 0$ and $d \leq a$.
\end{proof}

From this, we get our first wildly unintuitive theorem.

\begin{theorem}[Bezout's theorem]
    For integers $a,b$, $\gcd (a,b)$ is an integer combination of $a$ and $b$.
    That is, there exist integers $x$ and $y$ such that $ax + by = \gcd (a,b)$.

    \medskip
    \textit{Note: This integer combination is not unique.}
\end{theorem}

\begin{proof}
    Let $g = \gcd (a,b)$ and let $l = ax_0 + by_0$ be the smallest integer of the form $ax + by$.
    We will prove that $g = l$ by showing that (i) $g \leq l$ and (ii) $l \leq g$.
    \smallskip

    (i) By definition, $g \mid a$ and $g \mid b$.
    So by the integer combination theorem, $g \mid ax + by$ for all $x,y$, meaning $g \mid l$.
    Therefore, $g \leq l$.
    \smallskip

    (ii) We will show that $l$ is a common divisor of $a$ and $b$.
    Suppose, to the contrary, that $l \nmid a$; by the division algorithm,
    \[ a = lq + r, \quad 0 < r < l. \]
    We may rewrite this equation as
    \begin{align*}
            r &= a - lq \\
            &= a - (ax_0 + by_0)q \\
            &= a(1 - x_0q) + b(-y_0q)
    \end{align*}
    So $r$ is an integer combination of $a$ and $b$.
    But, since $0 < r < l$, this contradicts the definition of $l$ as the smallest positive integer combination of $a$ and $b$.
    Therefore $l$ is a common divisor of $a$ and $b$, and since $g$ is the greatest of these common divisors, $l \leq g$.
\end{proof}

This result is important in its own right, but it also has a very useful corollary.

\begin{corollary}
    $a$ and $b$ are relatively prime if and only if there exist $x,y \in \Z$ such that $ax + by = 1$.
\end{corollary}

\begin{proof}
    ($\Rightarrow$) Suppose $a$ and $b$ are relatively prime.
    It immediately follows from Bezout's theorem that there exist $x,y \in \Z$ such that $ax + by = 1$.
    \smallskip

    ($\Leftarrow$) Conversely, suppose there exist $x,y \in \Z$ such that $ax + by = 1$.
    Let $d$ be a common divisor of $a$ and $b$; by the integer combination theorem, $d \mid ax + by$, meaning $d \mid 1$.
    So the only common divisors of $a$ and $b$ are $\pm 1$, meaning $\gcd (a,b) = 1$ and $a$ and $b$ are relatively prime.
\end{proof}

Bezout's theorem is one of the two ``pillars'' of this section on GCDs.
The other is Euclid's theorem, stated below.

\begin{theorem}[Euclid's theorem]
    For $a,b,x \in \Z$, $\gcd (a,b) = \gcd (b, a - bx)$.
\end{theorem}

\begin{proof}
    By the integer combination theorem, any common divisor of $a$ and $b$ is a common divisor of $b$ and $a - bx$ and vice versa.
\end{proof}

Arguably the most useful special case of this theorem applies the division algorithm to quickly compute GCDs.

\begin{corollary}[Euclidean algorithm]
    $\gcd (a,b)$ can be found by repeatedly computing $\gcd (b, a \bmod b)$.
\end{corollary}

\begin{proof}
    By the division algorithm, for any integers $a,b$ there exist integers $q$ and $r$, $0 \leq r \leq b$, such that $a = bq + r$.
    Let $x = q$ in Euclid's theorem.
\end{proof}

Euclid's algorithm is very fast.
Its worst-case runtime is when it acts on two consecutive Fibonacci numbers (in which case the ``quotient'' is always zero), and even then it will (by Lame's theorem) only take at most $5k$ steps, where $k$ is the number of digits in $b \leq a$.

Not only does Euclid's algorithm find $(a,b)$, but it can also help us determine the coefficients of the integer combination that gives $(a,b)$.
There are two methods of doing this, both of which are demonstrated below.

\begin{example}[Bottom-up method]
    Suppose we want to find integers $x$ and $y$ that satisfy $(847, 203) = 847x + 203y$.
    First, we compute the GCD using Euclid's algorithm:
    \[ (847,203) = (203,35) = (35,28) = (28,7) = (7,0) = 7 \]
    Now, we'll back-trace the above computation to write each $a \bmod b$ in the form $a - bq$:
    \begin{align*}
        7 &= 35 - 28 \\
        &= 35 - (203 - 5 \cdot 35) \\
        &= 6 \cdot 35 - 1 \cdot 203 \\
        &= 6 \cdot (647 - 4 \cdot 203) - 1 \cdot 203 \\
        &= 6 \cdot 847 - 35 \cdot 203
    \end{align*}
    So the desired integers are $x = 6$ and $y = -35$.
\end{example}

\pagebreak

\begin{example}[Top-down method]
    Suppose we want to find integers $x$ and $y$ that satisfy $(847, 203) = 847x + 203y$.
    First, we compute the GCD using Euclid's algorithm, as we did above.
    Now, we'll trace the above computation and write each new number as a linear combination of the algorithm's ``roots''.
    \begin{alignat*}{3}
                     (1) \qquad &&      a \;\phantom{+}&&\;       =&\; 847 \\
                     (2) \qquad &&        \;\phantom{+}&&\;     b =&\; 203 \\
        (1) - 4(2) = (3) \qquad &&      a \;         - &&\;    4b =&\; 35 \\
        (2) - 5(3) = (4) \qquad &&    -5a \;         + &&\;   21b =&\; 28 \\
         (3) - (4) = (5) \qquad &&     6a \;         - &&\;   25b =&\; 7
   \end{alignat*}
   So the desired integers are $x=6$ and $y = -35$.
\end{example}

We'll finish this section with another important theorem which, as we will see, establishes an connection between GCDs and prime numbers.

\begin{theorem}[The important theorem]
    If $d \mid ab$ and $(d,a) = 1$ then $d \mid b$.
\end{theorem}

\begin{proof}
    Suppose $d \mid ab$ and $(d,a) = 1$.
    Then $ab = dq$ for some integer $q$, and by Bezout's theorem
    \begin{align*}
        (d,a) = 1 &\implies dx + ay = 1 \\
        &\implies dxb + aby = b \\
        &\implies dxb + dqy = b \\
        &\implies d (xb + qy) = b
    \end{align*}
    Therefore, $d \mid b$.
\end{proof}

\section{Primes}
Moving on from the basic notion of the GCD, we define another basic idea: primality.

\begin{definition}[Prime number]
    An integer $p > 1$ is prime if its only positive divisors are 1 and $p$.
\end{definition}

\begin{definition}[Composite number]
    If $n > 1$ is not prime, then $n$ is called composite.

    \medskip
    \textit{Note: This implies that 0 and 1 are neither prime nor composite.}
\end{definition}

Although they get decreasingly common at higher orders of magnitude, the primes are infinite.
Euclid gave a very simple proof of this, a variation on which is given below.

\begin{theorem}[Euclid's theorem]
    There are infinitely many primes.
\end{theorem}

\begin{proof}
    Suppose, to the contrary, that $p$ is the largest prime number.
    Notice that no number $2, \ldots, p$ divides $p! + 1$, meaning $p! + 1$ is either prime or has a prime factor that is greater than $p$.
    Either way, this contradicts our assumption that $p$ is the largest prime.
\end{proof}

Despite this, it can also be shown that, for any number $n$, there exists a list of $n$ consecutive prime numbers.
Namely, $(n+1)! + 2, \; (n+1)! + 3, \; \ldots, (n+1)! + (n+1)$.

It can be shown, by strong induction, that every number can be expressed as the product of primes.
Our aim is to show that this product is unique.
As a step toward this, we give a variation on the important theorem.

\begin{theorem}[Theorem of prime importance]
    If $p$ is prime and $p \mid ab$, then $p \mid a$ or $p \mid b$.
\end{theorem}

\begin{proof}
    Suppose $p$ is prime and $p \mid ab$.
    If $p \mid a$, then we're done; otherwise, $p \nmid a$, and therefore $(p,a) = 1$.
    So by the important theorem, $p \mid b$.
    Either way, $p \mid a$ or $p \mid b$.
\end{proof}

\begin{corollary}
    If $p$ is prime and $p \mid a_1 a_2 \cdots a_n$, then $p \mid a_1$ or $p \mid a_2$ or $\cdots$ or $p \mid a_n$.
\end{corollary}

Now, the centerpiece of the section.

\begin{theorem}[Fundamental theorem of arithmetic]
    Every integer $n \geq 2$ can be uniquely factored into primes.
\end{theorem}

\begin{proof}
    Suppose, to the contrary, that $m$ is the smallest number with at least two prime factorizations, say
    \[ p_1 p_2 \cdots p_r = m = q_1 q_2 \cdots q_s. \]
    Since $p_1 \mid m$, $p_1 \mid q_1 q_2 \cdots q_s$ and, by the previous corollary, $p \mid q_i$ for some $i$; without loss of generality, say $p_1 \mid q_1$.
    But since $q_1$ is prime, this means $p_1 = q_1$, so
    \[ p_2 \cdots p_r = \frac{m}{p_1} = q_2 \cdots q_s. \]
    So $\frac{m}{p_1}$ has two different prime factorizations, contradicting our assumption for $m$.
\end{proof}

We can use this fundamental theorem to solve a pretty neat counting problem.

\begin{theorem}[Number of positive divisors from prime factorization]
    Let $a = \prod\limits_{i=1}^r p_i^{\alpha_i}$, where $p_1, \cdots, p_r$ are distinct primes and $\alpha_i \geq 0$.
    $a$ has $\prod\limits_{i=1}^{r}(1 + \alpha_i)$ positive divisors.
\end{theorem}

Similarly, the exponents of this prime factorization can be used to quickly determine the greatest common divisor and least common multiple between two numbers.

\begin{theorem}[GCD and LCD from prime factorization]
    Let $a = \prod\limits_{i=1}^r p_i^{\alpha_i}$ and $b = \prod\limits_{i=1}^r p_i^{\beta_i}$,where $p_1, \cdots, p_r$ are distinct primes and $\alpha_i, \beta_i \geq 0$.
    \[ \gcd (a,b) = \prod_{i=1}^r p_i^{\min \left\{ \alpha_i, \beta_i \right\}} \,\text{ and }\, \pfn{lcm} [a,b] = \prod_{i=1}^r p_i^{\max \left\{ \alpha_i, \beta_i \right\}}. \]
\end{theorem}

Note that the LCM is commonly denoted $[a,b]$.
We give one last consequence of unique prime factorizations.

\begin{theorem}[gcd $\cdot$ lcm]
    For any $a,b \geq 1$, $(a,b) [a,b] = ab$.
\end{theorem}

\section{Modular Arithmetic}
Modular arithmetic will be essential to the rest of our discussion of number theory.
Informally, this is the arithmetic in which two numbers are called ``congruent'' if, when divided by some number, they have the same remainder.
A more formal definition is below.

\begin{definition}[Modular congruence]
    For $m > 0$, we say that $a$ is congruent to $b$ modulo $m$ if $m \mid a - b$.
    Symbolically, $a \equiv_m b$.
    (Here, $m$ is called the modulus.)
\end{definition}

A more common notation for modular congruence is $a \equiv b \pmod m$, but we won't use this here.

The relation $\equiv_m$ has a lot in common with the relation $=$.
In fact, $\equiv_m$ can be calle an equivalence relation since it has the following three properties.

\begin{theorem}[Modular congruence as an equivalence relation]
    For any integers $a, b, c$:
    \begin{enumerate}[label=(\alph*)]
        \item Reflexivity. $a \equiv_m$ a.
        \item Symmetry. If $a \equiv_m b$ then $b \equiv_m a$.
        \item Transitivity. If $a \equiv_m b$ and $b \equiv_m c$ then $a \equiv_m c$.
    \end{enumerate}
\end{theorem}

Now we'll explore some of the things we're ``allowed'' to do with this relation.

\begin{theorem}[Operations on $\equiv_m$]
    If $a \equiv_m b$ and $c \equiv_m d$, then
    \begin{enumerate}[label=(\alph*)]
        \item $a + c \equiv_m b + d$
        \item $ac \equiv_m bd$
    \end{enumerate}
\end{theorem}

\begin{proof}
    (a) Suppose $a \equiv_m b$ and $c \equiv_m d$.
    Then $m \mid a-b$ and $m \mid c-d$, so
    \[ m \mid (a - b) + (c - d) \implies m \mid (a + c) - (b + d). \]
    By definition, $a + c \equiv_m b + d$.
    \smallskip

    (c) By the same hypothesis and the integer combination theorem,
    \[ m \mid (a - b)c + (c - d)b \implies m \mid ac - bd. \]
    By definition, $ac \equiv_m bd$.
\end{proof}


\begin{corollary}[Power rule]
    If $a \equiv_m b$ then $a^n \equiv_m b^n$ for all $n \geq 0$.
\end{corollary}

Putting all this together, we get what may be a superficially surprising fact.

\begin{corollary}[Polynomial rule]
    Let $P(x)$ be a polynomial with integer coefficients.
    If $a \equiv_m b$, then $P(a) \equiv_m P(b)$.
\end{corollary}

With this polynomial theorem, here's a little aside.
Out of a variety of modular relationships that have to do with the digits of a number, we give a simple one.

\begin{theorem}
    Define $S(x)$ to be the sum of the digits of $x$.
    $x \equiv_9 S(x)$.
\end{theorem}

\begin{proof}
    Let $x$ have digits $a_n a_{n-1} \cdots a_1 a_0$.
    Then
    \[ x = \sum_{k=0}^n a_k 10^k \equiv_9 \sum_{k=0}^n a_k 1^k = S(x). \]
    Therefore, $x \equiv_9 S(x)$.
\end{proof}

Now we'll continue with our discussion of arithmetic.
We've seen that we can add (and thus subtract) and multiply on both sides of a congruence.
We can only divide, however, under certain conditions.

\begin{theorem}[Cancellation theorem]
    If $ax \equiv_m ay$ and $\gcd (a,m) = 1$, then $x \equiv_m y$.
\end{theorem}

\begin{proof}
    Suppose $ax \equiv_m ay$ and $(a,m) = 1$.
    By definition,
    \begin{align*}
        m &\mid ax - ay \\
        m &\mid a (x - y)
    \end{align*}
    By the important theorem, $m \mid x - y$ and $x \equiv_m y$.
\end{proof}

We can generalize this theorem slightly, if we allow ourselves to change the modulus.

\begin{theorem}
    If $ax \equiv_m ay$ and $\gcd (a,m) = d$, then $x \equiv_{m/d} y$.
\end{theorem}

\begin{proof}
    Suppose $ax \equiv_m ay$ and $\gcd (a,m) = d$.
    So $m \mid ax - ay$, meaning $\frac{m}{d} \mid \frac{a}{d} (x - y)$.
    Since $\left( \frac{m}{d}, \frac{a}{d} \right) = 1$, by the important theorem, $\frac{m}{d} \mid x - y$, and by definition $x \equiv_{m/d} y$.
\end{proof}

The cancellation theorem gives one idea of what it means to ``undo'' multiplication in a modular sense.
Alternatively, we can define the multiplicative inverse of a number mod $m$.

\begin{definition}[Multiplicative inverse]
    An integer $a$ has a multiplicative inverse if there exists an integer $x$ such that $ax \equiv_m 1$.
\end{definition}

Unfortunately, not every number has an inverse; fortunately, it is easy to check if a number has one!

\begin{theorem}[Existence and uniqueness of an inverse]
    $a$ has an inverse mod $m$ if and only if $(a,m) = 1$.
    The inverse is unique up to congruence mod $m$.
\end{theorem}

\begin{proof}
    $(\Rightarrow)$ Suppose $a$ has an inverse mod $m$.
    By definition, there exists an integer $x$ such that $ax \equiv_m 1$, so $m \mid ax - 1$.
    So there exists another integer $y$ such that
    \begin{gather*}
        my = ax - 1 \\
        ax - my = 1
    \end{gather*}
    By Bezout's theorem, $(a,m) = 1$.
    \smallskip

    $(\Leftarrow)$ All of the steps above are reversible, so if $(a,m) = 1$ then $a$ has an inverse mod $m$.
\end{proof}

If the modulus is prime, then we can make an even stronger statement.

\begin{corollary}
    If $p$ is prime, then the numbers $1, 2, \ldots, p-1$ have unique inverses mod $p$.
\end{corollary}

\section{Applications}
Here we state four theorems that utilize everything we've discussed so far, largely 
modular arithmetic.

\begin{theorem}[Wilson's theorem]
    If $p$ is prime then $(p-1)! \equiv_m -1$.
\end{theorem}

\begin{proof}
    Let $p$ be prime, so the numbers $1, 2, \ldots, p-1$ have inverses mod $p$.
    The only ones of these that are their own inverses are 1 and $p-1$.

    To see this, consider a number $a$ that is its own inverse, so $a^2 \equiv_p 1$ and $p \mid a^2 - 1$.
    By the theorem of prime importance, $p \mid a-1$ or $p \mid a+1$.
    The only integers (among $1, 2, \ldots, p-1$) satisfying this condition are 1 and $p-1$.

    Therefore, all of the other numbers $2, \ldots, p-2$ are inverses of each other, so when we compute $(p-1)!$ we get
    \[ (p-1)! \equiv_p (p-1) \cdot 1. \]
    Finally, since $p-1 \equiv_p -1$, we get $(p-1)! \equiv_p -1$.
\end{proof}

It isn't important for us, but Wilson's theorem is actually an iff theorem, since it can be shown that if $n$ is composite then $(n-1)! \not\equiv_n -1$.
So in theory, we could use Wilson's theorem to determine if $n$ is prime; in practice, $(n-1)!$ is big.
Like, really big.
So we must resort to other means.

\begin{theorem}[Fermat's little theorem]
    If $p$ is prime, then $a^p \equiv_p a$ for any $a \in \Z$.
    (Alternatively, $a^{p-1} \equiv_p 1$.)
\end{theorem}

\begin{proof}
    We have two cases.
    \smallskip

    (i) If $p \mid a$, then $a \equiv_p 0$ and $a^p \equiv_p 0^p$.
    Since $a^p = 0$, by transitivity, $a^p \equiv_p a$.
    (And, by the cancellation theorem, $a^{p-1} \equiv_p 1$.)
    \smallskip

    (ii) Notice that the numbers $0, 1, 2, \ldots, p-1$ are all distinct mod $p$.
    By the cancellation theorem contrapositive, since $(a,p) = 1$, the numbers $0, a, 2a, \ldots, (p-1)a$ are also distinct mod $p$.
    Now, since each of these lists covers all possible remainders mod $p$,
    \begin{align*}
        \{ 0, a, 2a, \ldots, (p-1)a \} &\equiv_p\{ 0, 1, 2, \ldots, (p-1) \} \\
        \{ a, 2a, \ldots, (p-1)a \} &\equiv_p\{ 1, 2, \ldots, (p-1) \}
    \end{align*}
    We can multiply all of the elements in each of these sets to get
    \begin{align*}
        a \cdot 2a \cdots (p-1)a &\equiv_p 1 \cdot 2 \cdots (p-1) \\
        (p-1)! a^{p-1} \equiv_p (p-1)! \\
        a^{p-1} \equiv_p 1
    \end{align*}
    Finally, multiplying by $a$ gives $a^p \equiv_p a$.
\end{proof}

Our next theorem will generalize Fermat's theorem.
First, however, we'll need some scaffolding in the form of a new definition.

\begin{definition}[Totient function]
    For $n \geq 1$, define the totient function $\phi (n)$ as the number of elements in the set $\{ 1, 2, \ldots, n \}$ that are relatively prime to $n$.
\end{definition}

This totient function has a couple of convenient properties that are useful in certain computations.

\begin{theorem}[Computing $\phi(n)$]
    If $n$ has distinct prime factors $p_1, p_2, \ldots, p_r$, then
    \[ \phi (n) = n \left( 1 - \frac{1}{p_1} \right) \left( 1 - \frac{1}{p_2} \right) \cdots \left( 1 - \frac{1}{p_r} \right). \]
\end{theorem}

\begin{proof}
    We use the principle of inclusion and exclusion to prove the $r=3$ case.
    We count in four phases:
    \begin{enumerate}
        \item Place no restrictions on the numbers we count.
        \item Subtract the numbers that are multiples of at least one prime.
        \item Add back the numbers that are multiples of at least two primes.
        \item Subtract the numbers that are multiples of all three primes.
    \end{enumerate}
    This gives
    \begin{align*}
        \phi (n) &= n - \frac{n}{p_1} - \frac{n}{p_2} - \frac{n}{p_3} + \frac{n}{p_1p_2} + \frac{n}{p_1p_3} + \frac{n}{p_2p_3} - \frac{n}{p_1p_2p_3} \\
        &= n \left( 1 - \frac{1}{p_1} - \frac{1}{p_2} - \frac{1}{p_3} + \frac{1}{p_1p_2} + \frac{1}{p_1p_3} + \frac{1}{p_2p_3} - \frac{1}{p_1p_2p_3} \right) \\
        &= n \left( 1 - \frac{1}{p_1} \right) \left( 1 - \frac{1}{p_2} \right) \left( 1 - \frac{1}{p_3} \right)
    \end{align*}
    The method here can be easily extended to other values of $r$, be it much more cumbersome.
\end{proof}

There's some other, less rigorous intuition to be gained from this formula.
Each factor represents a proportion of numbers that do not have $p_i$ as a factor (each of these proportions is independent), and multiplying all of these together gives the total proportion of numbers that do not have any $p_i$ as a factor.

This leads to one other nice property of totients, given below.

\begin{corollary}
    If $(x,y) = 1$, then $\phi (xy) = \phi (x) \phi (y)$.
\end{corollary}

Finally, we have one more ``pure'' result, also related to totients.
It is a generalization of Fermat's theorem.

\pagebreak

\begin{theorem}[Euler's theorem]
    If $(a,m) = 1$, then $a^{\phi (m)} \equiv_m 1$.
\end{theorem}

\begin{proof}
    Let $\left\{ r_1, \ldots, r_t \right\}$ be the subset of $\left\{ 1, \ldots, m \right\}$ containing the numbers that are relatively prime to $m$, so $t = \phi (m)$.
    Since $r_1, \ldots, r_t$ are distinct mod $m$ and $(a,m) = 1$, by the cancellation theorem's contrapositive the numbers $ar_1, \ldots, ar_t$ are also distinct mod $m$.

    Now, since $(a, m) = 1$ and $(r_i, m) = 1$, we also have $(ar_i, m) = 1$.
    By Euclid's theorem this means $(m, ar_i \bmod m) = 1$, so $ar_i \bmod m = r_j$ and $ar_i \equiv_m r_j$ for some $j$.

    Therefore, $\left\{ ar_1, \ldots, ar_t \right\} \equiv_m \left\{ r_1, \ldots, r_t \right\}$.
    Multiplying everything together:
    \begin{align*}
        (ar_1) \cdots (ar_t) &\equiv_m r_1 \cdots r_t \\
        a^t (r_1 \ldots r_t) &\equiv_m r_1 \ldots r_t \\
        a^t &\equiv_m 1 \\
        a^{\phi (m)} &\equiv_m 1,
    \end{align*}
    as desired.
\end{proof}

As an interlude, consider again Fermat's theorem, specifically its contrapositive: if $a^n \not\equiv_n a$ then $n$ is not prime.
This gives us a ``probable prime test''; it isn't perfect since Fermat's theorem isn't an iff statement, but it does filter out the vast majority of composite numbers.

A number $n$ that passes the Fermat test for all possible bases $a$ is called an industrial-grade prime.
Some of these IGPs, of course, are composite; such numbers are called Carmichael numbers.
(There are tests to weed these out, but they won't be covered here.)

We will, however, give a couple of practical methods for computing $a^n \bmod m$ for large $n$.
First, if $(a,m) = 1$ and $\phi (m)$ are known, we may be able to exploit Euler's theorem in some way (often using the division algorithm).
This method is simple, but it has limited use; the next one is much more widely applicable.

\begin{example}[Seed planting]
    Suppose we want to compute $6^{83} \bmod 79$.
    First, we decompose the exponent into powers of two:
    \[ 83 = 64 + 16 + 2 + 1. \]
    Now we'll go through all the powers of two from 64 to 1 and, starting with $6^0$ (the seed), successively square numbers, multiplying by an extra 6 whenever we encounter a power of two that is listed above.
    \begin{alignat*}{3}
        \boxed{64} &\rightarrow 6^0 \cdot 6 &&= 6 \\
        32 &\rightarrow 6^2 &&= 36 \\
        \boxed{16} &\rightarrow 36^2 \cdot 6 &&= 7776 &&\equiv_{79} 34 \\
        8 &\rightarrow 34^2 &&= 1156 &&\equiv_{79} 50 \\
        4 &\rightarrow 50^2 &&= 2500 &&\equiv_{79} 51 \\
        \boxed{2} &\rightarrow 51^2 \cdot 6 &&= 15606 &&\equiv_{79} 43 \\
        \boxed{1} &\rightarrow 43^2 \cdot 6 &&= 11094 &&\equiv_{79} 34
    \end{alignat*}
    So, $6^{83} \bmod 79 = 34$.
\end{example}

Using this, we can make a statement that is much more immediately practical in cryptography.

Consider a public key $\{ n, e \}$ and a private key $d$ known only to the recipient.
The idea is to construct a function that is easy to compute using the public key, but very difficult to invert without knowing the private key.
Specifically,
\begin{itemize}
    \item to encipher a message $m$ the sender computes $c = m^e \bmod n$, and
    \item to decipher the resulting ciphertext the recipient computes $m = c^d \bmod n$.
    (Note that, in order to unambiguously recover a message, we must have $m < n$.)
\end{itemize}
Now the problem becomes actually choosing values of $n$, $e$, and $d$ to make this work.
The process is simple:
\begin{enumerate}
    \item $n$ is the product of two (private) primes $p$ and $q$.
    \item $d$ is a random number that is relatively prime to $\phi (n) = (p-1)(q-1)$.
    \item $e \geq 0$ is an integer that satisfies $de - \phi (n) f = 1$ for some $f \geq 0$.
    (Note that this mean $de \equiv_{\phi (n)} 1$.)
\end{enumerate}
The proof that this works is similarly straightforward.

\begin{theorem}[RSA encryption]
    Let $n,d,e$ be chosen as described above.
    If $c = m^e \bmod n$, then $c^d \bmod n = m$.
\end{theorem}

\begin{proof}
    Suppose $c = m^e \bmod n$ and $n,d,e$ are chosen as described above.
    Since we require $0 \leq m < n$, we need only show that $c^d \bmod n \equiv_n m$.

    By definition
    \[ c = m^e \bmod n \equiv_n m. \]
    So
    \[ c^d \bmod n \equiv_n c^d \equiv_n \left( m^e \right)^d = m^{ed}. \]
    Since $ed - \phi (n) f = 1$,
    \[ m^{ed} = m^{1 + \phi(n) f} = m \cdot \left( m^{\phi (n)} \right)^f. \]
    By Euler's theorem,
    \[ m \cdot \left( m^{\phi (n)} \right)^f \equiv_n m \cdot 1^f = m. \]
    Therefore, $c^d \bmod n \equiv_n m$ and $c^d \bmod n = m$.
\end{proof}

\end{document}
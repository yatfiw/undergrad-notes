\documentclass[../m055main.tex]{subfiles}
\graphicspath{{\subfix{../figures/}}}

\begin{document}

\chapter{Combinatorics}

\section{Introduction}
We begin our discussion of combinatorics by giving two basic rules that will govern how we count things.

\begin{theorem}[Rule of sum]
    If $A_1, \ldots, A_n$ are disjoint sets, then
    \[ \bigcup_{i=0}^n A_i = \sum_{i=0}^n |A_i|. \]
\end{theorem}

\begin{theorem}[Rule of product]
    If an action is composed of $k$ steps such that there are $x_i$ choices for step $i$, then the action can be performed in $x_1 x_2 \cdots x_k$ different ways.
\end{theorem}

We'll apply these rules (especially the rule of product) to two different types of problems.
First, we can count the number of ways we can arrange a set of objects.

\begin{theorem}[Counting arrangements]
    The number of arrangements (or permutations) of $n$ objects is given by $n!$. % <3

    \medskip
    \textit{Note: We define $0! = 1$ since there is, technically, one way to arrange an empty set of objects.}
\end{theorem}

\begin{proof}
    Constructing a permutation of $n$ objects is an action that requires $n$ different steps.
    \begin{itemize}
        \item [1.] Step 1 is picking the first element in the permutation; there are $n$ ways to do this.
        \item [2.] Step 2 is picking the second element in the permutation; since we've already picked an element to be first, there are $n-1$ ways to pick this element.
        \item [3.] Step 3 is picking the third element; there are $n-2$ ways to do this.
        \item [$\vdots$\phantom{.}]
        \item [$n$.] Step $n$ is picking the final element; all but one element has already been chosen, so there is only one way to pick this element.
    \end{itemize}
    By the rule of product, there are $n \cdot (n-1) \cdot (n-2) \cdots 1 = n!$ ways to permute the elements in $S$.
\end{proof}

We can also count the number of subsets we can create using a set of objects, regardless of arrangement order.
This type of problem is so fundamental that it gets a special name and symbol, given below.

\begin{definition}[Combination]
    Consider a set $S$ that has size $n$.
    The number of size-$k$ subsets of $S$ is called ``$n$ choose $k$'' and is denoted by $\binom{n}{k}$.
    For $k < 0$ or $k > n$, we define $\binom{n}{k} = 0$.
\end{definition}

\pagebreak

\begin{theorem}[Counting subsets]
    For $0 \leq k \leq n$,
    \[ \binom{n}{k} = \frac{n!}{k! (n-k)!}. \]
\end{theorem}

\begin{proof}
    Let $S$ be a set with size $n$.
    We will count, in two different ways, the number of size-$k$ permutations of the elements in $S$.
    \smallskip

    (i) We first use the rule of product to directly compute the number of permutations:
    \[ n \cdot (n-1) \cdot (n-2) \cdots (n-k+1) = \frac{n!}{(n-k)!}. \]

    (ii) There are $\binom{n}{k}$ ways to choose the $k$ letters we're permuting.
    For each of these choices, there are $k!$ ways to arrange them.
    So, there are
    \[ k! \cdot \binom{n}{k} \]
    size-$k$ permutations of the elements in $S$.
    \smallskip

    Both of the expressions we've found count the same thing, so they must be equal.
    That is,
    \[ k! \cdot \binom{n}{k} = \frac{n!}{(n-k)!} \implies \binom{n}{k} = \frac{n!}{k!(n-k)!}, \]
    as desired.
\end{proof}

This is an example of a combinatorial proof---we can prove an equivalence between two expressions by showing that they count the same thing.
This will be a useful (and enlightening!) technique when proving other statements later on.
But first, we have one more result that may illustrate the surprisingly far-reaching power of combinatorics.

\begin{theorem}[]
    For $n \geq 1$,
    \[ \frac{n(n+1)}{2} = \binom{n+1}{2}. \]
\end{theorem}

\begin{proof}
    Consider an equilateral triangle formed by rows of circles, each row with one more circle than the last.
    If this triangle has $n$ rows, then there are $\frac{n(n+1)}{2}$ circles in the triangle.

    Now add one more row (of length $n+1$) to the bottom of the triangle.
    There is a one-to-one correspondence between pairs of circles in the bottom row and individual circles in the upper $n$ rows---we might illustrate this by drawing diagonal lines from any dot in the upper rows to a pair of dots in the bottom row.
    There are $\binom{n+1}{2}$ ways to pick pairs of circles from the bottom row, so 
    \[ \frac{n(n+1)}{2} = \binom{n+1}{2}, \]
    as desired.
\end{proof}

\section{Fibonacci Numbers and Tilings}
The Fibonacci numbers are often seen in the context of sequences as one of the first examples of a more ``complex'' rule for going from one term to the next.

\begin{definition}[Fibonacci numbers]
    The Fibonacci numbers are defined recursively by $F_0 = 0$, $F_1 = 1$, and
    \[ F_n = F_{n-1} + F_{n-2}, \quad n \geq 2. \]
\end{definition}

We can quickly go through some interesting facts about these numbers.

\begin{theorem}[Partial sums of Fibonacci numbers] \label{partialFibSum}
    For $n \geq 0$,
    \[ \sum_{k=0}^{n} F_n = F_{n+2} - 1. \]
\end{theorem}

\begin{proof}
    We do induction on $n$.
    The base case $n=0$ is trivial.
    Now suppose, as our inductive hypothesis, that the statement is true for $n$.
    So for $n+1$,
    \begin{align*}
        F_0 + F_1 + \cdots + F_n + F_{n+1} &= (F_{n+2} - 1) + F_{n+1} \\
        &= F_{n+3} - 1,
    \end{align*}
    as desired.
\end{proof}

Surprisingly, there is a simple closed form for the $n$th Fibonacci number!

\begin{theorem}[Binet's formula]
    For $n \geq 0$,
    \[ F_n = \frac{1}{\sqrt{5}} \left[ \left( \frac{1 + \sqrt{5}}{2} \right)^{n} - \left( \frac{1 - \sqrt{5}}{2} \right)^{n} \right] = \frac{\varphi^{n} - \bar{\varphi}^{n}}{\sqrt{5}}, \]
    where $\varphi = (1 + \sqrt{5}) / 2$ and $\bar{\varphi} = (1 - \sqrt{5}) / 2$.
\end{theorem}

\begin{proof}
    We do induction on $n$.
    The base cases $n=0$ and $n=1$ can be shown via simple arithmetic.
    Now suppose, as our inductive hypothesis, that Binet's formula folds for both $n$ and $n-1$.
    (We can do this since we have two base cases.)
    By the definition of the Fibonacci numbers:
    \begin{align*}
        F_{n+1} &= F_n + F_{n-1} \\
        \intertext{By the inductive hypothesis,}
        &= \frac{1}{\sqrt{5}} \left( (\varphi^{n} + \varphi^{n-1}) - (\bar{\varphi}^{n} + \bar{\varphi}^{n-1}) \right) \\
        &= \frac{1}{\sqrt{5}} \left( \varphi (\varphi + 1) - \bar{\varphi}({\varphi} + 1) \right) \\
        \intertext{Since these constants solve $\varphi + 1 = \varphi^2$ and $\bar{\varphi} + 1 = \bar{\varphi}^2$, we get}
        &= \frac{1}{\sqrt{5}} \left( \varphi^{n+1} - \bar{\varphi}^{n+1} \right),
    \end{align*}
    as desired.
\end{proof}

To prove our last property, we introduce a slight variation to induction that we'll use more later.

\begin{theorem}[Fibonacci decomposition]
    Every positive integer can be expressed as the sum of distinct Fibonacci numbers.
\end{theorem}

\begin{proof}
    We do \textit{strong} induction on the positive integer $n$.
    We still mention that the base case, $n=1$, obviously holds because $F_2 = 1$.

    Now, for our inductive hypothesis, suppose the statement holds for \textit{all} numbers up to and including $n$.
    We have two cases.
    \begin{enumerate}[label=(\alph*)]
        \item If $n+1$ is a Fibonacci number, then we're done.
        \item Otherwise, let $F_L$ be the largest Fibonacci number that is less than $n+1$, so $F_L < n+1 < F_{L+1}$.
        Now, $n+1 = F_L + (n+1 - F_L)$, and since $n + 1 - F_L$, by the inductive hypothesis $n+1$ is the sum of Fibonacci numbers.
        These numbers are distinct because
        \[ n+1 - F_L < F_{L+1} - F_L = F_{L-1} \leq F_L, \]
        so $F_L$ cannot be included in the sum of Fibonacci numbers that creates $n+1 - F_L$.
    \end{enumerate}
    In either case, $n+1$ is the sum of Fibonacci numbers.
\end{proof}

Since we're discussing all this in a chapter about combinatorics, it may not be surprising that there's a very direct connection between the Fibonacci numbers and counting.

Suppose we have a board of length $n$, and that we wish to cover this board with tiles of lengths 1 and 2 (called squares and dominoes, respectively).
Define $f_n$ as the total number of ways we can do this tiling; as it turns out, we can use Fibonacci numbers to quickly determine $f_n$!

\begin{theorem}[Number of square-domino tilings]
    For $n \geq 0$, $f_n = F_{n+1}$.
\end{theorem}

\begin{proof}
    We do induction on the length $n$ of the tiling.
    The base cases $n=0$ and $n=1$ are trivial.

    Now suppose as our inductive hypothesis that $f_{n-1} = F_n$ and $f_n = F_{n+1}$.
    To determine $f_{n+1}$, we observe that all length-$(n+1)$ tilings can be broken into two categories: ones that end with a domino (there are $f_{n-1}$ of these), and ones that don't ($f_n$).
    Adding these together gives
    \[ f_{n+1} = f_{n-1} + f_n = F_n + F_{n+1} = F_{n+2}, \]
    as desired.
\end{proof}

This connection allows us to combinatorially prove many identities involving Fibonacci numbers!

\begin{theorem}[Squares of consecutive Fibonacci numbers]
    For $n \geq 1$, $f_{n-1}^2 + f_n^2 = f_{2n}$.
\end{theorem}

\begin{proof}
    We give a combinatorial proof by counting the number of ways a length-$2n$ board can be tiled using squares and dominoes.
    \smallskip

    (i) By definition, there are $f_{2n}$ ways to do this.
    \smallskip

    (ii) Each tiling can be split into two categories: ones that have a domino at their midpoint ($f_{n-1}^2$ of these), and ones that don't ($f_n^2$).
    Adding these together gives $f_{n-1}^2 + f_n^2$ total tilings.

    \smallskip

    Both expressions we've found count the same thing, so they must be equal.
\end{proof}

By a similar argument, we could generalize this statement to write $f_{m+n} = f_{m-1} f_{n-1} + f_m f_n$.
We could also prove this generalization quite easily using induction, but the combinatorial route (one may argue) offers more intuition as to \textit{why} the identity is the way it is.

In fact, we can even use combinatorics to prove something like Theorem 1.\ref{partialFibSum} by examining different ways in which the tilings can end (specifically, tilings that end with different numbers of consecutive squares).

\section{Pascal's Triangle}
One of the most widely recognizable objects in combinatorics is Pascal's triangle, an infinite arrangement of combinations.
\begin{gather*}
    \binom{0}{0} \\
    \binom{1}{0} \qquad \binom{1}{1} \\
    \binom{2}{0} \qquad \binom{2}{1} \qquad \binom{2}{2} \\
    \binom{3}{0} \qquad \binom{3}{1} \qquad \binom{3}{2} \qquad \binom{3}{3} \\
    \vdots
\end{gather*}
There's a lot of different patterns to uncover in this triangle.
This section is dedicated to proving them!
We'll start with an identity that many would call the defining feature of Pascal's triangle.

\begin{theorem}[Pascal's identity]
    For $0 < k \leq n$,
    \[ \binom{n}{k} = \binom{n-1}{k-1} + \binom{n-1}{k}. \]
\end{theorem}

\begin{proof}
    We give a combinatorial proof by counting the number of size-$k$ subsets that can be created using a set of $n$ elements.
    \smallskip

    (i) By definition, $\binom{n}{k}$ subsets can be created.
    \smallskip

    (ii) Each subset can be categorized based on the presence of element $n$.
    There are $\binom{n-1}{k-1}$ size-$k$ subsets that do contain $n$, and $\binom{n-1}{k}$ that don't.
    So there are $\binom{n-1}{k-1} + \binom{n-1}{k}$ size-$k$ subsets in all.
    \smallskip

    Both of the expressions we've found count the same thing, so they must be equal.
\end{proof}

Now we'll examine two patterns that exist within each row of the triangle.

\begin{theorem}[Row sums] \label{rowSums}
    For $n \geq 0$,
    \[ \sum_{k=0}^{n} \binom{n}{k} = 2^{n}. \]
\end{theorem}

\begin{proof}
    We give a combinatorial proof by counting the number of subsets (of any size) that can be created using a set of $n$ elements.
    \smallskip

    (i) We can simply add up the number of subsets of each size between 0 and $n$; this gives $\sum\limits_{k=0}^{n} \binom{n}{k}$.
    \vspace{-3pt}

    (ii) We can also ``iterate'' through the elements in the set and, for each one, decide whether or not to include it in the subset.
    There are $2^{n}$ ways to do this.
    \smallskip

    Both of the expressions we've found count the same thing, so they must be equal.
\end{proof}

\pagebreak

\begin{theorem}[Same-parity row sums] \label{parityRowSums}
    For $n \geq 1$,
    \[ \sum_{k\textrm{ even}} \binom{n}{k} = 2^{n-1} = \sum_{k\textrm{ odd}} \binom{n}{k}. \]
\end{theorem}

\begin{proof}
    We give a combinatorial proof by counting the number of even-sized subsets that can be created using a set of $n$ elements.
    \smallskip

    (i) We can simply add up the number of subsets of each even size between 0 and $n$; this gives $\sum\limits_{k\textrm{ even}} \binom{n}{k}$.
    \vspace{-10pt}

    (ii) We can also ``iterate'' through the first $n-1$ elements in the set and, for each one, decide whether or not to include it in the subset.
    When we arrive at element $n$, whether we include or exclude it is predetermined by the parity of the subset we've already created, so there's no decision to be made.
    There are $2^{n-1}$ ways to construct a subset in this way.
    \smallskip

    Both of the expressions we've found count the same thing, so they must be equal.
    (The odd case can be proved in the exact same way, just with odd-sized subsets instead.)
\end{proof}

Now consider a version of Pascal's triangle that is left-aligned, with the numeric values filled in.
\begin{align*}
    & 1 \quad \\
    & 1 \quad 1 \\
    & 1 \quad 2 \quad 1 \\
    & 1 \quad 3 \quad 3 \quad 1 \\
    & 1 \quad 4 \quad 6 \quad 4 \quad 1 \\
    & \phantom{1 \quad 4 \quad\,} \vdots
\end{align*}
Notice that, when we start at the top of a column and sum the first few terms, we get the number immediately down and to the right of the last addend.
(This forms a hockey stick-like shape.)
We formalize and prove this observation below.

\begin{theorem}[Hockey stick theorem]
    For $0 \leq k \leq n$,
    \[ \sum_{j=k}^{n} \binom{j}{k} = \binom{n+1}{k+1}. \]
\end{theorem}

\begin{proof}
    We count the number of size-$(k+1)$ subsets that can be created using a set of $n+1$ elements.
    \smallskip

    (i) By definition, $\binom{n+1}{k+1}$ such subsets can be created.
    \smallskip

    (ii) Each subset can be categorized based on which elements are required or disallowed to be in it.
    \begin{itemize}
        \item There are $\binom{n}{k}$ ways to construct a subset that requires the presence of element $n+1$.
        \item There are $\binom{n-1}{k}$ ways to construct a subset that requires element $n$, but forbids $n+1$.
        \item There are $\binom{n-2}{k}$ ways to construct a subset that requires $n-1$ but forbids $n$ and $n+1$.
        \item[$\vdots$]
        \item There are $\binom{k}{k}$ ways to construct a subset that requires $k+1$ but forbids anything greater.
    \end{itemize}
    In all, this gives $\sum\limits_{j=k}^{n} \binom{j}{k}$ distinct subsets.

    Both of the expressions we've found count the same thing, so they must be equal.
\end{proof}

We can do the same thing with diagonal sums---pick a row and start summing terms going up and to the left.
Doing this for many rows reveals an even more interesting pattern!

\begin{theorem}[Diagonal sums]
    For $n \geq 0$,
    \[ \sum_{k \geq 0} \binom{n-k}{k} = f_n. \]
\end{theorem}

\begin{proof}
    We give a combinatorial proof by counting the number of length-$n$ tilings that can be created using squares and dominoes.
    \smallskip

    (i) By definition, there are $f_n$ such tilings.
    \smallskip

    (ii) Consider a tiling that has exactly $k$ dominoes (and $n-2k$ squares).
    So there are $n-k$ tiles in all, and there are $\binom{n-k}{k}$ ways to arrange the tiles.
    Adding these up for all possible values of $k$ gives $\sum\limits_{k \geq 0} \binom{n-k}{k}$.

    Both of the expressions we've found count the same thing, so they must be equal.
\end{proof}

We've been sticking with combinatorial proofs in this section, but some of the theorems we've proved also have pretty easy algebraic proofs.
In order to uncover these, though, we need another theorem.

\begin{theorem}[Binomial theorem]
    For $n \geq 0$,
    \[ (x+y)^{n} = \sum_{k=0}^{n} \binom{n}{k} x^{k} y^{n-k}. \]
\end{theorem}

\begin{proof}
    The expanded form of $(x+y)^{n}$ includes $n$ factors of the form $(x+y)$.
    To create an $x^{k} y^{n-k}$ term, we pick $k$ factors to pull an $x$ from, and the rest contribute $y$'s; there are $\binom{n}{k}$ ways to do this.
\end{proof}

Using this, some proofs almost become trivial!
For example, to prove Theorem 1.\ref{rowSums} we can simply write $2^{n}$ as the binomial $(1+1)^{n}$ and apply the binomial theorem.
(We could do a similar thing to prove the equivalence between the summations in Theorem 1.\ref{parityRowSums}; we'll omit the details here.)

We can also prove new identities!
For example, we can differentiate both side of the equation
\[ (1+x)^{n} = \sum_{k=0}^{n} \binom{n}{k} x^{k} \]
and substitute $x=1$ to show that
\[ n \cdot 2^{n-1} = \sum_{k=0}^{n} k \binom{n}{k}. \]

\section{Multisets and Inclusion-Exclusion}
So far we've been using combinations, which count the number of subsets that can be created from a set.
Here, we'll extend our discussions into creating \textit{multisets}---that is, sets that can have repeated elements.

\begin{definition}[Multichoose]
    Consider a set $S$ that has size $n$.
    The number of size-$k$ multisets that can be created using the elements in $S$ is called ``$n$ multichoose $k$'' and is denoted by $\mchoose{n}{k}$.
\end{definition}

We can quickly establish a nice connection to normal combinations that illuminates a different way of looking at the multichoose function.

\begin{theorem}[Multichoose theorem]
    \[ \mchoose{n}{k} = \binom{n+k-1}{k}. \]
\end{theorem}

\begin{proof}
    Suppose we have a set of $n$ integers and want to create a size-$k$ multiset using them.
    \smallskip

    (i) By definition, we can create $\mchoose{n}{k}$ multisets.
    \smallskip

    (ii) We have $k$ slots in the multiset, each of which is represented below by an asterisk.
    \[ \underbrace{* \,\phantom{|}\, * \,\phantom{|}\, * \,\phantom{|}\, * \,\phantom{|}\, * \,\phantom{|}\, * \,\phantom{|}\, * \,\phantom{|}\, * \,\phantom{|}\, * \,\phantom{|}\,\, *}_{k \text{ stars}} \]
    Suppose we insert $n-1$ vertical bars into the spaces between the asterisks.
    \[ \underbrace{\phantom{|}\,* \,|\, * \,\phantom{|}\, * \,\phantom{|}\, * \,\phantom{|}\, * \,\phantom{|}\, * \,\phantom{|}\, * \,|\, * \,\phantom{|}\, * \,\phantom{|}\,\, *\,|}_{n \text{ bars}} \]
    Each bar represents a transition from one set element to the next.
    For example, the above represents the construction of a multiset that contains one instance of element 1, six instances of element 2, three instances of element 3, and no instances of element 4.

    In general, we'll always have $k$ stars and $n-1$ bars; these objects can be arranged in
    \[ \binom{n+k-1}{k} = \binom{n+k-1}{n-1} \]
    different ways.
    \smallskip

    Both of the expressions we've found count the same thing, so they must be equal.
\end{proof}

We can exploit this connection to state a Pascal-like theorem for the multichoose.

\begin{theorem}[]
    \[ \mchoose{n}{k} = \mchoose{n-1}{k} + \mchoose{n}{k-1}. \]
\end{theorem}

\begin{proof}
    A simple algebraic proof using Pascal's identity:
    \begin{align*}
        \mchoose{n-1}{k} + \mchoose{n}{k-1} &= \binom{n+k-2}{k} + \binom{n+k-2}{k-1} \\
        &= \binom{n+k-1}{k} \\
        &= \mchoose{n}{k},
    \end{align*}
    as desired.
\end{proof}

The multichoose describes allocations of $k$ multiset slots to $n$ elements.
(In class, we allocated kandies to ninjas.)
We can place restrictions on this restriction by saying, for example, that each element must appear in the multiset at least once; in this case, we simply subtract $n$ from the available multiset slots.

To use another restriction we might impose, we must first describe the principle of inclusion and exclusion.
Suppose we want to count the number of elements in the union $A_1 \cup A_2 \cup A_3$.
We do so in three steps:
\begin{enumerate}
    \item We first add up the magnitudes of the individual sets: $|A_1| + |A_2| + |A_3|$.
    \item We then recognize that we've double-counted the elements that are present in at least two sets, so we subtract off the 2-intersections: $- |A_1 \cap A_2| - |A_1 \cap A_3| - |A_2 \cap A_3|$.
    \item But now, while subtracting, we've triple counted the elements that are present in all three sets, so we add back the 3-intersection: $+ |A_1 \cap A_2 \cap A_3|$.
\end{enumerate}
Combining all of these together gives the size of $A_1 \cup A_2 \cup A_3$.
We generalize this below.

\begin{theorem}[Principle of inclusion and exclusion]
    Let $A_1, A_2, \ldots, A_n$ be subsets of $S$, and let the indices $i,j,k$ satisfy $1 \leq i \leq j \leq k \leq n$.
    Then
    \begin{align*}
        |A_1 \cup A_2 \cup \cdots \cup A_n| =& \phantom{+}\sum_{i} |A_i| \\
        & - \sum_{i,j} |A_i \cap A_j| \\
        & + \sum_{i,j,k} |A_i \cap A_j \cap A_k| \\
        & \;\,\vdots \\
        & + (-1)^{n-1} |A_1 \cap A_2 \cap \cdots \cap A_n|
    \end{align*}
\end{theorem}

\begin{proof}
    We'll show that each element $x$ in the union is counted exactly once.
    Suppose $x$ is in exactly $m$ of the $n$ sets (and, without loss of generality, these sets are $A_1, \ldots, A_m$).
    The first sum counts $x$ $m$ times, the second \textit{un}counts it $\binom{m}{2}$ times, the third counts it $\binom{m}{3}$ times, and so on; altogether, the number of times $x$ is counted is
    \[ \binom{m}{1} - \binom{m}{2} + \binom{m}{3} - \binom{m}{4} + \cdots \pm \binom{m}{m} = \sum_{k\text{ odd}} \binom{m}{k} - \sum_{k\text{ even}} \binom{m}{k} = \binom{m}{0} = 1, \]
    as desired.
\end{proof}

So if we wanted to construct multisets such that no element is repeated more than $r$ times, we simply apply the principle of inclusion and exclusion:
\[ \mchoose{n}{k} - n \mchoose{n}{k-r} + \binom{n}{2} \mchoose{n}{k-2r} - \cdots \]
Another application comes in the form of derangements, defined below.

\begin{definition}[Derangement]
    A derangement of $n$ objects is an arrangement of the objects such that no object is in its natural position.
    The number of derangements of $n$ objects is denoted by $D_n$.
\end{definition}

\begin{theorem}[]
    For $n \geq 1$, $D_n = \pfn{round} \left( \frac{n!}{e} \right)$.
\end{theorem}

\begin{proof}
    We can use the principle of inclusion and exclusion to show that
    \[ D_n = n! \sum_{k=0}^{n} \frac{(-1)^k}{k!} \implies \frac{D_n}{n!} = \sum_{k=0}^{n} \frac{(-1)^k}{k!}. \]
    Notice that this is a partial sum of the power series of $e^{-1}$.
    From calculus, we know that
    \[ \left| \frac{D_n}{n!} - \frac{1}{e} \right| < \left| \frac{1}{n+1} \right|. \]
    Therefore, we have the chain of relations
    \[ \left| D_n - \frac{n!}{e} \right| = n! \left| \frac{D_n}{n!} - \frac{1}{e} \right| < \frac{1}{n+1} \leq \frac{1}{2} \]
    Since the distance between $D_n$ and $\frac{n!}{e}$ is always less than $\frac{1}{2}$, $D_n =\pfn{round}\left( \frac{n!}{e} \right)$.
\end{proof}

\section{Linear Recurrences and Generating Functions}
Now we'll step away from combinatorics and, as a brief interlude, study a simple kind of sequence.

\begin{definition}[Linear recurrence]
    A $t$-th linear recurrence is one of the form
    \[ a_n = k_1 a_{n-1} + k_2a_{n-2} + \cdots + k_t a_{n-t}. \]
\end{definition}

First-order recurrences, also known as geometric sequences, have the closed form $a_n = c \cdot r^n$.
We hypothesize that any $t$-th order linear recurrence also has this form.
We can test this on a second-order recurrence:
\begin{align*}
    a_n &= k_1 a_{n-1} + k_2 a_{n-2} \\
    c r^{n} &= k_1 cr^{n-1} + k_2 cr^{n-2} \\
    0 &= r^2 - k_1 r - k_2
\end{align*}
This equation has solutions $r_1, r_2$.
So, any sequence of the forms $c \cdot r_1^{n}$ or $c \cdot r_2^{n}$ satisfies the recurrence.
But by linearity, any linear combination of solutions is also a solution!

This works in general for any linear recurrence, even ones whose ``characteristic polynomials'' have complex roots.
If there are repeated roots, then we have multiple terms corresponding to that root, each of which differs by a factor of $n$.

All of this is summarized below.

\begin{theorem}[Closed form of a linear recurrence]
    A $t$-th order linear recurrence whose characteristic polynomial has roots $r_1, \ldots, r_t$ (with no repeated roots), then
    \[ a_n = c_1 r_1^{n} + c_2 r_2^{n} + \cdots + c_t r_t^{n}. \]
    If there is a repeated root $r$ with multiplicity $m$, then some of the terms in the above equation have the form $k r^{n}, k n r^{n}, \ldots, k n^{m} r^{n}$.
\end{theorem}

We could've come to this same conclusion via the manipulation of generating functions, exemplified below.

\begin{example}[Closed form via generating functions]
    The recurrence
    \[ a_n = 5a_{n-1} - 6a_{n-2} \]
    has the generating function
    \[ a(x) = \sum_{n \geq 0} a_n x^{n} = 1 + 7x + 29x^2 + 103x^3 + \cdots. \]
    We can view $a(x)$ as the power series of some simpler function.
    We'll start by determining what that function is using some clever algebraic manipulation.
    \begin{align*}
        a(x) &= \sum_{n \geq 0} a_n x^{n} \\
        \intertext{Since this is a second-order recurrence, we'll take out the first two terms and expand the rest.}
        &= 1 + 7x + \sum_{n \geq 2} a_n x^{n} \\
        &= 1 + 7x + \sum_{n \geq 2} (5a_{n-1} - 6a_{n-2}) x^{n} \\
        &= 1 + 7x + 5x \sum_{n \geq 2} a_{n-1} x^{n-1} - 6x^2 \sum_{n \geq 2} a_{n-2} x^{n_-2} \\
        \intertext{The first sum is $a(x)$, minus the first term, while the second one is precisely $a(x)$.}
        a(x) &= 1 + 7x + 5x \big( a(x) - 1 \big) - 6x^2 a(x)
    \end{align*}
    Solving for $a(x)$ gives
    \[ a(x) = \frac{1 + 2x}{1 - 5x + 6x^2} = \frac{5}{1 - 3x} - \frac{4}{1 - 2x}. \]
    Each of these is a geometric series with common ratio $2x$ and $3x$, respectively.
    So the closed form of the sequence is
    \[ a_n = 5 \cdot 3^{n} - 4 \cdot 2^{n}, \]
    as expected.
\end{example}

\end{document}
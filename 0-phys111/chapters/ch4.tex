\documentclass[../p111main.tex]{subfiles}
\graphicspath{{\subfix{../figures/}}}

\begin{document}

\chapter{Rigid Bodies}
\section{Rotational Energy and Momentum}
The theory of oscillations, coupled or otherwise, proves to be very useful in simplifying the study of macroscopic systems.
Another way we might study these is by assuming the particles comprise a rigid body, in which case the particles are said to be fixed in place.
(In principle this isn't really true, of course, but we're not going to write down a Lagrangian with $10^{23}$ degrees of freedom to accommodate this technicality.)
This leaves us with six degrees of freedom: three for the body's position and three for its orientation.

Take our object and break it into chunks indexed by $\alpha$; the object's kinetic energy is
\begin{align*}
    T &= \frac{1}{2} \sum_{\alpha}^{} \Delta m_\alpha |\mbf{r}_\alpha|^2. \\
    \intertext{We'd like to break this up into translational and rotational components.
    To this end, we'll introduce a ``tagged point'' that specifies the position of the object.
    Let $\mbf{R}$ be the position of the tagged point and let $\mbf{r}_\alpha'$ be the position of $\alpha$ relative to the tagged point; with these our kinetic energy becomes}
    &= \frac{1}{2} \sum_{\alpha}^{} \Delta m_\alpha |\dot{\mbf{R}} + \dot{\mbf{r}}_\alpha'|^2. \\
    \intertext{Now, $\mbf{r}_\alpha'$ has constant magnitude for a rigid body, meaning $\dot{\mbf{r}}_\alpha'$ is pure rotation.
    In particular, if the object has angular velocity $\bm \omega$ at an angle $\theta$ to $\mbf{r}_\alpha'$ then $|\dot{\mbf{r}}_\alpha'| = (r_\alpha' \sin \theta) \,\omega = |\bm \omega \times \mbf{r}_\alpha'|$ and, by the right-hand rule, $\dot{\mbf{r}}_\alpha' = \bm \omega \times \mbf{r}_\alpha'$.
    So the kinetic energy becomes}
    &= \frac{1}{2} \sum_{\alpha}^{} \Delta m_\alpha |\dot{\mbf{R}} + (\bm \omega \times \mbf{r}_\alpha')|^2.
\end{align*}
Expanding and manipulating turns this into
\[ T = \frac{1}{2} M |\dot{\mbf{R}}|^2 + \dot{\mbf{R}} \cdot \left[ \bm \omega \times \left( \sum_{\alpha}^{} \Delta m_\alpha \mbf{r}_\alpha' \right) \right] + \frac{1}{2} \sum_{\alpha}^{} \Delta m_\alpha |\bm \omega \times \mbf{r}_\alpha'|^2, \]
where $M = \sum_{\alpha}^{} \Delta m_\alpha$ is the mass of the body.
Notice that the first and third terms are purely translation and purely rotational, respectively, and that the middle term goes to zero if $\dot{\mbf{R}} = 0$ or $\mbf{R} = \mbf{R}_\textrm{CM}$.
One of these will always be true for us, so we can ignore this term and write
\[ T = \frac{1}{2} M |\dot{\mbf{R}}|^2 + \frac{1}{2} \sum_{\alpha}^{} \Delta m_\alpha (\bm \omega \times \mbf{r}_\alpha') \cdot (\bm \omega \times \mbf{r}_\alpha'). \]
Let's now focus our attention on the rotational component, which we can write as
\begin{align*}
    T_\textrm{rot} &= \frac{1}{2} \sum_{\alpha}^{} \Delta m_\alpha \big[ \omega^2 r_\alpha^2 - (\bm \omega \cdot \mbf{r}_\alpha')^2 \big] \\
    \intertext{or, in the continuum limit,}
    &= \frac{1}{2} \int \rho(\mbf{r}') \,d^3\mbf{r}' \big[ \omega^2 r'^2 - (\bm \omega \cdot \mbf{r}')^2 \big].
\end{align*}
We'll write our dot products in index notation via
\[ \omega^2 = \sum_{i,j}^{} \omega_i \delta_{ij} \omega_j, \qquad (\bm \omega \cdot \mbf{r}')^2 = \sum_{i,j}^{} (\omega_i r'_i) (\omega_j r'_j), \]
and substitute them into $T_\textrm{rot}$ to get
\begin{align*}
    T_\textrm{rot} &= \frac{1}{2} \sum_{i,j}^{} \int \rho(\mbf{r}') \, d^3\mbf{r}' \big[ \omega_i \omega_j r'^2 \delta_{ij} - \omega_i \omega_j r'_i r'_j \big] \\
    &= \frac{1}{2} \sum_{i,j}^{} \omega_i \omega_j \left[ \int \rho(\mbf{r}') \, d^3 \mbf{r}' \big( r'^2\delta_{ij} - r'_i r'_j \big) \right].
\end{align*}
The bracketed expression is the $(i,j)$ component of the moment of inertia tensor $\mbf{I}$.
So the rotational kinetic energy becomes
\[ T_\textrm{rot} = \frac{1}{2} \sum_{i,j}^{} \mbf{I}_{ij} \omega_i \omega_j. \]
Notice how this reduces to the familiar $I \omega^2 / 2$ for rotation about a single fixed axis.
Now, the inertia tensor $\mbf{I}$ is a (time-dependent) symmetric matrix, and its diagonal components look like
\[ \mbf{I}_{xx} = \int \rho(\mbf{r}') \, d^3 \mbf{r}' \big[ r'^2 \cdot 1 - x^2 \big] = \int \rho(\mbf{r}') \, d^3 \mbf{r}' (y^2 + z^2). \]
Physically, this is a projection of $\mbf{r}'$ into the $yz$-plane a measure of how resistant our object is to rotations about $\hat x$.
The off-diagonal terms, however, have no clear physical meaning other than the fact that nonzero values tend to couple rotations about two axes---if we try to generate a rotation about $\hat x$ then we'd also get one about $\hat y$.
(We'll soon find that this happens when the object at hand lacks some kind of symmetry.)

For angular momentum we get a similar result by taking
\[ \bm \ell = \sum_{\alpha}^{} \Delta m_\alpha (\mbf{r}_\alpha \times \dot{\mbf{r}}_\alpha) = \sum_{\alpha}^{} \Delta m_\alpha (\mbf{R} + \mbf{r}_\alpha') \times (\dot{\mbf{R}} + \bm \omega \times \mbf{r}_\alpha'), \]
from which one could deduce that
\[ \ell_i = M (\mbf{R} \times \dot{\mbf{R}}) + \sum_{j}^{} \mbf{I}_{ij} \omega_j. \]
Note that this means conservation of angular momentum doesn't imply conservation of $\bm \omega$.

\section{Principal Axes}
The fact that the inertia tensor has off-diagonal entries suggests that the Cartesian basis isn't the most ``natural'' one for this problem.
We'd like to have one in which $\bm \ell = \lambda \bm \omega$ if $\bm \omega$ points along one of the axes; since $\bm \ell = \mbf{I} \bm \omega$ for non-translational frames we can write $\mbf{I} \bm \omega = \lambda \bm \omega$.
This is an eigenvalue problem, and because $\mbf{I}$ is symmetric we get real eigenvalues and orthogonal eigenvectors.
The eigenvalues $\lambda_i$ are the principal moments of our object, and the eigenvectors $\hat e_i$ are its principal axes.
Define $\hat x' = \hat e_1$, $\hat y' = \hat e_2$, and $\hat z' = \hat e_3$.

We can see that if $\bm \omega$ points along a principal axis then $\bm \ell \propto \bm \omega$ and $\bm \ell$-conservation implies $\bm \omega$-conservation.
More generally,
\[ \bm \ell = \mbf{I} \bm \omega = \lambda_{x'} \omega_{x'} \hat x' + \lambda_{y'} \omega_{y'} \hat y' + \lambda_{z'} \omega_{z'} \hat z'. \]
This indicates that, in the principal axis basis, the inertia tensor $\mbf{I}$ is diagonal with entries $\lambda_{x'}, \lambda_{y'}, \lambda_{z'}$.
This gives us
\[ T = \frac{1}{2} \sum_{i',j'}^{} \omega_{i'} \mbf{I}_{i'j'} \omega_{j'} = \frac{1}{2} \left( \lambda_{x'} \omega_{x'}^2 + \lambda_{y'} \omega_{y'}^2 + \lambda_{z'} \omega_{z'}^2 \right), \]
which looks a lot more like what we'd expect!
So principal axes serve to reduce the number of components of $\mbf{I}$ we need to deal with, which serves to make all the math a bit more physically intuitive.
Unfortunately, an object's principal axes are intrinsic to its shape and so vary with time as the body spins.
(This doesn't affect our calculation for $T$, though, since it's coordinate-independent and we did it in an inertial frame.)

Symmetry makes it very easy to identify some or all of an object's principal axes.
It isn't difficult to show that the existence of a reflection symmetry, say $x \to -x$, implies that $\hat x$ is a principal axis; it follows that $I_{xx}$ is a principal moment and $I_{xy} = I_{xz} = 0$.
Similarly, if there is a rotational axis of symmetry then that axis is a principal axis, and any other pair of mutually orthogonal axes are also principal.

\section{Rotating Frames of Reference}
Motivated by the fact that a body's principal axes generally define a frame of reference that rotates relative to the body's surroundings, we'll spend some time talking about how to work in rotating frames in general.

Consider a reference frame that rotates with angular velocity $\bm \omega$ relative to an inertial frame, and suppose a vector $\mbf{A}$ is stationary in this rotating frame.
If $\mbf{A}$ makes an angle $\theta$ with $\bm \omega$ then in a time $dt$ we have $(dA)_\textrm{in} = (A \sin \theta) \omega \,dt$ and $(d\mbf{A} / dt)_\textrm{in} = \bm \omega \times \mbf{A}$.
But if $\mbf{A}$ is not stationary in the rotating frame then by the Galilean velocity transformation
\[ \left( \frac{d\mbf{A}}{dt} \right)_\textrm{in} = \left( \frac{d\mbf{A}}{dt} \right)_\textrm{rot} + (\bm \omega \times \mbf{A}). \]
Now we'll derive the equation of motion for a point mass in a rotating frame.
This requires finding the Lagrangian, which should be written down in an inertial frame.
The kinetic energy in such a frame is
\begin{align*}
    T &= \frac{1}{2} m \big[ \mbf{v}_\textrm{rot} + (\bm \omega \times \mbf{r}) \big]^2 \\
    &= \frac{1}{2} m \big[ \mbf{v}_\textrm{rot} \cdot \mbf{v}_\textrm{rot} + 2 \mbf{v}_\textrm{rot} \cdot (\bm \omega \times \mbf{r}) + \omega^2 r^2 - (\bm \omega \cdot \mbf{r})^2 \big] \\
    &= \frac{1}{2} m \left[ \sum_{k}^{} v_{\textrm{rot}, \, k}^2 + 2 \sum_{k}^{} v_{\textrm{rot}, \, k} (\bm \omega \times \mbf{r})_k + \omega^2 \sum_{k}^{} r_k^2 - \sum_{j,k}^{} \omega_j \omega_k r_j r_k \right],
\end{align*}
and the Lagrangian is $L = T - U(\mbf{r})$.
Each component of the Euler-Lagrange equations looks like
\[ \frac{d}{dt} \frac{\partial L}{\partial v_{\textrm{rot}, \,i}} = \frac{\partial L}{\partial r_i}, \]
and computing these would give the equation of motion
\[ m \ddot{\mbf{r}}_\textrm{rot} = - \nabla U - m \bm \omega \times (\bm \omega \times \mbf{r}_\textrm{rot}) - 2m \bm \omega \times \mbf{v}_\textrm{rot} - m (\dot{\bm \omega} \times \mbf{r}_\textrm{rot}). \]
The $-\nabla U$ here is the inertial force, and it's all we'd see in an inertial frame.
The second term, called the centrifugal force, tends to push away from the axis of rotation; we could demonstrate this using the right-hand rule.
If we're at the equator of a planet with mass $M$ and radius $R$, we could show that
\[ \frac{|\mbf{F}|_\textrm{cent}}{|\mbf{F}|_\textrm{grav}} = \frac{\omega^2 r^3}{GM}. \]
For Earth this evaluates to about 0.3\%.
The third is the Coriolis force, and the fourth is the Euler term.
(We generally ignore this term since $\dot{\bm \omega}$ is usually either zero or near zero.)

\section{Euler's Equations}
We're now in a position to determine the equations of motion for a rotating rigid body.
If a torque $\bm \Gamma$ acts in the inertial frame then by the rotational vector transformation we have
\[ \dot{\mbf{L}}_\textrm{rot} + \bm \omega \times \mbf{L}_\textrm{rot} = \bm \Gamma. \]
In the case of free rotation ($\bm \Gamma = 0$) we can expand both nonzero terms to get Euler's equations:
\begin{align*}
    \lambda_1 \dot \omega_{x'} - (\lambda_2 - \lambda_3) \omega_{y'} \omega_{z'} &= 0, \\
    \lambda_2 \dot \omega_{y'} - (\lambda_3 - \lambda_1) \omega_{x'} \omega_{z'} &= 0, \\
    \lambda_3 \dot \omega_{z'} - (\lambda_1 - \lambda_2) \omega_{x'} \omega_{y'} &= 0.
\end{align*}
We can quickly leverage this set of equations to learn something about the stability of rotations about each principal axis.
If we have rotation purely about $\hat z'$ and introduce a small perturbation so that $\omega_{z'} \approx 0$ is preserved, the first of Euler's equations gives
\[ \lambda_1 \ddot \omega_{x'} = (\lambda_2 - \lambda_3) [\omega_{z'} \dot \omega_{y'} + \dot \omega_{z'} \omega_{y'}] \]
and after taking $\dot \omega_{z'} = 0$ and substituting for $\dot \omega_{y'}$ we get
\[ \ddot \omega_{x'} = \frac{(\lambda_2 - \lambda_3) (\lambda_3 - \lambda_1) \omega_{z'}^2}{\lambda_1 \lambda_2} \omega_{x'}. \]
Notice that, if $\lambda_1 < \lambda_3 < \lambda_2$, the above coefficient is positive and $\omega_x'$ grows exponentially!
In this case rotation about $\hat z'$ is unstable.
More generally, rotations about principal axes with ``intermediate'' moments are unstable; this is called the tennis racket theorem.

Now let's make a huge simplification and assume our body has cylindrical symmetry, so $\lambda_1 = \lambda_2 = \lambda$.
Looking at the Euler equations, we can see that this makes $\omega_{z'}$ constant, so that nonlinear system of equations has become linear!
Specifically, we get
\[ \dot \omega_{x'} = \Omega_b \omega_{y'}, \quad \dot \omega_{y'} = -\Omega_b \omega_{x'}, \qquad \Omega_b \equiv \frac{(\lambda - \lambda_3) \omega_{z'}}{\lambda}. \]
We could combine these equations to get $\ddot \omega_{x'} = -\Omega_b^2 \omega_{x'}$ which, with the initial conditions $\omega_{x'}(0) = \omega_0$ and $\omega_{y'}(0) = 0$, gives
\[ \omega_{x'}(t) = \omega_0 \cos \Omega_b t, \qquad \omega_{y'}(t) = -\omega_0 \sin \Omega_b t. \]
So both $\bm \omega$ and $\bm \ell$ precess about $\hat z'$ at a rate $\Omega_b$:
\begin{align*}
    \bm \omega &= \omega_0 \cos \Omega_b t \, \hat x' - \omega_0 \sin \Omega_b t \, \hat y' + \omega_{z'} \hat z', \\
    \bm \ell &= \lambda \big[ \omega_0 \cos \Omega_b t \, \hat x' - \omega_0 \sin \Omega_b t \, \hat y' \big] + \lambda_3 \omega_{z'} \hat z'.
\end{align*}
But remember, this is what the body sees in its own, rotating frame of reference.
In the inertial ``space frame'' we note that $\bm \ell$ is fixed, that it makes a constant angle with $\bm \omega$, and that these two vectors lay in the same plane as $\hat z'$.
Put together, it turns out that both $\hat z'$ and $\bm \omega$ precess about $\bm \ell$ at a rate $\Omega_s = \ell / \lambda$!
(This will be easier to show via the Lagrangian approach that we'll develop soon.)

\section{Euler Angles}
The Euler equations do a good job at describing motion in the body frame, but we really seek equations of motion relative to the inertial frame.
To make this happen we'll need to define a set of generalized coordinates that specify the orientation of our body relative to such a frame.
We'll once again focus on objects with cylindrical symmetry.

The standard choice is to use the Euler angles, which are defined as follows.
Start with the principal axes aligned with the inertial-frame axes.
\begin{itemize}[topsep=0pt]
    \item Rotate the body an angle $\phi$ about the $z$-axis.
    \item Tilt the body an angle $\theta$ from the $z$-axis (about $\hat x$').
    \item Spin the body an angle $\psi$ about the $z'$-axis.
\end{itemize}
In practical terms, $\theta$ and $\phi$ are the polar and azimuthal angles of the $z'$-axis, while $\psi$ is the amount that $\hat x'$ has rotated out of the $xy$-plane due to the spin of the object about $\hat z'$.

This gives rise to three kinds of motion: spin ($\dot \psi$), precession ($\dot \phi$), and nutation ($\dot \theta$).
Each of these contributes to the full rotation of the body in different ways.
\begin{itemize}[topsep=0pt]
    \item ($\dot \psi$)
    This is purely rotation about $\hat z'$, so $\omega_{z' (\psi)} = \dot \psi$.
    
    \item ($\dot \phi$)
    This is rotation about $\hat z = \hat z' \cos \theta + \hat y' \sin \theta$, so $\omega_{z'(\phi)} = \dot \phi \cos \theta$ and $\omega_{y'(\phi)} = \dot \phi \sin \theta$.
    
    \item ($\dot \theta$)
    By cylindrical symmetry we can choose $\hat x'$ to lie in the $xy$-plane, in which case $\dot \theta$ is purely rotation about $\hat x'$ meaning $\omega_{x'(\theta)} = \dot \theta$.
\end{itemize}
With this we have the kinetic energy
\begin{align*}
    T &= \frac{1}{2} \left( \lambda \omega_{x'}^2 + \lambda \omega_{y'}^2 + \lambda_3 \omega_{z'}^2 \right) \\
    &= \frac{1}{2} \lambda \left( \dot \theta^2 + \dot \phi^2 \sin^2 \theta \right) + \frac{1}{2} \lambda_3 \left( \dot \psi + \dot \phi \, \cos \theta \right)^2
\end{align*}
which, along with a potential, is just what we need to write down a Lagrangian!
If we again focus our attention on the free rotation case (in which $L = T$) we can immediately see a couple of conserved quantities:
\begin{align*}
    p_\psi &= \frac{\partial L}{\partial \dot \psi} = \lambda_3 \left( \dot \psi + \dot \phi \cos \theta \right), \\
    p_\phi &= \frac{\partial L}{\partial \dot \phi} = \lambda \dot \phi \sin^2 \theta + \lambda_3 \cos \theta \left( \dot \psi + \dot \phi \cos \theta \right).
\end{align*}
Also, if we define $\hat z$ such that $\bm \ell  = \ell \hat z$, then by conservation of angular momentum we have
\begin{align*}
    \ell \, \hat z &= \lambda \omega_{x'} \hat x' + \lambda \omega_{y'} \hat y' + \lambda \omega_{z'} \hat z', \\
    \ell \cos \theta \,\hat z' + \ell \sin \theta \,\hat y' &= \lambda \dot \theta \,\hat x' + \lambda \dot \phi \sin \theta \,\hat y' + \lambda_3 (\dot \psi + \dot \phi \cos \theta) \hat z'.
\end{align*}
From this it follows that $\dot \theta = 0$, so we have no nutation, and $\dot \phi = \ell / \lambda$, which is exactly the precession frequency we pointed out previously.
We can also say something about spin---since $\ell \cos \theta = \lambda_3 (\dot \psi + \dot \phi \cos \theta)$,
\[ \dot \psi = \frac{1}{\lambda_3} \left( \ell \cos \theta - \lambda_3 \dot \phi \cos \theta \right) = \frac{1}{\lambda_3} \left( \ell \cos \theta - \lambda_3 \frac{\ell}{\lambda} \cos \theta \right) = \frac{\ell \cos \theta}{\lambda \lambda_3} (\lambda - \lambda_3), \]
and because $\ell \cos \theta = \lambda_3 \omega_{z'}$, this becomes
\[ \dot \psi = \frac{\omega_{z'} (\lambda - \lambda_3)}{\lambda}, \]
which is exactly the frequency at which we found $\bm \ell$ to precess in the body frame.

We'll finish off with physical interpretations for the generalized momenta $p_\psi, p_\phi$, along with a relationship between the two.
In the body frame we have
\begin{align*}
    p_\psi &= \lambda_3 \left( \dot \psi + \dot \phi \cos \theta \right) = \lambda_{z'}, \\
    p_\phi &= \lambda \dot \phi \sin^2 \theta + \lambda_3 \cos \theta \left( \dot \psi + \dot \phi \cos \theta \right) = \ell_{y'} \sin \theta + \ell_{z'} \cos \theta,
\end{align*}
and by moving into the inertial frame with $\bm \ell = \ell \hat z$ we get $p_\psi = \ell \cos \theta$ and
\[ p_\phi = \ell \sin^2 \theta + \ell \cos^2 \theta = \ell. \]
Note that $p_\phi \cos \theta = p_\psi$ here.
Lastly, let's take a look at the Euler-Lagrange equation for $\theta$:
\begin{align*}
    \lambda \ddot \theta &= \lambda \dot \phi^2 \sin\theta \cos\theta + \lambda_3 \left( \dot \psi + \dot \phi \cos \theta \right) \left( - \dot \phi \sin \theta \right) \\
    &= \sin \theta \dot \phi \left( \lambda \dot \phi \cos \theta - p_\psi \right) \\
    \intertext{Because $p_\phi = \lambda \sin^2 \theta \,\dot \phi + p_\psi \cos \theta$, we can substitute for the innner $\dot \phi$ and, after some manipulation, get}
    &= \frac{\dot \phi}{\sin \theta} \left( p_\phi \cos \theta - p_\psi \right).
\end{align*}
So we can see that $\ddot \theta = 0$ if $p_\phi \cos \theta = p_\psi$, which occurs when $\bm \ell$ aligns with $\hat z$.

\section{Gyroscope Precession}
Now we'll consider a problem in which there is an external torque---a gyroscope under the influence of gravity.
There's actually only a couple of modifications we'll need to make from the free rotation problem.
\begin{itemize}[topsep=0pt]
    \item Our tagged point will be the stationary point of the gyroscope in contact with the ground.
    This turns out to not affect our kinetic energy $T$ at all if we go through the same motions as before.

    \item If $R$ is the distance from the tagged point to the center of mass then $U = MgR \cos \theta$.
\end{itemize}
With these the Lagrangian is $L = T - U$ and the Euler-Lagrange equation in $\theta$ is
\[ \lambda \ddot \theta = \lambda \dot \phi^2 \sin \theta \cos \theta - \lambda_3 \left( \dot \psi + \dot \phi \cos \theta \right) \dot \phi \sin \theta + MgR \sin \theta. \]
In the case of no nutation ($\dot \theta = 0$) we get $\lambda \dot \phi^2 \cos \theta - p_\psi \dot \phi + MgR = 0$.
If $p_\psi$ is large (i.e., if the top is spinning very quickly), solving the quadratic for $\dot \phi$ would give
\[ \dot \phi = \frac{MgR}{p_\psi} \;\textrm{ or }\; \dot \phi = \frac{p_\psi}{\lambda \cos \theta} \]
as the respective slow and fast solutions.
This is called steady precession---all angles change at constant rates!
Now, if there is nutation we'll need to use the fact that the Hamiltonian is conserved and equal to
\[ H = \frac{1}{2} \lambda \left( \dot \phi^2 \sin^2 \theta + \dot \theta^2 \right) + \frac{1}{2} \lambda_3 \left( \dot \psi + \dot \phi \cos \theta \right)^2 + MgR \cos \theta.\]
Noting that $p_\phi$ and $p_\psi$ are conserved, we can substitute $p_\psi = \lambda_3 (\dot \psi + \dot \phi \cos \theta)$ and $\lambda \dot \phi \sin^2 \theta = p_\phi - p_\psi \cos \theta$ to get
\[ H = \frac{1}{2} \lambda \dot \theta^2 + \frac{(p_\phi - p_\psi \cos \theta)^2}{2\lambda \sin^2 \theta} + \frac{p_\psi^2}{2\lambda_3} + MgR \cos \theta, \]
which allows us to define an effective potential as before.
Solving as we did for the two-body problem would reveal that
\[ \dot \phi = \frac{p_\phi - p_\psi \cos \theta}{\lambda \sin^2 \theta}. \]
Notice that if $p_\phi > p_\psi$ then $\dot \theta$ is always positive, but if $p_\phi < p_\psi$ then the sign of $\dot \phi$ may change depending on the values of $p_\psi - p_\psi \cos \theta$ at the minimum and maximum $\theta$.

\end{document}
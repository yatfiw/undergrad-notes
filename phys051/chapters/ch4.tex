\documentclass[../p051main.tex]{subfiles}
\graphicspath{{\subfix{../figures/}}}

\begin{document}

\chapter{Optics}
\section{Light and Materials}
Now we'll turn our attention to a study of light in its own right.
To begin, we'll need to modify Maxwell's equations a little bit.
Recall that they were constructed in free space, so in order to make them valid in materials we must substitute $\epsilon_0 \to \kappa_E \epsilon_0$ and $\mu_0 \to \kappa_B \mu_0$.
Consequentially, in matter we have the wave equation
\[ \nabla^{2} \mathbf{E} = (\kappa_e \kappa_m) \mu_0 \varepsilon_0 \frac{\partial^{2} \mbf{E}}{\partial t^2}. \]
If we define the material's index of refraction $n = \sqrt{\kappa_e \kappa_m}$, then we find that light travels at
\[ v = \frac{1}{\sqrt{\kappa_e \kappa_m}} \frac{1}{\sqrt{\mu_0 \epsilon_0}} = \frac{c}{n} \]
when in a medium.
It follows that electromagnetic waves satisfy $B_0 = E_0 n / c$, and that their intensities are given by $I = E_0^2 n / 2\mu_0 c$.

A point at which materials with different refractive indices meet is called an interface.
When an electromagnetic wave is incident on an interface, it gets split into a reflected wave and a transmitted wave.
For normal incidence, the equations describing this split are
\begin{align*}      
    \mathbf{E}_i &= E_i \,\hat{y}\, \cos (k_1 x - \omega_1 t), & \mathbf{B}_i &= \frac{E_i n_1}{c} \,\hat{z}\, \cos (k_1 x - \omega_1 t), \\
    \mathbf{E}_r &= E_r \,\hat{y}\, \cos (-k_1 x - \omega_1 t), & \mathbf{B}_r &= \frac{E_r n_1}{c} \,(-\hat{z})\, \cos (-k_1 x - \omega_1 t), \\
    \mathbf{E}_t &= E_t \,\hat{y}\, \cos (k_2 x - \omega_2 t), & \mathbf{B}_t &= \frac{E_t n_2}{c} \,\hat{z}\, \cos (k_2 x - \omega_2 t),
\end{align*}
where $n_1$ is the refractive index of the first material and $n_2$ is that of the second.
We can relate all of these equations using Ampere's law and Faraday's law; if we take thin loops that are parallel to $\mbf{E}$ and $\mbf{B}$, respectively, we get
\[ \oint \mbf{E} \cdot d\mbf{l} = 0, \qquad \oint \mbf{B} \cdot d\mbf{l} = 0. \]
So if we have an interface at $x=0$, then the relationship between the electric fields is
\[ E_i \cos(-\omega_1 t) + E_r \cos(-\omega_1 t) - E_t \cos (-\omega_2 t) = 0. \]
The only way for this equation to be true for all $t$ is if $\omega_1 = \omega_2 - \omega$. meaning the frequency of light does not change in a medium!
All that changes, then, is the wavenumber $k = \omega / v = 2\pi n / \lambda_\textrm{vac}$.

Anyway, cancelling the cosines gives $E_i + E_r = E_t$.
A similar analysis for magnetic fields would give $E_i n_1 - E_r n_1 = E_t n_2$, and solving the resulting system of equations gives
\[ E_r = \left( \frac{n_1 - n_2}{n_1 + n_2} \right) E_i, \qquad E_t = \left( \frac{2n_1}{n_1 + n_2} \right) E_i.  \]
Note that $E_r$ is negative when $n_1 < n_2$, so in this case the reflected wave is out of phase with the transmitted wave.
Otherwise the waves are in phase.
Here's a handy mnemonic to keep track:
\begin{center}
    "Low to high, add a $\pi$." \qquad "High to low, let it go."
\end{center}
For scenarios with multiple interfaces we simply apply these rules several times, often ignoring light that bounces back and forth between interfaces many times.

\section{Polarization}
The orientation of an electromagnetic wave, called its polarization, can be described entirely by the direction in which its electric field points.
Most common sources of light produce unpolarized light, in which the direction of $\mbf{E}$ is basically random and time-varying.
When light is polarized, though, it can come in three different forms: linear, circular, and elliptical.
\begin{itemize}
    \item For light that is linearly polarized, $\mbf{E}$ points in a constant direction as the wave propagates.
    A wave propagating in the $x$-direction may be polarized along $\hat y$, $\hat z$, or some linear combination of the two.

    \item Circularly polarized light is comprised of two identical components that are out of phase by precisely a quarter-cycle, so that the electric field appears to trace out a circle in space.
    Confusingly, optics people tend to specify the ``handedness'' of this tracing with respect to the receiving end of the light, so we point our thumb opposite the direction of propagation and curl our fingers in such a way that matches the twist of the $\mbf{E}$-field.
    This gives us the following parameterizations.
    \begin{align*}
        \textrm{RHCP: } \mathbf{E} &= E_0 \,\hat{y} \cos(kx - \omega t) + E_0 \,\hat{z} \sin(kx - \omega t) \\
        \textrm{LHCP: } \mathbf{E} &= E_0 \,\hat{y} \sin(kx - \omega t) - E_0 \,\hat{z} \cos(kx - \omega t)
    \end{align*}
    \vspace{-18pt}

    \item Elliptical polarization occurs we have two identical components that are out of phase by a different amount, so that the electric field appears to trace out an ellipse in space.
    Linear and circular polarization can be seen as special cases of elliptical polarization.
\end{itemize}
The polarization of light can be controlled using filters called polarizers.
In the case of a linear polarizer, the incident electric field is split into two components $\mbf{E}_\textrm{in} = \mbf{E}_\parallel + \mbf{E}_\perp$ which are parallel and perpendicular to the direction of polarization, respectively.
As $\mbf{E}_\textrm{in}$ passes through, $\mbf{E}_\parallel$ is transmitted entirely while $\mbf{E}_\perp$ is absorbed entirely.
So if $\mbf{E}_\textrm{in}$ is polarized at an angle $\theta$ with respect to the polarization direction, then $E_\textrm{out} = E_\textrm{in} \cos \theta$ and we have Malus's law
\[ I_\textrm{out} = I_\textrm{in} \cos^2 \theta. \]
Linear polarizers are the basis for a host of technologies, like polarizing sunglasses and LCD screens!
They also exhibit some strange behavior due to Malus's law.
Two polarizers oriented perpendicular to one another will block out all light, but if a third is added at a $45^\circ$ angle to both of them, then suddenly some light will pass through.

There are other ways light can be polarized, too.
One is through scattering: if some light traveling in the $\hat z$-direction and is incident on a collection of dust, then this dust is excited and begins to vibrate in the $xy$-plane (i.e., in the plane of the electromagnetic field).
When light is emitted its propagation direction and polarization direction both remain in this plane and, of course, are orthogonal to each other.

Polarization by reflection may also occur when light reflects on an interface between materials.
Define the light's plane of incidence to be that spanned by the vector normal to the interface and that which points in the direction of incident light.
For any interface, the incidence-plane component of the reflected electromagnetic wave is dependent not only on the materials' refraction indices, but also on the angle of incidence.
Thus there is a critical angle, called the Brewster's angle, such that all light polarized in the plane of incidence is transmitted.
(This also happens to be the angle at which the reflected and transmitted light are orthogonal to one other.)
As a result, the reflected light is polarized orthogonal to the plane of incidence.

Finally, polarization can occur when light passes through a birefringent material, the details of which don't concern us.
In the big picture, some materials have refractive indices that depend on the direction in which light propagates through them.
Such a material has an intrinsic optic axis with index $n_o$, while the perpendicular axis has $n_e$; the perpendicular components of the normally incident wave (the $o$- and $e$-rays) undergo a relative phase shift
\[ \Delta \phi = \frac{2\pi}{\lambda} (n_o - n_e)d, \]
where $\lambda$ is the light's wavelength and $d$ is the width of the material.
If $\Delta \phi = \pi / 4$ or a coterminal equivalent, then we have a quarter-wave plate that can be used to convert $45^\circ$-incident linearly polarized light into circularly polarized light, or vice versa.
In the case of $\Delta = \pi / 2$ we have a half-wave plate which ``mirrors'' the polarization of the incident light.

\end{document}
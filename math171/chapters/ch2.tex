\documentclass[../m171main.tex]{subfiles}
\graphicspath{{\subfix{../figures/}}}

\begin{document}

\chapter{Quotient Groups}
\section{Fibers and Kernels}
At this point we've gotten a broad overview of groups and the subgroups embedded within them.
Now we'll do something a little more sophisticated---we'll take a group and collapse it into a smaller group based on a set of characteristics decided ahead of time.
We'll begin this discussion in the context of fibers and kernels.

\begin{definition}[Fiber]
    Let $\varphi : G \to H$ be a homomorphism.
    We can define an equivalence relation $\sim$ on $G$ where if $x,y \in G$, then
    \[ x \sim y \;\textrm{ iff }\; \varphi(x) = \varphi(y). \]
    The resulting equivalence classes are called fibers.
\end{definition}

\begin{theorem}[]
    Let $\varphi : G \to H$ be a homomorphism with kernel $K$, and let $x,y \in G$.
    We have the equivalent statements
    \[ \varphi(x) = \varphi(y) \;\iff\; y^{-1} x \in K \;\iff\; xy^{-1} \in K \;\iff\; x^{-1}y \in K \;\iff\; yx^{-1} \in K. \]
\end{theorem}

The latter two bits here can be used to show something important about what fibers have to do with kernels.

\begin{theorem}[]
    Let $\varphi : G \to H$ be a homomorphism with kernel $K$.
    The fiber containing $x \in G$ can be expressed in two different ways:
    \[ xK = \left\{ xk \mid k \in K \right\}, \qquad Kx = \left\{ kx \mid k \in K \right\}. \]
    In particular, note that
    \begin{enumerate}[label=(\alph*)]
        \item $xK = Kx$ (so $x$ is in the normalizer of $K$) and
        \item $\varphi(x) = \varphi(y)$ if and only if $xK = yK$.
    \end{enumerate}
\end{theorem}

In this way, each fiber of a homomorphism $\varphi$ is a ``translate'' of $\ker(\varphi)$, and these fibers partition $G$.
It turns out that they interact in a very nice way!

\begin{definition}[]
    Let $\varphi : G \to H$ be a homomorphism with kernel $K$.
    The set of associated fibers, called ``$G$ mod $K$'', is denoted by
    \[ G / K = \left\{ gK \mid g \in G \right\}. \]
\end{definition}

\begin{theorem}[]
    Let $\varphi : G \to H$ be a homomorphism with kernel $K$.
    Then $G / K$ forms a group where the binary operation is given by
    \[ (uK)(vK) = (uv)K. \]
\end{theorem}

\begin{proof}
    Showing that the binary operation satisfies the group axioms is straightforward, but we must also show that it is well-defined---that is, it shouldn't depend on which elements we take from $uK$ and $vK$.

    Say $uK = u'K$ and $vK = v'K$, so $\varphi(u) = \varphi(u')$ and $\varphi(v) = \varphi(v')$, and
    \[ \varphi(uv) = \varphi(u) \varphi(v) = \varphi(u') \varphi(v') = \varphi(u'v'). \]
    Thus $(uK)(vK) = (uv)K = (u'v')K = (u'K)(v'K)$ and the binary operation is well-defined.
\end{proof}

As a concrete example, consider $\varphi : D_8 \to \Z / 2\Z$ where $\varphi(s^{i} r^{j}) = \bar i$.
We can see that $\ker(\varphi) = \left< r \right>$ and
\[ D_8 / \left< r \right> \cong \varphi(D_8) = Z / 2\Z. \]
Note that we'll sometimes write $\bar u$ to denote the fiber $uK$, in which case we have $\bar u \bar v = \overline{uv}$.

\section{Quotients}
We can generalize much of the above discussion to any subgroup of the group we're working with, rather than just the kernel of some homomorphism.

\begin{definition}[Coset]
    Let $N \leq G$.
    If $g \in G$, then
    \[ gN = \left\{ gn \mid n \in N \right\} \;\text{ and }\; Ng = \left\{ ng \mid g \in N \right\} \]
    are left and right cosets of $N$ in $G$, respectively.
\end{definition}

Note that this time, the left and right cosets are not necessarily equal.
We will make statements mostly about left cosets, but bear in mind that every such statement has an analog for right cosets.

\begin{theorem}[]
    If $N \leq G$ then the left cosets of $N$ in $G$ partition $G$.
    Furthermore, for $u,v \in G$ then $uN = vN$ if and only if $v^{-1} u \in N$.
    % "we're in the same coset if our 'difference' is in N".
\end{theorem}

\begin{proof}
    First note that $1 \in N$, meaning $g \in gN$ for all $g \in G$ and the set of all $gN$ ``covers'' $G$.
    Now suppose $uN \cap vN \neq \varnothing$ for some $u,v \in G$; we will show that $uN = vN$.

    Let $x \in uN \cap vN$, so $x = un = vm$ for some $n,m \in N$ and $u = v \cdot mn^{-1}$.
    Thus for a $t \in N$ we can write $ut = v \cdot mn^{-1} t$, meaning $uN \subseteq vN$.
    By a symmetric argument we also have $vN \subseteq uN$ and $uN = vN$.
    It follows that the left cosets of $N$ in $G$ partition $G$.

    Finally, by what we saw above, $uN = vN$ if and only if $u = vn$ for some $n \in N$.
    This happens if and only if $v^{-1} u \in N$.
\end{proof}

For the next two theorems, let $G / N$ (read aloud as ``$G$ mod $N$'') denote the left cosets of $N$ in $G$.

\begin{theorem}[] \label{thm:cosets_well_defined}
    If $N \leq G$ then the binary operation
    \[ (uN) \cdot (vN) = (uv)N \]
    defined on $G / N$ is well-defined if and only if $gNg^{-1} \subseteq N$ for all $g \in G$.
\end{theorem}

\pagebreak

\begin{proof}
    ($\Rightarrow$)
    Suppose the binary operation is well-defined.
    Then both $1,n \in N$ are representatives of the same coset, so for $g \in G$ the expression
    \[ (1N) \cdot (g^{-1} N) = g^{-1}  \]
    must be equivalent to
    \[ (nN) \cdot (g^{-1} N) = (ng^{-1}) N. \]
    Hence $\left( g^{-1} \right)^{-1} = gng^{-1} \in N$.
    It follows that $gNg^{-1} \subseteq N$ for all $g \in G$.
    \medskip

    ($\Leftarrow$)
    Suppose $gNg^{-1} \subseteq N$ for all $g \in G$, and let $uN = vN$ and $xN = yN$.
    So $u = nv$ and $x = ym$ for some $m,n \in N$, and
    \[ ux = vn \cdot ym = vy \cdot \left( y^{-1} ny \right) m, \]
    and since the parenthetical is in $N$ we have $(ux)N = \left[ vy \cdot \left( y^{-1} ny \right) m \right] N = (vy)N$.
\end{proof}

\begin{theorem}[]
    If the binary operation above is well defined on $G / N$, then $G / N$ forms a group with this binary operation.
    The identity is $N = 1N$, and $(gN)^{-1} = g^{-1} N$.
\end{theorem}

We'll often refer to $G / N$ as the quotient of $G$ by $N$.
Now, notice that the condition in Theorem 2.\ref{thm:cosets_well_defined} is stronger than it seems because conjugation is bijective.
So really, our binary operation is well defined if and only if $gNg^{-1} = N$ for all $g \in G$.
This evokes the normalizer we discussed earlier!

\begin{definition}[Normal subgroup]
    If $N \leq G$, we say $N$ is normal in $G$ if $gNg^{-1} = N$ for all $g \in G$.
    In this case we write $N \trianglelefteq G$.
\end{definition}

And now, a result that ties the structure we've built here to the theory from earlier.

\begin{theorem}[]   \label{thm:normal_kernel}
    $N \trianglelefteq G$ if and only if $N$ is the kernel of some homomorphism.
\end{theorem}

\begin{proof}
    ($\Rightarrow$)
    If $N \trianglelefteq G$ then $N$ is the kernel of the ``natural projection'' homomorphism $\varphi : G \to G / N$ where $\varphi(g) = gN$.
    \medskip

    ($\Leftarrow$)
    If $N$ is the kernel of some homomorphism then $gN = Ng$ and $gNg^{-1} = N$ for all $g \in G$.
\end{proof}

\section{More on Cosets}
We'll begin here with a powerful result about the structure of cosets in a group and a couple of corollaries.

\begin{theorem}[Lagrange's theorem]
    If $G $is a finite group and $H \leq G$, then $|H|$ divides $|G|$ and there are $|G| / |H|$ left cosets of $H$ in $G$.
\end{theorem}

\begin{proof}
    Suppose there are $k$ left cosets of $H$ in $G$; these left cosets partition $G$.
    Now, the map $H \to gH$ defined by $h \mapsto gh$ is a surjection, and it is injective since $gh_1 = gh_2$ implies $h_1= h_2$.
    So there is a bijection form $H$ to $gH$ and $|H| = |gH|$.

    $G$ is partitioned into $k$ disjoint subsets, each of cardinality $|H|$, so $|G| = k|H|$.
    Thus $k = |G| / |H|$.
\end{proof}

\begin{corollary}[]
    If $G$ is a finite group and $x \in G$, then $|x|$ divides $|G|$ and $x^{|G|} = 1$.
\end{corollary}

\begin{corollary}[]
    If $G$ is a finite group of prime order $p$, then $G$ is cyclic and $G \cong \Z / p\Z$.
\end{corollary}

Note that the converse to Lagrange's theorem is not, in general, true.
Before we give a counterexample, we'll make another straightforward statement about cosets.

\begin{definition}[Index of a subgroup]
    If $G$ is a (potentially infinite) group and $H \leq G$, then the number of left cosets of $H$ in $G$ is the index of $H$ in $G$.
    It is denoted by $|G : H|$.
\end{definition}

\begin{theorem}[]
    If $H \leq G$ and $|G : H| = 2$ then $H \trianglelefteq G$.
\end{theorem}

\begin{proof}
    The left and right cosets of $H$ and $G$ are equal.
    In particular, if $g \in G - H$, then the set of left cosets of $H$ in $G$ is $\left\{ H, gH \right\}$, the set of right cosets is $\left\{ G, Hg \right\}$.
\end{proof}

Now let $A$ be the group of symmetries of a regular tetrahedron (so $|A| = 12$), and suppose $A$ had a subgroup of order 6.
Then $|A : H| = 2$, meaning $H$ is a normal subgroup of $A$ and $A / H \cong Z / 2\Z$.
It follows that $(aH)^2 = H$ for all $a \in A$, implying that $a^2 H = H$ and $a^2 \in H$ for all such $a$.
But if $|a| = 3$ then $a = (a^2)^2$ and $a \in H$, so $H$ contains all elements in $A$ of order $3$.
The fact that $A$ has at least eight elements of order 3 brings us to a contradiction.

Happily, Lagrange's theorem has some good partial converses, neither of which we will prove now.

\begin{theorem}[Cauchy's theroem]
    If $G$ is a finite group and $p$ is a prime divisor of $|G|$, then $G$ has an element of order $p$.
\end{theorem}

\begin{theorem}[First Sylow theorem]
    If $G$ is a finite group of order $p^{\alpha} m$ where $p$ is prime and $p$ does not divide $m$, then $G$ has a subgroup of order $p^{\alpha}$.
\end{theorem}

It turns out that this result from Sylow is provably the strongest partial converse to Lagrange's theorem.
We'll come back to this it later---for now, we'll finish off the section with some more useful facts about cosets.

\begin{theorem}[]
    Let $H$ and $K$ be subgroups of $G$, and define $HK = \left\{ hk \mid h \in H, \, k \in K \right\}$.
    \begin{enumerate}[label=(\alph*)]
        \item If $H$ and $K$ are finite then $|HK| = |H| |K| / |H \cap K|$.
        \item $HK$ is a subgroup of $G$ if and only if $HK = KH$.
        \item If $H \leq N_G(K)$ then $HK \leq G$.
        \item If $K \trianglelefteq G$ then $HK \leq G$ for any $H \leq G$.
    \end{enumerate}
\end{theorem}

\section{The Isomorphism Theorems}
Here we'll give a few important results surrounding quotient groups and homomorphisms.
The first is essentially a formalization of the ideas we started the chapter with.

\begin{theorem}[First isomorphism theorem]
    If $\varphi : G \to H$ is a homomorphism then
    \begin{enumerate}[label=(\alph*)]
        \item $\ker(\varphi) \trianglelefteq G$ and
        \item $G / \ker(\varphi) \cong \varphi(G)$.
    \end{enumerate}
\end{theorem}

\begin{proof}
    (Sketch)
    We've already proved (a) as Theorem 1.\ref{thm:normal_kernel}.
    For (b), define $\Phi : G / \ker (\varphi) \to \varphi(G)$ via $\Phi(g \ker (\varphi)) = \varphi(g)$.
    We'd show, from here, that $\Phi$ is well-defined and is a bijective homomorphism.
\end{proof}

For our next theorem, let $B$ normalize a subgroup $A$ of $G$ and suppose we want to say something about the ``quotient'' of $A$ by $B$.
Since $B$ isn't necessarily a subgroup of $A$, we have to do one of two things first: expand $A$ to include the elements in $B$, or contract $B$ to exclude elements not in $A$.

\begin{theorem}[Second isomorphism theorem]
    Let $A,B$ be subgroups of a group $G$, where $A \leq N_G(B)$.
    Then
    \begin{enumerate}[label=(\alph*)]
        \item $B \trianglelefteq AB$ and $A \cap B \trianglelefteq A$.
        \item $AB / B \cong A / (A \cap B)$.
    \end{enumerate}
\end{theorem}

\begin{proof}
    The first part of (a) will be proved soon.
    For the second part, if $c \in A \cap B$ and $A \in A$ then $aca^{-1} \in A$ (because $c \in A$) and $aca^{-1} \in B$ (because $a \in N_G(B)$).
    Thus $aca^{-1} \in A \cap B$, so $A \cap B \trianglelefteq A$.

    For (b), define a map $\varphi : AB \to A / (A \cap B)$ via
    \[ \varphi(ab) = a (A \cap B), \quad a \in A, \; b \in B. \]
    We quickly show that $\varphi$ is well-defined: if $ab = a_1b_1$ then $a_1^{-1} a = b_1b^{-1}$, meaning $a_1^{-1} a \in A \cap B$ and $a (A \cap B) = a_1 (A \cap B)$.
    Now, $\varphi$ is a homomorphism because
    \begin{align*}
        \varphi \big( (ab) (a_1 b_1) \big) &= \varphi \big( aa_1 \cdot a_1^{-1} b a_1 \cdot b_1 \big) \\
        &= aa_1 (A \cap B) \\
        &= \big( a (A \cap B) \big) \big( a_1 (A \cap B) \big) \\
        &= \varphi(ab) \varphi(a_1 b_1).
    \end{align*}
    The kernel of $\varphi$ is, by definition,
    \[ \ker(\varphi) = \left\{ ab \mid a \in A, \, b \in B, \, a \in A \cap B \right\} = B. \]
    Finally, $\varphi(AB) = A / (A \cap B)$ because each coset of $A \cap B$ has a representative in $A$.
    So by the first isomorphism theorem, $B \trianglelefteq AB$ and $AB / B \cong A / (A \cap B)$.
\end{proof}

Our third theorem concerns taking quotients of quotients and is a bit more intuitive.

\begin{theorem}[Third isomorphism theorem]
    Let $H$ and $K$ be normal subgroups of a group $G$ where $H \leq K$.
    Then
    \begin{enumerate}[label=(\alph*)]
        \item $K / H \trianglelefteq G / H$.
        \item $(G / H) / (K / H) \cong G / K$.
    \end{enumerate}
\end{theorem}

\begin{proof}
    Define $\varphi : G / H \to G / K$ by $\varphi(gH) = gK$.
    Notice that $gH = g_1H$ implies $g_1^{-1} g \in H \leq K$, meaning $gK = g_1 K$ and $\varphi$ is well-defined.
    It is easy to see that $\varphi$ is surjective, and the kernel is
    \begin{align*}
        \ker(\varphi) &= \left\{ gH \in G / H \mid gK = K \right\} \\
        &= \left\{ gH \in G / H \mid g \in K \right\} \\
        &= K / H.
    \end{align*}
    Both (a) and (b) follow from the first isomorphism theorem.
\end{proof}

The final theorem here relates the subgroup structure of $G$ to that of $G / N$.
In particular, the subgroups of $G / N$ have the same structure as the subgroups of $G$ containing $N$.

\begin{theorem}[Fourth isomorphism theorem]
    Let $G$ be a group and let $N$ be a normal subgroup of $G$.
    Then there is a bijection from the set of subgroups $A$ of $G$ containing $N$ onto the set of subgroups $\overline A = A / N$ of $\overline G = G / N$.
    This bijection has the following properties: for all $A,B \leq G$ with $N \leq A$ and $N \leq B$,
    \begin{enumerate}[label=(\alph*),topsep=0pt]
        \item $A \leq B$ if and only if $\overline A \leq \overline B$,
        \item if $A \leq B$ then $|B : A| = |\overline B : \overline A|$,
        \item $\overline{\left< A,B \right>} = \left< \overline A, \overline B \right>$,
        \item $\overline{A \cap B} = \overline A \cap \overline B$, and
        \item $A \trianglelefteq G$ if and only if $\overline A \trianglelefteq \overline G$.
    \end{enumerate}
\end{theorem}

%<3
% QUOTES
% "Let's get your mathematical spleen going here."
% "Sylow's theorem-- *sledgehammer noise*"
% "Every time I hear `coset' I think of Cosette from Les Mis"
% "If this were a hike you'd still be able to see the parking lot. And we're about to turn a corner, so you need to rest up."
% "So wouldn't it be cool, and the answer is yes, ..."

\end{document}
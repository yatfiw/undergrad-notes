\documentclass[../p052main.tex]{subfiles}
\graphicspath{{\subfix{../figures/}}}

\begin{document}

\chapter{Wave Mechanics}
\section{Introduction}
Our exploration of quantum mechanics began after a variety of experiments suggested that light behaves as both a wave and a particle.
The particles that compose light are called photons.
Each photon has energy $E = h\nu$, where $\nu$ is the photon's frequency and $h$ is Planck's constant.
The inherently probabilistic nature of the photon is described by complex numbers called probability amplitudes.

Using these probability amplitudes, we were able to describe the results of the double-slit experiment.
Suppose light with wavenumber $k = \frac{2\pi}{\lambda}$ passes through two infinitesimal slits spaced a distance $d$ apart; the probability amplitude of detecting a photon an angle $\theta$ above the slits is given by
\[ z = re^{ikd_1} \left( 1 + e^{ikd \sin \theta} \right), \]
where $d_1$ is the total distance traveled by light coming from the top slit and $r$ is the probability amplitude for light diffracting in the correct direction.
This gives the probability of detection
\[ z^*z = 4r^2 \cos^2 \left( \frac{kd \sin \theta}{2} \right). \]
Using a similar line of reasoning (and some calculus), for a single slit of width $a$, the detection probability is
\[ z^*z = r^2 \frac{\sin^2 \alpha}{\alpha^2}, \]
where $\alpha = \frac{1}{2}ka\sin\theta$.

Shockingly, both of these results hold for massive particles!
In order to describe the interference that leads to these equations, though, we need a wavelength.
Taking inspiration from the fact that an individual photon has energy $E = h\nu = pc$ and thus wavelength $\lambda = h/p$, we can define a particle's de Broglie wavelength as
\[ \lambda_\text{dB} = \frac{h}{p} = \frac{h}{mv}. \]
Mathematically, this wavelength correctly predicts the way in which matter behaves in the double-slit experiment and others.
This suggests that the wave-particle duality is not just true for light, but also for matter.

An immediate application of this revelation comes in the form of crystal diffraction.
In an ideal crystal, there are many very thin layers of atoms that are spaced a very small distance $d$ apart.
Each layer of atoms acts as a mirror that reflects some incident atoms and reflects others.

If a stream of, say, electrons is incident on a two-layer crystal at an angle $\theta$ above the horizontal, then some electrons are reflected off of the top layer and some off of the bottom layer.
The bottom electrons travel an extra $2d\sin\theta$, so in order for the top and bottom electrons to leave the crystal in phase with each other, this number must be an integer number of wavelengths.
Mathematically,
\[ 2d \sin\theta = n \lambda. \]
This is called the Bragg relation.

\section{The Schrödinger Equation}%<3
We've seen that light (i.e., an electroamgnetic field) obeys the wave equation
\[ \frac{\partial^2 \mathcal{E}_z}{\partial x^2} = \frac{1}{c^2} \frac{\partial^2 \mathcal{E}_z}{\partial t^2}, \]
where $\mathcal{E}_z$ is the $z$-component of the electric field.
Solutions to this equation include linear combinations of the oscillating functions $\cos (kx \pm \omega t)$ and $\sin (kx \pm \omega t)$, where $k = \frac{2\pi}{\lambda}$ and $\omega = \frac{2\pi}{T}$ are the light's spatial and temporal angular frequencies, respectively.
We can't just pick any values of $k$ and $\omega$, however; specifically, the quantities must satisfy the equation
\[ \omega = ck. \]
This is called the dispersion relation for the photon wave equation.
If we can find such a relation for matter waves, it would be very useful in finding a matter wave equation.

Recall from quantum optics that, for light, $E = h\nu$ and $p = \frac{h}{\lambda}$.
Taking inspiration from de Broglie (whose hypothesis we know to be sound), suppose that these relations also hold for massive particles.
If we define the reduced Planck constant $\hbar = \frac{h}{2\pi}$, we can write
\[ E = \hbar \omega \,\text{ and }\, p = \hbar k. \]
We relate the (kinetic) energy and momentum of the particle via the equation
\[ E = \frac{mv^2}{2} = \frac{p^2}{2m}; \]
substituting our de Broglie relations gives the dispersion relation
\[ \omega = \frac{\hbar k^2}{2m}. \]
Any matter wave equation we construct must have oscillatory solutions that satisfy this equation.
As it turns out, the correct equation is the Schrödinger equation,
\[ -\frac{\hbar^2}{2m} \frac{\partial^2 \Psi(x,t)}{\partial x^2} + V(x) \Psi(x,t) = i \hbar \frac{\partial \Psi(x,t)}{\partial t}, \]
where $V(x)$ is the particle's potential energy (not potential!).
$\Psi(x,t)$ is called the wave function, and it encodes the wave property of the particle.

For a free particle ($V = 0$),
\[ -\frac{\hbar^2}{2m} \frac{\partial^2 \Psi(x,t)}{\partial x^2} = i \hbar \frac{\partial \Psi(x,t)}{\partial t}. \]
Something like $\Psi(x,t) = \cos (kx - \omega t)$ doesn't quite work as a solution since it doesn't have any imaginary parts to it.
There's no way it satisfies our dispersion relation.
However, a complex exponential
\[ \Psi(x,t) = e^{i(kx - \omega t)} \]
works just fine!
But this means solutions to the Schrödinger equation are irreducibly complex, so there's no immediate physical interpretation for them.
So what do they represent?

\section{Wave Functions}
Recall that the behavior of photons is characterized by complex probability amplitudes.
When we take the magnitude of an amplitude, we get real probabilities.

Wave functions are probability amplitude \textit{density} functions.
So $|\Psi (x,t)|^2$ is a probability density function, that is, a probability per unit length.
Specifically, $|\Psi(x,t)|^2$ gives the probability of measuring a particle in $[x, x + dx]$.
Mathematically, we have the Born rule,
\[ dP(x,t) = |\Psi(x,t)|^2 dx. \]
It follows that the units of $\Psi(x,t)$ are $L^{-1/2}$.

So we can calculate the probability of finding a particle in an interval of space using the integral
\[ P(a \leq x \leq b, t) = \int_a^b |\Psi(x,t)|^2dx. \]
Also, we must find the particle \textit{somewhere}, so
\[ 1 = \int_{-\infty}^\infty |\Psi(x,t)|^2dx. \]
This means the wave function goes to zero in the infinite limits.
(This is not a consequence of the Schrödinger equation, but rather the physical interpretation of the wave function.)

\begin{example}[Wave function normalization]
    Suppose we want to normalize the wave function
    \[ \psi(x) = \begin{cases} Nx(L-x) & 0 < x < L \\ 0 & \text{elsewhere}. \end{cases} \]
    We know that
    \begin{align*}
        \int_{-\infty}^\infty |\Psi(x,t)|^2dx = 1 \\
        \int_0^L |N|^2 x^2 (L^2 - 2xL + x^2)dx = 1 \\
        |N|^2 L^5 \cdot \frac{1}{30}
    \end{align*}
    Therefore, $N = \sqrt{\frac{30}{L^5}} e^{i\theta}$, where $\theta \in \mathbb{R}$ is an arbitrary phase.
\end{example}

We can use the Schrödinger equation to show that probability is conserved both locally and globally.
Let's start by finding the time derivative of probability density:
\[ \frac{\partial |\Psi|^2}{\partial t} = \frac{\partial (\Psi^*\Psi)^2}{\partial t} = \Psi^* \frac{\partial \Psi}{\partial t} + \Psi \frac{\partial \Psi^*}{\partial t}. \]
From the Schrödinger equation we get
\begin{align*}
    \frac{\partial \Psi}{\partial t} &= \frac{1}{i\hbar} \left( -\frac{\hbar^2}{2m}\frac{\partial^2 \Psi}{\partial x^2} + V(x) \Psi \right), \\
    \frac{\partial \Psi^*}{\partial t} &= \frac{-1}{i\hbar} \left( -\frac{\hbar^2}{2m}\frac{\partial^2 \Psi^*}{\partial x^2} + V(x) \Psi^* \right).
\end{align*}
Substituting gives
\begin{align*}
    \frac{\partial |\Psi|^2}{\partial t} &= \frac{i\hbar}{2m} \left( \Psi^* \frac{\partial^2 \Psi}{\partial x^2} - \Psi \frac{\partial^2 \Psi^*}{\partial x^2} \right) \\
    &= \frac{\partial}{\partial x} \left[ \frac{i\hbar}{2m} \left( \Psi^* \frac{\partial \Psi}{\partial x} - \Psi \frac{\partial \Psi^*}{\partial x} \right) \right]. \\
    \intertext{When we rewrite this as}
    &= -\frac{\partial}{\partial x} \left[ \frac{\hbar}{2mi} \left( \Psi^* \frac{\partial \Psi}{\partial x} - \Psi \frac{\partial \Psi^*}{\partial x} \right) \right],
\end{align*}
it becomes more clear that we've just written an equation describing a local conservation of probability!
Recall the local conservation of charge equation, $d\rho/dt = -\nabla \cdot \mbf{J}$, where $\rho$ is charge density and $\mbf{J}$ is current density.
In a similar fashion, we can define a probability current
\[ j_x(x,t) = \frac{\hbar}{2mi} \left( \Psi^* \frac{\partial \Psi}{\partial x} - \Psi \frac{\partial \Psi^*}{\partial x} \right) \]
that describes the flow of probability throughout space.
So any change in probability density at a point in space is matched with an inward flow of probability.

We can use this to show that probability is conserved globally:
\[ \frac{d}{dt} \int_{-\infty}^{\infty} |\Psi(x,t)|^2 dx = -\int_{-\infty}^{\infty} \frac{\partial j_x}{\partial x}dx = -j_x \Big|_{-\infty}^\infty = 0. \]
(The last step follows from the fact that $\Psi(x,t) \to 0$ as $x \to \pm \infty$.)

\section{Physical Wave Solutions}
We've seen that $\Psi(x,t) = Ae^{i(kx - \omega t)}$ is a solution to the Schrödinger equation.
However, it's easy to show that this wave function cannot be normalized!
So this solution doesn't align with our physical interpretation of $\Psi$ very well.

We can exploit the fact that the Schrödinger equation is linear to write a solution of the form
\[ \Psi(x,t) = \sum_{n}^{} A_n \sin (k_n x - \omega t). \]
But this still doesn't work, because for any finite number of terms we'll still have some overall periodic behavior which does not converge.
So instead, we need an infinite number of terms, which we can express using an integral (for now at a snapshot in time):
\[ \Psi(x,0) = \int_{-\infty}^{\infty} A(k) e^{i(kx - 0)}dk. \]
This integral produces a wave packet, a localized ``bump'' of probability.

\begin{example}[Wave packets]
    Suppose we want to find $\Psi(x,0)$ for
    \[ A(k) = \begin{cases} A & k_0 - \frac{\Delta k}{2} < k < k_0 + \frac{\Delta k}{2}, \\ 0 & \text{elsewhere}. \end{cases} \]
    We simply integrate:
    \begin{align*}
        \Psi(x,0) &= \int_{k_0 - \frac{\Delta k}{2}}^{k_0 + \frac{\Delta k}{2}} Ae^{ikx}dk \\
        &= \frac{A}{ix} \left( e^{i(k_0 + \frac{\Delta k}{2})x} - e^{i(k_0 - \frac{\Delta k}{2})x} \right) \\
        &= \frac{2Ae^{ik_0x}}{x} \sin\left( \frac{\Delta k x}{2} \right)
    \end{align*}
    We can show that this results in a probability density function that converges:
    \[ |\Psi|^2 = \frac{4|A|^2}{x^2} \sin^2 \left( \frac{\Delta k}{2}x \right). \]
\end{example}

Notice that there is an inverse relationship between the width $\Delta x$ of the wave packet and the range $\Delta k$ of wavenumbers we're integrating over.
It can be shown that, in general, this relationship is
\[ \Delta x \Delta k \geq \frac{1}{2}. \]
In quantum mechanics we have $p = \hbar k$, so we get the Heisenberg uncertainty principle
\[ \Delta x \Delta p_x \geq \frac{\hbar}{2}. \]
Here we can interpret $\Delta x$ and $\Delta p$ as uncertainties in a particle's position and momentum, respectively, when we take a measurement.

Let's allow time to move forward again.
We'd like for the speed of $\Psi(x,t)$ to be the same as the speed of the particle it represents.
We have two options: the velocity of the individual wavelengths (the phase velocity) and the velocity of the envelope enclosing them (the group velocity).
These are given by
\[ v_\textrm{p}= \frac{\omega}{k} \;\text{ and }\; v_\textrm{g} = \lim_{\Delta \to 0} \frac{\Delta \omega}{\Delta k} = \frac{d\omega}{dk}, \]
respectively.
(Recall how, when we superpose two waves, the resulting envelope has wavenumber $\Delta k$ and frequence $\Delta \omega$; to find the group velocity, we take the speed of the envelope enclosing the superposition two very closely-spaced wavelengths.)
The phase velocity doesn't work because, applying the dispersion relation,
\[ v_\textrm{p} = \frac{\omega}{k} = \frac{\hbar k}{2m} = \frac{p}{2m} = \frac{1}{2}v. \]
However, the group velocity gives
\[ v_\textrm{g} = \frac{d\omega}{dk} = \frac{\hbar k}{m} = \frac{p}{m} = v, \]
so this is the velocity we seek!

\section{Quantum Averages and the Classical Limit}
Recall how, for discrete variables, we define the expectation value
\[ \left< n \right> = \sum_{n=0}^{\infty} nP(n) \]
and uncertainty (standard deviation)
\begin{align*}
    (\Delta n)^2 &= \sum_{n=0}^{\infty} (n - \left< n \right>)^2P(n). \\
    \intertext{We can do some simplification to make this a bit less unwieldy:}
    &= \sum_{n=0}^{\infty} P(n) n^2 - \sum_{n=0}^{\infty} P(n) 2n\left< n \right> + \sum_{n=0}^{\infty}P(n) \left< n \right>^2 \\
    &= \left< n^2 \right> - 2\left< n \right>^2 + \left< n \right>^2 \\
    (\Delta n)^2 &= \left< n^2 \right> - \left< n \right>^2.
\end{align*}
These definitions and results generalize nicely to the continuous case:
\[ \left< x^{\alpha} \right> = \int_{-\infty}^{\infty} x^{a} |\Psi(x,t)|^2 dx, \qquad (\Delta x)^2 = \left< x^2 \right> - \left< x \right>^2. \]

\begin{example}[Expectation value and uncertainty]
    Determine $\left< x \right>$, $\Delta x$ for
    \[ \Psi(x) = \begin{cases} \sqrt{\frac{30}{L^{5}}} x (L - x) & 0 < x < L, \\ 0 & \text{elsewhere}. \end{cases} \]
    By symmetry, we can see that $\left< x \right> = L/2$.
    To calculate $\Delta x$, we also need to know $\left< x^2 \right>$.
    Integrating:
    \begin{align*}
        \left< x^2 \right> &= \int_{-\infty}^{\infty} x^2 |\Psi|^2dx \\
        &= \int_{0}^{L}x^2 \frac{30}{L^{5}} x^2 (L - x)^2 dx \\
        &= \frac{2}{7}L^2
    \end{align*}
    Finally, we can calculate the square of the uncertainty
    \[ (\Delta x)^2 = \left< x^2 \right> - \left< x \right>^2 = \frac{1}{28}L^2. \]
    This gives $\Delta x = \frac{1}{\sqrt{28}}L$.
\end{example}

We've seen how microscopic objects like photons and atoms obey the principles of quantum mechanics, but for this to be a truly accurate theory it must also apply to macroscopic objects in some limit.
This is called the principle of correspondence.

For large systems, the average position and momentum should obey the classical relationship between momentum and velocity; that is, we expect to see that
\[ \left< p_x \right> = m \frac{d \left< x \right>}{dt}. \]
This allows us to find an expression for average momentum!
But first, we must integrate to determine $\frac{d \left< x \right>}{dt}$.
\begin{align*}
    \frac{d\left< x \right>}{dt} &= \frac{d}{dt} \int_{-\infty}^{\infty} x |\Psi|^2 dx \\
    &= \int_{-\infty}^{\infty} x \frac{\partial |\Psi|^2}{\partial t}dx \\
    \intertext{By conservation of probability:}
    &= -\int_{-\infty}^{\infty} x \frac{\partial j_x}{\partial x} dx \\
    \intertext{Proceeding with integration by parts:}
    &= -\left( x j_x \Big|_{-\infty}^\infty - \int_{-\infty}^{\infty} j_x \,dx \right) \\
    \intertext{Since $j_x$ goes to zero in the infinite limits, that first term disappears. Substituting the probability current gives}
    &= \frac{\hbar}{2mi} \int_{-\infty}^{\infty} \left( \Psi^* \frac{\partial \Psi}{\partial x} - \Psi \frac{\partial \Psi^*}{\partial x} \right) dx \\
    &= \frac{\hbar}{2mi} \left( \int_{-\infty}^{\infty} \Psi^* \frac{\partial \Psi}{\partial x} dx - \int_{-\infty}^{\infty} \Psi \frac{\partial \Psi^*}{\partial x} dx \right) \\
    \intertext{Proceeding again with integration by parts on the second term::}
    &= \frac{\hbar}{2mi} \left[ \int_{-\infty}^{\infty} \Psi^* \frac{\partial \Psi}{\partial x} dx - \left( \Psi \Psi^* \Big|_{-\infty}^\infty - \int_{-\infty}^{\infty} \frac{\partial \Psi}{\partial x} \Psi^* dx \right) \right] \\
    &= \frac{\hbar}{mi} \int_{-\infty}^\infty \Psi^* \frac{\partial \Psi}{\partial x} dx
\end{align*}
Multiplying by $m$ gives us an expression for $\left< p_x \right>$!
We could follow a similar line of reasoning for $\left< p_x^2 \right>$, but doing it here is unproductive so we just state
\[ \left< p_x \right> = \int_{-\infty}^\infty \Psi^* \frac{\hbar}{i} \frac{\partial \Psi}{\partial x} dx, \qquad \left< p_x^2 \right> = -\int_{-\infty}^\infty \Psi^* \hbar^2 \frac{\partial^2 \Psi}{\partial x^2}. \]

\end{document}
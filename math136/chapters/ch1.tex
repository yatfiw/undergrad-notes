\documentclass[../m136main.tex]{subfiles}
\graphicspath{{\subfix{../figures/}}}

\begin{document}

\chapter{Complex Numbers}
We'll start with the absolute basics of complex numbers.

\begin{definition}[Complex number]
    A complex number is an expression of the form $a + bi$, with $a,b \in \R$ and $i^2 = -1$; the set of complex numbers is denoted $\C$.
    The real part $\Re(z) = a$ and the imaginary part $\Im(z) = b$.
\end{definition}

Adding and subtracting complex numbers looks exactly how we'd expect---if $z = a + bi$ and $w = c + di$, then $z \pm w = (a + c) + i (b + d)$.
Multiplication and division look like
\[ zw = (ac - ba) + i(ad + bc), \qquad \frac{z}{w} = \left( \frac{ac + bd}{c^2 + d^2} \right) + i \left( \frac{bc - ad}{c^2 + d^2} \right), \]
where we require $w \neq 0$ for division.
To get a more geometric sense for these, we make a couple more definitions motivated by framing complex numbers as vectors in the real-imaginary plane.

\begin{definition}[Norm and argument]
    Let the number $z = a + bi$ make an angle $\theta$ with the positive-real axis.
    The norm (or magnitude, modulus) and argument of $z$ are, respectively,
    \[ |z| = \sqrt{a^2+ b^2}, \arg(z) = \left\{ \theta + 2\pi k \mid k \in \Z \right\}, \]
    where $\arg(0)$ is undefined.
    The principal argument $\Arg(z)$ is the unique value of $\arg(z)$ in $(-\pi, \pi]$.
\end{definition}

We can therefore express complex numbers in polar form via
\[ z = r \cos \theta + i r \sin \theta = r \cis \theta, \]
and we could use some trigonometric identities to show that multiplication should be interpreted as multiplying magnitudes and adding arguments.
(Note that arguments add, but Arguments might not.)
We can thus state de Moivre's formula
\[ (\cos \theta + i \sin \theta)^{n} = \cos (n \theta) + i \sin(n \theta), \quad n \in \N, \]
which has a straightforward proof by induction.
This also allows us to reason about complex roots!
Suppose $z = r \cis \theta$; we seek $w = \rho \cis \phi$ such that $w^{n} = z$ for some $n \in \N$, that is,
\[ \rho^{n} \cis n\phi = r \cis \theta. \]
We can see that $\rho = r^{1 / n}$, the unique positive root of $r > 0$, and $\phi = (\theta + 2\pi k) / n = (\theta / n) + (2\pi / n)k$ for $k \in \Z$.
These roots are distinct for $k \in \left\{ 0, \, 1, \, \ldots, \, n-1 \right\}$, and together they define our $n$th roots!
(Note that an $n$th root is called primitive if there is no $k \in \left\{ 1, \, \ldots, \, n-1 \right\}$ for which $w^{k} = 1$.)

Now we'll introduce a new, distinctly complex feature of this number system.

\begin{definition}[Complex conjugate]
    The complex conjugate of $z = a + bi$ is
    \[ \overline z = a - bi. \]
\end{definition}

It is clear that $\overline{z \pm w} = \overline z \pm \overline w$ and $z \overline z = |z|^2$, and if we interpret conjugation as reflection over the real axis, we get
\[ |\overline z| = |z|, \quad \overline{zw} = \overline z \overline w, \quad \overline{\left( \frac{z}{w} \right)} = \frac{\overline z}{\overline w}, \quad \overline{\bar z} = z. \]
We also have a few handy identities:
\[ \Re(z) = \frac{z + \overline z}{2}, \qquad \Im(z) = \frac{z - \overline z}{2i}, \qquad \frac{z}{w} = \frac{z \overline w}{|w|^2}. \]
Finally, we can use the notion of conjugation to arrive at a familiar result!

\begin{theorem}[Triangle inequality]
    For any $z,w \in \C$,
    \[ |z + w| \leq |z| + |w|. \]
\end{theorem}

\begin{proof}
    We could prove this by interpreting complex numbers as vectors, but alternatively, we can simply write
    \begin{align*}
        |z + w| &= (z + w) \cdot \overline{z + w} \\
        &= z \overline z + z \overline w + w \overline z + w \overline w \\
        &= |z|^2 + |w|^2 + 2 \Re(z \overline w). \\
        \intertext{By the triangle inequality in $\R$,}
        &\leq |z|^2 + |w|^2 + |z \overline w| \\
        &= (|z| + |w|)^2,
    \end{align*}
    as desired.
\end{proof}

We'll end by constructing an interesting model of the complex plane.
Let $\Sigma$ be a diameter-1 sphere whose south pole is at the origin.
If the $xy$-plane is identified with $\C$ and the $u$-axis is vertical, the sphere is then described by
\[ x^2 + y^2 + \left( u - \frac{1}{2} \right)^2 = \frac{1}{4}. \]
Now, for each $z \in \C$ we draw a line segment to the north pole $N$ and notice that the segment intersects with the sphere at exactly one other point.
This defines a bijection from $C \to \Sigma \setminus \left\{ N \right\}$, and to map to all of $\Sigma$ we add a point $\infty$ defined by $\sigma \leftrightarrow N$.
We therefore define the extended complex plane $\hat \C = \C \cup \left\{ \infty \right\}$, and we have $\hat \C \simeq \Sigma$.

To make this bijection explicit we first parametrize the line via
\[ \ell(t) = (x,y,0) + t (-x, -y, 1), \quad t \in [0,1] \]
and notice that it intersects with the sphere when
\[ x^2 (1-t)^2 + y^2(1-t)^2 + \left( t - \frac{1}{2} \right)^2 = \frac{1}{4}. \]
This equation has roots
\[ t = 1, \qquad t = \frac{x^2 + y^2}{1 + x^2 + y^2}, \]
and so the nontrivial intersection is at
\[ \left( \frac{x}{1 + x^2 + y^2}, \; \frac{y}{1 + x^2 + y^2}, \; \frac{x^2 + y^2}{1 + x^2 + y^2} \right) \in \Sigma. \]
This characterization of $\Sigma$ is called the Riemann sphere.
(In topological terms, we call $\hat \C$ a one-point compactification of $\C$.)

\end{document}
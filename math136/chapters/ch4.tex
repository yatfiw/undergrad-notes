\documentclass[../m136main.tex]{subfiles}
\graphicspath{{\subfix{../figures/}}}

\begin{document}

\chapter{Series Representations}
\section{Taylor Series}
We'll begin our discussion of series representations of functions with a few familiar definitions.

\begin{definition}[Sequence and series]
    Let $\left\{ a_n \right\}_{n=0}^\infty$ be a sequence of complex numbers.
    We say $a_n$ converges to $L$ if for every $\varepsilon > 0$ there exists an $N \in \N$ such that
    \[ m > N \;\implies\; |a_m - L| < \varepsilon. \]
    In this case we say $a_n \to L$ and $\lim_{n \to \infty} a_n = L$.
    We also say the series
    \[ \sum_{n=0}^{\infty} a_n = L \;\text{ if }\; \lim_{N \to \infty} \sum_{n=0}^{N} a_n = L. \]
\end{definition}

\begin{definition}[Uniform convergence]
    The series $\sum_{k=0}^{\infty} f_k(z)$ converges uniformly to $f(z)$ on a set $K$ if for every $\varepsilon > 0$ there exists an $n \in N$ such that
    \[ m \geq n \;\implies\; \left| f(z) - \sum_{k=0}^{m} f_k(z) \right| < \varepsilon. \]
\end{definition}

An important example of a series that converges is the geometric series.
Let $z \in \C$ with $|z| < 1$; then
\[ \sum_{k=0}^{\infty} z^{k} = \frac{1}{1-z}. \]
This will be useful in our analysis of Taylor series, the key (patently complex) result about which is below.

\begin{theorem}[Taylor series]
    Let $f$ be analytic on a domain $\Omega$.
    If $z_0 \in \Omega$ and $B(z_0, R) \subset \Omega$, then for all $z$ interior to $B(z_0, R)$
    \[ f(z) = \sum_{k=0}^{\infty} \frac{f^{(k)}(z_0)}{k!} (z - z_0)^{k}. \]
\end{theorem}

\begin{proof}
    We'll prove the statement for $z_0 = 0$, noticing that we can generalize our result to any $z_0$ by translation.
    (The chain rule doesn't introduce anything new into the picture.)

    Suppose $0 \in \Omega$ and $B(0,R) \subset \Omega$, and let $C$ be a circle defined by $|z| = R_0 < R$.
    For $z$ inside $C$,
    \begin{align*}
        f(z) &= \frac{1}{2\pi i} \oint_C \frac{f(s)}{s-z} \,ds \\
        &= \frac{1}{2\pi i} \oint_C \frac{f(s)}{s} \left[ \frac{1}{1 - z / s} \right] ds \\
        &= \frac{1}{2\pi i} \oint_C \frac{f(s)}{s} \left[ \sum_{k=0}^{N-1} \left( \frac{z}{s} \right)^{k} + \frac{(z / s)^{N}}{1 - z / s} \right], \\
        \intertext{since $|z| < |s|$. This effectively allows us to ``swap'' the order of the integral and the sum:}
        &= \sum_{k=0}^{N-1} \left( \frac{1}{2\pi i} \oint_C \frac{f(s)}{s^{k+1}} \,ds \right) z^{k} + \frac{z^{N}}{2\pi i} \oint_C \frac{f(s)}{s^{N} (s-z)} \,ds \\
        &= \sum_{k=0}^{N-1} \frac{f^{(k)}(0)}{k!} z^{k} + \rho_N(z),
    \end{align*}
    where we've defined the last term to be $\rho_N(z)$.
    We need only show that $|\rho_N(z)| \to 0$.
    Let $M = \max_{s \in C} |f(s)| < \infty$, which we know exists because $f$ is continuous.
    Then by the ML lemma
    \begin{align*}
        |\rho_N(z)| &\leq \frac{|z|^{N}}{2\pi} \frac{M}{R_0^{N} (R_0 - |z|)} \cdot 2\pi R_0 = \frac{MR_0}{R_0 - |z|} \left( \frac{|z|}{R} \right)^{N},
    \end{align*}
    and since $N$ can be arbitrarily large, $|\rho_N(z)| < \varepsilon$.
    Thus the Taylor series of $f$ converges uniformly to $f$ for all closed ``sub-disks'' of $B(z_0, R)$, meaning the convergence is uniform on $B(z_0, R)$.
\end{proof}

Note that this theorem says nothing about convergence \textit{on} the disk.
Also note that the Taylor series of, say, $f'$ is equivalent to term-by-term derivative of the Taylor series for $f$.
Finally, we can generate new Taylor series from known ones using the Cauchy product terms
\[ \frac{(fg)^{(k)}(z_0)}{k!} = \frac{1}{k!} \sum_{\ell = 0}^{k} \binom{k}{\ell} f^{(k - \ell)}(z_0) g^{(\ell)}(z_0) = \sum_{\ell = 0}^{k} \frac{f^{(k - \ell)}(z_0)}{(k - \ell)!} \frac{g^{(\ell)}(z_0)}{\ell !} = \sum_{\ell = 0}^{k} a_{k - \ell} b_\ell = c_k, \]
where $a_k$ and $b_k$ are the coefficients of two Taylor series and $c_k$ are the coefficients of the product.

\section{Power Series}
Now we'll discuss the convergence (and divergence) of power series more generally, starting with the following.

\begin{lemma}[Convergence lemma]    % attributed to Abel-Weierstrass
    Given $R_0 > 0$, if there exists an $M > 0$ such that
    \[ |a_k| \leq \frac{M}{R_0^{k}} \vspace{-6pt} \]
    for all $k$, then $\ds \sum_{k=0}^{\infty} a_k (z - z_0)^{k}$ converges absolutely and uniformly for all $z$ satisfying $|z - z_0| < R_0$.
\end{lemma}

\begin{proof}
    Let $z \in \C$ such that $|z - z_0| \leq r_\star < R_0$ then for all $k$,
    \[ \sum_{k=0}^{\infty} |a_k (z - z_0)^{k}| \leq \sum_{k=0}^{\infty} M \left( \frac{r_\star}{R_0} \right)^{k} = \frac{1}{M (1 - r_\star / R_0)}, \]
    which is bounded because $|r_\star / R_0| < 1$.
\end{proof}

% Explaining the alternating series test:
    % "We start with nothing... like we all do in this ephemeral life..." - Jakes

This leads to a couple of nice corollaries---roughly speaking, if a series converges at a point then it also converges everywhere ``interior'' to that point, and if it diverges at a point then it diverges everywhere ``exterior'' to that point.

\pagebreak

\begin{corollary}[]
    \vspace{-4pt} If $\,\ds \sum_{k=0}^{\infty} a_k (z - z_0)^{k}$ converges at $z_1 \neq z_0$, then it converges uniformly and absolutely for all  \vspace{-6pt}
    \[ |z - z_0| < R_0 = |z_1 - z_0|. \]
\end{corollary}

\begin{proof}
    Let $R_0 = |z_1 - z_0|$ and pick an $r_\star < R_0$.
    Since $\sum_{k=0}^{\infty} a_k (z_1 - z_0)^{k}$ converges, the terms $a_k (z_1 - z_0)^{k} \to 0$, so given $\varepsilon > 0$ there is an $N$ such that $m \geq N$ implies
    \[ |a_m (z_1 - z_0)^{m}| = |a_m| R_0^{m} < \varepsilon. \]
    With this in mind, let $M = \max \left\{ |a_0|, \, |a_1| R_0, \, |a_2| R_0^2, \, \ldots, \, |a_{N-1}| R_0^{N-1}, \, \varepsilon \right\}$.
    It follows from the existence of such an $M$ that $|a_k| \leq M / R_0^{k}$ for all $k$, and by the previous lemma $\sum_{k=0}^{\infty} a_k (z - z_0)^{k}$ converges absolutely and uniformly.
\end{proof}

\begin{corollary}[]
    \vspace{-4pt} If $\,\ds \sum_{k=0}^{\infty} a_k (z - z_0)^{k}$ diverges at $z_1 \neq z_0$, then it also diverges for all $z$ satisfying   \vspace{-6pt}
    \[ |z - z_0| > |z_1 - z_0|. \]
\end{corollary}

Now, given $z_0$ we define the radius of convergence
\[ R = \sup \left( |z - z_0| \;\middle|\; \sum_{k=0}^{\infty} a_k(z - z_0)^k \;\text{ converges at $z$} \right). \]
The circle $|z - z_0| = R$ is called the circle of convergence.
Everything inside converges, but on this circle the behavior is not definite.
We can say a couple of things, though:
\begin{itemize}[topsep=0pt]
    \item if a point on the circle converges absolutely then everything on the circle converges absolutely, and consequentially
    \item if a point on the circle diverges then nothing on the circle converges absolutely.
\end{itemize}
When determining convergence we generally use the ratio test, though sometimes we also resort to the root test.
Now we'll look at a series of lemmas that together reveal a deep relationship between power series and analytic functions.
(The first is a standard result from analysis---see Math 131 for proof.)

\begin{lemma}[]
    If $f_k \to f$ uniformly on a set $K$ then $f$ is continuous on $K$.
\end{lemma}

\begin{lemma}[]
    Suppose $f_k$ are continuous on a set $K$ and $\Gamma$ is a contour in $K$.
    If $f_k \to f$ uniformly on $K$ then
    \[ \int_\Gamma f_k(z) \,dz \longrightarrow \int_\Gamma f(z) \,dz. \]
\end{lemma}

\begin{proof}
    Given $\varepsilon > 0$ there is an $N \in \N$ such that $m \geq N$ implies $|f_m(z) - f(z)| < \varepsilon / \ell$ for $z \in K$, where $\ell$ is the length of $\Gamma$.
    Then
    \[ \left| \int_\Gamma f_m(z) \,dz - \int_\Gamma f(z) \,dz \right| \leq \ell \cdot \max_{z \in \Gamma} |f_m(z) - f(z)| < \ell \cdot \frac{\varepsilon}{\ell}, \]
    as desired.
\end{proof}

\begin{lemma}[]
    Suppose $f_k \to f$ uniformly on a simply connected domain $\Omega$.
    If the $f_k$ are analytic on $\Omega$ then $f$ is analytic on $\Omega$.
\end{lemma}

\begin{proof}
    From the first lemma, $f$ is continuous.
    Since $\Omega$ is simply connected and $f_k$ analytic, $\oint_\Gamma f_k(z) \,dz = 0$ for any closed loop $\Gamma$ in $\Omega$.
    It follows from the previous lemma that $\oint_\Gamma f(z) \,dz = 0$, and by Morera's theorem $f$ is analytic.
\end{proof}

Now suppose
\[ S(z) = \sum_{k=0}^{\infty} a_k (z - z_0)^{k} \]
converges on the set $\Omega$ defined by $|z - z_0| < R$.
This final lemma shows that $S(z)$ is analytic on $\Omega$!
But then it is also equal to its Taylor series, which is a potentially different representation of $S(z)$.
To reconcile the two, we can do some partial-sum manipulation with the second lemma to get
\[ \sum_{k=0}^{\infty} \oint_\Gamma a_k(z - z_0)^k \,dz = \oint_\Gamma \left( \sum_{k=0}^{\infty} a_k (z - z_0)^k \right) dz. \]
But by Cauchy's integral formula, for some integral $C$ within the circle of convergence we also have
\begin{align*}
    S^{(k)}(z_0) &= \frac{k!}{2\pi i} \oint_C \frac{S(z)}{(z - z_0)^{k+1}} \,dz \\
    &= \frac{k!}{2\pi i} \oint_C \frac{1}{(z - z_0)^{k+1}} \left( \sum_{\ell = 0}^{\infty} a_\ell (z - z_0)^\ell \right) dz \\
    &= \frac{k!}{2\pi i} \sum_{\ell = 0}^{\infty} \oint_C a_\ell (z - z_0)^{\ell - k - 1} dz \\
    &= \frac{k!}{2\pi i} \cdot a_k \cdot 2\pi i.
\end{align*}
Thus $a_k = S^{(k)}(z_0) / k!$, meaning our power-series representations are the same!
So in $\C$ there is a precise equivalence between analytic functions (defined in terms of complex derivatives) and convergent power series.

As a side note, all this also gives us a new proof of Liouville's theorem.
If an analytic function $f(z)$ is bounded (by $M$) and entire then by Cauchy's estimate
\[ \frac{1}{k!} \left| f^{(k)}(z_0) \right| \leq \frac{M}{R^{k}}, \]
meaning the higher-degree coefficients of the Taylor series of $f$ go to zero as $R \to \infty$.
This leaves $f(z) = a_0$ as the only possibilities for $f$.

\section{Laurent Series}
Many of the functions we care about have singularities, points about which there is no faithful power series expansion.
This is not to say, however, that there's no way to represent these functions as series!
Take, for example, the function
\[ f(z) = \frac{1}{z(z-1)}. \]
We can use our knowledge of power series to expand $f(z)$ in the regions $|z| > 1$ and $0 < |z| < 1$, respectively:
\begin{align*}
    f(z) &= \frac{1}{z^2} \frac{1}{1 - 1 / z} = \frac{1}{z^2} \sum_{k=0}^{\infty} \left( \frac{1}{z} \right)^{k} = \frac{1}{z^2} + \frac{1}{z^3} + \frac{1}{z^{4}} + \cdots, \qquad |z| > 1, \\
    f(z) &= -\frac{1}{z} \frac{1}{1-z} = -\frac{1}{z} \sum_{k=0}^{\infty} z^{k} = -\frac{1}{z} - 1 - z - z^2 \cdots, \qquad 0 < |z| < 1.
\end{align*}
These are examples of Laurent series---they look like Taylor series, just with possibly negative powers.
Notice how we define these series on an annulus rather than a neighborhood, reflecting how the function is potentially ill-defined at the center of the expansion.

\begin{theorem}[Laurent series]
    If $f$ is analytic on an annular domain $A = \left\{ z \mid r < |z - z_0| < R \right\}$ then
    \[ f(z) = \sum_{k=-\infty}^{\infty} c_k(z - z_0)^{k}, \qquad c_k = \frac{1}{2\pi i} \oint_\Gamma \frac{f(s)}{(s - z_0)^{k+1}} \,ds, \qquad z \in A, \quad k \in \Z \]
    for any simple closed positively oriented curve $\Gamma$ about $z_0$ in $A$.
    The convergence is uniform on closed sub-annuli, and the series representation is unique.
\end{theorem}

\begin{proof}
    Let $z \in A$ with $r < r' \leq |z - z_0| \leq R' < R$ with the circles $C_{r'}, C_{R'}$.
    If $\Gamma$ is a small circle about $z$ then we can deform $\Gamma$ to get
    \[ f(z) = \frac{1}{2\pi i} \oint_\Gamma \frac{f(s)}{s-z} \,ds = \frac{1}{2\pi i} \oint_{C_{R'}} \frac{f(s)}{s-z} \,ds - \frac{1}{2\pi i} \oint_{C_{r'}} \frac{f(s)}{s-z} \,ds. \]
    Call these integrals $I_1$ and $I_2$, respectively.
    Noting that $|(z - z_0) / (s - z_0)| < 1$ for $s \in C_{R'}$, we could do some algebra and exploit uniform convergence to get
    \begin{align*}
        I_1 &= \frac{1}{2\pi i} \oint_{C_{R'}} \frac{f(s)}{s - z_0} \sum_{k=0}^{\infty} \left( \frac{z - z_0}{s - z_0} \right)^{k} ds \\
        &= \sum_{k=0}^{\infty} \left( \frac{1}{2\pi i} \oint_{C_{R'}} \frac{f(s)}{(s - z_0)^{k+1}} \,ds \right) (z - z_0)^{k}.
    \end{align*}
    This sum converges uniformly on $|z - z_0| \leq R' < R$.
    Now, $|(s - z_0) / (z - z_0)| < 1$ for $s \in C_{r'}$, so using similar methods     \vspace{-12pt}
    \begin{align*}
        I_2 &= \frac{1}{2\pi i} \oint_{C_{r'}} \frac{f(s)}{z - z_0} \sum_{k=0}^{\infty} \left( \frac{s - z_0}{z - z_0} \right)^{k} ds \\
        &= \sum_{k=0}^{\infty} \left( \frac{1}{2\pi i} \oint_{C_{r'}} \frac{f(s)}{(s - z_0)^{-k}} \,ds \right) (z - z_0)^{-k-1} \\
        &= \sum_{k=1}^{\infty} \left( \frac{1}{2\pi i} \oint_{C_{r'}} \frac{f(s)}{(s - z_0)^{-k+1}} \,ds \right) (z - z_0)^{-k}.
    \end{align*}
    This sum converges uniformly on $|z - z_0| \geq r' > r$.
    Thus if we define $c_k$ as in the theorem statement we get
    \[ f(z) = \sum_{k=0}^{\infty} c_k (z - z_0)^{k} + \sum_{k=1}^{\infty} \frac{c_{-k}}{(z - z_0)^{k}} = \sum_{k=-\infty}^{\infty} c_k (z - z_0)^{k}, \]
    which converges uniformly on $r' \leq |z - z_0| \leq R'$.
    We can deform $C_{R'}$ and $C_{r'}$ into any contour $\Gamma$ in $A$.
    To prove uniqueness we consider another Laurent series with coefficients $d_k$; in this case,
    \[ \oint_\Gamma \frac{f(s)}{(s - z_0)^{n+1}} \,ds = \sum_{k=-\infty}^{\infty} \oint_\Gamma d_k \frac{(s - z_0)^{k}}{(s - z_0)^{n+1}} \,ds = d_n \cdot 2\pi i, \]
    meaning
    \[ d_n = \frac{1}{2\pi i} \oint_\Gamma \frac{f(s)}{(s - z_0)^{n+1}} \,ds = c_n. \]
    The Laurent series is therefore unique.
\end{proof}

Note that if $f$ is analytic on the entire disk $|z - z_0| < R$, then by uniqueness $c_k = 0$ for negative $k$ and the remaining coefficients are the Taylor coefficients.
Typically, however, $f$ is not analytic at $z_0$, so the positive-indexed terms are not generally the Taylor coefficients (because the derivatives at $z_0$ do not exist).
Also, this analysis doesn't work on branch cuts, since in that case no sufficient annulus exists.

\section{Zeros and Singularities}
We'll begin by quickly classifying the zeroes of a function.

\begin{definition}[Classifying zeroes]
    A point $z_0$ is called a zero of order $m$ for the function $f$ if $f$ is analytic at $z_0$ and it, along with its first $m-1$ derivatives, vanish at $z_0$ but $f^{(m)}(0) \neq 0$.
    A zero of order 1 is called a simple zero.
\end{definition}

In this case the Taylor series for $f$ about $z_0$ looks like
\begin{align*}
    f(z) &= a_m (z - z_0)^{m} + a_{m+1} (z - z_0)^{m+1} + a_{m+2} (z - z_0)^{m+2} + \cdots \\
    &= (z - z_0)^{m} \left[ a_m + a_{m+1} (z - z_0) + a_{m+2} (z - z_0)^2 + \cdots \right], \qquad a_m \neq 0.
\end{align*}
The bracketed series defines an analytic function in a neighborhood of $z_0$, so we deduce the following.

\begin{theorem}[]
    Let $f$ be analytic at $z_0$.
    Then $z_0$ is a zero of order $m$ if and only if
    \[ f(z) = (z - z_0)^{m} g(z) \]
    for some $g$ that is analytic at $z_0$ and satisfies $g(z_0) \neq 0$.
\end{theorem}

By continuity there exists an $\varepsilon > 0$ for which $|z - z_0| < \varepsilon$ implies $g(z) \neq 0$.
Thus $z_0$ is the only zero of $f$ in this neighborhood, and in general the zeroes of an analytic function are isolated!

\begin{definition}[Classifying singularities]
    Let $f$ have the Laurent expansion
    \[ f(z) = \sum_{k=-\infty}^{\infty} a_k (z - z_0)^{k}. \]
    If $f$ is not analytic at $z_0$ but is analytic on some punctured disk $0 < |z - z_0| < \varepsilon$, then we say $f$ has an isolated singularity at $z_0$.
    This singularity is classified as follows.
    \begin{itemize}[topsep=0pt]
        \item If $a_k = 0$ for all $k < 0$ then $z_0$ is a removable singularity.
        
        \item If $a_{-m} \neq 0$ for some positive integer $m$ but $a_k = 0$ for all $j < -m$ then $z_0$ is a pole of order $m$.
        (If $m=1$ then we call $z_0$ a simple pole.)

        \item If $a_k \neq 0$ for an infinite number of negative $k$ then $z_0$ is an essential singularity of $f$.
    \end{itemize}
\end{definition}

Now we'll look a bit more closely at the behavior of a function $f$ at each of these kinds of singularities.
\begin{itemize}[topsep=0pt]
    \item If $f$ has a removable singularity at $z_0$ it has a Laurent series
    \[ f(z) = c_0 + c_1 (z - z_0) + c_2 (z  - z_0)^2 + \cdots, \qquad 0 < |z - z_0| < \varepsilon. \]
    Strictly speaking this series is defined a punctured neighborhood of $z_0$, but it's analytic at $z_0$, too!
    So we can simply ``remove'' the singularity at $z_0$ by defining $f(z_0) = c_0$.
    We can also see that $f$ is bounded on a punctured neighborhood around $z_0$, and the extended $f$ is analytic on some potentially larger neighborhood.

    \item If $f$ has a pole of order $N$ at $z_0$ then there exists an analytic $g$ satisfying $g(z_0) \neq 0$ such that
    \[ f(z) = \frac{g(z)}{(z - z_0)^{N}}. \]
    Limits approaching $z_0$ are now unbounded as $f$ is not bounded around $z_0$, but $f$ is analytic around $z_0$.

    \item If $f$ has an essential singularity at $z_0$ then near $z_0$ the output $f(z)$ can be made close to any point in the complex plane!
    This is formalized by the following.
\end{itemize}

\begin{theorem}[Casorati-Weirerstrass]
    If $f$ has an essential singularity at $z_0$, then for every $w \in \C$ there exists a sequence $\left\{ z_k \right\}$ such that $z_k \to z_0$ and $f(z_k) \to w$.
    (In other words, $f(B(z_0, \varepsilon) \setminus \left\{ z_0 \right\})$ is dense in $\C$.)
\end{theorem}

\begin{proof}
    If not, then there exist $\varepsilon > 0$ and $\delta > 0$ such that $|f(z) - w| > \varepsilon$ for $0 < |z - z_0| < \delta$.
    Then
    \[ R(z) = \frac{1}{f(z) - w} \]
    is analytic on the $\delta$-ball, and $|R(z)| < 1 / \varepsilon$ there.
    Also, if $R(z)$ has a Laurent expansion with coefficients $d_k$ then
    \[ |d_k| = \left| \frac{1}{2\pi i} \oint_\gamma \frac{R(s)}{(s - z_0)^{k+1}} \,ds \right| \leq \frac{1}{2\pi} \frac{1 / \varepsilon}{r^{k+1}} \cdot 2\pi r = \frac{r^{-k}}{\varepsilon}. \]
    for a radius-$r$ circle $\gamma$ around $z_0$.
    Thus as $r \to 0$ we have $|d_k| \to 0$ for $k < 0$, meaning $R(z)$ has a removable singularity at $z_0$.
    But then
    \[ f(z) = w + \frac{1}{R(z)}, \qquad 0 < |z - z_0| < \delta \]
    either has a pole of finite order at $z_0$ or $R(z_0) \neq 0$, both of which disallow $f$ from having an essential singularity there.
\end{proof}

A stronger statement, called the Picard theorem, shows that the image of $f$ around $z_0$ includes the entire complex plane barring at most one point.

Now let's bring our discussion into the extended complex plane, which includes a point at $\infty$.
If $f$ has a pole at $z_0$ then $\lim_{z \to z_0} |f(z)| = +\infty$, in which case we say that $f(z_0) = \infty$.
We can, however, also study the behavior of $f$ at $z = \infty$ by looking at how $\tilde f(w) = f(1 / w)$ behaves at $w=0$.
In particular, if $f$ is analytic in a neighborhood $|z| > R$ of $z=\infty$ then $\tilde f$ is analytic in a punctured neighborhood $0 < |w| < R$ of $w=0$, and so we can use the familiar classification of isolated singularities at $w=0$ to see what's going on at $z=\infty$.

If $f$ is entire then
\[ f(z) = a_0 + a_1 z + a_2 z^2 + \cdots, \qquad z \in \C \]
is valid in every neighborhood of $\infty$, meaning this is the Laurent series for $f$ at $\infty$!
We have three cases.
\begin{itemize}[topsep=0pt]
    \item If $f(z) = a_0$ then $f$ is analytic at $\infty$.
    \item If $f(z)$ is a degree-$N$ polynomial then $f$ has a pole of order $N$ at $\infty$.
    \item If $f(z)$ is an infinite power series then $f$ has an essential singularity at $\infty$.
\end{itemize}

\end{document}
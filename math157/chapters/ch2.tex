\documentclass[../m157main.tex]{subfiles}
\graphicspath{{\subfix{../figures/}}}

\begin{document}

\chapter{Multivariate Distributions}
\section{Joint Probability Functions}
So far we've been talking about distributions of individual variables.
Now we'll look at the slightly more general case of two variables.

\begin{definition}[Joint probability mass function]
    Let $X$ and $Y$ be discrete random variables.
    The function $P(X = x, \, Y = y)$ is a joint PMF if
    \begin{itemize}
        \item $P(X = x, \; Y = y) \geq 0$ for all $x,y$ and
        \item $\sum_{x,y} P(X = x, \, Y = y) = 1$.
    \end{itemize}
    The marginal PMF for $X$ is
    \[ P(X = x) = \sum_{y} P(X = x, \, Y = y); \]
    $X$ and $Y$ are independent if $P(X = x, \, Y = y) = P(X = x) P(Y = y)$ for all $x,y$.
\end{definition}

\begin{definition}[Joint probability density function]
    Let $X$ and $Y$ be continuous random variables.
    The function $f_{X,Y}(x,y)$ is a joint PDF if
    \begin{itemize}
        \item $f_{X,Y}(x,y) \geq 0$ for all $x,y$ and
        \item $\iint_{\R^2} f_{X,Y}(x,y) \,dxdy = 1$.
    \end{itemize}
    the marginal PDF for $X$ is
    \[ f_X(x) = \int_\R f_{X,Y}(x,y) \,dy; \]
    $X$ and $Y$ are independent if $f_{X,Y}(x,y) = f_X(x) f_Y(y)$ for all $x,y$.
\end{definition}

Also, note that knowing the marginals of two random variables is not necessarily enough to obtain their full joint PF.
To do this, we must also require that $X$ and $Y$ are independent.

\begin{definition}[Conditional probability distribution]
    Let $X$ and $Y$ be random variables.
    For discrete and continuous variables we have, respectively,
    \[ P(Y = y \,|\, X = x) = \frac{P(Y = y, \, X = x)}{P(X = x)}, \qquad f_{Y|X}(y \,|\, x) = \frac{f_{X,Y}(x,y)}{f_X(x)}. \]
\end{definition}

\section{Multivariate Distributions}
We can generalize the above discussion to probability functions of several random variables.
Rather than give a general treatment for the discrete case, though, we'll focus on one particularly important distribution.

\pagebreak

\begin{definition}[Multinomial distribution]
    If an experiment with $m$ outcomes (with probabilities $p_1, \ldots, p_m$) is independently performed $n$ times and $X_i$ is the number of times outcome $i$ occurs, then
    \[ P(X_1 = x_1, \, \ldots, \, X_m = x_m) = \binom{n}{x_1,\, \ldots,\, x_m} \; p_1^{x_1} \cdots p_m^{x_m}, \qquad \binom{n}{x_1,\, \ldots,\, x_m} = \frac{n!}{x_1! \cdots x_m!}, \]
    where the coefficient is called a multinomial coefficient.
\end{definition}

We'll spend much more time looking at the continuous case.

\begin{definition}[Continuous multivariate distributions]
    If $X_1, \ldots, X_n$ have joint pdf $f_{X_1, \ldots, X_n}$ then
    \[ P \big( (x_1, \ldots, x_n) \in A \big) = \int_A f_{X_1, \ldots, X_n} (x_1, \ldots, x_n) \, dx_1 \cdots dx_n. \]
    The marginal density of $X_1$ is found by ``integrating out'' $X_2, \ldots, X_n$.
    $X_1, \ldots, X_n$ are independent if
    \[ f_{X_1, \ldots, X_n} (x_1, \ldots, x_n) = f_{X_1}(x_1) \cdots f_{X_n}(x_n) \]
    with a rectangular feasible region.
\end{definition}

When working with univariate distributions we were often tasked with transforming from one variable to another.
We'll see the same problems with multivariate distributions, and the procedure generalizes nicely.
When transforming from several variables to one variable we simply write the cdf for the new variable and differentiate to get its pdf---the methods for doing this might be more involved, but the general idea is there.

We can do the same when we transform to several variables.
In the single-variable case, if $Y = g(X)$ and $g$ is invertible then
\[ f_Y(y) = f_X \big( g^{-1}(y) \big) \left| \frac{d}{dy} g^{-1}(y) \right| = f_X(x) \left| \frac{dx}{dy} \right|. \]
More generally, say we want to transform from some $\mbf{X}$ to some $\mbf{Y}$ given a $\mbf{y} = \mbf{g}(\mbf{x})$.
We start by inverting $\mbf{g}$ and writing $\mbf{x} = \mbf{h}(\mbf{y})$, use this to rewrite the pdf of $\mbf{X}$ in terms of $\mbf{h}(\mbf{y})$, and compute
\[ \mbf{f}_\mbf{Y}(\mbf{y}) = \mbf{f}_\mbf{X} \big( \mbf{h}(\mbf{y}) \big) \,|J|. \]
Here $J = \det (d\mbf{h} / d\mbf{y})$ is the Jacobian determinant of the transformation from $\mbf{Y}$ to $\mbf{X}$:
\[ J = \begin{vmatrix} \frac{\partial x_1}{\partial y_1} & \frac{\partial x_1}{\partial y_2} & \cdots & \frac{\partial x_1}{\partial y_n} \\[6pt] \frac{\partial x_2}{\partial y_1} & \frac{\partial x_2}{\partial y_2} & \cdots & \frac{\partial x_2}{\partial y_n} \\ \vdots & \vdots & \ddots & \vdots \\ \frac{\partial x_n}{\partial y_1} & \frac{\partial x_n}{\partial y_2} & \cdots & \frac{\partial x_n}{\partial y_n} \end{vmatrix}. \]
Finally, don't forget the feasible region!

So far we've seen a very natural generalization of the two-variable case discussed earlier.
We can get similarly natural results about expected value.

\begin{definition}[Expected value]
    If $\mbf{X} = (X_1, \ldots, X_n)$ have joint PDF $f_\mbf{X}$ and $g : \R^n \to \R$ is a real-valued function, we define
    \[ E[g(\mbf{x})] = \int_{\R^n} g(\mbf{x}) f_{\mbf{X}}(\mbf{x}) \,dx_1 \cdots dx_n. \]
\end{definition}

\begin{theorem}[Expected value of a sum]
    For any random variables $X_1, \ldots, X_n$,
    \[ E[X_1 + \cdots + X_n] = E[X_1] + \cdots + E[X_n]. \]
\end{theorem}

Note that this holds when $X_1, \ldots, X_n$ are not independent.
If they are independent, then we get something familiar.

\begin{theorem}[Expected value of a product]
    If $X_1, \ldots, X_n$ are independent, then
    \[ E(X_1 \cdots X_n) = E(X_1) \cdots E(X_n). \]
\end{theorem}

We can leverage these results to solve some neat problems.

\begin{example}[The matching problem]
    Suppose $n$ students do an assignment and their instructor wants them to grade each other's work.
    We'd like to find the expected number of students who get their own submission back, assuming random distribution (under the condition that each student grades one submission).

    We can define an ``indicator variable''
    \[ X_i = \begin{cases} 1 & \textrm{student $i$ gets own submission}, \\ 0 & \textrm{otherwise}, \end{cases} \]
    so $E(X_i) = 1 / n$.
    We might think of the assignment distribution process as a sequence of $X_i$s.
    The $X_i$ are not independent, but we can still say that
    \[ E(X) = E(X_1 + \cdots + X_n) = E(X_1) + \cdots + E(X_n) = 1, \]
    where $X$ is the number of students who get their own submission.
\end{example}

We could follow a very similar line of reasoning to prove what our intuition previously told us about the binomial expected value.

\section{Variance and Covariance}
Now we seek similarly convenient results about variance.
To make this happen, we'll need some scaffolding.

\begin{definition}[Covariance]
    The covariance of $X$ and $Y$ is
    \[ \pfn{cov}(X,Y) = E[(X - \mu_X)(Y - \mu_Y)], \]
    where $E(X) = \mu_X$ and $E(Y) = \mu_Y$.
\end{definition}

\begin{theorem}[Covariance via expected value]
    For any random variables $X$ and $Y$,
    \[ \pfn{cov}(X,Y) = E(XY) - E(X) E(Y). \]
\end{theorem}

It follows that the covariance of two independent variables is zero, and that the covariance of a variable with itself (a scenario with maximum dependence) is simply its variance.

This suggests some intuition as to what the covariance is quantifying.
If $\pfn{cov} (X,Y) > 0$ then we say $X,Y$ are positively correlated, meaning higher values of $X$ are likely to correspond to higher $Y$.
If $\pfn{cov}(X,Y) < 0$ then we analogously have negative correlation, and if $\pfn{cov}(X,Y) = 0$ then $X,Y$ are uncorrelated.

We'll come back to this in a moment.
First, let's look at some of the nice results covariance gives us.

\begin{theorem}[Covariance is bilinear]
    For any random variables $X,Y,Z$,
    \begin{enumerate}[label=(\alph*)]
        \item $\pfn{cov}(aX, bY) = ab \,\pfn{cov}(X,Y)$ and
        \item $\pfn{cov} (X, Y+Z) = \pfn{cov}(X,Y) + \pfn{cov}(X,Z)$.
    \end{enumerate}
\end{theorem}

\begin{theorem}[Variance of a sum]
    For any random variables $X_1, \ldots, X_n$,    \vspace{-6pt}
    \[ \pfn{var} (X_1 + \cdots + X_n) = \sum_{i=1}^{n} \sum_{j=1}^{n} \pfn{cov}(X_i, X_j) \]
\end{theorem}

\begin{corollary}[]
    For independent random variables $X_1, \ldots, X_n$,
    \[ \pfn{var}(X_1 + \cdots + X_n) = \pfn{var}(X_1) + \cdots + \pfn{var}(X_n). \]
\end{corollary}

At this point we could use indicator variables to show that, for the matching problem, $\pfn{var}(X) = 1$.
We could similarly derive the variance of a binomial random variable.
More novel, though, is a connection we can make to correlation.

\begin{definition}[Correlation]
    Let $X,Y$ have variances $\sigma_X^2,\sigma_Y^2$.
    The correlation between $X$ and $Y$ is
    \[ \rho(x,y) = \frac{\pfn{cov}(X,Y)}{\sigma_X \sigma_Y}. \]
\end{definition}

\begin{theorem}[Restrictions on correlations]
    For any random variables $X,Y$,
    \begin{enumerate}[label=(\alph*)]
        \item $|\rho(X,Y)| \leq 1$ and
        \item $\rho(X,Y) = \pm 1$ if and only if $Y$ is a linear function of $X$.
    \end{enumerate}
\end{theorem}

\begin{proof}
    Variance is strictly non-negative, so we have
    \begin{align*}
        0 &\leq \pfn{var} \left( \frac{X}{\sigma_X} + \frac{Y}{\sigma_Y} \right) \\
        &= \frac{1}{\sigma_X^2} \pfn{var}(X) + \frac{1}{\sigma_Y^2} \pfn{var}(Y) + \frac{2 \pfn{cov}(X,Y)}{\sigma_X \sigma_Y} \\
        &= 1 + 1 + 2\rho(X,Y),
    \end{align*}
    meaning $\rho(X,Y) \geq -1$.
    If we instead started with $\pfn{var}(X / \sigma_X - Y / \sigma_Y)$ we'd get $\rho(X,Y) \leq 1$.
    This proves statement (a).
    If we'd used equalities rather than inequalities we'd get the forward direction for (b); the reverse is just some algebra.
\end{proof}

\section{Moment Generating Functions}
We'll look at one last really important characteristic of random variables.
We state uniqueness without proof.

\begin{definition}[Moment generating function]
    A random variable $X$ has moment generating function
    \[ M_X(t) = E [e^{tX}]. \]
\end{definition}

\begin{theorem}[Uniqueness of an MGF]
    If $M_{X_1}(t) = M_{X_2}(t)$ then $f_{X_1}(x) = f_{X_2}(x)$.
\end{theorem}

Notice, in general, that
\[ M_X^{(n)}(t) = E[X^{n} e^{tX}] \;\text{ and }\; M_X^{(n)}(0) = E[X^{n}] \]
for $n \geq 0$.
We will sometimes refer to $E[X^n]$ as the $n$th moment of $X$.

It'll be important for later to know that the MGF corresponding to $X \sim n(0,1)$ is $M_X(t) = e^{t^2 / 2}$.
We could use this next result to quickly obtain the MGF for a general normal distribution.

\begin{theorem}[]
    If $Y = aX + b$ then $M_Y(t) = e^{bt} M_X(at)$.
\end{theorem}

Finally, a way to find the MGF of a sum of independent variables.

\begin{theorem}[MGF of a sum]
    If $X_1, \ldots, X_n$ are independent, where $X_i$ has MGF $M_{X_i}(t)$, then $Y = X_1 + \cdots + X_n$ has MGF
    \[ M_Y(t) = \prod_{i=1}^{n} M_{X_i}(t). \]
\end{theorem}

We could use this result with indicator variables to determine the MGF of, say, the binomial distribution.
We could also use it to prove the claim we made earlier about summing independent normal random variables.

\section{Theoretical Results}
To finish things off, we'll look at some of the more important theoretical results that are accessible to us now.

\begin{theorem}[Markov's inequality]
    For any non-negative random variable $X$, and for any real $t > 0$,
    \[ P(X \geq t) \leq \frac{E(X)}{t}. \]
\end{theorem}

\begin{proof}
    Fix $t > 0$ and write, for a discrete random variable,
    \[ E(x) = \sum_{k \geq 0}^{} k \,P(X = k) \geq \sum_{k \geq t}^{} k \, P(X = k) \geq \sum_{k \geq t}^{} t \, P(X = k) = t \,P(X \geq t). \]
    So $P(X \geq t) \leq E(X) / t$, as desired.
    The continuous case is completely analogous.
\end{proof}

\begin{theorem}[Chebyshev's inequality]
    If $E(X) = \mu$, then for any $t > 0$,
    \[ P(|X - \mu| \geq t) \leq \frac{\pfn{var}(X)}{t^2}. \]
\end{theorem}

\begin{proof}
    Let $Y = (X-\mu)^2$.
    Note that $Y \geq 0$ and $E(Y) = \pfn{var}(X)$, so using Markov's inequality we can write
    \[ P(|x - \mu| \geq t) = P(Y \geq t^2) \leq \frac{E(Y)}{t^2} = \frac{\pfn{var}(X)}{t^2}, \]
    as desired.
\end{proof}

We were able to get this slightly more powerful result from a less powerful one because of our added assumption that the variance exists.
Chebyshev's inequality is a workhorse in probability theory, but we can also prove something much more familiar.

\begin{definition}[Sample mean]
    Let $X_1, \ldots, X_n$ be independent and identically distributed random variables.
    Their sample mean is
    \[ \overline{X}_n = \frac{X_1 + \cdots + X_n}{n}. \]
\end{definition}

\begin{theorem}[EV and variance of a sample mean]
    Let $X_1, \ldots, X_n$ be independent and identically distributed random variables, each with mean $\mu$ and variance $\sigma^2$.
    Then
    \[ E(\overline{X}_n) = \mu, \qquad \pfn{var}(\overline{X}_n) = \frac{\sigma^2}{n}. \]
\end{theorem}

\begin{theorem}[Law of large numbers]
    $\overline{X}_n$ converges to $\mu$ in probability ($\overline{X}_n \overset{P}{\to} \mu$).
    That is, for any $\epsilon > 0$,
    \[ \lim_{n \to \infty} P \Big( |\overline{X}_n - \mu| < \varepsilon \Big) = 1. \]
\end{theorem}

\begin{proof}
    For $\epsilon > 0$, by Chebyshev's inequality
    \[ P \Big( |\overline{X}_n - \mu| \geq \epsilon \Big) \leq \frac{\pfn{var}(\overline{X}_n)}{\epsilon^2} = \frac{\sigma^2}{N \epsilon^2}. \]
    and
    \[ P \Big( |\overline{X}_n - \mu| < \epsilon \Big) = 1 - \frac{\sigma^2}{N\epsilon^2}. \]
    Thus
    \[ \lim_{n \to \infty} P \Big( |\overline{X}_n - \mu| < \varepsilon \Big) = \lim_{n \to \infty} \left( 1 - \frac{\sigma^2}{N \epsilon^2} \right) = 1, \]
    as desired.
\end{proof}

\pagebreak

\section{The Central Limit Theorem}
For our last theoretical result, we have one of the foundational theorems of probability and statistics.

\begin{theorem}[Central limit theorem]
    Suppose the random variables $X_1, \ldots, X_n$ are independent and identically distributed with mean $\mu$ and variance $\sigma^2$.
    The sample mean $\overline{X}_n$ is normally distributed with mean $\mu$ and variance $\sigma^2 / n$.
\end{theorem}

\begin{proof}
    Define
    \[ Y = \frac{\overline{X}_n - \mu}{\sigma / \sqrt{n}}. \]
    We will show that $M_Y(t) \to e^{t^2 / 2}$ as $n \to \infty$ so that $Y$ is a standard normal random variable.
    To this end, let $Y_i = (X_i - \mu) / \sigma$ so that
    \[ \sum_{i=1}^{n} \frac{Y_i}{\sqrt{n}} = \frac{\sum X_i - n\mu}{\sigma \sqrt{n}} = \frac{\overline{X}_n - \mu}{\sigma / \sqrt{n}} = Y. \]
    Now we need only find the moment generating function of each $Y_i$.
    We can easily show that $E(Y_i) = 0$ and $\pfn{var}(Y_i) = 1$, meaning $E(Y_i^2) = 1$ and by Taylor's theorem the MGF of $Y_i$ is
    \begin{align*}
        M(t) &= M(0) + M'(0) t + M''(0) \frac{t^2}{2} + M'''(c) \frac{t^3}{6} \\
        &= 1 + E(Y_i) t + E(Y_i^2) \frac{t^2}{2} + M'''(c) \frac{t^3}{6} \\
        &= 1 + \frac{t^2}{2} + M'''(c) \frac{t^3}{6}
    \end{align*}
    for some $0 < c < t$.
    Now note that a random variable $X / a$ has MGF $M_X(t / a)$ for any constant $a$, so we can write
    \begin{align*}
        M \left( \frac{t}{\sqrt{n}} \right) &= 1 + \frac{(t / \sqrt{n})^2}{2} + \frac{M'''(c_n)}{6} (t / \sqrt{n})^3 \\
        &= 1 + \frac{t^2}{2n} + \frac{d_n t^3}{n \sqrt{n}} \\
        &= 1 + \frac{t^2 / 2 + d_n t^3 / \sqrt{n}}{n},
    \end{align*}
    where we'de defined $d_n = M'''(c_n) / 6$.
    We could show that $d_n$ is bounded, meaning the numerator above tends to $t^2 / 2$ as $n \to \infty$.
    Thus
    \[ M_Y(t) = \left( 1 + \frac{t^2 / 2}{n} \right)^{n} = e^{t^2 / 2}, \]
    as desired.
\end{proof}

\end{document}
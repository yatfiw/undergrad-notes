\documentclass[../a122main.tex]{subfiles}
\graphicspath{{\subfix{../figures/}}}

\begin{document}

\chapter{Unsorted Notes}
\section{Lecture 1}
Some bullets:
\begin{itemize}[topsep=0pt]
    \item High-energy: anything above the visible. After a point we stop talking about wavelengths, and just about energies
    \item Visible: approx 1 eV; X-ray: approx > 1 keV; gamma ray: > 1 MeV
    \item (Recall that E = hv = hc/lambda)
    \item (If the photons are being generated by a blackbody (thermal plasma), the peak has $E \approx kT$)
\end{itemize}
Review of radiative transfer. Some important quantities:
\begin{itemize}[topsep=0pt]
    \item $L \equiv$ luminosity, energy/time -> erg/s or J/s. L\_e =- specific luminosity, energy/(time * energy) = erg/(s * keV)
    \item $F \equiv$ flux, energy/(time * area) and F = L/4pi*d\^2. we can also define specific flux
    \item $\mathcal F_E \equiv$ photon flux = specific flux / photon energy, number of photons per time per area per energy
    \item $I \equiv$ intensity, energy/(time * solid angle * normal area)
\end{itemize}
A solid angle is an area on a unit sphere! (This is why 4pi steradians is full.)
Also, phi is the azimuthal angle and theta is the angle from the vertical.
With this, $d\Omega = \sin \theta \, d\theta d\phi$.

More mathematically, we have
\[ F = \frac{dE}{dt \cdot dA}, \qquad I = \frac{dE}{dt \cdot d\Omega \cdot dA_\perp}, \]
where $dA_\perp = dA \cos \theta$ ($\theta = 0$ corresponds to normal incidence).
So in all,
\[ F = \int_{\Omega}^{} I \cos \theta \, d\Omega. \]
For an emitter that produces isotropically into a hemisphere we can evaluate this integral to get $F = \pi I$.
(Importantly, intensity is independent of distance!)

Now onto blackbody radiation.
See intro astro and quantum for derivations.
The intensity due to a blackbody is given by
\[ I_\nu^\textrm{BB}(T) = B_\nu(T) = \frac{2h\nu^3 / c^2}{e^{h\nu / kT} - 1} \]
and, with the change of variables $B_\nu d\nu = B_E dE$, we get
\[ B_E = B_\nu \frac{d\nu}{dE} = \frac{B_\nu}{h} \]
and so
\[ I_E^\textrm{EE} = B_E = \frac{2\nu^3 / c^2}{e^{h\nu / kT} - 1} = \frac{2E^3 / h^3 c^2}{e^{E / kT} - 1} \]
In the (small) Rayleigh-Jean limit, the blackbody curve takes the shape of $E^2$.
Wien's peak is at around $E_\textrm{peak} = 2.82 kT \approx (2.43 \textrm{ keV}) \cdot T / (10^{7} \textrm{ K})$.
The total emission of a blackbody is given by
\[ I = B(T) = \int_{0}^{\infty} B_E dE = \frac{\sigma T^{4}}{\pi}, \]
where $\sigma$ is the Stefan-Boltzmann constant, and so $F = \sigma T^{4}$.

The blackbody assumption is fine for opaque bodies in thermal equilibrium.
When it isn't fine, we have to deal with radiative transfer.
Let's say we have a cloud with (emitter/absorber) number density $n$.
If an intensity $I_{E,0}$ comes in from the front, we can determine the intensity $I_E(x)$ as a function of the distance from the entrance point.
We can intuit
\[ dI_E = -\alpha_E I dx + j_E dx, \]
where $\alpha_E$ is the absorption coefficient and $j_E$ is the emission coefficient.
Note that we can write $\alpha_E = n \sigma_E$, where $\sigma_E$ is the absorption cross section.
We also have the dimensionless optical depth $d\tau_E = \alpha_E dx = n \sigma_E dx$.
In all,
\[ \tau_E = \int_{x_0}^{x} n \sigma_E dx; \]
if $\tau_E \gg 1$ then our cloud is optically thick (opaque), and if $\tau_E \ll 1$ then it's optically thin (transparent).

Throwing stuff at us: differential equation
\[ \frac{dI_E}{d\tau_E} = I_E - S_E, \]
solution is
\[ I_E(\tau_E) = I_E(0) e^{-\tau_E} + S_E(1 - e^{-\tau_E}). \]
Here, $S_E = j_E / \alpha_E$.
If $\tau_E \gg 1$,
\[ I_E = S_E = \frac{j_E}{\alpha_E} = B_E \]
for thermal equilibrium.

\section{Lecture 2}
For a charge that has accelerated for a very short period of time we have
\[ \frac{dW}{dt \, d \Omega} = \frac{q^2 a^2}{4\pi c^3} \sin^2 \theta, \qquad \frac{dW}{dt} = \frac{2q^2 a^2}{3c^3} = L, \]
where $dW / dt$ is the total power emitted.
We can see that most of the radiation is produced normal to the charge's direction of motion, forming two ``lobes'' of energy output.
(See the Lecture 2 handout for a derivation.)
Some notes:
\begin{itemize}[topsep=0pt]
    \item The kink in the electric field propagates radially at the speed of light.
    \item As it propagates the radial component of the field is constant, but the tangential component is not.
\end{itemize}

Now we will determine the radiation spectrum by breaking down the luminosity by frequency.
If we have the acceleration time signal $a(t)$, then we can use a Fourier transform to decompose it into different modes of oscillation described by $\hat a(\omega)$; each of these modes generates a different frequency of radiation.

Recall(?) Parseval's theorem:
\[ \int_{-\infty}^{\infty} |a(t)|^2 dt = 2\pi \int_{-\infty}^{\infty} |\hat a(\omega|^2 d\omega. \]
So we have the total energy emitted:
\begin{align*}
    W &= \int_{-\infty}^{\infty} \frac{dW(t)}{dt} \\
    &= \int_{-\infty}^{\infty} \frac{2q^2 a^2(t)}{3c^3} dt \\
    &= \frac{2q^2}{3c^3} \int_{-\infty}^{\infty} a^2(t) dt \\
    &= 2 \cdot \frac{2q^2}{3c^3} \cdot 2\pi \int_{0}^{\infty} |\hat a(\omega)|^2 d\omega,
\end{align*}
and since $W = \int_{0}^{\infty} \frac{dW}{d\omega} d\omega$ we get
\[ \frac{dW}{d\omega} = \frac{8\pi}{3} \frac{q^2}{c^3} \hat a(\omega)^2. \]

This has been for a particle moving at non-relativistic speeds.
For relativistic (and-nonrelativistic) particles we have
\[ \left[ \frac{dW}{dt} \right]_R = \frac{2q^2}{3c^3} \gamma^{4} (a_\perp^2 + \gamma^2 a_\parallel^2), \]
where $\gamma$ is the Lorentz factor $\gamma = 1 / \sqrt{1 - (v^2 / c^2)} \gg 1$.
The acceleration components are defined with respect to the direction of motion.
Qualitatively, the difference here is that radiation is emitted primarily in the direction of motion (if this motion is orthogonal to acceleration, otherwise things get weird).

Now we'll consider three different kinds of radiation (not just line emission anymore!).
\begin{enumerate}[topsep=0pt]
    \item Bremsstrahlung (braking) radiation is emission by charged particles interacting with other charged particles.
    If an electron $e^-$ passes by an ion $+Ze$, its trajectory is deflected and it emits as it accelerates.
    The electron's initial velocity is $v$ and its closest approach has distance $b$; the time of interaction is on the order of $\Delta t \sim b / v$.

    The acceleration $a(t)$ looks sort of like a Gaussian, so its Fourier transform $\hat a(\omega)$ also sort of looks like a Gaussian!
    The transformed Gaussian starts to really fall off at around $\omega_c \sim 1 / \Delta t \sim v / b$, and so the spectrum falls off here, too.

    Now, the Maxwell-Boltzmann velocity distribution is
    \[ d \mathcal N = \mathcal N_0 v^2 e^{-mv^2 / 2kT}, \]
    which rises like $v^2$ until around $kT$ and then drops off exponentially.
    This produces specific power per unit volume
    \[ \left( \frac{dW}{dt \, dE \, dV} \right)_{ff} \propto n_e n_\textrm{ion} e^{-E / kT}. \]
    so the BR spectrum starts off looking like a blackbody at low frequencies (a self-absorbed region), flattens off until around $E = kT$, and then drops off super quickly.
    
    At low energies the spectrum follows the Rayleigh-Jeans law(?)

    \item Cyclotron radiation is emission from non-relativistic charged particles interacting with magnetic fields.
    If an electron travels with velocity $\mbf{v}$ in a magnetic field $B$ it experiences an acceleration
    \[ a = \frac{F_L}{m_e} = \frac{q}{c} \frac{v_\perp B}{m_e}, \]
    and so the electron follows a helical trajectory that is counterclockwise along the field lines.
    The radius of this helix is called the Larmor radius $r_L$, and since $a = v^2 / r_L$ we get
    \[ r_L = \frac{v_\perp m_e c}{eB}, \quad \omega_L = \frac{2\pi}{2\pi r_L / v_\perp} = \frac{eB}{m_e c}. \]
    A person watching this happen observes a spectrum comprised of a single frequency, so the spectrum is a delta at $E = \hbar \omega_L$.

    If we now take this and make it relativistic, the orbital frequency turns out to be $\omega_B = \omega_L / \gamma = eB / \gamma m_e c$.
    ((slide photo))
    For such a particle, the spectrum is broadened and has characteristic (peak) frequency $\omega_c \simeq \gamma^2 \omega_L$.

    \item (from next lecture) Inverse Compton scattering involves a relativistic electron with which an electron interacts head-on.
    If the photon gets scattered in the opposite direction then the electron is slowed down a bit, and the photon is scattered with a higher energy.

    In the rest frame of the electron the photon has energy
    \[ E_0' = E_0 \sqrt{\frac{1 + \beta}{1 - \beta}} = E_0 (1 + \beta_e) \gamma_e. \]
    If $E_0' \ll m_e c^2$ then the scattered energy is $E_0'' = E_0'$.
    Back in the lab frame we have $E = E_0'' (1 + \beta_e) \gamma_e$, meaning
    \[ E = E_0 (1 + \beta)^2 \gamma_e^2. \]
    If we average over all angles of incidence and of scattering we get
    \[ \left< \Delta E \right> = \frac{4}{3} \beta^2 \gamma^2 E_0. \]
    Summed over multiple scatterings (per electron),
    \[ \frac{dW}{dt} = \frac{4}{3} \beta^2 \gamma^2 \mu_\gamma \sigma_T. \]
    
    
    % involves an energy loss per electron, per scattering:
    % \[ \left< \Delta E_\gamma \right> = \left< E_{\gamma, \textrm{ fin}} - E_{\gamma, \textrm{ in}} \right> = \frac{4}{3} \beta^2 \gamma^2 E_{\gamma, \textrm{ in}}. \]
    % Summed over multiple scatterings:
    % \[ \frac{dW}{dt} = \frac{4}{3} \beta^2 \gamma^2 \mu_\gamma c \sigma_T. \]
\end{enumerate}

\section{Lecture 3}
A few pointers about neutron stars:
\begin{itemize}[topsep=0pt]
    \item n:p:e ratio is 8:1:1
    \item Masses range from 1-2 solar masses
    \item Radii range from 10 to 15 kilometers
    \item Densities $\rho_\textrm{center} \approx 7 \times 10^{15}$ g/cm$^3$ and $\rho_\textrm{nuc} \approx 2 \times 10^{14}$ g/cm$^3$
    \item Periods: 1.4 ms to 10's of sec (due to conservation of angular momentum)
    \item Magnetic fields: $10^8$-$10^{15}$ G (where 1 G is about the Earth's field strength)
\end{itemize}

Where do the magnetic fields come from?
Flux freezing!
Recall that magnetic fields are generated by currents, in our case plasma currents.
If this plasma gets compressed, the field lines ``stick'' to the plasma, get denser, and the magnetic flux increases.
Due to this conservation of flux we get
\[ B_c R_c^2 = B_{NS} R_{NS}^2 \;\implies\; B_{NS} \simeq B_c \left( \frac{R_c}{R_{NS}} \right)^2. \]
We typically see stellar core fields of $B_c \approx 1$ G.
This isn't entirely consistent with the above formula, so the field must have been amplified somehow.
    % my talk: by what mechanism are these fields amplified? the dynamo effect!

Neutron stars still obey the equation of hydrostatic equilibrium:
\[ \frac{dP}{dr} = -\frac{G m(r) \rho}{r^2}, \]
but rather than being held up by thermal pressure, they exhibit degeneracy pressure with $P \propto \rho^{5 / 3}$ (for a non-relativistic gas).
For our purposes, it will often be convenient to say the star has a uniform density $\rho = \overline \rho$.

In reality, we also have relativistic neutrons with $P \propto \rho^{4 / 3}$.
When we try to get a mass-radius relation for this we get $M^{4 / 3} \propto M^2$, which gives a maximum allowed mass of about 5.7 solar masses.

\section{Lecture 4}
We derived the minimum period for a star to stay intact in Astro 62.
If we were to write this in terms of the density $\rho = 3M / 4\pi R^3$, we'd get
\[ P_\textrm{min} = \sqrt{\frac{3\pi}{\rho G}}. \]
So a pulsar with period 10 ms has minimum density $10^{12} \textrm{ g/cm}^3$.

Plasma frequency when an EM wave passes through plasma:
\[ \nu_p = \left( \frac{e^2 n_e}{\pi m_e} \right)^{1 / 2}. \]
This is typically around 1270 Hz.

For an EM wave with frequency $\nu$ traveling through a plasma, we have the phase and group velocities
\[ v_p = \frac{c}{n}, \quad v_g = cn, \qquad n = \sqrt{1 - \frac{\nu_p^2}{\nu^2}}. \]
With this, we get the signal travel time
\[ t = \frac{d}{v_g} = \frac{d}{c \sqrt{1 - \nu_p^2 / \nu^2}}, \]
so the difference in the travel times of two signals is
\[ \Delta t = t(\nu_1) - t(\nu_2) = \frac{e^2}{2 \pi m_e c^2} dn_e \left( \frac{1}{\nu_1^2} - \frac{1}{\nu_2^2} \right), \]
where $dn_e$ is called the dispersion measure.
($d$ denotes the distance to the object.)

Stuff from Danny's presentation:
\begin{itemize}[topsep=0pt]
    \item Derivation: power radiated from a pulsar in terms of the (second derivative of) the magnetic moment
    \item Solving for the magnetic moment, expression in terms of B and omega and alpha
    \item Derive the spin-down (omega-dot)
    \item Determining the ``dipole age'' of a pulsar
    \item Deriving B propto sqrt(P P-dot)
    \item Interpreting P-Pdot diagrams
\end{itemize}

And now my presentation!!!

\section{Lecture 5}
Let's start with a review of the principle of equivalence.
\begin{enumerate}[topsep=0pt]
    \item The inertial mass of any object is equal to its gravitational mass.
    \item All local, freely falling laboratories are fully equivalent for the performance of physics experiments, i.e., free-falling bodies do not feel the presence of gravity.
\end{enumerate}
We can show some neat things from these principles.
\begin{itemize}[topsep=0pt]
    \item If we shoot a photon across a room accelerating perpendicularly with $a = g$, the photon will be detected at a different height relative to the room's floor.
    Thus light gets bent under the influence of gravity.

    \item If we instead shoot a photon antiparallel to the direction of acceleration, special relativity dictates that the photon's frequency will be blueshifted.
    In order to offset this effect, there must be something else redshifting the photon!

    \item This result about redshift means that an observer far away from a gravity well will perceive clocks inside the gravity well to be ticking more slowly than normal.
\end{itemize}
The Schwarzschild radius corresponds to the smallest radius at which light is still able to escape from the gravity well.
By the Newtonian conservation of energy:
\[ \frac{GMm}{R_S} = \frac{mv_c^2}{2} \;\implies\; R_S = \frac{2GM}{c^2} = (3 \textrm{ km}) \left( \frac{M}{M_\odot} \right). \]
All black holes can be completely described by three parameters: mass $M$, angular momentum $L$, and charge $Q$.
The Minkowski metric
\begin{align*}
    ds^2 &= c^2 d\tau^2 = c^2 dt^2 - dx^2 - dy^2 - dz^2 \\
    &= c^2 dt^2 - dr^2 - r^2 d\theta^2 - r^2 \sin^2 \theta \, d\phi^2
\end{align*}
is what we get when we take $M = L = Q = 0$.
If we relax the assumption that $M = 0$ and we consider a space with a point mass, we get the Schwarzschild metric:
\[ ds^2 = \left( 1 - \frac{R_S}{r} \right)(c\,dt)^2 - \frac{(dr)^2}{1 - R_S / r} - (r\,d\theta)^2 - (r\sin\theta\,d\phi)^2. \]
Expressions for gravitational redshift and time dilation are in ASTR062 notes.
Also, photons orbit a Schwarzschild black hole at $R_\textrm{ps} \simeq 1.5 R_S$, which we see (from infinity) as around $R_\textrm{ps, obs} = (3\sqrt{3} / 2) R_S$.

Now, if we relax another assumption and require only that the black hole has $Q = 0$, we get the Kerr metric.
Check out the slides for that one.
The horizon radius(?) of such a black hole turns out to be
\[ R_h = \frac{R_S}{2} \left( 1 + \sqrt{1 - a^2} \right), \]
and the ergosphere radius is
\[ R_E = \frac{R_S}{2} \left( 1 + \sqrt{1 - a^2 \cos^2 \theta} \right). \]
The ergosphere radius is that within which space rotates around the black hole, along with everything on it.

Now let's look at the dynamics of orbits around black holes.
In theoretical mechanics we wrote
\[ E = \frac{mv_r^2}{2} + \frac{L^2}{2mr^2} - \frac{GMm}{r} = \frac{mv_r^2}{2} + U_\textrm{eff}(r). \]
The the radius at which $U_\textrm{eff}$ is minimized is that at which circular orbits occur.
In the GR treatment, the total energy is
\[ E = \frac{mv_r^2}{2} + \frac{L^2}{2mr^2} - \frac{GMm}{r} - \frac{GML^2}{mc^2 r^3}, \]
where we again lump up the last three terms in $U_\textrm{eff}$.
The $1 / r^3$ term dominates at small distances. 
So as we approach $r=0$ from infinity $U_\textrm{eff}(r)$ decreases for a bit due to $1 / r$, has a little bump due to $1 / r^2$, and then falls off again due to $1 / r^3$.
The bump gets smaller as $L$ decreases, and finally there is a critical $L$ below which there is no point with $U_\textrm{eff}'(r) = 0$.
The $r$ at which this happens is the innermost stable circular orbit.
(We'll calculate this on the homework.)

\section{Lecture 6}
Consider a test mass $m$ in a rotating binary frame, whose position vector $\mbf{r}$ is the displacement from the binary's center of mass.
The net force on $m$ is
\[ \mbf{F} = -\frac{Gmm_1}{|\mbf{r} - \mbf{r}_1|^3} (\mbf{r} - \mbf{r}_1) - \frac{Gmm_2}{|\mbf{r} - \mbf{r}_2|^3} (\mbf{r} - \mbf{r}_2) + m\omega^2 \mbf{r}, \]
where the last term captures the centrifugal force that arises from looking in a rotating frame.
The gravitational potential $U(\mbf{r}) / m = \Phi(r)$ corresponding to this force is
\[ \Phi(\mbf{r}) = -\frac{Gm_1}{|\mbf{r} - \mbf{r}_1|} - \frac{Gm_2}{|\mbf{r} - \mbf{r}_2|} - \frac{1}{2} \omega^2 r^2. \]
This is called the Roche potential.
Each finite-mass body sits in a gravitational well called a Roche lobe, and accretion happens when one of them gets large enough (in volume) to spill over into the other lobe.

Suppose a donor $m_1$ transfers mass to an accretor $m_2$, so $\dot m_2 = -\dot m_1$.
Energy is not conserved in the process (since it turns out that a lot of dissipation happens), but angular momentum is!
We have
\[ L = \mu \sqrt{G M a} \;\implies a = \frac{L^2}{\mu^2 GM} \;\implies\; \frac{da}{dt} = -\frac{2L^2}{GM \mu^3} \frac{d\mu}{dt}, \]
where
\[ \frac{d\mu}{dt} = \frac{d}{dt} \frac{m_1m_2}{M} = \frac{\dot m_1 m_2 + m_1 \dot m_2}{M}. \]
Substituting this and doing some other manipulation gives
\[ \frac{da}{dt} = 2 a \dot m_2 \frac{(m_2 - m_1)}{m_1 m_2}. \]
When $m_2 > m_1$ (so the accretor is more massive), $da / dt > 0$ and the orbit expands.
This is called a low-mass x-ray binary (LMXB), and it is a stable configuration.
If the opposite is true then the orbit shrinks, in which case the Roche lobes get even smaller (and we have an unstable HMXB).

((characteristics of the two in lecture notes))

To see where these x-rays come from, suppose a mass $m$ falls onto the surface of $M$ from a large distance $r \gg R$.
By conservation of energy,
\[ -\frac{GMm}{r} = K - \frac{GMm}{R} \;\implies\; K \simeq \frac{GMm}{R}. \]
So the smaller the star, the higher the energy upon impact.
This is the amount of energy lost to radiation, so $K = E_\text{rad}$ and the efficiency of accretion is
\[ \eta = \frac{E_\text{rad}}{mc^2} = \frac{GM}{Rc^2} = \frac{R_S}{2R}. \]
Eddington mass accretion rate (differentiate $\eta$, set $dE_\textrm{rad} / dt = L$):
\[ \dot M_\textrm{Edd} = \frac{L_\textrm{Edd}}{\eta c^2} = 1.41 \times 10^{18} \left( \frac{M}{M_\odot} \right) \left( 0.\frac{1}{\eta} \right) \text{ g/s}. \]

((spectrum of emitted radiation from slides))

\section{Lecture 7}
General properties of LMXBs:
\begin{itemize}[topsep=0pt]
    \item The accreting star can either be a NS (70\%) or a BH (30\%)
    \item Orbital periods range from hours to days, depending on what kind of star the donor is
    \item Low-mass donors evolve slowly, meaning these systems are long-lived
    \item Roche-lobe overflow leads to the formation of an accretion disk
\end{itemize}
Let $\ell_2$ be defined as in the slides.
The initial angular momentum of a mass $m$ getting overflowed is $\ell_\textrm{in} = \omega_\textrm{orb} \ell_2^2 m$, and the final angular momentum is $\ell_\textrm{in} = v_k R_\textrm{circ} m$, where $R_\textrm{circ}$ is the radius of the orbit it falls into about the accretor.
We know a fre things:
\[ v_k = \sqrt{\frac{GM_2}{R_\textrm{circ}}}, \quad \omega_\textrm{orb} = \frac{2\pi}{P_\textrm{orb}}, \]
and with the definition of $\ell_2$ and Kepler's third law
\[ P_\textrm{orb}^2 = \frac{4\pi^2 a^3}{m_1 + m_2} \]
we get
\[ \frac{R_\textrm{circ}}{\ell_2} = \left( 1 + \frac{m_1}{m_2} \right) \left[ 0.5 - 0.277 \log_{10} \frac{m_1}{m_2} \right]^3. \]

((picture from class))

Suppose the resulting accretion disk has height $H \ll R$.
The energy of a mass $m$ at radius $r$ is
\[ E(r) = \frac{1}{2} m v_k^2(r) - \frac{GMm}{r} = -\frac{GMm}{2r}, \]
and if it spirals inward and experiences a change in orbital distance $dr$, the corresponding change in energy is
\[ dE = E (r + dr) - E(r) = \frac{dR}{dr} dr = \frac{GMm}{2r^2} dr. \]
Now, in a time $t$, the total mass passing through this annulus (corresponding to spiral-inward) is $m = \dot M t$, and it emits a luminosity given by $dE = dL \cdot t$.
So
\[ dL \cdot t = \frac{gM}{2r^2} \, dr \, \dot M t, \]
and recalling that for an blackbody emitter $dL = \sigma T^{4} \cdot 2(2\pi r \, dr)$, we can solve for the temperature to get
\[ T(r) = \left( \frac{GM \dot M}{8\pi r^3 \sigma} \right)^{1 / 4}. \]
But because of an assumption we've made about energy loss, this expression isn't quite right; see the slides for the correct one.

Also check the slides for a correction regarding NS emission.
Energy lost due to radiation in the accretion disk does not make it to the NS surface, so the amount of kinetic energy left over is a factor of 2 smaller.

%% JACOB'S PRESENTATION:

    % Good balance of qualitative and qualitative discussion of ideas and equations! I don't feel like I initially came away with a great understanding of what actually causes the common envelope, though, but the response to Elliot's question at the end helped with that, though!

    % I liked that the slides were so minimalist. Never gave too much to pay attention to (and get distracted by) at any given time. I would, however, have liked a bit more language on the slides. Just a few words to describe what is being shown without dumping all the information about it. 

%% ELLIOT'S PRESENTATION:

    % Focusing on a small subset of flash mechanisms for the bulk of the presentation was a good idea.
    
    % The flow between slides was steady, and I could follow the trajectory of the talk pretty easily. The clean presentation helped with that---a nice balance between visuals, equations, and words.

\section{Lecture 8}
General properties of HMXBs:
\begin{itemize}[topsep=0pt]
    \item The accreting star is usually a NS (there are only 6 BH candidates)
    \item Orbital periods range from days to months, depending on what kind of star the donor is
    \item High-mass donors evolve quickly, meaning these systems are short-lived (typically tens of Myr)
    \item Accreting NSs have strong magnetic fields
    \item Accretion may be due to RL overflow or wind accretion
\end{itemize}
Change in orbital separation:
\[ \frac{da}{dt} = \dot m_2 \frac{m_2 - m_1}{m_1 m_2}. \]
So if $\dot m_2$ is reasonably small, $da / dt$ is slow when compared with the evolutionary timescale of the donor.

% Mickey's presentation!

Supergiant HMXBs have mostly circular orbits due to tidal circularization---tidal bulges in the donor SG facilitate a transfer in angular momentum between the two objects in such a way that the NS orbital and SG rotational frequencies converge and $e \to 0$.

% Madeleine's presentation!

The magnetosphere radius is that at which pressure due to the accretor's magnetic field is able to overcome the pressure due to the donor's wind:
\[ R_M = \left( \frac{B^{4} R_\text{NS}^{12}}{8G M_\text{NS} \dot M^2} \right)^{1 / 7}. \]

Also---almost all pulsars do not burst!
Also also, accretion increases the angular momentum of the NS (this is where millisecond pulsars come from)!

\section{Lecture 9}
% Auggie's presentation!

((mass function!))

The initial energy of an object falling into a BH is $E_\textrm{in} = 0$.
Also, $E_\textrm{in} = E_\textrm{fin} + E_\textrm{rad}$, where
\[ E_\textrm{fin} = -\frac{GMm}{R_\textrm{ISCO}} + \frac{GMm}{2R_\textrm{ISCO}}, \]
potential plus kinetic.
So we have
\[ E_\textrm{rad} = E_\textrm{disk} = \frac{GMm}{2R_\textrm{ISCO}}, \]
with the first equality holding since there's nothing ``hitting'' the surface of the BH and emitting.
The luminosity is therefore
\[ L_\textrm{disk} = \frac{GM \dot M}{2R_\textrm{ISCO}}. \]
For a Schwarzschild BH we have $R_\textrm{ISCO} = 3R_S$ (?), meaning the (classical) efficiency of accretion is
\[ \eta_\textrm{BH} = \frac{L}{\dot M c^2} = \frac{1}{2} \cdot \frac{R_S}{2 (3R_S)} = \frac{1}{12}. \]

Surrounding NS and BH is a corona which produces hard x-ray emission via inverse Compton scattering.
This serves to cool the corona.
But if there's no surface emission then there's less ICS, meaning there's a hotter corona and more hard x-rays.
So to identify BH, we can see where there are more hard x-rays and no BB peak at all.(?)

\section{Lecture 10}
There're a few types of supernovae:
\begin{itemize}[topsep=0pt]
    \item Type Ia: WD exceeds the Chandrasekhar mass and does something Bad
    \item Type II, Ib, Ic: core collapse in a star $M \gtrsim 10 M_\odot$
\end{itemize}
We'll focus on the second bullet, specifically Type II for now.
If the supernova results in a neutron star, we could show that the energy release is
\[ \Delta E = -\frac{3}{10} \frac{GM_\textrm{NS}^2}{R_\text{NS}}, \]
and about 0.5-1\% of this goes back into the envelope.
So material in the envelope gets ejected at around 3000 km/s (and the fastest material can go 10,000-20,000 km/s).
Is this faster than the speed of sound in the ISM?

The speed of sound is given by
\[ c_s = \left( \frac{\Gamma P}{\rho} \right)^{1 / 2}, \]
where $\Gamma = 5 / 3$ is the ratio of specific heats, $P$ is the pressure, and $\rho$ is the density.
For an ideal gas we can write (from the IGL)
\[ P = \frac{\rho kT}{m} \;\implies\; \frac{P}{\rho} = \frac{kT}{m}. \]
Also, the average thermal speed of particles in a gas is
\[ \frac{mv_\textrm{th}^2}{2} = \frac{3}{2} kT \;\implies\; v_\textrm{th} = \sqrt{\frac{3kT}{m}}. \]
So, putting all this together we get
\[ c_s = \left( \frac{5}{3} \frac{kT}{m} \right)^{1 / 2} \approx v_\textrm{th}. \]
For the ISM, $c_s$ ranges from 1-3 km/s.
So supernova ejecta are highly supersonic (moving at around Mack 1000-10,000)!

Let 2-quantities describe the shocked ISM and let 1-quantities describe the undisturbed ISM.
In the shock frame, the ISM moves leftward with $u = v_1$ and the shock front is moving at $v_2 = u - v_\text{ej}$.
To analyze this shock, we can look at a few conserved quantities:
\begin{itemize}[topsep=0pt]
    \item Mass conservation.
    The mass per unit time, per unit area is the same on both sides of the shock front:
    \[ \rho_1 v_1 = \rho_2 v_2. \]
    \item Momentum conservation.
    There is no net force on the shock front---pressure and ``momentum flow'' $dp / dt$ cancel on either side.
    \[ P_1 + \rho_1 v_1^2 = P_2 + \rho_2 v_2^2. \]
    \item Energy conservation.
    Change in kinetic and thermal energy is due to work by the gas.
    (Derivation using stat mech stuff.)
    \[ \frac{v_1^2}{2} + \frac{5}{2} \frac{P_1}{\rho_1} = \frac{v_2^2}{2} + \frac{5}{2} \frac{P_2}{\rho_2}. \]
\end{itemize}
We could go through a bunch of algebra to get
\begin{align*}
    \frac{\rho_2}{\rho_1} &= \frac{v_1}{v_2} = \frac{4}{1 + 3 / M_1^2}, \\
    \frac{P_2}{P_1} &= \frac{5M_1^2 - 1}{4}, \\
    \frac{T_2}{T_1} &= \frac{(5 M_1^2 - 1) (3 + M_1^2)}{16M_1^2}.
\end{align*}
For strong shocks the Mack number $M_1 \gg 1$, which reduces these to
\[ \frac{\rho_2}{\rho_1} = \frac{v_1}{v_2} = 4, \qquad \frac{P_2}{P_1} = \frac{5}{4} M_1^2, \qquad \frac{T_2}{T_1} = \frac{5}{16} M_1^2. \]
We could use this to show that $u = (4 / 3) v_\textrm{ej}$.

Particles that bounce back and forth through the shock experience a net energy increase, so they get accelerated!
(1st order Fermi acceleration.)

\section{Lecture 11}
thing here

\section{Lecture 12}
Some bullets.
\begin{itemize}[topsep=0pt]
    \item The distribution of GRBs in the sky is isotropic, so these sources must be cosmological.
    They're insanely energetic!
\end{itemize}

\section{Lecture 13}
thing here

\end{document}
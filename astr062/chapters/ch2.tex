\documentclass[../a062main.tex]{subfiles}
\graphicspath{{\subfix{../figures/}}}

\begin{document}

\chapter{Stellar Properties}
\section{Spectral Formation}
If we were to zoom into the blackbody spectrum of a star like the Sun we'd notice some deviations from the ideal blackbody spectrum.
Specifically, there are many wavelengths where the emission is smaller than we'd expect from a blackbody.
These features are called \textbf{absorption lines}.
But some objects like planetary nebulae are characterized more by \textbf{emission lines}, which are spikes of emission in a sea of none.
In this section we'll try to explain these phenomena!

\subsection*{Quantized Energy States}
We'll start with \textbf{Kirchhoff's laws}, which is a set of empirical laws that seem to predict spectral formation.
\begin{enumerate}
    \item[0.] Each chemical element produces a unique pattern of spectral lines that have the same wavelengths in emission and absorption.
    \item[1.] A hot, dense gas or solid object produces a continuous spectrum with no dark spectral lines.
    (This is a blackbody spectrum!)
    \item[2.] A hot, diffuse gas produces bright spectral lines (emission lines).
    \item[3.] A cold, diffuse gas in front of a hotter continuous-spectrum source produces dark spectral lines (absorption lines) in the continuous spectrum.
\end{enumerate}
The first solar spectrum was published by Fraunhofer in the early nineteenth century.
A few decades later, Balmer discovered an empirical formula for hydrogen absorption lines, now called \textbf{Balmer lines:}
\[ \frac{1}{\lambda} = R_H \left( \frac{1}{4} - \frac{1}{n^2} \right), \quad n = 3,4, \ldots, \]
where $R_H$ is called the Rydberg constant.

The explanation for all these empirical observations is quantum mechanical.
The energy in electromagnetic radiation comes in discrete packets called photons, each with energy $E_\gamma = h\nu = hc / \lambda$.
Electrons can only occupy predetermined, discrete energy levels within atoms.
When an electron absorbs or emits a photon it transitions to a different energy state; the resulting change in the electron's energy is equal to the photon's energy, meaning electrons can only interact with photons of certain wavelengths.

In the early twentieth century Bohr devised the first model of the hydrogen atom that correctly predicted the quantized energy levels for an electron.
We now know this model to be completely wrong, but it remains a useful tool for our analysis.
He proposed that the electron orbits around the proton (due to the electrostatic force) at radii $r_n$ that are multiples of its de Broglie wavelength, $\lambda_\textrm{dB} = h / m_e v$.
This led to the energies
\[ \boxed{E_n = -\frac{m_e e^{4}}{8 \varepsilon_0^2 h^2 n^2} = -\frac{E_1}{n^2}}, \]
where $e$ is the charge of an electron and $E_1 = 2.18 \times 10^{-18} \textrm{ J} = 13.6 \textrm{ eV}$.

This can be used to recover a more general version of Balmer's equation!
Suppose that a photon with energy $E_\gamma$ is emitted as an electron moves between its $k$th and $n$th energy states.
So $E_\gamma = hc / \lambda = E_n - E_k$ and
\[ \frac{1}{\lambda} = \frac{m_e e^{4}}{8 \varepsilon_0^2 h^3 c} \left( \frac{1}{k^2} - \frac{1}{n^2} \right). \]
The constant in front evaluates to the Rydberg constant, and setting $k=2$ gives the Balmer equation!

Originally stars were classified by letter (A, B, C, etc.) based on the strength of their hydrogen lines.
A few decades later, the spectra were re-understood as a temperature sequence: from hottest to coolest,
\begin{center}
    O B A F G K M
\end{center}
Nowadays each spectral type is subdivided into ten subtypes (e.g., B0, B1, ..., B9).
Each type corresponds to a very specific pattern of spectral lines, their relative strengths, and thus a variety of elements and molecules!

\subsection*{Excitation and Ionization}
So a star's spectrum is a function of its temperature.
Now, in order for a spectral line to form, there must be a large enough number of atoms in both the proper excitation and ionization states---that is, it must have an electron at the proper energy level, and it must also have the proper amount of electrons.
How can we quantify these things in terms of temperature?

Generally, excitations and ionizations are caused by high-speed collisions between atoms.
These velocities are related to temperature by the equation
\[ \left< \frac{1}{2}mv^2 \right> = \frac{3}{2} kT, \]
where $m$ is the mass of an emitter.
So whatever we end up with should give us higher excitation and ionization states with increasing temperature.

Let's start with excitation states.
Consider two energy levels $a$ and $b$ inside an atom such that $E_B > E_a$.
Let $P(a)$ is the probability that an electron occupies level $a$, and let $N_a$ be the number of electrons in this level.
Using statistical mechanics one could show that
\[ \frac{N_b}{N_a} = \frac{P(b)}{P(a)} = \frac{e^{-E_b / kT}}{e^{-E_a / kT}} = e^{-\Delta E / kT}, \]
where $\Delta E = E_b - E_a$.
Notice that we have $N_a > N_b$ for all $\Delta E > 0$, meaning higher energy states will never have a higher probability that lower ones.

But this doesn't quite tell the whole story because energy levels can be degenerate---that is, there can be several states with the same energy.
For example, hydrogen has two ground-state levels (2s) and eight first excited-state levels (2s + 6p).
Let $g_a$ be the number of states with energy $E_a$ (for hydrogen $g_n = 2n^2$); making a slight modification to our equation to account for degeneracy gives the \textbf{Boltzmann equation}:
\[ \boxed{\frac{N_b}{N_a} = \frac{P(b)}{P(a)} = \frac{g_b e^{-E_b / kT}}{g_a e^{-E_a / kT}} = \frac{g_b}{g_a} e^{-\Delta E / kT}}. \]
Now let's talk ionization.
We represent ionization states using Roman numerals---for example, the first three ionization states for calcium are Ca I, Ca II, and Ca III, in which zero, one, and two electrons have been removed, respectively.
Let $N_i$ be the number of atoms in their $i$th ionization state and $\chi_i$ the energy required to bring an atom from its $i$th ionization state to its $(i+1)$th state; we might expect that $N_{i+1} / N_i \propto e^{-\chi_i / kT}$.
This will actually be the case!
To account for degeneracy, we define the partition function
\[ Z_i = \sum_{k=1}^{\infty} g_k e^{-(E_k - E_1) / kT}. \]
This is a weighted average of all the (infinitely many) degeneracies, with the Boltzmann factor there to give more weight to the higher-probability energies.
Using statistical mechanics magic we get the \textbf{Saha equation},
\[ \boxed{\frac{N_{i+1}}{N_i} = \frac{2}{n_e} \left( \frac{2\pi m_e kT}{h^2} \right)^{\frac{3}{2}} \frac{Z_{i+1}}{Z_i} e^{-\chi_i / kT}}, \]
where $n_e$ is the number density of electrons in space.
Note that a larger $T$ corresponds to more collisions and thus a higher $N_{i+1}$, and that a larger $n_e$ implies more recombinations and thus a lower $N_{i+1}$. % TODO what is chi?

The Boltzmann and Saha equations together describe why spectral lines appear and then disappear again as temperature increases!
Take the H$\upalpha$ line for example, which corresponds to a transition between the first and second excited states of a neutral hydrogen atom.
At low temperatures there are no excited hydrogens, so the line is invisible; as temperature increases, more hydrogens become excited and H$\upalpha$ becomes stronger; but as temperature increases further the hydrogens begin to ionize and the line fades.

\section{Radiative Transfer}
Let's get to work on a physical model for stars, starting with \textbf{radiative transfer}---the production of radiation in the stellar atmosphere and its path to Earth.
We'll focus on absorption for now.

\subsection*{Absorption}
Suppose a photon passes through a cloud on its way to an observer.
If we want to determine the probability that this photon gets absorbed along the way, we need only consider the electrons in the cloud---only high-energy gamma rays can interact directly with atomic nuclei and other elementary particles.
Let's say our cloud contains $n_e$ electrons per unit volume, and a photon gets absorbed if it comes within the \textbf{cross section} $\sigma_\lambda$ (the ``target area'') of an electron.
This cross section is a function of the photon's wavelength.

We can simplify this scenario slightly by thinking of the electrons as point particles and of the photon as having an effective cross section $\sigma_\lambda$.
In a time $\Delta t$, the photon sweeps out a cylinder that contains $N = n_e \sigma_\lambda \,c\Delta t$ electrons, so the collision rate is
\[ R = \frac{N}{\Delta t} = \sigma c n_e. \]
We can use this to define the \textbf{mean free time} and \textbf{mean free path} between collisions:
\[ t_\textrm{mf} = \frac{1}{R} = \frac{1}{\sigma_\lambda c n_e} \qquad l_\textrm{mf} = ct_\textrm{mf} = \frac{1}{n \sigma_\lambda}. \]
In practice things will not be so simple.
The medium in which the photon travels may have many absorbing species, each with their own number densities $n$.
So rather than deal with the cross section $\sigma_\lambda$ (an effective area per absorber), we'll often work with the \textbf{opacity}
\[ \boxed{\kappa_\lambda = \sigma_\lambda \frac{n}{\rho}}, \]
which is an effective area per unit mass.
(In this equation $\rho$ is the mass density of absorbers.)
A particularly convenient property of this quantity is that, when working with several absorbing species, we can determine the total opacity just by summing up the individual opacities!

We're now in a position to determine the effect of absorption on light as it propagates through space.
We choose intensity as our measure of brightness because, according to \textbf{Liouville's theorem}, it remains constant as radiation propagates from source to observer.
To show this, consider a source that creates a solid angle $\Omega = A / d^2$, where $A$ is the area that ``fits'' inside the solid angle at a distance $d$ from the observer.
For a small source, $I_\textrm{obs} = F / \Omega = Fd^2 / A$; but flux is proportional to $A / d^2$, so $I_\textrm{obs}$ is constant over $d$!

Consider a very short distance $ds$ in the absorptive medium.
The probability that a photon gets absorbed in this length is $ds / l_\textrm{mf}$, so the expected change in intensity
\[ dI_\lambda = -I_\lambda(s) \frac{ds}{l_\textrm{mf}}. \]
This is a separable differential equation whose solution is
\[ \boxed{I_\lambda(s) = I_\lambda(0) e^{-\int_{0}^{s} ds / l_\textrm{mf}} = I_\lambda(0) e^{-s / l_\textrm{mf}}}, \]
where the last expression represents the special case in which the mean free length is constant.
The dimensionless quantity in the exponent is of huge importance---we call it the \textbf{optical depth} $\tau_\lambda = s / l_\textrm{mf}$, and we might interpret it as the number of collisions a photon will experience if it travels in a straight line.
We usually define $\tau_\lambda = 0$ on the side of the medium facing the observer.
So we can repackage this intensity equation as
\[ d\tau_\lambda =-\frac{ds}{l_\textrm{mf}} \implies \boxed{I_\lambda(0) = I_\lambda(\tau_{\lambda,c}) e^{-\tau_\lambda}}, \]
where $\tau_{\lambda,c}$ is the optical depth at the side of the medium opposite the observer.
In the optically thick limit $\tau_\lambda \gg 1$ most of the light is absorbed so $I_\lambda(0) \approx 0$, but in the optically thin limit $\tau_\lambda \ll 1$ a Taylor approximation gives $I_\lambda(0) \approx I_\lambda(\tau_{\lambda,c}) (1 - \tau_\lambda)$.

\subsection*{The Equation for Radiative Transfer}
We take a similar path to account for emission.
Define the \textbf{emission coefficient} $j_\lambda$
as the energy emitted per unit time, per unit solid angle, per unit mass, per unit wavelength.
In order to go from an emission coefficient to an intensity, we'll have to multiply by a mass and divide by an area to get the right units:
\[ dI_\lambda = j_\lambda \frac{\rho A \,ds}{A} = j_\lambda \rho \,ds. \]
Combined with absorption
, we get the differential equation
\[ dI_\lambda = -I_\lambda \frac{ds}{l_\textrm{mf}} + j_\lambda \rho ds \;\implies\; dI_\lambda = I_\lambda d\tau_\lambda - \frac{j_\lambda}{\kappa_\lambda} d\tau_\lambda \]
The local quantity $S_\lambda = j_\lambda / \kappa_\lambda$, called the \textbf{source function}, describes the removal and replacement of traveling photons.
So we can write
\[ \boxed{\frac{dI_\lambda}{d\tau_\lambda} = I_\lambda - S_\lambda}. \]
One could solve this differential equation via multiplication by the integrating factor $e^{-\tau_\lambda}$.
This gives the observed intensity
\begin{empheq}[box=\fbox]{align*}
    I_\lambda(0) &= I_\lambda (\tau_\lambda) e^{-\tau_\lambda} + \int_{0}^{\tau_\lambda} S_\lambda e^{-\tau_\lambda'} d\tau_\lambda' \\
    &= I_\lambda (\tau_\lambda) e^{-\tau_\lambda} - S_\lambda \left( e^{-\tau_\lambda} - 1 \right),
\end{empheq}
where the last expression is the special case in which $S_\lambda$ is constant.
The first term in each accounts for the absorption
of some initial intensity, while the second accounts for emission and the potential for re-absorption.

In the optically thick limit we can see that $I_\lambda(0) = S_\lambda$.
But if the cloud is in thermal equilibrium then it acts like a blackbody, meaning $S_\lambda = B_\lambda(T)$.
(In fact, since $S_\lambda$ is a local property, this relationship holds for \textit{any} cloud in thermal equilibrium!)
In the optically thin limit, a Taylor expansion gives
\[ I_\lambda(0) = I_\lambda(\tau_\lambda) (1-\tau_\lambda) + S_\lambda(\tau_\lambda) \tau_\lambda. \]

\subsection*{Kirchhoff's Laws and Line Widths}
We can use this to prove Kirchhoff's second and third laws quite easily.
Suppose we have an optically thick star sitting behind an optically thin cloud at temperatures $T_B$ and $T_C$, respectively.
The observed intensity is
\begin{align*}
    I_\lambda(0) &= B_\lambda(T_B)(1-\tau_\lambda) + B_\lambda(T_C)\tau_\lambda \\
    &= B_\lambda(T_B) + [B_\lambda (T_C)- B_\lambda(T_B)] \tau_\lambda.
\end{align*}
If $T_C > T_B$ then the stuff in the brackets is positive.
Also, $\tau_\lambda$ peaks at atomic transition wavelengths, so overall we get an intensity spectrum with spikes at particular wavelengths, which we recognize as emission lines.
If $T_B < T_C$, then by an analogous argument we get absorption lines.

It is important to note that these spectral lines are not infinitely thin as one might expect.
In fact, they are roughly Gaussian in shape.
Three big reasons for this are given below.
\begin{itemize}
    \item \textbf{Natural broadening}.
    Excited states have finite lifetimes $\Delta t$, and there is an uncertainty relation in quantum mechanics which relates this to the uncertainty in energy $\Delta E$ (which is easily related to $\Delta \lambda$):
    \[ \Delta E \Delta t \geq \frac{\hbar}{2}. \]

    \item \textbf{Doppler broadening}.
    If an object's velocity has a radial component with respect to an observer then the light it gives off will be redshifted or blueshifted.
    Specifically,
    \[ \frac{\Delta \lambda}{\lambda} \approx \frac{\left< v_r \right>}{c} = \frac{2}{c} \sqrt{\frac{2kT \ln 2}{m}}. \]

    \item \textbf{Pressure broadening}.
    As atoms run into each other, they distort each others' electron orbitals and thus their energy eigenvalues.
    The effects of these collisions are captured by the equation
    \[ \Delta \lambda \approx \frac{\lambda^2}{c} \frac{n\sigma}{\pi} \sqrt{\frac{2kT}{m}}. \]
\end{itemize}
In a group of stars that all have the same surface temperatures, larger stars have lower densities near their photospheres and thus narrower lines.
So we can use a star's stellar spectrum not only to determine its spectral type, but also to determine its luminosity class!
(Some common classes are V for the main sequence, III for giants, and I for supergiants; the Sun is a G2V star.)

\subsection*{The Photosphere}
All stellar spectra have absorption lines, which might suggest that stars have some emitting surface upon which there is a cooler atmosphere.
But reality is much more complicated---mainly, stars have no well-defined surface.
So where does their light come from?

Let's imagine the simplest possible source function that is monotonically increasing: $S_\lambda = a + b\tau_\lambda$.
(We might think of this as a Taylor approximation for a more complex source function.)
Let $\tau_\lambda^*$ be some point deep inside the star, so we have
\[ I_\lambda(0) = I_\lambda(\tau_\lambda^*)e^{-\tau_\lambda^*} + \int_{0}^{\tau_\tau^*} (a + b\tau_\lambda)e^{-\tau_\lambda} d\tau_\lambda. \]
But stars are opaque, so $\tau_\lambda^*$ is a very good approximation.
This gives
\[ I_\lambda(0) = \int_{0}^{\infty} (a + b\tau_\lambda) e^{-\tau_\lambda} d\tau_\lambda = a + b. \]
So the observed intensity is $I_\lambda(0) = S_\lambda(\tau_\lambda = 1)$.
We can, therefore, interpret $\tau_\lambda$ as the ``surface'' of the star, called the \textbf{photosphere}.

The precise configuration of the photosphere is relative to the observer.
Consider our view of the Sun as a disk in the sky.
If we pick a point near the center of this circle and go $\tau_\lambda$ inward, we'll end up deeper within the Sun than if we had picked a point near the disk's edge.
Thus the photosphere actually gets more shallow as the distance from the center of the solar disk increases, and the solar temperature and intensity of emitted light decrease similarly.
This phenomenon in which the solar disk gets redder and darker near its edges is called \textbf{limb darkening}.

\section{Stellar Interiors}
All this describes what goes on in the ``atmosphere'' of a star, and it's pretty easy to observe since this is precisely where the light reaching us comes from.
We cannot, however, see into the inside of a star, so we'll have to construct a reasonable model to describe it.

\subsection*{Hydrostatic Equilibrium}
Let's start with the fairly reasonable requirement that the star does not collapse in on itself due to gravity.
The only thing that stops this from happening is pressure---to quantify it, consider a cylindrical chunk of mass whose axis is aligned with the star's radius.
The chunk is at a distance $r$ from the center of the star and has mass
\[ dm = A \,dr \,\rho(r), \]
where $\rho$ is the mass density of the star and $A$ and $dr$ are the base area and height of the chunk, respectively.
If $P$ is the pressure acting on the cylinder, then the equation of the cylinder's radial motion is
\[ F_\textrm{net} = P(r) A - \left[ P(r + dr)A + dm\,g(r) \right], \]
where $g$ is the local acceleration due to gravity.
(We need not concern ourselves with the perpendicular directions because those forces cancel each other out.)
In equilibrium $F_\textrm{net} = 0$, and we can rearrange to get the \textbf{hydrostatic equilibrium equation}
\[ \boxed{\frac{dP}{dr} = -\rho(r) g(r)}. \]
Note that we can use Newton's shell theorem to write $g(r) = GM_r(r) / r^2$, where $M_r(r)$ is the mass contained within a distance $r$ of the star's center.
But either way we have three unknowns and only one equation.
We can add another, the \textbf{conservation of mass equation}
\[ \boxed{\frac{dM_r}{dr} = 4\pi r^2 \rho(r)}, \]
but we're still one short.
We'll take some time to discuss other stellar behaviors with the aim of writing down more equations that, hopefully, will comprise a complete model for the interiors of stars.

Let's start with pressure.
We know that pressure can arise from the thermal motions of gas molecules, and that their behavior is (with few exceptions) described by the ideal gas law $PV = NkT$, where $N$ is the number of gas molecules.
Written in terms of number density, this is
\[ P_\textrm{gas} = nkT. \]
To avoid introducing $n$ as a new variable, we can write $n = \rho / \overline m$, with the average mass per particle $\overline m = \mu m_p$.
($m_p$ is the proton mass and $\mu$ is the mean molecular mass).
So we have
\[ \boxed{P_\textrm{gas}(r) = \frac{\rho(r) kT(r)}{\mu m_p}}. \]
This is the equation of state that's most relevant to main sequence stars.
But photons, too, carry momentum and can thus produce pressures that, in the brightest and most massive stars, can be even greater than that produced by gas molecules.
As a slight simplification, suppose $N$ photons are incident upon a reflective area $A$ in the star, each with initial momentum $p_\gamma$ and collision time $\Delta t$.
The total pressure on the area is
\[ P_\textrm{rad} = \frac{\textrm{force}}{A} = \frac{N \cdot 2p_\gamma}{A \,\Delta t}. \]
Now, by the energy-momentum relation $p_\gamma = E_\gamma / c$ and the definition of flux $N E_\gamma = F A \,\Delta t$,
\[ P_\textrm{rad} = \frac{N \cdot 2E_\gamma}{c A \,\Delta t} = \frac{2F}{c}. \]
In terms of temperature, this is $P_\textrm{rad} = 2 \sigma T^{4} / c$.
If we were to take into account the different directions from which a photon may collide with the surface, we would instead get
\[ \boxed{P_\textrm{rad} = \frac{4}{3} \frac{\sigma T^{4}(r)}{c}}. \]
Which of these \textbf{equations of state} we use depends on the scenario.
For some objects, the effects of gas pressure and radiation pressure are comparable enough to use both equations (by adding them).

\subsection*{Energy Generation}
We now have a new equation to work with, but we've also introduced a new unknown: temperature.
A reasonable next step, then, is to work out where this kinetic energy comes from.

For a while the leading theory was \textbf{gravitational contraction}.
Stars would start out large and over time shrink to smaller radii, converting gravitational potential energy to kinetic energy in the process.
It can be shown that the potential energy of a uniform sphere of mass $M$ and radius $R$ is given by
\[ U(R) = -\frac{3}{5} \frac{GM^2}{R}, \]
and by the virial theorem the total mechanical energy is
\[ E(R) = -\frac{3}{10} \frac{GM^2}{R}. \]
So if a star shrinks from radius $R_i$ to $R_f$ (where $R_i \gg R_f$), then the amount of energy liberated is $\Delta E_g = -E(R_f)$.
But there's a problem with this: assuming a roughly constant luminosity over the Sun's lifetime, it implies that the estimated age of the Sun is
\[ t_\textrm{KH} = \frac{\Delta E_g}{L_\odot} \sim 10^{7} \text{ yr}. \]
This is called the Kelvin-Helmholtz timescale, and it's at least four hundred times smaller than even the estimated age of Earth.
So clearly gravitational contraction cannot play a major role in energy production in main-sequence stars, though it is important in earlier and later stages of life.

Today we know that the main source of energy is \textbf{nuclear fusion}.
Specifically, in the nuclear process
\[ \ch{4 H + 2 e^- -> He} \]
some amount $\Delta m$ of it is converted to energy according to $\Delta E = \Delta m \,c^2$.
We define the \textbf{efficiency parameter} $\eta = \Delta m / m_i$ to quantify this conversion.

Since only about ten percent of the Sun is hot enough to do fusion, we have the nuclear timescale
\[ t_\textrm{nuc} = \frac{E_\textrm{em}}{L_\odot} = \frac{\Delta m}{4m_p} \frac{0.1 M_\odot c^2}{L_\odot} \sim 10^{10} \text{ yr}, \]
ignoring the negligible mass of the electrons.
This lines up much more nicely with observations on Earth---nuclear fusion, unlike gravitational contraction, works as a mechanism for energy production!

The fusion is not a simple as four protons smashing together to create a helium nucleus.
The main reaction chain in Sun-like stars is called the \textbf{proton-proton chain}, and it is described below.
\begin{align*}
    \ch{^11H + ^11H &-> ^21H + e^+ + \nu_e} \\
    \ch{^21H + ^11H &-> ^32H + \gamma} \\
    \ch{^32He + ^32He &-> ^42He + 2 ^11H}
\end{align*}
Here $\gamma$ is a photon, $e^+$ is a positron (the antiparticle of an electron), and $\nu_e$ is called an electron neutrino.
% Neutrinos interact via the weak force and gravity, but the weak force has a very short range and neutrinos have a very small mass so they typically pass through matter unimpeded and undetected.
Putting all these together gives
\[ \ch{4 ^11H -> ^42He + 2 e^+ + 2 \nu_e + 2 \gamma}. \]
Note that the positrons will be annihilated with electrons, creating more photons.
Reactions like these follow five key conservation laws: conservation of nucleons, charge, lepton number, momentum, and energy.
(Neutrinos have a lepton number of $1$, and positrons $-1$.)

How hot must it be for this fusion to occur?
The rate-limiting step is the first one, in which an incoming proton must overcome the Coulomb potential of another to enter the strong force's domain of influence.
In particular, the kinetic energy of the incoming proton must satisfy
\[ K \geq \frac{e^2}{4\pi \varepsilon_0 r_0}, \]
where the strong force dominates for $r < r_0$.
But $\left< K \right> = (3/2) kT$, so on average we have
\[ T \geq \frac{e^2}{6\pi \varepsilon_0 r_0} \sim 10^{10} \textrm{ K}. \]
But it's known that the core temperature of the Sun is closer to $10^{7} \textrm{ K}$, so what gives?
Quantum tunneling!
The incoming proton is likely to tunnel through the Coulomb potential at temperatures above $T \sim 10^{6} \text{ K}$, which the Sun satisfies.

All this leads us to another equation for our stellar model.
Define $\varepsilon$ as the energy generated by fusion per unit mass per unit time.
In general $\varepsilon$ has a very complex form, but for Sun-like cores we can approximate $\varepsilon_\textrm{ppc} \propto \rho T^{4}$.
So we get the \textbf{energy conservation} (or \textbf{luminosity gradient}) \textbf{equation}
\[ \boxed{\frac{dL_r}{dr} = 4\pi r^2 \rho(r) \varepsilon(\rho, T)}, \]
where $L_r(r)$ is the net energy flow through a sphere of radius $r$.

\subsection*{Energy Transport}
Let's consider one last aspect of stars that will yield an equation that introduces no new unknowns.
There are three broad mechanisms by which energy can be transported:
\begin{itemize}
    \item conduction, in which energy is transferred via particle-particle interactions;
    \item radiation, in which energy is transferred via photon absorption and emission; and
    \item convection, which is facilitated by the motion of macroscopic parts of the system.
\end{itemize}
In regular stars conduction is not important, but the other two play major roles.
Let's start with radiative transport.
Consider a cylindrical parcel of gas somewhere inside the star with its axis aligned in the radial direction.
For energy to be transported outward, there is a net flux through the caps that changes as a function of photon absorption.
(Emission is not a factor because it is isotropic.)
In particular,
\begin{align*}
    F_\textrm{net}(r + dr) - F_\textrm{net}(r) &= F_\textrm{net}(r) \left( e^{-d\tau} - 1 \right),
    \intertext{where $d\tau = dr / l_\textrm{mf}$ is the optical depth through the parcel. A first-order Taylor expansion gives}
    F_\textrm{net}(r + dr) - F_\textrm{net}(r) &= -F_\textrm{net}(r) \frac{dr}{l_\textrm{mf}} \\
    \frac{dF_\textrm{net}}{dr} &= -F_\textrm{net}(r) \kappa \rho,
    \intertext{where $\kappa$ is the opacity. Now, the radiation pressure on an absorbing surface is $P_\textrm{rad} = F_\textrm{net} / c$; substituting the derivative gives}
    \frac{dP_\textrm{rad}}{dr} &= -\frac{\kappa \rho}{c} F_\textrm{net}(r) \\
    \frac{dP_\textrm{rad}}{dr} &= -\frac{\kappa \rho}{c} \frac{L_r(r)}{4\pi r^2} \\
    \intertext{Finally, substituting the derivative of $P = (4 / 3)(\sigma T^{4} / c)$ gives}
    \Aboxed{\frac{dT}{dr} &= -\frac{3}{16} \frac{\kappa \rho}{\sigma T^{3}} \frac{L_r(r)}{4\pi r^2}}
\end{align*}
This is the \textbf{radiative energy transport equation}.
The relationships between all these variables should make sense: a large $L_r(r)$ causes a large $dT / dr$ to drive more radiation outward, and a large opacity requires a large $dT / dr$ to drive the same $L_r$.
(Opacity acts as the ``friction'' for radiation.)

Let's move on to convection. % Kayla was here
Consider another cylindrical bubble of gas with density and temperature $\rho_B, T_B$; the surroundings have $\rho_S, T_S$.
Our gaseous equation of state says that $P \propto \rho T$, so in order for the bubble to maintain a constant pressure relative to its surroundings, any change in temperature must be matched with a multiplicatively equal and opposite change in density.
If $T_B > T_S$, then $\rho_B < \rho_S$ and the bubble rises. % oooo gas

As the bubble rises, we have three scenarios for the relative changes in temperature and density.
\begin{itemize}
    \item Vigorous convection.
    The bubble's temperature drops more slowly than that of its surroundings, so it continues to rise with increasing speed.

    \item Limited convection.
    The bubble's temperature drops more quickly than that of its surroundings, so its rise eventually slows to a stop.

    \item Marginal convection.
    The bubble's temperature drops at the same rate as that of its surroundings.
\end{itemize}
But all forms of convection move heat around in such a way that promotes marginal convection, so we can take
\[ \frac{dT_B}{dr} = \frac{dT_s}{dr}. \]
Now, from thermodynamics it is known that $P_B = a T_B^{\gamma / (\gamma - 1)}$, where $a$ is some constant and $\gamma$ is called the adiabatic exponent of the gas, which depends only on its composition---no heat is exchanged with its surroundings.
(For a monatomic ideal gas, $\gamma = 5 / 3$.)
We can differentiate this expression to get
\[ \frac{dP_B}{dr} = a \left( \frac{\gamma}{\gamma - 1} \right) \frac{T_B^{\gamma / (\gamma - 1)}}{T_B} \frac{dT_B}{dr} = \frac{P_B}{T_B} \frac{\gamma}{\gamma - 1} \frac{dT_B}{dr}. \]
Since the bubble is always in pressure equilibrium with its surroundings, we can define $P = P_B = P_S$.
Also, in practice our bubble is only going to be slightly hotter than its surroundings, so we can approximate $T = T_B \approx T_S$ to get
\[ \boxed{\frac{dP}{dr} = \frac{P}{T} \frac{\gamma}{\gamma - 1} \frac{dT}{dr}}. \]
This is the \textbf{convection energy transport equation}, and it's the last one we need to get a complete set of differential equations.

Convection, of course, will not always occur---the temperature gradient in the star must be greater than that in a bubble.
In particular,
\begin{align*}
    \frac{dT}{dr} &< \frac{\gamma - 1}{\gamma} \frac{T}{P} \frac{dP}{dr} \\
    \frac{dT}{T} &< \frac{\gamma - 1}{\gamma} \frac{dP}{P} \\
    d(\ln T) &< \frac{\gamma - 1}{\gamma} d(\ln P) \\
    \intertext{Since both differentials are negative,}
    \Aboxed{\frac{d(\ln P)}{d(\ln T)} &< \frac{\gamma}{\gamma - 1}}
\end{align*}
Convection occurs if and only if this condition is satisfied.
In this case, since convection is much more efficient than radiation, we may use only the convection equation.
Otherwise, we'll have to stick to radiation.

\section{Building Stellar Models}
We now have a complete set of equations that can be solved to obtain functional forms for the key properties of a star.
At $r=0$ we have the initial conditions $M_r = L_r = 0$ and at $r = R$ we have $\rho = T = 0$.

According to the \textbf{Vogt-Russel theorem}, the mass and composition structure of a star completely determines its radius, luminosity, internal structure, and subsequent evolution.
For example, the mass determines the energy transport structure of a star.
\begin{itemize}
    \item Smaller and cooler stars tend to have more neutral atoms and thus higher opacities in their outer layers.
    This leads to a steeper temperature gradient and thus convection.
    The coolest stars exhibit full convection over its entire radius.

    \item More massive stars tend to have higher luminosities and thus higher temperatures in their cores.
    This once again leads to a steeper temperature gradient and convection in the inner layers.
\end{itemize}
Mass also determines the nature of the nuclear fusion in stellar cores.
High temperatures facilitate the carbon-nitrogen-oxygen cycle, which fuses hydrogen into helium with a much higher energy generation rate.

We can even predict the lifetime of a star using its mass!
For main-sequence stars we have the relations
\[ \frac{L}{L_\odot} = \left( \frac{M}{M_\odot} \right)^{3.5} \,\text{ and }\, t_\textrm{nuc} = \frac{0.1 \eta M c^2}{L}, \]
where $\eta$ is the efficiency of nuclear fusion in a mass-$M$ star with luminosity $L$.
If we take the star's nuclear lifetime $t_\textrm{nuc}$ and divide by that of the Sun we get
\[ \frac{t_\textrm{nuc}}{t_{\textrm{nuc, }\odot}} = \frac{M}{M_\odot} \frac{L_\odot}{L} = \left( \frac{M}{M_\odot} \right)^{-2.5}. \]
So the star's fusion lifetime is inversely related to its mass!

It will be fruitful to put bounds on what a star's mass can physically be.
On the lower end the limiting factor is temperature---a main-sequence star must be hot enough to facilitate nuclear fusion.
This depends a bit on composition.
Let $Z$ be the mass proportion of ``metals'' in a star (so $X$ and $Y$ are the hydrogen and helium proportions); a metal-rich star with $Z = 0.02$ has $M_\textrm{min} \approx 0.072M_\odot$, while a metal-poor one with $Z = 0.0$ has $M_\textrm{min} \approx 0.09M_\odot$.

The upper bound is set by radiation pressure.
Consider another cylindrical parcel of gas that contains only $N$ hydrogen atoms.
For a very massive star, temperatures are high enough for hydrogen gas to be completely ionized at all radii.
The gravitational force on the parcel, significant only for the protons, is
\[ F_\textrm{g} = \frac{GM(Nm_p)}{R^2}, \]
and the force due to radiation pressure, applicable only to the electrons, is
\[ F_\textrm{rad} = \frac{4}{3} \frac{F}{c} \sigma_\textrm{tot} = \frac{4}{3} \frac{L}{4\pi R^2 c} N \sigma_T, \]
where the Thompson cross section $\sigma_T$ is that for the interaction between electrons and photons.
Though these two forces act on different sets of particles, the protons and electrons are bound together by the Coulomb force, so they move as a unit.

For the star to stay together we must have $F_\textrm{rad} < F_\textrm{g}$, so
\[ \frac{L}{3 \pi R^2 c} N\sigma_T < \frac{GM_\star N m_p}{R^2} \implies L < \frac{3\pi G c m_pM}{\sigma_T}. \]
This is called the \textbf{Eddington luminosity}.
Any brighter star will blow material off from its surface.
Skipping many details, it follows that $M_\textrm{max} \sim 70M_\odot$.

\end{document}
\documentclass[../p116main.tex]{subfiles}
\graphicspath{{\subfix{../figures/}}}

\begin{document}

\chapter{Wave Mechanics}
\section{Position States and Translations}
Now we'll leave the world of two-state systems to discuss a particle's position and momentum in one dimension.
We begin by introducing the position states $\ket{x}$, which satisfy $\hat x \ket{x} = x \ket{x}$ for $x \in \R$.
Of course, it isn't possible to obtain a single, definite value when we go to measure a particle's positions (a measuring device cannot have infinite precision).
Any physical ket is a superposition of position states with coefficients $dx \braket{x}{\psi}$:
\[ \ket{\psi} \int dx \ket{x} \braket{x}{\psi}. \]
(Integrals are over all space unless otherwise stated.)
Notice that if $\ket{\psi}$ is itself a position state $\ket{x'}$ then
\[ \ket{x'} = \int dx \ket{x} \braket{x}{x'}, \]
meaning $\braket{x}{x'} = \delta(x - x')$.
This $\delta$-function appears non-physical in isolation, but in all physical scenarios it appears as part of an integral.
For example, in normalizing $\ket{\psi}$ we write
\begin{align*}
    1 = \braket{\psi}{\psi} &= \left( \int dx \braket{\psi}{x} \bra{x} \right) \left( \int dx \ket{x} \braket{x} {\psi} \right) \\
    &= \iint dx' dx \braket{\psi}{x} \braket{x}{x'} \braket{x'}{\psi},
    \intertext{and since the $\braket{x}{x'} = \delta(x - x') \neq 0$ only for $x = x'$,}
    &= \int dx \braket{\psi}{x} \braket{x}{\psi}.
\end{align*}
We therefore identify $dx = \left| \braket{x}{\psi} \right|^2$ with the probability of finding a particle in the interval $[x, \, x + dx]$.
The amplitude density function $\braket{x}{\psi}$, often denoted $\psi(x)$, is called the wave function; we can use it to rewrite the normalization condition:
\[ 1 = \int dx \, \psi^*(x) \psi(x) = \int dx \, |\psi(x)|^2. \]
The expected position of a particle in a state $\ket{\psi}$ is given by
\begin{align*}
    \langle x \rangle = \bra{\psi} \hat x \ket{\psi} &= \int dx \bra{\psi} \hat x \ket{x} \braket{x}{\psi}, \\
    &= \int dx \braket{\psi}{x} x \braket{x}{\psi} \\
    &= \int dx \, |\psi(x)|^2 x,
\end{align*}
where in the second step we wrote $\ket{\psi}$ as a superposition (or, alternatively, inserted the identity operator $1 = \int dx \ket{x} \bra{x}$).
This matches with our understanding of continuous expectation values from probability!

Now we introduce a translation operator $\hat T(a)$ which has the effect $\hat T(a) \ket{x} = \ket{x+a}$.
For an arbitrary state the action is
\[ \hat T(a) \ket{\psi} = \hat T(a) \int dx' \ket{x'} \braket{x'}{\psi} \int dx' \ket{x' + a} \braket{x'}{\psi}, \]
so to shift a state is to shift all the basis kets it's comprised of by an amount $a$.
Note that $\hat T(a)$ must be unitary to conserve probability.

Using the above, the amplitude of finding the translated state $\ket{\psi'}$ at a position $x$ is
\begin{align*}
    \psi'(x) = \braket{x}{\psi'} &= \int dx' \braket{x}{x' + a} \braket{x'}{\psi} \\
    &= \int dx' \, \delta[x - (x' + a)] \braket{x'}{\psi} \\
    &= \braket{x-a}{\psi} = \psi(x-a),
\end{align*}
which is, of course, equivalent to shifting the wave function rightward by $a$.
Alternatively, we could take advantage of the fact that $\hat T(a)$ is unitary to write
\[ \bra{x} \hat T(a) \ket{\psi} = \braket{x-a}{\psi} = \psi(x-a), \]
as before.
We might view this through the lens of active and passive transformations---we may literally translate the ket rightward by $a$, or we can find the amplitude in a space that's been shifted leftward by $a$.

\section{The Momentum Operator}
Now we define a generator of translations $\hat p_x$ using an infinitesimal translation operator:
\[ \hat T(dx) = 1 - \frac{i}{\hbar} \hat p_x dx \;\implies\; \hat T(a) = e^{-i \hat p_x a / \hbar}. \]
Given that $\hat p_x$ is Hermitian (since $\hat T(dx)$ is unitary) and that it has units of linear momentum, we identify $\hat p_x$ as the operator corresponding to the $x$-component of linear momentum.

It is easy to show, using the definition of $\hat T(\delta x)$, that
\[ \left( \hat x \, \hat T(\delta x) - \hat T(\delta x) \hat x \right) \ket{\psi} = \left( \frac{-i \,\delta x}{\hbar} \right) [\hat x, \hat p_x] \ket{\psi}. \]
We can rewrite the left-hand side using the superposition expansion for $\psi$:
\begin{align*}
    & \left( \hat x \, \hat T(\delta x) - \hat T(\delta x) \hat x \right) \int dx \ket{x} \braket{x}{\psi} \\
    &\qquad\qquad = \hat x \int dx \ket{x + \delta x} \braket{x}{\psi} - \hat T(\delta x) \int dx \, x \ket{x} \braket{x}{\psi} \\
    &\qquad\qquad = \int dx (x + \delta x) \ket{x + \delta x} \braket{x}{\psi} - \int dx \, x \ket{x + \delta x} \braket{x}{\psi} \\
    &\qquad\qquad = \delta x \int dx \ket{x + \delta x} \braket{x}{\psi} = \delta x \int dx \ket{x} \braket{x}{\psi} = \delta x \ket{\psi},
\end{align*}
where in the penultimate step we've kept only the leading order in $\delta x$.
(Alternatively, we could substitute $x' = x + \delta x$ and Taylor-expand the wave function to leading order.)
Comparing this expression with $(-i \, \delta x / \hbar) [\hat x, \hat p_x] \ket{\psi}$ gives
\[ [\hat x, \hat p_x] = i\hbar. \]
The Heisenberg uncertainty principle $\Delta x \Delta p_x \geq \hbar / 2$ follows.
With this commutation relation in mind, let's now look at time evolution---if our particle has mass $m$ then the Hamiltonian is
\[ \hat H - \frac{\hat p^2}{2m} + V(\hat x) \]
for some potential $V$.
So the position expectation value changes like
\begin{align*}
    \frac{d \langle x \rangle}{dt} &= \frac{i}{2m\hbar} [\hat p_x^2, \hat x] \ket{\psi} \\
    &= \frac{i}{2m\hbar} \bra{\psi} \left( \hat p_x [\hat p_x, \hat x] + [\hat p_x, \hat x] \hat p_x \right) \ket{\psi} \\
    &= \frac{\bra{\psi} \hat p_x \ket{\psi}}{m} = \frac{\langle p_x \rangle}{m}.
\end{align*}
This is the familiar expression for classical linear momentum!
Further, we could show that
\[ \frac{d \langle p_x \rangle}{dt} = \frac{i}{\hbar} \bra{\psi} [\hat H, \hat p_x] \ket{\psi} = \langle -\frac{dV}{dx} \rangle, \]
which is just Newton's second law!
Note that the momentum is a constant of motion only when $[\hat H, \hat p_x]$, that is, when the Hamiltonian is translationally invariant (when $V(x)$ is independent of $x$).

Now we'll determine the representation of the momentum operator in position space.
The action of the infinitesimal translation operator is
\begin{align*}
    \hat T(\delta x) \ket{\psi} &= \int dx \ket{x + \delta x} \braket{x}{\psi} \\
    &= \int dx' \ket{x'} \braket{x' - \delta x}{\psi} \\
    \intertext{Taylor-expanding the wave function to first order,}
    &= \int dx' \ket{x'} \left( \braket{x'}{\psi} - \delta x \frac{\partial}{\partial x'} \braket{x'}{\psi} \right) \\
    &= \ket{\psi} - \delta x \int dx' \ket{x'} \frac{\partial}{\partial x'} \braket{x'}{\psi}.
\end{align*}
Comparing with the definition of $\hat T(\delta x)$ gives
\[ \hat p_x \ket{\psi} = \frac{\hbar}{i} \int dx' \ket{x'} \frac{\partial}{\partial x'} \braket{x'}{\psi}, \]
and so
\[ \bra{x} \hat p_x \ket{\psi} = \frac{\hbar}{i} \frac{\partial}{\partial x} \braket{x}{\psi}. \]
For an arbitrary state the expectation value is
\[ \bra{\psi} \hat p_x \ket{\psi} = \int dx' \braket{\psi}{x'} \frac{\hbar}{i} \frac{\partial}{\partial x'} \braket{x'}{\psi}. \]
All of this suggests that
\[ \hat p_x \xrightarrow[x \textrm{ basis}]{} \frac{\hbar}{i} \frac{\partial}{\partial x}. \]

\section{Momentum Space}
We now introduce a new set of momentum eigenstates satisfying $\hat p_x \ket{p} = p \ket{p}$.
Arbitrary states look like
\[ \ket{\psi} = \int dp \ket{p} \braket{p}{\psi}, \]
and of course $\braket{p'}{p} = \delta(p' - p)$.
The normalization condition suggests that we identify $dp \left| \braket{p}{\psi} \right|^2$ as the probability that a particle in $\ket{\psi}$ has a momentum in $[p, \, p + dp]$; the amplitude density $\braket{p}{\psi}$ is called the wave function in momentum space.

It will be useful to determine the momentum eigenstates' position-space wave functions $\braket{x}{p}$.
First we write
\[ \bra{x} \hat p_x \ket{p} = p \braket{x}{p} = \frac{\hbar}{i} \frac{\partial}{\partial x} \braket{x}{p}, \]
a differential equation which has solution $\braket{x}{p} = N e^{ipx / \hbar}$.
We could evaluate (using $\delta(x) = \frac{1}{2\pi} \int dk \, e^{ikx}$)
\[ \braket{p'}{p} = \int dx \braket{p'}{x} \braket{x}{p} \]
to find that $N = 1 / \sqrt{2\pi \hbar}$, and so
\[ \braket{x}{p} = \frac{1}{\sqrt{2\pi \hbar}} e^{ipx / \hbar}. \]
Inserting the identity operator here allows us to transform back and forth between position and momentum space:
\begin{align*}
    \braket{p}{\psi} &= \int dx \braket{p}{x} \braket{x}{\psi} = \int dx \frac{1}{\sqrt{2\pi \hbar}} e^{-ipx / \hbar} \braket{x}{\psi} \\
    \braket{x}{\psi} &= \int dp \braket{x}{p} \braket{p}{\psi} = \int dp \frac{1}{\sqrt{2\pi \hbar}} e^{ipx / \hbar} \braket{p}{\psi}
\end{align*}
These together show that $\braket{p}{\psi}$ and $\braket{x}{\psi}$ form a Fourier transform pair.

As an application, consider the Gaussian wave packet
\[ \braket{x}{\psi} = \psi(x) = N e^{-x^2 / 2a^2}, \quad N = \frac{1}{\sqrt{a \sqrt{\pi}}}. \]
By symmetry $\langle x \rangle = 0$, and by integration $\langle x^2 \rangle = a^2 / 2$ and so $\Delta x = a / \sqrt{2}$, which reflects the fact that the Gaussian narrows as $a \to 0$.
We could show, using the above integrals, that
\[ \braket{p}{\psi} = \sqrt{\frac{a}{\hbar \sqrt{\pi}}} e^{-p^2 a^2 / 2\hbar^2}. \]
Once again, by symmetry $\langle p_x \rangle = 0$, by integration $\langle p_x^2 \rangle = \hbar^2 / 2a^2$, and $ \Delta p_x = \hbar / a \sqrt{2}$.
Note that $\Delta x \Delta p_x = \hbar / 2$, so this Gaussian wave packet is a minimum uncertainty state.
(We could prove that it is \textit{the} minimum uncertainty state by requiring $\ket{\beta} = c \ket{\alpha}$ and $\bra{\psi} (\hat O + \hat O^\dagger) \ket{\psi} = 0$ in the proof of the uncertainty principle.)

One advantage of expressing states in momentum space is that free particles have $\hat H = \hat p_x^2 / 2m$, meaning momentum states are also energy eigenstates!
Thus
\[ \ket{\psi(t)} = e^{i \hat H t / \hbar} \int dp \ket{p} \braket{p}{\psi(0)} = \int dp \, e^{-ip^2 t / 2m\hbar} \ket{p} \braket{p}{\psi(0)}. \]
We put this into position space by prepending $\bra{x}$ which, after integrating, gives
\[ \psi(x,t) = \frac{1}{\sqrt{[a + (i\hbar t / ma)] \sqrt{\pi}}} \exp \left( -\frac{x^2}{2a^2 [1 + (i\hbar t / ma^2)]} \right). \]
If we compare $\psi^*(x,t) \psi(x,t)$ with the time-independent form from before, we see that
\[ \Delta x = \frac{a}{\sqrt{2}} \left( 1 + \frac{\hbar^2 t^2}{m^2 a ^{4}} \right)^{1 / 2}. \]

\section{Energy in a Harmonic Oscillator}
Given how ubiquitous harmonic oscillators are in physics, in this section we will ``solve'' these oscillators---that is, we will determine the eigenvalues and eigenkets of the Hamiltonian
\[ \hat H = \frac{p_x^2}{2m} + \frac{1}{2} m \omega^2 \hat x^2 \]
given the commutation relation $[\hat x, \hat p_x] = i\hbar$.
The operator methods employed here will be very similar in spirit to the ones we used to solve the angular momentum problem.

We start by defining two non-Hermitian operators that are reminiscent of our $\hat J_\pm = \hat J_x \pm i \hat J_y$, noting that we can't simply add $\hat x \pm i \hat p_x$ because of the unit discrepancy:
\[ \hat a = \sqrt{\frac{m\omega}{2\hbar}} \left( \hat x + \frac{i}{m\omega} \hat p_x \right), \qquad \hat a^\dagger = \sqrt{\frac{m\omega}{2\hbar}} \left( \hat x - \frac{i}{m\omega} \hat p_x \right). \]
The factor out front is just there to make the operators dimensionless; this produces $[\hat a, \hat a^\dagger] = 1$.
We also get
\[ \hat x = \sqrt{\frac{\hbar}{2m\omega}} (\hat a + \hat a^\dagger), \qquad \hat p_x = -i \sqrt{\frac{m\omega \hbar}{2}} (\hat a - \hat a^\dagger), \]
and so the Hamiltonian can be expressed as
\[ \hat H = \frac{\hbar \omega}{2} (\hat a^\dagger \hat a + \hat a \hat a^\dagger) = \hbar \omega \left( \hat a^\dagger \hat a + \frac{1}{2} \right). \]
Thus finding the eigenstates of $\hat H$ is equivalent to finding the eigenstates of the operator $\hat N = \hat a^\dagger \hat a$.
We'll do this now!
Denote the eigenstates by $\ket{\eta}$.
Note that the eigenvalues are non-negative---we have the expectation value
\[ \bra{\eta} \hat N \ket{\eta} = (\ket{\eta} \hat a^\dagger) (\hat a \ket{\eta}) = \eta \braket{\eta}{\eta}, \]
and since both of these inner products are zero we must have $\eta \geq 0$.
To see the effects of $\hat a$ and $\hat a^\dagger$ on the $\ket{\eta}$, we first determine the commutation relations
\begin{align*}
    [\hat N, \hat a] &= [\hat a^\dagger \hat a, \hat a] & [\hat N, \hat a^\dagger] &= [\hat a^\dagger \hat a, \hat a^\dagger] \\
    &= [\hat a^\dagger, \hat a] \hat a &&= \hat a^\dagger [\hat a, \hat a^\dagger] \\
    &= -\hat a, &&= \hat a^\dagger,
\end{align*}
and evaluate $\hat N \hat a^\dagger \ket{\eta}$ and $\hat N \hat a \ket{\eta}$:
\begin{align*}
    \hat N \hat a^\dagger \ket{\eta} &= (\hat a^\dagger \hat N + \hat a^\dagger) \ket{\eta} & \hat N \hat a \ket{\eta} &= (\hat a \hat N - \hat a) \ket{\eta} \\
    &= (\hat a^\dagger \eta + \hat a^\dagger) \ket{\eta} &&= (\hat a \eta - \hat a) \ket{\eta}, \\
    &= (\eta + 1) \hat a^\dagger \ket{\eta} &&= (\eta - 1) \hat a \ket{\eta},
\end{align*}
where we've used the commutation relations above to swap the order of the operations.
We can see, then, that $\hat a^\dagger \ket{\eta}$ is an eigenket of $\hat N$ with eigenvalue $\eta + 1$, and that $\hat a \ket{\eta}$ has eigenvalue $\eta - 1$, meaning these are raising and lowering operators.

The fact that $\eta \geq 0$ puts a restriction on how far we can lower our eigenvalues.
The minimum-eigenvalue ket must satisfy $\hat a \ket{\eta_\textrm{min}} = 0$, and if we apply the raising operator to this we get
\[ \hat a^\dagger \hat a \ket{\eta_\textrm{min}} = \eta_\textrm{min} \ket{\eta_\textrm{min}} = 0, \]
we can see that we must have $\eta_\textrm{min} = 0$.
So we label the lowest state as $\ket{0}$, and applying the $\hat a^\dagger$ $n$ times yields the state satisfying
\[ \hat N \ket{n} = n \ket{n}, \qquad n = 0, 1, 2, \ldots. \]
Thus the eigenvalues of $\hat N$---now called the number operator---are the non-negative integers, and those of the Hamiltonian are
\[ \hat H \ket{n} = \hbar \omega \left( n + \frac{1}{2} \right) \ket{n}, \qquad n = 0, 1, 2, \ldots. \]
So we've quantized the energy of the harmonic oscillator!
The spacing is uniform with step size $\hbar \omega$.

Like before, it will be useful to determine the matrix representations of our raising and lowering operators.
We'd like to solve for the $c_+, c_-$ satisfying $\hat a^\dagger \ket{n} = c_+ \ket{n+1}$ and $\hat a \ket{n} = c_- \ket{n-1}$.
If we take the inner product of the two equations
\[ \hat a^\dagger \ket{n} = c_+ \ket{n+1}, \qquad \bra{n} \hat a = c_+^* \bra{n+1} \]
we get $(n+1) \braket{n}{n} = c_+^* c_+ \braket{n+1}{n+1}$.
(Note that on the left we get an $\hat a \hat a^\dagger$, and so we use the commutation relation to swap the order of the operators.)
If the eigenstates are normalized then we must have $c_+ = \sqrt{n+1}$.
Similarly, $c_- = \sqrt{n}$, meaning
\[ \hat a^\dagger \ket{n} = \sqrt{n+1} \ket{n+1}, \qquad \hat a \ket{n} = \sqrt{n} \ket{n-1}, \]
and the matrix elements of these operators are
\[ \bra{n'} \hat a^\dagger \ket{n} = \sqrt{n+1} \,\delta_{n', n+1}, \qquad \bra{n'} \hat a \ket{n} = \sqrt{n} \, \delta_{n', n-1}. \]
The matrix representations in the energy eigenbasis, therefore, look like
\[ \hat a^\dagger \rightarrow \begin{bmatrix} 0 & 0 & \cdots \\ \sqrt{1} & 0 & \cdots \\ 0 & \sqrt{2} \\ \vdots & \vdots & \ddots \end{bmatrix}, \qquad \hat a \rightarrow \begin{bmatrix} 0 & \sqrt{1} & 0 & \cdots \\ 0 & 0 & \sqrt{2} & \cdots \\ \vdots & \vdots & \vdots & \ddots \end{bmatrix}. \]
We could construct the matrix representations of the position and momentum operators from here (again, in the energy eigenbasis).
At this point we can also establish. via induction, that a normalized ket $\ket{n}$ can be expressed as
\[ \ket{n} = \frac{(\hat a^\dagger)^n}{\sqrt{n!}} \ket{0}. \]

\section{Analysis of the Energy Eigenvalues and Eigenfunctions}
Now let's connect this discussion back to wave mechanics by determining the position-space eigenfunctions for the harmonic oscillator.
Projecting the ground-state $\hat a \ket{0} = 0$ into position space gives
\[ 0 = \bra{x} \hat a \ket{0} = \sqrt{\frac{m\omega}{2\hbar}} \bra{x} \left( \hat x + \frac{i}{m\omega} \hat p_x \right) \ket{0} = \sqrt{\frac{m\omega}{2\hbar}} \left( x \braket{x}{0} + \frac{i}{m\omega} \cdot \frac{\hbar}{i} \frac{\partial \braket{x}{0}}{\partial x} \right), \]
a differential equation which has the (normalized) solution
\[ \braket{x}{0} = \left( \frac{m\omega}{\pi \hbar} \right)^{1 / 4} e^{-m\omega x^2 / 2\hbar}. \]
We can use this to generate all of the position-space energy eigenfunctions:
\[ \braket{x}{n} = \frac{1}{\sqrt{n!}} \bra{x} (a^\dagger)^n \ket{0} = \frac{1}{\sqrt{n!}} \left( \frac{m\omega}{2\hbar} \right)^{n / 2} \left( x - \frac{\hbar}{m\omega} \frac{d}{dx} \right)^{n} \left( \frac{m\omega}{\pi \hbar} \right)^{1 / 4} e^{-m\omega x / 2\hbar}. \]
Note that both of the energies' expectation values
\begin{align*}
    \frac{\langle p_x^2 \rangle}{2m} &= -\frac{\hbar^2}{2m} \int dx \braket{n}{x} \frac{d^2}{dx^2} \braket{x}{n}, \\
    \langle V(x) \rangle &= \frac{1}{2} m\omega^2 \int dx \braket{n}{x} x^2 \braket{x}{n}.
\end{align*}
get larger with increasing $n$ due to the wave function's increasing curvature (for the kinetic energy) and widening domain of applicability (for the potential).

Let's hone in on the ground-state energy.
In a classical oscillator the lowest energy occurs when the particle is at rest, but for us this is disallowed by the Heisenberg uncertainty principle.
These uncertainties influence the energy via
\[ \langle E \rangle = \frac{(\Delta p_x)^2 + \langle p_x \rangle^2}{2m} + \frac{1}{2} m\omega^2 [(\Delta x)^2 + \langle x \rangle^2]. \]
It's easy to show, using the raising and lowering operators, that $\langle x \rangle = \langle p_x \rangle = 0$ in an energy eigenstate, and so in such a state
\[ \langle E \rangle = \frac{(\Delta p_x)^2}{2m} + \frac{1}{2} m\omega^2 (\Delta x)^2. \]
To minimize the energy nature picks the combination of $(\Delta x)^2 = \langle x^2 \rangle$ and $(\Delta p_x)^2 = \langle p_x^2 \rangle$ that minimizes this expression.
In the ground state these quantities are
\begin{align*}
    (\Delta x)^2 &= \frac{\hbar}{2m \omega} \bra{0} (\hat a + \hat a^\dagger)^2 \ket{0} & (\Delta p_x)^2 &= -\frac{m\omega \hbar}{2} \bra{0} (\hat a - \hat a^\dagger)^2 \ket{0} \\
    &= \frac{\hbar}{2m \omega} \bra{0} (\hat a^2 + (\hat a^\dagger)^2 + \hat a \hat a^\dagger + \hat a^\dagger \hat a) \ket{0} &&= -\frac{m\omega \hbar}{2} \bra{0} (\hat a^2 + (\hat a^\dagger)^2 - \hat a \hat a^\dagger - \hat a^\dagger \hat a) \ket{0} \\
    &= \frac{\hbar}{2m\omega} \bra{0} \hat a \hat a^\dagger \ket{0} = \frac{\hbar}{2m\omega}, &&= \frac{m\omega \hbar}{2} \bra{0} \hat a \hat a^\dagger \ket{0} = \frac{m\omega\hbar}{2},
\end{align*}
which notably produces $\Delta x \Delta p_x = \hbar / 2$.
So the ground state, as we found before---our $\braket{x}{0}$ is a Gaussian, after all---is a minimum-uncertainty state!
For the excited states we similarly have
\[ \Delta x = \sqrt{\left( n + \frac{1}{2} \right) \frac{\hbar}{m\omega}}, \qquad \Delta p_x = \sqrt{\left( n + \frac{1}{2} \right) m\omega \hbar}, \]
which produce $\Delta x \Delta p_x = (n + 1 / 2) \hbar$.

Now let's zoom back out.
The correspondence principle states that we should expect that the predictions of quantum mechanics to agree with those of classical physics on macroscopic scales---that is, scales on which our quantum number $n$ is very large.
On such a scale, the classical motion of a particle with energy $E$ is restricted to lie within the classical turning points at which the kinetic energy is zero:
\[ E_n = \left( n + \frac{1}{2} \right) \hbar \omega = \frac{1}{2} m \omega^2 x_n^2. \]
This agrees with the (visual) observation that the eigenfunctions' excursions beyond these turning points becomes less pronounced as $n$ increases.

For a classical oscillator, the probability of finding an energy-$E$ particle in $[x, x + dx]$ is proportional to the amount of time $dx / v$ it spends in this interval.
We can express this probability as
\[ P_\textrm{cl} \, dx \propto \frac{dx}{(2 / m)(E_n - V(x))} = \frac{dx}{\sqrt{2 E_n / m - \omega^2 x^2}} = \frac{dx}{\omega \sqrt{x_n^2 - x^2}}, \]
and normalizing gives a factor of $1 / \sqrt{\pi}$ out front.
Plotting this over a a $\left| \braket{x}{n} \right|^2$ (for reasonably large $n$) reveals that the $P_\text{cl}$ we've found approximates the mean value of $\left| \braket{x}{n} \right|^2$ at any given position!

\section{Time Dependence and Coherent States}
The energy eigenstates of the harmonic oscillator are stationary states.
So time dependence, as we've come to expect, results from superpositions of eigenstates with different energies.
If this superposition is of two adjacent energy states $\ket{\psi(0)} = c_n \ket{n} + c_{n+1} \ket{n+1}$ then the time evolution is
\[ \ket{\psi(t)} = e^{-i(n + 1 / 2) \omega t} \left( c_n \ket{n} + c_{n+1} e^{-i\omega t} \ket{n+1} \right), \]
and we could use the raising and lowering operators to compute $\langle x \rangle = A \cos(\omega t + \delta)$.

Another example of a state with interesting time evolution is the coherent state $\ket{\alpha}$, which satisfies $\hat a \ket{\alpha} = \alpha \ket{\alpha}$.
If we start by writing such a state as a superposition of the energy eigenstates with coefficients $c_n$, the eigenvalue equation becomes   \vspace{-6pt}
\begin{align*}
    \alpha \sum_{n=0}^{\infty} c_n \ket{n} &= \sum_{n=1}^{\infty} \sqrt{n} \,c_n \ket{n-1} \\
    &= \sum_{n'=0}^{\infty} \sqrt{n'+1} \,c_{n'+1} \ket{n'}.
\end{align*}
So all of the coefficients must satisfy the recurrence $\sqrt{n+1} \, c_{n+1} = \alpha c_n$, the solution of which is $c_n = (\alpha^n / \sqrt{n!}) c_0$ and thus
\[ \ket{\alpha} = c_0 \sum_{n=0}^{\infty} \frac{\alpha^n}{\sqrt{n!}} \ket{n}. \]
To normalize we could evaluate $\braket{\alpha}{\alpha} = |c_0|^2 e^{|\alpha|^2}$, meaning $c_0 = e^{-|\alpha|^2 / 2}$ up to an overall phase and so
\[ \ket{\alpha} = e^{-|\alpha|^2 / 2} \sum_{n=0}^{\infty} \frac{\alpha^{n}}{\sqrt{n!}} \ket{n}. \]
The time evolution is given by
\begin{align*}
    \ket{\alpha(t)} &= e^{-|\alpha|^2 / 2} \sum_{n=0}^{\infty} \frac{\alpha^{n} e^{-i(n + 1 / 2) \omega t}}{\sqrt{n!}} \ket{n} \\
    &= e^{i \omega t / 2} e^{-|\alpha|^2 / 2} \sum_{n=0}^{\infty} \frac{( e^{-i \omega t})^{n}}{\sqrt{n!}} \ket{n} \\
    &= e^{-i \omega t / 2} \ket{\alpha e^{-i \omega t}},
\end{align*}
so the eigenvalue $\alpha$ of the lowering operator becomes $\alpha e^{i \omega t}$ as time progresses.

As a first step in investigating the behavior of this coherent state, let's compute the expectation values:
\begin{align*}
    \langle x \rangle &= \sqrt{\frac{\hbar}{2m\omega}} \bra{\alpha(t)} (\hat a + \hat a^\dagger) \ket{\alpha(t)} & \langle p_x \rangle &= -i\sqrt{\frac{m\omega\hbar}{2}} \bra{\alpha(t)} (\hat a - \hat a^\dagger) \ket{\alpha(t)} \\
    &= \sqrt{\frac{\hbar}{2m\omega}} \left( \alpha e^{-i\omega t} + \alpha^* e^{i\omega t} \right) &&= -i\sqrt{\frac{m\omega\hbar}{2}} \left( \alpha e^{-i \omega t} - \alpha^* e^{i\omega t} \right) \\
    &= \sqrt{\frac{\hbar}{2m\omega}} \, 2|\alpha| \cos (\omega t + \delta), &&= -\sqrt{\frac{m\omega\hbar}{2}} \, 2|\alpha| \sin (\omega t + \delta),
\end{align*}
where $\alpha = |\alpha| e^{-i\delta}$.
We can see that the position and momentum are oscillating back and forth, as we'd expect from motion in a harmonic oscillator.
Note that the phase $\delta$ of the eigenvalue determines the phase of these expectation values.
Now we'll compute the uncertainties, using the commutator $[\hat a, \hat a^\dagger] = 1$ to swap the order of operators as needed:
\begin{align*}
    \langle x^2 \rangle &= \frac{\hbar}{2m\omega} \bra{\alpha(t)} (\hat a + \hat a^\dagger)^2 \ket{\alpha(t)} & \langle p_x^2 \rangle &= -\frac{m\omega \hbar}{2} \bra{\alpha(t)} (\hat a - \hat a^\dagger)^2 \ket{\alpha(t)} \\
    &= \frac{\hbar}{2m\omega} \bra{\alpha(t)} (\hat a + (\hat a^\dagger)^2 + 2 \hat a^\dagger \hat a + 1) \ket{\alpha(t)} &&= -\frac{m\omega\hbar}{2} \bra{\alpha(t)} (\hat a^2 + (\hat a^\dagger)^2 - 2\hat a^\dagger \hat a - 1) \ket{\alpha(t)} \\
    &= \frac{\hbar}{2m\omega} \left( \alpha(t) + \alpha^*(t)^2 + 2|\alpha(t)|^2 + 1 \right), &&= \frac{m\omega\hbar}{2} \left( 2|\alpha(t)|^2 + 1 - \alpha(t)^2 - \alpha^*(t)^2 \right),
\end{align*}
which combined with the previous expectation values gives
\[ (\Delta x)^2 = \frac{\hbar}{2m\omega}, \qquad (\Delta p_x)^2 = \frac{m\omega\hbar}{2}. \]
The product of the uncertainties is $\Delta x \Delta p_x = \hbar / 2$.
Thus the coherent state is a minimum uncertainty state, and unlike the Gaussian wave packet for the free particle, this coherent Gaussian does not spread with time.
This is just about as close as we can get to a quantum state that might represent, say, a classical pendulum!

\section{The Schrödinger Equation and Inversion Symmetry}
As an alternative way of solving the harmonic oscillator, we can write down the energy eigenvalue equation as a differential equation:
\[ -\frac{\hbar^2}{2m} \frac{d^2}{dx^2} \braket{x}{E} + \frac{1}{2} m \omega^2 x^2 \braket{x}{E} = E \braket{x}{E}. \]
If we substitute $y = x \sqrt{m\omega / \hbar}$ we get
\[ \frac{d^2 \psi}{dy^2} + (\varepsilon - y^2) \psi = 0, \qquad \varepsilon = \frac{2E}{\hbar \omega}. \]
As a first step toward finding physical solutions, we note that the solutions in the $|y| \to \infty$ limit look like $\psi(y) = A e^{-y^2 / 2} + B e^{y^2 / 2}$.
So to ``factor out'' this limiting behavior we can express the wave function as
\[ \psi(y) = h(y) e^{-y^2 / 2}, \]
where we have dropped the non-normalizable term.
Substituting this into the Schrödinger equation reveals that $h$ must satisfy
\[ \frac{d^2 h}{dy^2} - 2y \frac{d h}{dy} + (\varepsilon - 1) h = 0, \]
which has the power series solution
\[ h(y) = \sum_{k=0}^{\infty} a_k y^{k}, \qquad \frac{a_{k+2}}{a_k} = \frac{2k+1 - \varepsilon}{(k+2)(k+1)}. \]
But there's a problem---this solution exhibits the same behavior as $e^{y^2}$ in the large-$k$ limit, so it diverges.
To reconcile this we note that $h(y)$ is finite so long as $\varepsilon = 2n + 1$ for some integer $n$, so $\psi(y)$ becomes a finite polynomial multiplied by a decreasing exponential.
Since $\varepsilon = 2E / \hbar \omega$ the corresponding energies are
\[ E_n = \left( n + \frac{1}{2} \right) \hbar \omega, \qquad n = 0, 1, 2, \ldots. \]
Each $h(y)$ is called the Hermite polynomial of order $n$, and we may express it either in power series form or using our previous eigenfunctions for the harmonic oscillator.

One of the most striking features of the energy eigenfunctions is that they're all either even or odd.
This behavior is rooted in a symmetry in the Hamiltonian.
To formalize this symmetry we introduce the parity operator $\hat \Pi$, which acts on position states via
\[ \hat \Pi \ket{x} = \ket{-x}. \]
An eigenstate of this operator satisfies $\hat \Pi \ket{\psi_A} = \lambda \ket{\psi_A}$, and since $\hat \Pi^2$ is the identity operator, the eigenvalues are $\lambda = \pm 1$.
For the corresponding eigenstates note that for an arbitrary state $\ket{\psi}$ we have $\bra{x} \hat \Pi^2 \ket{\psi} = \braket{-x}{\psi} = \psi(-x)$, so the parity eigenfunctions satisfy
\[ \psi_\lambda(-x) = \lambda \psi_\lambda(x). \]
Thus $\lambda = 1$ corresponds to the even eigenfunctions and $\lambda = -1$ the odd ones.

Importantly, the harmonic oscillator Hamiltonian and the parity operator commute:
\begin{align*}
    \bra{x} \hat \Pi \hat H \ket{\psi} &= \bra{-x} \hat H \ket{\psi} \\
    &= \left( -\frac{\hbar^2}{2m} \frac{d^2}{dx^2} + V(-x) \right) \psi(-x) \\
    &= \left( -\frac{\hbar^2}{2m} \frac{d^2}{dx^2} + V(x) \right) \psi(-x) \\
    &= \bra{x} \hat H \hat \Pi \ket{\psi}.
\end{align*}
In fact, this relation holds for every even choice $V$!
By observing the symmetry of the Hamiltonian under this inversion (among others), we can deduce some of the properties of the eigenfunctions before we've even solved the eigenvalue equation.
This fact will come in handy in the next chapter.

\end{document}
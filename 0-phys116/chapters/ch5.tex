\documentclass[../p116main.tex]{subfiles}
\graphicspath{{\subfix{../figures/}}}

% listen to led zepplin (start with 4, untitled)

\begin{document}

\chapter{More Complex Systems}
\section{Time-Independent Perturbation Theory}
Lots of the systems we care about don't have Hamiltonians that we can solve exactly.
But in many cases they are sufficiently close to ones that we can solve, in which case we can treat the differences as small ``perturbations'' and real with them in a systematic way.
We begin by expressing such a Hamiltonian in the form
\[ \hat H = \hat H_0 + \hat H_1, \]
where we say the ``big'' $\hat H_0$ is the unperturbed Hamiltonian and the ``small'' $\hat H_1$ is the perturbing Hamiltonian.
The unperturbed part has known eigenstates and eigenvalues that satisfy
\[ \hat H_0 \ket{\varphi_n^{(0)}} = E_n^{(0)} \ket{\varphi_n^{(0)}}, \]
and we seek an approximate solution to $\hat H \ket{\psi_n} = E_n \ket{\psi_n}$.
To this end we introduce a ``tuning'' parameter $\lambda$ via $\hat H = \hat H_0 + \lambda \hat H_1$ and assume that we can express the exact eigenstates and eigenvalues as power-series expansions in $\lambda$:  \vspace{-10pt}
\begin{align*}
    \ket{\psi_n} &= \ket{\varphi_n^{(0)}} + \lambda \ket{\varphi_n^{(1)}} + \lambda^2 \ket{\varphi_n^{(2)}} + \cdots, \\
    E_n &= E_n^{(0)} + \lambda E_n^{(1)} + \lambda^2 E_n^{(2)} + \cdots.
\end{align*}
We require that as $\lambda \to 0$, the energies and eigenstates go smoothly to the exact values for the unperturbed Hamiltonian.
Now, if we substitute these expressions into the energy eigenvalue equation we get
\begin{align*}
    \left( \hat H_0 + \lambda \hat H_1 \right) & \left( \ket{\varphi_n^{(0)}} + \lambda \ket{\varphi_n^{(1)}} + \lambda^2 \ket{\varphi_n^{(2)}} + \cdots \right) \\
    &= \left( E_n^{(0)} + \lambda E_n^{(1)} + \lambda^2 E_n^{(2)} + \cdots \right) \left( \ket{\varphi_n^{(0)}} + \lambda \ket{\varphi_n^{(1)}} + \lambda^2 \ket{\varphi_n^{(2)}} + \cdots \right).
\end{align*}
Expanding and collecting like terms yields
\begin{align*}
    \hat H_0 \ket{\varphi_n^{(0)}} \phantom{\;\; + \hat H_0 \ket{\varphi_n^{(0)}}} &= E_n^{(0)} \ket{\varphi_n^{(0)}}, \\
    \hat H_0 \ket{\varphi_n^{(1)}} + \hat H_1 \ket{\varphi_n^{(0)}} &= E_n^{(0)} \ket{\varphi_n^{(1)}} + E_n^{(1)} \ket{\varphi_n^{(0)}}, \\
    \hat H_0 \ket{\varphi_n^{(2)}} + \hat H_1 \ket{\varphi_n^{(1)}} &= E_n^{(0)} \ket{\varphi_n^{(2)}} + E_n^{(1)} \ket{\varphi_n^{(1)}} + E_n^{(2)} \ket{\varphi_n^{(0)}}
\end{align*}
for the $\lambda^{0}$, $\lambda^{1}$, and $\lambda^2$ terms, respectively.
A pattern emerges in higher-order terms, but these suffice for us.
These equations encode information about the corrections we must make to the eigenstates and energies!

The first-order correction to the energy is found by applying $\bra{\varphi_n^{(0)}}$ to the $\lambda^{1}$ equation:
\begin{align*}
    \bra{\varphi_n^{(0)}} \hat H_0 \ket{\varphi_n^{(1)}} + \bra{\varphi_n^{(0)}} \hat H_1 \ket{\varphi_n^{(0)}} &= \bra{\varphi_n^{(0)}} E_n^{(0)} \ket{\varphi_n^{(1)}} + \bra{\varphi_n^{(0)}} E_n^{(1)} \ket{\varphi_n^{(0)}}, \\
    \bra{\varphi_n^{(0)}} \hat H_1 \ket{\varphi_n^{(0)}} &= E_n^{(1)}.
\end{align*}
Thus the first-order shift in the $n$th energy level is simply the perturbing $\langle E \rangle$ for the $n$th unperturbed state.
If we were to instead apply some other $\bra{\varphi_k^{(0)}}$ with $k \neq n$ we'd get \vspace{-6pt}
\[ \bra{\varphi_k^{(0)}} \hat H_0 \ket{\varphi_n^{(1)}} + \bra{\varphi_k^{(0)}} \hat H_1 \ket{\varphi_n^{(0)}} = E_n^{(0)} \braket{\varphi_k^{(0)}}{\varphi_n^{(1)}} \;\implies\; \braket{\varphi_k^{(0)}}{\varphi_n^{(1)}} = \frac{\bra{\varphi_k^{(0)}} \hat H_1 \ket{\varphi_n^{(0)}}}{E_n^{(0)} - E_k^{(0)}}, \vspace{-6pt} \]
meaning $\ket{\varphi_n^{(1)}}$ generally has a nonzero component along $\ket{\varphi_k^{(0)}}$ for $k \neq n$.
For the $k=n$ case we use the expansion from before to write
\[ 1 = \braket{\psi_n}{\psi_n} = \braket{\varphi_n^{(0)}}{\varphi_n^{(0)}} + \lambda \braket{\varphi_n^{(0)}}{\varphi_n^{(1)}} + \lambda \braket{\varphi_n^{(1)}}{\varphi_n^{(0)}} + \cdots, \]
meaning $\braket{\varphi_n^{(0)}}{\varphi_n^{(1)}}$ is purely imaginary.
Inserting the identity operator into the $\ket{\psi_n}$ expansion gives
\begin{align*}
    \ket{\psi_n} &= \ket{\varphi_n^{(0)}} + ia \, \lambda \ket{\varphi_n^{(0)}} + \lambda \sum_{k \neq n}^{} \ket{\varphi_k^{(0)}} \braket{\varphi_k^{(0)}}{\varphi_n^{(1)}} + \mathcal O (\lambda^2) \\
    &= e^{ia \lambda} \ket{\varphi_n^{(0)}} + \lambda \sum_{k \neq n}^{} \ket{\varphi_k^{(0)}} \braket{\varphi_k^{(0)}}{\varphi_n^{(1)}} + \mathcal O (\lambda^2). \\
    \intertext{Even after normalizing $\ket{\psi_n}$ it turns out that $a$ is not determined, so we can set $a = 0$ so that the first-order correction is orthogonal to the unperturbed state and}
    &= \ket{\varphi_n^{(0)}} + \lambda \sum_{k \neq n}^{} \ket{\varphi_k^{(0)}} \frac{\bra{\varphi_k^{(0)}} \hat H_1 \ket{\varphi_n^{(0)}}}{E_n^{(0)} - E_k^{(0)}} + \mathcal O(\lambda^2).
\end{align*}
This summation is the $\ket{\varphi_n^{(1)}}$ we sought.
For the second-order energy shift we take the $\lambda^2$ equation and apply $\bra{\varphi_n^{(0)}}$; after some cancellation we can substitute this expression for $\ket{\varphi_n^{(1)}}$ to get
\begin{align*}
    E_n^{(2)} &= \bra{\varphi_n^{(0)}} \hat H_1 \ket{\varphi_n^{(1)}} \\
    &= \sum_{k \neq n}^{} \frac{\bra{\varphi_n^{(0)}} \hat H_1 \ket{\varphi_k^{(0)}} \bra{\varphi_k^{(0)}} \hat H_1 \ket{\varphi_n^{(0)}}}{E_n^{(0)} - E_k^{(0)}} = \sum_{k \neq n}^{} \frac{\left| \bra{\varphi_k^{(0)}} \hat H_1 \ket{\varphi_n^{(0)}} \right|^2}{E_n^{(0)} - E_k^{(0)}}.
\end{align*}
Everything we've done so far has assumed that there is no degeneracy in $E_n^{(0)}$.
But if there is, we then face the additional problem of figuring out which combinations of the degenerate states will converge to the correct eigenstates and energies of $\hat H$.
To solve this problem we first allow for degeneracy in writing
\[ \ket{\psi_n} = \sum_{i=1}^{N} c_i \ket{\varphi_{n,i}^{(0)}} + \lambda \ket{\varphi_n^{(1)}} + \cdots, \]
and if we substitute this into the energy eigenvalue equation we obtain, for the $\lambda^{1}$ term,
\[ \hat H_0 \ket{\varphi_n^{(1)}} + \hat H_1 \sum_{i=1}^{N} c_i \ket{\varphi_{n,i}^{(0)}} = E_n^{(0)} \ket{\varphi_n^{(1)}} + E_n^{(1)} \sum_{i=1}^{N} c_i \ket{\varphi_{n,i}^{(0)}}. \]
Note the similarities to the $\lambda^{1}$ equation from before.
Now we apply each of the degenerate $\bra{\varphi_{n,j}^{(0)}}$ to get
\[ \sum_{i=1}^{N} \bra{\varphi_{n,j}^{(0)}} \hat H_1 \ket{\varphi_{n,i}^{(0)}} c_i = E_n^{(1)} \sum_{i=1}^{N} \braket{\varphi_{n,j}^{(0)}}{\varphi_{n,i}^{(0)}} = E_n^{(1)} \sum_{i=1}^{N} \delta_{ji} c_i = E_n^{(1)} c_j, \]
where we have assumed that the degenerate states are orthonormal.
We can recognize this as the $j$th component of an energy eigenvalue equation!
In the $N=2$ case the equation looks like
\[ \begin{bmatrix} \bra{\varphi_{n,1}^{(0)}} \hat H_1 \ket{\varphi_{n,1}^{(0)}} & \bra{\varphi_{n,1}^{(0)}} \hat H_1 \ket{\varphi_{n,2}^{(0)}} \\ \bra{\varphi_{n,2}^{(0)}} \hat H_1 \ket{\varphi_{n,1}^{(0)}} & \bra{\varphi_{n,2}^{(0)}} \hat H_1 \ket{\varphi_{n,2}^{(0)}} \end{bmatrix} \begin{bmatrix} c_1 \\ c_2 \end{bmatrix} = E_n^{(1)} \begin{bmatrix} c_1 \\ c_2 \end{bmatrix}. \]
Notice howe we're looking only at how the Hamiltonian behaves in the subspace of degenerate states
The eigenvalues are the first-order shifts, and the eigenstates are the ``correct'' linear combinations of the degenerate states that we sought.

\section{Perturbing Electric Fields}
As a first example we will apply an external electric field $\mbf{E} = |\mbf{E}| \mbf{z}$ to the hydrogen atom.
The relevant Hamiltonians are
\[ \hat H_0 = \frac{\hat{\mbf{p}}^2}{2\mu} - \frac{q_e^2}{|\hat{\mbf{r}}|}, \qquad \hat H_1 = q_e |\mbf{E}| \hat z. \]
The ground-state first-order shift can be easily computed using the nondegenerate theory: we could show that the eigenstates $\ket{n,l,m}$ have definite parity $(-1)^{l}$, meaning the position expectation value
\[ E_1^{(1)} = q_e |\mbf{E}| \bra{1,0,0} \hat z \ket{1,0,0} = 0. \]
The second-order shift is
\[ E_1^{(2)} = \sum_{\substack{n,l,m \\ n \neq 1}}^{} \frac{q_e^2 |\mbf{E}|^2 \left| \bra{n,l,m} \hat z \ket{1,0,0} \right|^2}{E_1^{(0)} - E_n^{(0)}}. \]
As for the first-excited states, the degenerate subspace is spanned by $\left\{ \ket{2,0,0}, \ket{2,1,0}, \ket{2,1,1}, \ket{2,1,-1} \right\}$.
In principle this gives us sixteen matrix elements to compute, but we can use symmetry to simplify things considerably.
For one, all of the elements in which the bra and ket have the same $l$-value vanish immediately, since the parity $(-1)^{l}$ depends only on $l$.

For most of the remaining states, we can also take advantage of the fact that $[\hat H_1, \hat L_z] = 0$ to write 
\begin{align*}
    m' \hbar \bra{n, l', m'} \hat z \ket{n,l,m} &= \bra{n, l', m'} \hat L_z \hat z \ket{n,l,m} \\
    &= \bra{n, l', m'} \hat z \hat L_z \ket{n,l,m} \\
    &= m\hbar \bra{n, l', m'} \hat z \ket{n,l,m}.
\end{align*}
Thus the matrix elements with different $m$-values vanish.
(This is true in general---if an operator $\hat A$ commutes with $\hat H_1$ then matrix elements with different eigenvalues $a$ vanish.)
Thus the only entry we must evaluate is
\[ \bra{2,0,0} \hat H_1 \ket{2,1,0} = q_e |\mbf{E}| \int_{0}^{\infty} r^2 dr \int_{0}^{\pi} \sin \theta \,d\theta \int_{0}^{2\pi} d\phi \, R_{2,0}^* Y_{0,0}^* r \cos \theta \, R_{2,1} Y_{1,0} = -3q_e |\mbf{E}| a_0, \]
where $a_0$ is the Bohr radius.
We can use this to construct the matrix representation of $\hat H_1$, and some computations give the first-order shifts $E_2^{(1)} = 0, \; 0, \; 3q_e|\mbf{E}| a_0, \; -3q_e|\mbf{E}| a_0$ with corresponding kets
\[ \ket{2,1,1}, \quad \ket{2,1,-1}, \quad \frac{1}{\sqrt{2}} \big( \ket{2,0,0} - \ket{2,1,0} \big), \quad \frac{1}{\sqrt{2}} \big( \ket{2,0,0} + \ket{2,1,0} \big), \]
respectively.
This splitting is known as the Stark effect.
Notice that, as a consequence of $[\hat H_1, \hat L_z] = 0$, eigenkets with different $m$-values are not mixed by the perturbation.
Commutativity therefore would have allowed us to focus only on the $m=0$ states, analyzing the others using the nondegenerate theory.

As a second example, consider again the ammonia molecule in a weak external electric field.
We have, in the energy eigenbasis,
\[ \hat H_0 \to \begin{bmatrix} E_0 - A & 0 \\ 0 & E_0 + A \end{bmatrix}, \qquad \hat H_1 \to \begin{bmatrix} 0 & \mu_e |\mbf{E}| \\ \mu_e |\mbf{E}| & 0 \end{bmatrix}. \]
The vanishing first-order shift is explained by $\hat \Pi \ket{I} = \ket{I}$ and $\hat \Pi \ket{II} = -\ket{II}$, as the electric dipole moment operator has a vanishing expectation value in a state of definite parity.
Notice how this conclusion is in agreement with the Taylor expansion
\[ E = E_0 \pm A \pm A \left[ \frac{1}{2} \left( \frac{\mu_e |\mbf{E}|^2}{A} \right)^2 - \frac{1}{8} \left( \frac{\mu_e |\mbf{E}|^2}{A} \right)^4 + \cdots \right]. \]
But in the case of a strong electric field $\mu_e |\mbf{E}| \gg A$ the expansion is
\[ E = E_0 \pm \mu_e |\mbf{E}| \pm \frac{A^2}{2\mu_e |\mbf{E}|} \mp \cdots, \]
which has a term that is linear in $|\mbf{E}|$.
We therefore have a nonzero first-order shift, suggesting that we cannot use the nondegenerate theory here.
Roughly speaking, this is because the states $\ket{I}$ and $\ket{II}$ are very close together in energy, so the factor
\[ \frac{\bra{II} \hat H_1 \ket{I}}{E_{II}^{(0)} - E_I^{(0)}} = \frac{\mu_e |\mbf{E}|}{2A} \]
diverges.
Thus we use the degenerate theory on the $\ket{1}$-$\ket{2}$ representations of $\hat H_0$ and $\hat H_1$ to determine the relevant information.

\section{Relativistic Perturbations to the Hydrogen Atom}
We've already studied the hydrogen atom extensively, but now we'll discuss three relativistic perturbations to the Hamiltonian that together create a fine structure to our previously-derived energy levels.
First, we write the relativistically correct expression for the kinetic energy in the system (using $m_p \gg m_e$):
\begin{align*}
    \hat K &= m_e c^2 \left( \sqrt{1 + (\hat{\mbf{p}}_e^2 / m_e^2 c^2)} - 1 \right) + \frac{\hat{\mbf{p}}_p^2}{2m_p} \\
    &= \frac{\hat{\mbf{p}}_e^2}{2m_e} - \frac{(\hat{\mbf{p}}_e^2)^2}{8m_e^3 c^2} + \cdots + \frac{\hat{\mbf{p}}_p^2}{2m_p} \;=\; \frac{\hat{\mbf{p}}^2}{2\mu} - \frac{(\hat{\mbf{p}}^2)^2}{8m_e^3 c^2} + \cdots.
\end{align*}
Our perturbing Hamiltonians is therefore $\hat H_K = -(\hat{\mbf{p}}^2)^2 / 8m_e^3 c^2$.
Conveniently, $[\hat H_K, \hat{\mbf{L}}^2] = 0$, so we can use the nondegenerate theory to compute the first-order shift
\begin{align*}
    E_{n,l}^{(1)} &= - \bra{n,l,m} \frac{(\hat{\mbf{p}}^2)^2}{8m_e^3 c^2} \ket{n,l,m} \\
    &= -\frac{1}{2m_e c^2} \bra{n,l,m} \left( \hat H_0 + \frac{Zq_e^2}{|\hat{\mbf{r}}|} \right)^2 \ket{n,l,m}.
\end{align*}
Going through the motions here reveals that
\[ E_K^{(1)} = -\frac{1}{2} m_e c^2 Z^{4} \alpha^{4} \left( -\frac{3}{4n^{4}} + \frac{1}{n^3 (l + 1 / 2)} \right). \]
Now we'll look at the interaction between the electron's spin and its orbital angular momentum.
In the electron's rest frame the proton generates a magnetic field $\mbf{B}$, and so the interaction energy is
\[ -\bm \mu \cdot \mbf{B} = - \left( -\frac{ge}{2m_e c} \mbf{S} \cdot \frac{-Z q_e \mbf{v} \times \mbf{r}}{c r^3} \right) = \frac{Zq_e^2}{m_e^2 c^2 r^3} \mbf{S} \cdot \mbf{L}, \]
where $g = 2$ for the electron, and $\mbf{L}$ and $\mbf{v}$ are the electron's angular momentum and velocity (in the proton's frame).
An effect called Thomas precession reduces this energy by a factor of two, so
\[ \hat H_\textrm{S-O} = \frac{Zq_e^2}{2m_e^2 c^2 |\hat{\mbf{r}}|^3} \hat{\mbf{L}} \cdot \hat{\mbf{S}}. \]
The work we'll do here is very similar to what we did much earlier for the spin-spin interaction between the proton and electron.
% We first note that $\hat{\mbf{L}}^2$, $\hat L_z$, $\hat{\mbf{S}}^2$, and $\hat S_z$ all commute with one another, so we can form simultaneous eigenstates $\ket{l, m_l, \pm \mbf{z}} = \ket{l,m_l} \otimes \ket{\frac{1}{2}, \pm \frac{1}{2}}$.
Rather than work with the total spin $\hat{\mbf{S}} = \hat{\mbf{S}}_1 + \hat{\mbf{S}}_2$, we will use the total angular momentum $\hat{\mbf{J}} = \hat{\mbf{L}} + \hat{\mbf{S}}$:   \vspace{-10pt}
\begin{align*}
    \hat{\mbf{J}}^2 &= \hat{\mbf{L}}^2 + \hat{\mbf{S}}^2 + 2 \hat{\mbf{L}} \cdot \hat{\mbf{S}}, \\
    \hat J_z &= \hat L_z + \hat S_z,
\end{align*}
We can see that $\hat{\mbf{L}} \cdot \hat{\mbf{S}}$ commutes with $\hat J_z$---the latter commutes with $\hat{\mbf{J}}^2$, $\hat{\mbf{L}}^2$, and $\hat{\mbf{S}}^2$, so it must also commute with $\hat{\mbf{L}} \cdot \hat{\mbf{S}}$.
Also notice that each of these three ``standalone'' operators commute with one another, and so they must have simultaneous eigenstates with $\hat{\mbf{L}} \cdot \hat{\mbf{S}}$; in particular, $[\hat{\mbf{L}} \cdot \hat{\mbf{S}}, \, \hat{\mbf{L}}^2] = 0$.

This simplifies things considerably.
We now know that only states with the same values of $l$ and $m_j$ can be mixed in the spin-orbit perturbation.
For fixed $l$ and $m_j = (m_l + 1 / 2) \hbar$, the only two such states are
\[ \ket{l, m_l, +\mbf{z}}, \qquad \ket{l, \, m_l + 1, \, -\mbf{z}}, \]
unless $m_l = l$ (in which case only the former state exists).
We'll now determine the two linear combinations of these states that are eigenstates of $2 \hat{\mbf{L}} \cdot \hat{\mbf{S}} = \hat L_+ \hat S_- + \hat L_- \hat S_+ + 2 \hat L_z \hat S_z$.
The matrix representation in the above basis is
\[ 2 \hat{\mbf{L}} \cdot \hat{\mbf{S}} \to \hbar^2 \begin{bmatrix} m & \sqrt{l(l+1) - m(m+1)} \\ \sqrt{l(l+1) - m(m+1)} & -(m+1) \end{bmatrix}, \]
and the eigenvalues are $\lambda = l$ and $\lambda = -(l+1)$.
So the total angular momentum $j$ satisfies
\[ j(j+1) = l(l+1) + \tfrac{1}{2} (\tfrac{1}{2} + 1) + \begin{cases} l, \\ -(l+1), \end{cases} \]
meaning $j = l \pm 1 / 2$.
We therefore label the eigenstates
\begin{align*}
    \ket{j = l + \tfrac{1}{2}, \, m_j} &= \sqrt{\frac{l + m_l + 1}{2l + 1}} \; \ket{l, m, +\mbf{z}} + \sqrt{\frac{l - m_l}{2l + 1}} \; \ket{l, \, m+1, \, -\mbf{z}}, \\
    \ket{j = l - \tfrac{1}{2}, \, m_j} &= \sqrt{\frac{l - m_l}{2l + 1}} \; \ket{l, m, +\mbf{z}} - \sqrt{\frac{l + m_l + 1}{2l + 1}} \; \ket{l, \, m+1, \, -\mbf{z}}.
\end{align*}
Note that we can use $m_j = m_l + 1 / 2$ to write the above directly in terms of $m_j$.
The first order shift for each of these is
\[ E_\textrm{S-O}^{(1)} = \frac{m_e c^2 Z^{4} \alpha^{4}}{4n^3 l(l + 1 / 2)(l+1)} \begin{cases} l & j = l + 1 / 2, \\ -(l+1) & j = l - 1 / 2. \end{cases} \]
To label these states we use the spectroscopic notation $1s_{1 / 2}, \, 2s_{1 / 2}, \, 2p_{1 / 2}, \, 2p_{3 / 2}$, and so on---the coefficient denotes $n$, the letter $l$, and the subscript $j$.

But there's a problem here: evaluating the above for $l=0$ gives a nonzero result, but a state with zero orbital angular momentum should have no spin-orbit coupling!
Solving using the more exact Dirac equation gives a result consistent with this, but it also spits out an additional perturbation
\[ \hat H_D = -\frac{1}{8m_e^2 c^2} \Big[ \hat{\mbf{p}} \,\cdot, \, \big[ \hat{\mbf{p}}, V(|\hat{\mbf{r}}|) \big] \Big], \]
where the momentum operators are dotted with one another in the expansion.
When we go to compute the first-order shift due to this Darwin term, we find that it contributes only for $s$ states---in fact, its magnitude
\[ E_D^{(1)} = \bra{n,0,0} \hat H_D \ket{n,0,0} = \frac{m_e c^2 Z^{4} \alpha^{4}}{2n^3} \]
is exactly the same as the seemingly nonphysical $l=0$ contribution from the spin-orbit Hamiltonian!
Finally, if we were to add together all of these first-order shifts we would get
\[ E_K^{(1)} + E_\textrm{S-O}^{(1)} + E_D^{(1)} = E_{n,j}^{(1)} = -\frac{m_e c^2 (Z\alpha)^{4}}{2n^3} \left( \frac{1}{j + 1 / 2} - \frac{3}{4n} \right), \]
which is $(Z\alpha)^2$ times the unperturbed energy (hence the term ``fine structure'').
The surprising degeneracy in $l$ is, as it turns out, maintained to all orders!
(We could show this by finding an exact solution the Dirac equation with a Coulomb potential.)

There's one more phenomenon here that we haven't yet considered: a very small but observable energy difference between the $2s_{1 / 2}$ and $2p_{1 / 2}$ levels.
This Lamb shift, which is on the order of $m_e c^2 (Z\alpha)^{4} \alpha \log \alpha$, is explained by quantum electrodynamics.
Taken with the hyperfine structure discussed much earlier, all this gives us a much more complete look at the energy structure of the hydrogen atom.

\section{The Zeeman Effect}
We've seen what happens when the hydrogen atom is put in an external electric field.
For a magnetic field $\mbf{B} = B\mbf{z}$, \vspace{-10pt}
\begin{align*}
    \hat H_B &= -\bm \mu \cdot \mbf{B} \\
    &= \frac{q_e B}{2m_e c} (\hat L_z + 2\hat S_z) = \frac{q_e B}{2m_e c} (\hat J_z + \hat S_z),
\end{align*}
which clearly commutes with $\hat J_z$.
We can therefore compute the shifts in the (nondegenerate) $j = l \pm 1 / 2$ states:
\begin{align*}
    E_B^{(1)} &= \frac{q_e B}{2m_e c} \bra{j = l \pm \tfrac{1}{2}, \, m_j} (\hat J_Z + \hat S_z) \ket{j = l \pm \tfrac{1}{2}, \, m_j} \\
    &= \frac{q_e \hbar B}{2m_e c} m_j \left( 1 \pm \frac{1}{2l + 1} \right) = \frac{g(j,l) q_e \hbar B}{2m_e c} m_j, \qquad g(j = l \pm \tfrac{1}{2}, \, l) \,=\, 1 \pm \frac{1}{2l+1},
\end{align*}
where $g(j,l)$ is called the Landé $g$ factor.

\section{Identical Particles}
For the remainder of the chapter we will focus on systems of identical particles.
The key here is that, unlike in the macroscopic world, identical particles in quantum mechanics are entirely indistinguishable.
There is no measurement we can take to distinguish, say, one electron from any other.

To formalize this we introduce the exchange operator $\hat P_{12}$, which ``exchanges'' the roles of particles 1 and 2 in a system.
In other words, we swap how the kets are labeled:
\[ \hat P_{12} \ket{a,b} = \ket{b,a} \quad\text{or}\quad \hat P_{12} (\ket{a}_1 \otimes \ket{b}_2) = \ket{b}_1 \otimes \ket{a}_2. \]
The ``exchanged'' must be the same overall physical state as the original state, meaning the states must differ by at most an overall phase; further, since $\hat P_{12}^2 = 1$, it follows that the exchange operator has eigenvalues $\pm 1$.
A state with $\lambda = 1$ is called symmetric, while one with $\lambda = -1$ is called antisymmetric.

Since we are doing nonrelativistic quantum mechanics we must state the spin-statistics theorem as a law of nature:
\begin{itemize}[topsep=0pt]
    \item particles with integral spin, called bosons, are found only in symmetric states and obey Bose-Einstein statistics; and
    \item those with half-integral spin, called fermions, are found only in antisymmetric states and obey Fermi-Dirac statistics.
\end{itemize}
If two identical particles are in the same state then clearly the state is symmetric under exchange; it follows that no two fermions can be in the same state.
This is known as the Pauli exclusion principle.

\section{The Helium Atom}
As an example of a system with two identical fermions, consider an atom with two electrons and a nuclear charge $Z$.
We will take the unperturbed and perturbed Hamiltonians to be
\[ \hat H_0 = \frac{\hat{\mbf{p}}_1^2}{2m_e} - \frac{Zq_e^2}{|\hat{\mbf{r}}_1|} + \frac{\hat{\mbf{p}}_2^2}{2m_e} - \frac{Zq_e^2}{|\hat{\mbf{r}}_2|}, \qquad \hat H_1 \frac{q_e^2}{|\hat{\mbf{r}}_1 - \hat{\mbf{r}}_2|}. \]
In the atom's ground state the identical electrons occupy the same spatial state, forcing their spin state to be antisymmetric:
\[ \ket{1s, 1s} = \ket{1,0,0}_1 \ket{1,0,0}_2 \, \frac{1}{\sqrt{2}} \big( \ket{+\mbf{z}}_1 \ket{-\mbf{z}}_2 - \ket{-\mbf{z}}_1\ket{+\mbf{z}}_2 \big). \]
The unperturbed energy is known to be $E_{1s, 1s}^{(0)} = 2(-\frac{1}{2} m_e c^2 Z^2 \alpha^2)$.
The first-order shift is the integral
\[ E_{1s,1s}^{(1)} = \bra{1s,1s} \frac{q_e^2}{|\hat{\mbf{r}}_1 - \hat{\mbf{r}}_2|} \ket{1s,1s} = \iint d^3 r_1 d^3 r_2 \left| \braket{\mbf{r}_1}{1,0,0} \right|^2 \left| \braket{\mbf{r}_2}{1,0,0} \right|^2 \frac{q_e^2}{|\mbf{r}_1 - \mbf{r}_2|}, \]
where we have inserted the symmetric spatial state
\begin{align*}
    \ket{\psi_S} &= \frac{1}{2} \iint d^3 r_1 d^3 r_2 \left( \frac{1}{\sqrt{2}} \ket{\mbf{r}_1, \mbf{r}_2} + \frac{1}{\sqrt{2}} \ket{\mbf{r}_2, \mbf{r}_1} \right) \left( \frac{1}{\sqrt{2}} \braket{\mbf{r}_1, \mbf{r}_2}{\psi_S} + \frac{1}{\sqrt{2}} \braket{\mbf{r}_2, \mbf{r}_1}{\psi_S} \right) \\
    &= \frac{1}{2} \iint d^3 r_1 d^3 r_2 \big( \ket{\mbf{r}_1. \mbf{r}_2} \braket{\mbf{r}_1, \mbf{r}_2}{\psi_S} + \ket{\mbf{r}_2, \mbf{r}_1} \braket{\mbf{r}_1, \mbf{r}_2}{\psi_S} \big) \\
    &= \iint d^3 r_1 d^3 r_2 \ket{\mbf{r}_1, \mbf{r}_2} \braket{\mbf{r}_1, \mbf{r}_2}{\psi_S}.
\end{align*}
We could get a similar integral for $\ket{\psi_A}$.
To evaluate, we'd substitute the known $R_{1,0}(r) Y_{0,0}(\theta,\phi)$ from our previous discussion of the hydrogen atom; after doing this we'd find a sizable discrepancy with the experimental value due to the fact that the Coulomb repulsion between the electrons isn't actually that small.
We'll look at a more accurate way of determining the energies shortly.

As for the first-excited states, one of the particles is in $\ket{1,0,0}$ while the other has $\ket{2,l,m}$; there are four states that fit this description.
The unperturbed energy is $E_{1s, \, 2s \text{ or } 2p} = -\frac{1}{2} m_e c^2 Z^2 \alpha^2 (1 + 1 / 2^2)$.
Now, taking advantage of the fact that $\hat H_1$ commutes with the exchange operator in both space and spin, we can diagonalize $\hat H_1$ by choosing states that are (overall antisymmetric) eigenstates of both $\hat P_{12}^\textrm{spin}$ and $\hat P_{12}^\textrm{space}$.

We'd end up with a first-order shift of the form $E^{(1)} = J \pm K$, where the $\pm$ is correlated with the value of the total spin so that the overall state is antisymmetric: we have $J - K$ for the triplet states, while the singlet has $J + K$.
(This reflects how the antisymmetric wave functions vanish at $\mbf{r}_1 = \mbf{r}_2$, causing increased separation due to Coulomb repulsion.)

For atoms with more electrons the Hamiltonian is much less wieldy.
We can hone in on what the potential looks like by taking a guess (based on what we know about the limiting values), numerically solving the Schrödinger equation for this guess, determining the charge distribution $q_e \left| \braket{\mbf{r}_1}{n,l,m} \right|^2$ corresponding to the resulting wave function, and using that to determine a new guess potential, all in succession until a self-consistent solution materializes.
The states fill according to the Pauli exclusion principle.
\begin{itemize}[topsep=0pt]
    \item The $1s$, $2s$, $2p$, $3s$, and $3p$ states fill like normal.
    \item The $4s$ level is very slightly below the $3d$ level, so those fill next.
    \item By the time the $3d$ level is filled with four electrons, the $4s$ level has been raised enough to be slightly above the $3d$ level; chromium therefore has $4s^1 \,3d^5$.
    \item Manganese, iron, cobalt, and nickel have $4s^2 \,3d^5$ through $4s^2 \,3d^8$, but then again the pattern changes and copper gets $4s^1 \,3d^{10}$.
    (This configuration is so close in energy to $4s^2 \,3d^9$, however, that copper may be one or the other depending on its environment.)
    \item The rest of the row fills like normal.
\end{itemize}

\section{The Variational Method}
In practice, we estimate the energy levels of helium (and heavier) atoms using the variational method.
The idea is to exploit the fact that every state $\ket{\psi}$ is a linear combination of energy eigenstates $\ket{E_n}$, each of which satisfies $E_n \geq E_0$, meaning the expectation value $\bra{\psi} \hat H \ket{\psi} \geq E_0$.

The method itself is also pretty simple.
We begin by picking a trial state $\ket{\psi(\alpha_1, \alpha_2, \ldots)}$ which depends on parameters $\alpha_1, \alpha_2, \ldots$.
We then compute $\left< E \right>$ for this state and then minimize it with respect to the free parameters to get as close as possible to the ground-state $E_0$.

When we go to pick a trial state, we generally choose one that closely resembles what we think the actual state should look like.
Two common wave functions are the (radial) ground state of the hydrogen atom
\[ \psi(r) = \frac{b^{3 / 2}}{\sqrt{\pi}} e^{-br} \]
and that of the one-dimensional harmonic oscillator
\[ \psi(x) = \left( \frac{2a}{\pi} \right)^{1 / 4} e^{-ax^2}. \]
Note that we could use a similar method to determine higher energy eigenvalues $E_n$ simply by picking a trial state that is orthogonal every eigenstate $\ket{E_0}, \, \ldots, \, \ket{E_{n-1}}$ below it.
We may do this either by simply making a clever choice, or we might construct one; for example, if $\ket{\psi}$ is the trial ground state then we can build an excited trial state via
\[ \ket{\varphi} \longrightarrow \frac{\ket{\varphi} - \ket{\psi} \braket{\psi}{\varphi}}{\sqrt{1 - \left| \braket{\psi}{\varphi} \right|^2}}. \]
However, since $\ket{\psi}$ is only a trial state rather than the true ground state, minimizing $\bra{\varphi} \hat H \ket{\varphi}$ will not necessarily yield an upper bound for $E_1$.

\section{Covalent Bonding}
Ionic bonding can be explained classically as the exchange of electrons, but covalent bonding is pure quantum mechanics.
Suppose two nuclei are spaced a fixed distance $R$ apart; if we allow this distance to appear as a parameter in our wave functions, we can minimize $\left< E \right>$ with respect to $R$ to approximate both the energy and size of a molecule held together by a covalent bond.

Consider the hydrogen molecule ion $\text{H}_2^+$.
If the protons are spaced far apart, there are clearly at least two possible states: the electron is bound to one proton or the other.
The corresponding wave functions are
\[ \braket{\mbf{r}}{1} = \frac{1}{\sqrt{\pi a_0^3}} e^{-|\mbf{r} - \mbf{R} / 2| / a_0}, \qquad \braket{\mbf{r}}{2} = \frac{1}{\sqrt{\pi a_0^3}} e^{-|\mbf{r} + \mbf{R} / 2| / a_0}, \]
where $\mbf{r}$ is the position of the electron and $\mbf{R}$/2 is the position of each proton with respect to their midpoint.
The linear combinations of $\ket{1}$ and $\ket{2}$ that diagonalize the Hamiltonian turn out to be
\[ \ket{\pm} = \frac{1}{\sqrt{2 \pm 2 \braket{1}{2}}} \big( \ket{1} \pm \ket{2} \big), \qquad E_\pm \frac{1}{1 \pm \braket{1}{2}} \big( H_{11} \pm H_{12} \big), \]
where $H_{11}$ and $H_{12}$ are the matrix elements of $\hat H$ (expressed as integrals).
Notice that $\braket{\mbf{r}}{+}$ and $\braket{\mbf{r}}{-}$ have even and odd parity, respectively.
Of course, only the even parity state has a minimum, and so we call $\ket{+}$ a bonding molecular orbital; the state $\ket{-}$ is called an antibonding molecular orbital.
Note that these molecular orbitals are linear combinations of atomic orbitals.
(Also note that our simple trial wave functions, despite producing pretty good qualitative results, undershoot the binding energy and overshoot the separation.)

Adding another electron into the mix produces $\textrm{H}_2$, which has a higher binding energy and a smaller internuclear separation.
The electrons' spatial state, of course, is symmetric, meaning their spin state has to be antisymmetric in a spin-0 state.
Bonding \textit{only} occurs in such a state---in a spin-1 state the spatial state would be antisymmetric, meaning the electrons wouldn't exist in the space between the protons.

This explains why, say, hydrogen and helium do not bind together to form HHe.
The electrons in He are already paired up in the spatial state $1s$ and the total-spin-0 state, so if the H electron were to form a covalent bond by pairing up with one of the He electrons it would expend a great deal of energy in kicking the other one up to the $2s$.
If the He electrons were instead in a total-spin-1 state we would see repulsion between the two atoms.
Helium is therefore inert, along with all the other noble gases.

\end{document}
\documentclass[../p116main.tex]{subfiles}
\graphicspath{{\subfix{../figures/}}}

\begin{document}

\chapter{Central Potentials}
\section{Wave Mechanics in Three Dimensions}
Now w'll extend our discussion of wave mechanics to three dimensions.
The position eigenstates are specified by three coordinates $\ket{\mbf{r}} = \ket{x,y,z}$, each dimension with its own position operator:
\[ \hat x \ket{\mbf{r}} = x \ket{\mbf{r}}, \quad \hat y \ket{\mbf{r}} = y \ket{\mbf{r}}, \quad \hat z \ket{\mbf{r}} = z \ket{\mbf{r}}. \]
An arbitrary state is a superposition of position states,
\[ \ket{\psi} = \int d^3 \mbf{r} \ket{\mbf{r}} \braket{\mbf{r}}{\psi}, \]
and like before $\braket{\mbf{r}}{\mbf{r}'} = \delta^3(\mbf{r} - \mbf{r}')$, where $\delta^3$ is the product of three deltas (one for each dimension).
The normalization condition $\braket{\psi}{\psi} = 1$ suggests that we identify $d^3 \mbf{r} \left| \braket{\mbf{r}}{\psi} \right|^2$ as the probability of finding the particle in some $d^3 \mbf{r}$ of space.

Now we introduce a three-dimension translation operator $\hat T(\mbf{a}) \ket{\mbf{r}} = \ket{\mbf{r} + \mbf{a}}$ which is generated by three operators:
\[ \hat T(a_x \mbf{i}) = e^{-i \hat p_x a_x / \hbar}, \qquad \hat T(a_y \mbf{j}) = e^{-i \hat p_y a_y / \hbar}, \qquad \hat T(a_z \mbf{k}) = e^{-i \hat p_z a_z / \hbar}. \]
In contrast with rotations, translations in different directions certainly commute with one another.
Their generators do, too---we could Taylor-expand the elements of $[\hat T(a_x \mbf{i}), \,\hat T(a_y \mbf{j})] = 0$ to second order, for example, to find that $[\hat p_x, \hat p_y] = 0$.
We can therefore express the translation operator as
\[ \hat T(\mbf{a}) = e^{-i \hat p_x a_x / \hbar} e^{-i \hat p_y a_y / \hbar} e^{-i \hat p_z a_z / \hbar} = e^{-i \mbf{p} \cdot \mbf{a} / \hbar}. \]
In each dimension we have the same commutation relation $[\hat x, \hat p_x] = i\hbar$ as before, but across dimensions the position and momentum operators do commute:
\[ \hat T(a_x \mbf{i}) \, \hat y \ket{x,y,z} = y \ket{x + a_x, \, y, z} = \hat y \, \hat T(a_x \mbf{i}) \ket{x,y,z}. \]
As shorthand for this, we write
\[ [\hat x_i, \hat p_j] = i \hbar \delta_{ij}. \]
Now, because the momentum operators all commute with one another, we can form three-dimensional momentum states $\ket{\mbf{p}} = \ket{p_x, p_y, p_z}$ normalized by $\braket{\mbf{p}'}{\mbf{p}} = \delta^3(\mbf{p}' - \mbf{p})$ with $d^3 \mbf{p} \left| \braket{\mbf{p}}{\psi} \right|^2$ as the probability of finding a particle in some $d^3 \mbf{p}$ of momentum space.

The generalization of the momentum operator in the position basis is
\[ \bra{\mbf{r}} \hat{\mbf{p}} \ket{\psi} = \frac{\hbar}{i} \nabla \braket{\mbf{r}}{\psi}. \]
If we take $\psi = \mbf{p}$ then this becomes a differential equation with solution
\[ \braket{\mbf{r}}{\mbf{p}} = \frac{1}{(2\pi \hbar)^{3 / 2}} e^{i \mbf{p} \cdot r / \hbar}, \]
just the product of three of the one-dimensional eigenfunctions we found before.

\section{Translational Symmetry and Linear Momentum}
We'll take a special interest in systems with central potentials, which have Hamiltonians of the form
\[ \hat H = \frac{\hat{\mbf{p}}^2}{2m_1} + \frac{\hat{\mbf{p}}^2}{2m_2} + V(|\hat{\mbf{r}}_1 - \hat{\mbf{r}}_2|). \]
The most important example of such a two-body system is the one where $V$ is the Coulomb potential, but for now we'll keep our analysis general.
We introduce the two-body position basis states $\ket{\mbf{r}_1, \mbf{r}_2} = \ket{\mbf{r}_1}_1 \otimes \ket{\mbf{r}_2}_2$.
The translation operators $\hat T_1(\mbf{a}), \hat T_2(\mbf{a})$ commute since we can translate the particles independent of one another; it follows that $[\hat{\mbf{p}}_1, \hat{\mbf{p}}_2] = 0$ and that the ``joint'' translation operator is
\[ \hat T_1(\mbf{a}) \hat T_2(\mbf{a}) = e^{-i \hat{\mbf{p}}_1 \cdot \mbf{a} / \hbar} e^{-i \hat{\mbf{p}}_2 \cdot \mbf{a} / \hbar} = e^{-i \hat{\mbf{P}} \cdot \mbf{a} / \hbar}, \qquad \hat{\mbf{P}} = \hat{\mbf{p}}_1 + \hat{\mbf{p}}_2. \]
Applying this operator to a pair of particles doesn't affect the distance between them, so we expect that it commutes with the Hamiltonian.
We can immediately see that $\hat T_1(\mbf{a}) \hat T_2(\mbf{a})$ commutes with the kinetic energy bit; as for the potential term, the process is a bit mechanical:
\begin{align*}
    \hat T_1(\mbf{a}) \hat T_2(\mbf{a}) V(|\hat{\mbf{r}}_1 - \hat{\mbf{r}}_2|) \ket{\mbf{r}_1, \mbf{r}_2} &= \hat T_1(\mbf{a}) \hat T_2(\mbf{a}) V(|\mbf{r}_1 - \mbf{r}_2|) \ket{\mbf{r}_1. \mbf{r}_2} \\
    &= V(|\mbf{r}_1 - \mbf{r}_2|) \ket{\mbf{r}_1 + \mbf{a}. \mbf{r}_2 + \mbf{a}} \\
    &= V(|\hat{\mbf{r}}_1 - \hat{\mbf{r}}_2|) \ket{\mbf{r}_1 + \mbf{a}. \mbf{r}_2 + \mbf{a}} \\
    &= V(|\hat{\mbf{r}}_1 - \hat{\mbf{r}}_2|) \hat T_1(\mbf{a}) \hat T_2(\mbf{a}) \ket{\mbf{r}_1, \mbf{r}_2}.
\end{align*}
Thus $[\hat H, \, \hat T_1(\mbf{a}) \hat T_2(\mbf{a})] = 0$, as expected.
It follows that time-evolving the translated state $\hat T(\mbf{a}) \ket{\psi(0)}$ is equivalent to translating the time-evolved state $\ket{\psi(t)}$.
This is precisely what we'd expect from (Galilean) relativity.
Also, because translations are generated by $\hat{\mbf{P}}$ we can write
\[ [\hat H, \hat{\mbf{P}}] = 0, \]
meaning $d \left< \mbf{P} \right> / dt = 0$---translational invariance implies conservation of momentum!
This is another example (following parity) of the deep connection between symmetries of the Hamiltonian and conservation laws.
Notice that we can turn the argument around, too: conservation of momentum implies $[\hat H, \hat{\mbf{P}}] = 0$ and so $\hat H$ exhibits translational symmetry.
Such an argument holds whenever there  are no interactions external to the system we're working with.

We can use this discussion to simplify our Hamiltonian considerably.
The natural coordinates for our problem aren't the positions standard Cartesian coordinates, but rather the center-of-mass position $\mbf{R}$ and the relative position $\mbf{r}$ defined by
\[ \hat{\mbf{r}} = \hat{\mbf{r}}_1 - \hat{\mbf{r}}_2, \qquad \hat{\mbf{R}} = \frac{m_1 \hat{\mbf{r}}_1 + m_2 \hat{\mbf{r}}_2}{m_1 + m_2}. \]
It's easy to see that the relative position and total momentum operators satisfy $[\hat x_i, \hat P_j] = 0$.
We could also show that $[\hat X_i, \hat P_j] = i\hbar \delta_{ij}$, as we'd expect from the usual canonical commutation relations.
If we introduce the relative momentum
\[ \hat{\mbf{p}} = \frac{m_2 \hat{\mbf{p}}_1 - m_1 \hat{\mbf{p}}_2}{m_1 + m_2}, \]
we could show that it satisfies $[\hat x_i, \hat p_j] = i\hbar \delta_{ij}$ and $[\hat X_i, \hat p_j] = 0$ in a symmetric fashion.
Thus relative operators obey the usual commutation relations when paired with other relative operators (as do the center-of-mass operators), but operators from different groups commute.

In these coordinates we could show that the Hamiltonian is
\[ \hat H = \frac{\hat{\mbf{P}}^2}{2M} + \frac{\hat{\mbf{p}}^2}{2\mu} + V(|\hat{\mbf{r}}|), \qquad M = m_1 + m_2, \quad \mu = \frac{m_1 m_2}{m_1 + m_2}. \]
We view this as the sum of the center-of-mass kinetic energy $\hat H_\textrm{cm} = \hat{\mbf{P}}^2 / 2M$ and the energy due to the relative motion $\hat H_\textrm{rel} = \hat{\mbf{p}}^2 / 2\mu + V(|\hat{\mbf{r}}|)$.
These operators commute, so they have simultaneous eigenstates $\ket{E_\textrm{cm}, E_\textrm{rel}}$; in particular, the eigenstates of $\hat H_\textrm{cm}$ are those of $\hat{\mbf{P}}$, so in momentum space
\[ \braket{\mbf{R}}{\mbf{P}} = \frac{1}{(2\pi\hbar)^{3 / 2}} e^{i \mbf{P} \cdot \mbf{R} / \hbar}, \]
as we saw before.
To simplify things we'll usually analyze the two-body problem in the center-of-mass frame, where $\mbf{P} = 0$:
\[ \hat H = \frac{\hat{\mbf{p}}^2}{2\mu} + V(|\hat{\mbf{r}}|). \]
We've therefore reduced the problem to a single body in a central potential $V(r)$, as seen in classical mechanics.

\section{Rotational Symmetry and Angular Momentum}
Another observation we can make about this $\hat H$ is that it is invariant under rotations; we'll once again verify this using operators.
We first introduce
\[ \hat R(d\phi \,\mbf{k}) = \left( 1 - \frac{i}{\hbar} \hat L_z \, d\phi \right), \]
which rotates a position state through an angle $d\phi$ about the $z$-axis.
Its effect can be seen by
\begin{align*}
    \hat R(d\phi \,\mbf{k}) \ket{x,y,z} &= \ket{x - y \,d\phi, \, y + x \,d\phi, \, z} \\
    &= \left[ 1 - \frac{i}{\hbar} \hat p_x (-y \,d\phi) \right] \left[ 1 - \frac{i}{\hbar} \hat p_y (x \,d\phi) \right] \ket{x,y,z} \\
    &= \left[ 1 - \frac{i}{\hbar} (\hat x \hat p_y - \hat y \hat p_x) d\phi \right] \ket{x,y,z},
\end{align*}
and so we identify
\[ \hat L_z = \hat x \hat p_x - \hat y \hat p_x \]
as the $z$-component of the orbital angular momentum operator $\hat{\mbf{L}} = \hat{\mbf{r}} \times \hat{\mbf{p}}$.
(Orbital angular momentum appears here because we're now generating rotations of position states.)
Using this form of $\hat L_z$ it's easy to directly verify the intuitive relations
\begin{gather*}
    [\hat L_z, \hat p_x] = i\hbar \hat p_y, \qquad [\hat L_z, \hat p_y] = -i\hbar \hat p_x, \qquad [\hat L_z, \hat p_z] = 0, \\
    [\hat L_z \hat x] = i\hbar \hat y, \qquad [\hat L_z, \hat y] = -i\hbar \hat x, \qquad [\hat L_z, \hat z] = 0,
\end{gather*}
meaning $[\hat L_z, \hat{\mbf{p}}^2] = 0$ and $[\hat L_z, \hat{\mbf{r}}^2] = 0$.
Thus $\hat L_z$ commutes with our Hamiltonian.
Alternatively, we may have used known eigenvalue equations:
\begin{align*}
    \hat R(d\phi \,\mbf{k}) V(|\hat{\mbf{r}}|) \ket{r, \theta, \phi} &= V(r) \ket{r, \, \theta, \, \phi + d\phi} \\
    &= V(|\hat{\mbf{r}}|) \hat R(d\phi \,\mbf{k}) \ket{r, \theta, \phi}.
\end{align*}
Since $\hat H$ commutes with $\hat L_z$ and our axis of the $z$-axis was arbitrary, our Hamiltonian must commute with the generator of rotations about any axis!
Thus the system is invariant under rotations; it follows that angular momentum is conserved.

We'll once again leverage this discussion in our analysis of $\hat H$, this time to specify the energy eigenstates and an eigenvalue equation that these states must satisfy.
Just like we saw with spin, we can form simultaneous eigenstates of $\hat{\mbf{L}}^2$ and $\hat L_z$, and the Hamiltonian commutes with both of these.
So we can write the following.
\begin{align*}
    \hat H \ket{E, l, m} &= E \ket{E, l, m}, \\
    \hat{\mbf{L}}^2 \ket{E, l, m} &= l(l+1) \hbar^2 \ket{E, l, m}, \\
    \hat L_z \ket{E, l, m} &= m\hbar \ket{E, l, m}.
\end{align*}
We wish to write down the energy eigenvalue equation in position space.
The only nontrivial bit here is the kinetic energy term, so really we seek the position-space representation of $\hat{\mbf{p}}^2$.
To this end, we note the identity
\[ (\hat{\mbf{r}} \times \hat{\mbf{p}}) \cdot (\hat{\mbf{r}} \times \hat{\mbf{p}}) = \hat{\mbf{r}}^2 \hat{\mbf{p}}^2 - (\hat{\mbf{r}} \cdot \hat{\mbf{p}})^2 + i\hbar \hat{\mbf{r}} \cdot \hat{\mbf{p}} \]
and compute
\begin{align*}
    \bra{\mbf{r}} \hat{\mbf{L}}^2 \ket{\psi} &= \bra{\mbf{r}} [\hat{\mbf{r}}^2 \hat{\mbf{p}}^2 - (\hat{\mbf{r}} \cdot \hat{\mbf{p}})^2 + i\hbar \hat{\mbf{r}} \cdot \hat{\mbf{p}}] \ket{\psi} \\
    &= r^2 \bra{\mbf{r}} \hat{\mbf{p}}^2 \ket{\psi} + \hbar^2 r \frac{\partial}{\partial r} \left( r \frac{\partial}{\partial r} \right) \braket{\mbf{r}}{\psi} + \hbar^2 \frac{\partial}{\partial r} \braket{\mbf{r}}{\psi}.
\end{align*}
Solving for $\bra{\mbf{r}} \hat{\mbf{p}}^2 \ket{\psi}$ and substituting into the position-space representation of the energy eigenvalue equation gives
\[ -\frac{\hbar^2}{2\mu} \left( \frac{\partial^2}{\partial r^2} + \frac{2}{r} \frac{\partial}{\partial r} \right) \braket{\mbf{r}}{\psi} + \frac{\bra{\mbf{r}} \hat{\mbf{L}}^2 \ket{\psi}}{2\mu r^2} + V(r) \braket{\mbf{r}}{\psi} = E \braket{\mbf{r}}{\psi}. \]
We recognize the second term as the rotational kinetic energy, so the first term must be the radial part.
Now, if $\ket{\psi}$ is a simultaneous eigenstate $\ket{E,l,m}$ the equation becomes
\[ \left[ -\frac{\hbar^2}{2\mu} \left( \frac{\partial^2}{\partial r^2} + \frac{2}{r} \frac{\partial}{\partial r} \right) + \frac{l(l+1)\hbar^2}{2\mu r^2} + V(r) \right] \braket{\mbf{r}}{E,l,m} = E \braket{\mbf{r}}{E,l,m}. \]
If we write $\braket{\mbf{r}}{E,l,m} = R(r) \Theta(\theta) \Phi(\phi)$, we can divide out the angular part of the wave function to get the radial equation
\[ \left[ -\frac{\hbar^2}{2\mu} \left( \frac{d^2}{dr^2} + \frac{2}{r} \frac{d}{dr} \right) + \frac{l(l+1)\hbar^2}{2\mu r^2} + V(r) \right] R(r) = ER(r), \]
and we can simplify it using the substitution $R(r) = u(r) / r$:
\[ \left[ -\frac{\hbar^2}{2\mu} \frac{d^2}{dr^2} + \frac{l(l+1)\hbar^2}{2\mu r^2} + V(r) \right] u(r) = E u(r). \]
This has the exact same form as the one-dimensional Schrödinger equation, only this time with an effective potential $V_\textrm{eff}(r) = l(l+1)\hbar^2 / 2\mu r^2 + V(r)$.
Notably, the radial equation has no dependence on the quantum number $m$; this is a consequence of the rotational invariance of the Hamiltonian, as there is no preferred axis upon which we should measure $L_z$.
So solutions to this equation do not necessarily specify states uniquely---we must specify $m$ (and the spin) to do so.

The set of commuting operators that are necessary to label states uniquely is called a complete set of commuting observables.
For any given system there may exist several such sets---for example, we might have taken $\{ \hat H, \hat{\mbf{L}}^2, \hat L_x \}$ instead of $\{ \hat H, \hat{\mbf{L}}^2, \hat L_z \}$.
(Again, in reality neither of these are complete set since we haven't yet seen how spin enters into $\hat H$.)
For a system with cylindrical symmetry we might take $\{ \hat H, \hat L_z, \hat p_z \}$.

\section{The Total Angular Momentum Operator}
We seek a position-space representation of $\hat{\mbf{L}}^2$.
We'll go component-by-component---for $\hat L_z$ we'll start by Taylor-expanding    \vspace{-10pt}
\begin{align*}
    \bra{r, \theta, \phi} \hat R(d\phi \,\mbf{k}) \ket{\psi} &= \braket{r, \theta, \, \phi - d\phi}{\psi} \\
    &= \braket{r, \theta, \phi}{\psi} - d\phi \, \frac{\partial \braket{r, \theta, \phi}{\psi}}{d\phi}, 
\end{align*}
which we can compare wth the definition of $\hat R(d\phi \,\mbf{k})$ to write
\[ \hat L_z \longrightarrow \frac{\hbar}{i} \frac{\partial}{\partial \phi}. \]
Note the similarity with the linear momentum operator.
Now, for an eigenstate of $\hat L_z$ we have the differential equation
\[ \frac{\hbar}{i} \frac{\partial}{\partial \phi} \braket{r, \theta, \phi}{l, m} = m\hbar \braket{r, \theta, \phi}{l,m}, \]
meaning the $\phi$-dependence of the eigenfunction is $e^{im \phi}$ for $m \in \Z$ (so that the function is single-valued with period $2\pi$).
It follows, from our much earlier discussion of angular momentum, that $l = 0, 1, 2, \ldots$.

To find the other components of angular momentum we use the gradient in spherical coordinates:
\begin{align*}
    \hat{\mbf{L}} \to \mbf{r} \times \frac{\hbar}{i} \nabla &= r \mbf{u}_r \times \frac{\hbar}{i} \left( \mbf{u}_r \frac{\partial}{\partial r} + \mbf{u}_\theta \frac{1}{r} \frac{\partial}{\partial \theta} + \mbf{u}_\phi \frac{1}{r \sin \theta} \frac{\partial}{\partial \phi} \right) \\
    &= \frac{\hbar}{i} \left( \mbf{u}_\phi \frac{\partial}{\partial \theta} - \mbf{u}_\theta \frac{1}{\sin \theta} \frac{\partial}{\partial \phi} \right).
\end{align*}
If we split the $x$ and $y$ components of these unit vectors we find that
\begin{align*}
    \hat L_x &\to \frac{\hbar}{i} \left( -\sin \phi \frac{\partial}{\partial \theta} - \cot \theta \cos \phi \frac{\partial}{\partial \phi} \right), \\
    \hat L_y &\to \frac{\hbar}{i} \left( \cos \phi \frac{\partial}{\partial \theta} - \cot \theta \sin \phi \frac{\partial}{\partial \phi} \right).
\end{align*}
The total angular momentum operator is therefore represented by
\[ \hat{\mbf{L}} \to -\hbar^2 \left[ \frac{1}{\sin \theta} \frac{\partial}{\partial \theta} \left( \sin \theta \frac{\partial}{\partial \theta} \right) + \frac{1}{\sin^2 \theta} \frac{\partial}{\partial \phi^2} \right]. \]
This agrees entirely with the Laplacian in spherical coordinates!
In fact, we can write the Schrödinger equation as
\[ E \braket{\mbf{r}}{\psi} = \left[ -\frac{\hbar^2}{2\mu} \nabla^2 + V(r) \right] \braket{\mbf{r}}{\psi} \]
which, after expanding, agrees entirely with the energy eigenvalue equation we found before, provided we identify $\hat{\mbf{L}}^2$ as we have.

The position-space representations of the angular momentum operators depend only on $\theta$ and $\phi$, meaning we can isolate the angular dependence of our eigenfunctions and determine the amplitudes $\braket{\theta,\phi}{l,m} = Y_{l,m}(\theta,\phi)$ called the spherical harmonics.
Expressed this way, the energy eigenfunctions are
\[ \braket{r, \theta, \phi}{E, l, m} = R(r) Y_{l,m}(\theta, \phi). \]
The normalization condition for these eigenfunctions is
\begin{align*}
    1 &= \int d^3 \mbf{r} \, |R(r)|^2 |Y_{l,m}(\theta, \phi)|^2 \\
    &= \left[ \int_0^\infty r^2 dr \, |R(r)|^2 \right] \left[ \int_0^\pi \sin \theta \,d\theta \int_0^{2\pi} d\phi \, |Y_{l,m}(\theta, \phi)|^2 \right],
\end{align*}
where we have substituted $d^3 r = r^2 dr \,d\Omega = r^2 dr \, \sin \theta \, d\theta \, d\phi$.
We normalize the radial and angular parts separately, so we can interpret
\[ \left| \braket{\theta, \phi}{l,m} \right|^2 d\Omega = |Y_{m,l}(\theta, \phi)|^2 d\Omega \]
as the probability of finding a particle in $\ket{l,m}$ within the solid angle $d\Omega$.

To determine the spherical harmonics we start by using the position-space representations of the angular momentum operators to write
\[ \hat L_\pm \to \frac{\hbar}{i} e^{\pm i \phi} \left( \pm i \frac{\partial}{\partial \theta} - \cot \theta \frac{\partial}{\partial \phi} \right), \]
and so when we raise the highest angular momentum state we get
\[ 0 = \bra{\theta, \phi} \hat L_+ \ket{l,l} = \frac{\hbar}{i} e^{i\phi} \left( i \frac{\partial}{\partial \theta} - \cot \theta \frac{\partial}{\partial \phi} \right) \braket{\theta, \phi}{l,l}. \]
With the known dependence on $e^{il\phi}$ the equation becomes $(\partial / \partial \theta - l \cot \theta) \braket{0, \phi}{l,l} = 0$, and our solution is
\[ Y_{l,l}(\theta, \phi) = \braket{\theta, \phi}{l,l} = c_l e^{il\phi} \sin^{l} \theta, \qquad c_l = \frac{(-1)^{l}}{2^l l!} \sqrt{\frac{(2l+1)!}{4\pi}}. \]
Successive applications of the lowering operator reveal that
\[ Y_{l,m}(\theta, \phi) = \frac{(-1)^{l}}{2^l l!} \sqrt{\frac{(2l + 1)(l+m)!}{4\pi (l-m)!}} e^{im\phi} \frac{1}{\sin^m \theta} \frac{d^{l-m}}{d(\cos \theta)^{l-m}} \sin^{2l} \theta. \]
This choice of phase factor ensures that $Y_{1,0}(\theta, \phi)$ has a real positive value for $\theta = 0$.
For negative $m$, note that
\[ Y_{l, -m}(\theta, \phi) = (-1)^{m} [Y_{l,m}(\theta, \phi)]^*. \]
The $l=0$ and $l=1$ states are called $s$ and $p$ states, respectively.
The $s$ state is spherically symmetric; the $p$ states are a bit smashed on the $xy$-plane for $m = \pm 1$, but appears oriented on the $z$-axis for $m=0$.
A similar pattern exists for higher values of $l$.
We can create harmonics oriented along different axes by taking superpositions of the ``$z$-harmonics''.

\section{Miscellaneous Results in Atoms and Molecules}
Now we'll look at a couple of ideas that will be useful in our future discussion of central potentials.
First we estimate the energy scale for the hydrogen atom without actually solving the energy eigenvalue equation.
The Hamiltonian is
\[ \hat H = \frac{\hat{\mbf{p}}^2}{2\mu} - \frac{q_e^2}{|\hat{\mbf{r}}|}, \]
the ground-state expectation value is $E_1 = \left< \mbf{p}^2 / 2\mu - q_e^2 / r \right>$, and using dimensional analysis we express $\left< q_e^2 / r \right> = q_e^2 / a$, where $a$ is a length characteristic of the atom's size.
The uncertainty in $r$ is at most of order $a$, so from the Heisenberg uncertainty principle $|\Delta p| \gtrsim \hbar / a$.

Now, the expectation value of the kinetic energy is
\[ \frac{\left< \mbf{p}^2 \right>}{2\mu} = \frac{\Delta \mbf{p}^2 + \left< \mbf{p} \right>^2}{2\mu} = \frac{\Delta \mbf{p}^2}{2\mu}, \]
since $\left< \mbf{p} \right>$ is time-independent in an eigenstate and $\left< \mbf{p} \right> \neq 0$ would give an unbounded system (it wouldn't stay within a particular region of space).
The total energy estimate is thus
\[ E_1 \gtrsim \frac{\hbar^2}{2\mu a^2} - \frac{q_e^2}{a}, \]
and it is minimized at
\[ a = \frac{\hbar^2}{m_e q_e^2} \;\implies\; E_1 \simeq -\frac{m_e q_e^{4}}{2\hbar^2}, \]
where we've taken $\mu \simeq m_e$.
This gives $a$ on the order of angstroms and $E_1$ on the order of 10 eV.

Now we'll look at diatomic molecules.
The potential energy $V(r)$ of such a molecule can be approximated as a harmonic oscillator with $\mu \omega^2 = d^2 V / dr^2$.
This energy is on the order of $q_e^2 / a$, so by dimensional analysis
\[ \frac{d^2 V}{dr^2} \sim \frac{e^2}{a^3} \;\implies\; \hbar \omega = \hbar \left[ \frac{1}{\mu} \left( \frac{d^2 V}{dr^2} \right)_{r = r_0} \right]^{1 / 2} \sim \left( \frac{m_e}{M_N} \right)^{1 / 2} \left( \frac{m_e e^{4}}{\hbar^2} \right), \]
where $\mu$ (the reduced mass of the nucleus-nucleus system) is on the order of the nuclear mass $M_N$.
We recognize the second factor as the electronic energy scale we found earlier, and it's been scaled down by roughly $(m_e / M_N)^{1 / 2} \approx 1 / 40$.
Thus the wavelengths of photons absorbed or emitted when the system moves between vibrational energy levels are about 40 times longer than they are for a typical atomic transition.
In particular, the purely vibrational energies are $E_{n_\nu} = (n_\nu + 1 / 2) \hbar \omega$.

When we look at the harmonic oscillator energy eigenfunctions we see that, at low excitations, the molecule vibrates on a distance scale
\[ \frac{\sqrt{\hbar}}{\mu \omega} \sim \left( \frac{m_e}{M_N} \right)^{1 / 4} \frac{\hbar^2}{m_e e^2} = a \left( \frac{m_e}{M_N} \right)^{1 / 4}, \]
which is a small fraction of the equilibrium separation $r_0$ between the nuclei.
We therefore approximate the molecule as rigid and separate the rotational motion from the vibrational motion.
The rigid rotator Hamiltonian is    \vspace{-4pt}
\[ \hat H = \frac{\hat{\mbf{L}}^2}{2I}, \qquad I = \mu r_0^2, \]
and so its eigenstates are just the angular momentum eigenstates $\ket{l,m}$ with $E_l = l(l+1)\hbar^2 / 2I$.
The spacing between energy levels is
\[ E_l - E_{l-1} = \frac{l\hbar^2}{I} \sim l \frac{\hbar^2}{M_N a^2} = l \frac{m_e}{M_N} \left( \frac{m_e e^{4}}{\hbar^2} \right), \]
which corresponds to wavelength a factor of $(M_n / m_e)^{1 / 2}$ longer, for low $l$, than for vibrational transitions.
This is only slightly below order-$k_B T$ at room temperature, so many levels are excited here.
We can take advantage of this to experimentally determine $r_0$; setting $E_l - E_{l-1} = hc / \lambda$ gives the observable
\[ \lambda l = 2\pi I c / \hbar, \]
which we've found to be constant in experiment in alignment with our treatment of the molecule as a rotator!

In practice, it's difficult to produce the wavelengths necessary for observations of purely rotational transitions.
But putting rotational energy alongside vibrational energy gives
\[ E_{n_\nu, l} = \left( n_\nu + \frac{1}{2} \right) \hbar \omega + \frac{l(l+1) \hbar^2}{2I}, \]
and since it turns out that most transitions satisfy $\Delta n_\nu = \pm 1$ and $\Delta l = \pm 1$, the frequencies resulting from vibrational-rotational transitions are more accessible than we may believe.

\section{Solutions in the Coulomb Potential}
Now we will solve the radial equation for a variety of potentials, but first we'll do a more general analysis of the behavior of our solutions near the origin.
If $V(r)$ is not more singular than $1 / r^2$ at the origin then we are guaranteed a power series solution, so to determine the leading behavior of $u_{E,l}(r)$ we substitute $r^s$ into the radial equation to get
\[ -\frac{\hbar^2}{2\mu} s(s - 1) r^{s-2} + \frac{l(l+1) \hbar^2}{2\mu} r^{s-2} + V(r) r^s = E r^s. \]
The $r^{s-2}$ terms dominate here, meaning $-s(s-1) + l(l+1) = 0$ and so $s = l+1$ or $s = -l$.
But we must discard the $r^{-l}$ solutions for normalizability, so we must have
\[ u_{E,l}(r) \longrightarrow r^{l+1} \;\implies\; R_{E,l}(r) \longrightarrow r^{l}. \]
Our wave functions therefore vanish at the origin for all states except for $l=0$.
Notice how this dependence on $l$ is in line with the ``centrifugal term'' in the effective potential $V_\textrm{eff}(r)$ from earlier!

Now let's solve the radial equation for the Coulomb potential $V(r) = -Zq_e^2 / r$.
All bound states of such a system have negative energy, so it is convenient to write $E = -|E|$ and substitute $\rho = \sqrt{8\mu |E| / \hbar^2} \, r$ to get
\[ \frac{d^2u}{d\rho^2} - \frac{l(l+1)}{\rho^2} u + \left( \frac{\lambda}{\rho} - \frac{1}{4} \right) u = 0, \qquad \lambda = \frac{Ze^2}{\hbar} \sqrt{\frac{\mu}{2|E|}}. \]
Solving this using power series methods gives a nasty recurrence relation.
But we can observe that the solutions look like $u(\rho) = A e^{-\rho / 2}$ in the $\rho \to \infty$ limit, so we make the substitution $u(\rho) = \rho^{l+1} e^{-\rho / 2} F(\rho)$ to get
\[ \frac{d^2 F}{d\rho^2} = \left( \frac{2l+2}{\rho} - 1 \right) \frac{dF}{d\rho} - \left( \frac{\lambda}{\rho} - \frac{l+1}{\rho} \right) F = 0, \]
which leads to the first-order
\[ \frac{c_{k+1}}{c_k} = \frac{k + l + 1 - \lambda}{(k+1)(k + 2l + 2)}. \]
This has exponential limiting behavior, so we truncate the series by requiring $\lambda = 1 + l + n_r$ for any $n_r = 0, 1, 2, \ldots$.
From our definition of $\lambda$ we therefore have the quantized energies
\[ E_n = -\frac{\mu Z^2 q_e^{4}}{2\hbar^2 n^2}, \quad n = 1, 2, 3, \ldots, \]
where $n = l + 1 + n_r$ is called the principal quantum number.
(We often write $E_n$ in terms of the dimensionless fine-structure constant $\alpha = q_e^2 / \hbar c$.)
This leads to a shocking amount of degeneracy---not only does each $l$ have $2l+1$ values of $m$ all with the same energy, but the $l = 0, 1, \ldots, n-1$ also all have the same energy, despite the different centrifugal terms!
The total degeneracy is therefore $\sum_{l=0}^{n-1} (2l+1) = n^2$.
(There is no ``obvious'' symmetry causing this degeneracy in $l$, so we call it an ``accidental degeneracy''.)

In the ground state the power series $F$ is constant, and so    \vspace{-4pt}
\[ u_{1,0}(\rho) = c_0 \rho e^{-\rho / 2} \;\implies\; R_{1,0}(r) = 2 \left( \frac{Z}{a_0} \right)^{3 / 2} \! e^{-Zr / a_0}, \]
where we've used our expressions for $\rho$ and $|E|$ to write $\rho = (2Z / n) (r / a_0)$.
The length $a_0 = \hbar / \mu c \alpha$, called the Bohr length, is a convenient length scale for expressing the wave functions.
Note that the eigenfunctions $R_{n, n-1}$ only have one ``bump'' in their probability distributions $r^2 |R_{n,n-1}|^2$, adding one more bump for each drop in $l$.
These bumps will be key in determining the order in which states fill up in multielectron atoms.

\section{Solutions in the Finite Spherical Well}
Now we'll move into the realm of nuclear physics by investigating the bound state of a proton and neutron (called a deuteron).
The finite spherical well
\[ V(r) = \begin{cases} -V_0 & r < a, \\ 0 & r > a. \end{cases} \]
provides a crude approximation of the strong interaction between the two particles.
For the ground state we have $l=0$, so the radial equation is
\begin{align*}
    \frac{d^2u}{dr^2} = -k_0^2 u, &\quad r < a, \qquad k_0^2 = \frac{2\mu (V_0 + E)}{\hbar^2}, \\
    \frac{d^2u}{dr^2} = q^2 u, &\quad r > a, \qquad q^2 = -\frac{2\mu E}{\hbar^2}.
\end{align*}
Solving and imposing the boundary condition $u(0) = 0$ yields
\begin{align*}
    u(r) &= A \sin k_0 r, \quad r < a, \\
    u(r) &= C e^{-qr}, \qquad r > a.
\end{align*}
Continuity of $u(r)$ and $u'(r)$ at $r=a$ yields
\begin{align*}
    A \sin k_0 a &= C e^{-qa}, \\
    A k_0 \cos k_0 a &= -q C e^{-qa},
\end{align*}
and dividing these gives $\tan k_0 a = -k_0 / q$.
If we define $k_0 a = \zeta$ and $qa = \eta$ we can therefore write
\[ \zeta \cot \zeta = -\eta, \qquad \zeta^2 + \eta^2 = r_0^2, \quad r_0^2 = \frac{2\mu V_0 a^2}{\hbar^2} \]
Each solution to this system corresponds to a different bound state.
A graphical analysis reveals that there are no bound states unless $r_0 > \pi / 2$, that there is exactly one bound state for $\pi / 2 < r_0 < 3\pi / 2$, and so on.
This is in contrast with the one-dimensional finite square well, which always has at least one bound state.

The finite spherical well obviously isn't a very realistic model for the deuteron's interaction potential---experiment has shown, for example, that the deuteron has exactly one bound state.
It can still tell us some useful things, though!
For example, the deuteron has an intrinsic spin of one, and the proton and neutron do not bind in a spin-0 state, meaning the strong nuclear force is spin-dependent.

\section{Solutions in the Infinite Spherical Well}
For heavier nuclei, we imagine each nucleon as being confined to its own spherical potential well.
This well gets wider as we add more nucleons, but given that (for $l=0$) the effect of increasing $a$ is the same as that for increasing $V_0$, we'll just examine the infinite spherical well
\[ V(r) = \begin{cases} 0 & r < a, \\ \infty & r > a. \end{cases} \]
We begin with the radial equation in $r < a$:
\[ \frac{d^2 R}{dr^2} + \frac{2}{r} \frac{dR}{dr} - \frac{l(l+1)}{r^2} R + k^2 R = 0, \qquad k ^2= \frac{2\mu E}{\hbar^2}. \]
We could solve this equation using power series, but we can define the dimensionless variable $\rho = kr$ to get
\[ \frac{d^2 R}{d \rho^2} + \frac{2}{\rho} \frac{dR}{d \rho} + \left[ 1 - \frac{l(l+1)}{\rho^2} \right] R = 0, \]
which we can solve in terms of simple functions (applying the proper boundary conditions):
\begin{align*}
    j_l(\rho) &= \phantom{-}(-\rho)^{l} \left( \frac{1}{\rho} \frac{d}{d\rho} \right)^{l} \left( \frac{\sin \rho}{\rho} \right), \\
    \eta_l(\rho) &= -(-\rho)^{l} \left( \frac{1}{\rho} \frac{d}{d\rho} \right)^{l} \left( \frac{\cos \rho}{\rho} \right).
\end{align*}
These are called the spherical Bessel and Neumann functions, respectively, but since $u(r) = r R(r)$ must vanish at $r=0$ we can discard the Neumann function.
We also require that $R(a) = 0$ for continuity, so the energy eigenvalues are determined by
\[ j_l(ka) = 0. \]
The $l=0$ condition, $j_0(ka) = (\sin ka) / ka = 0$, is satisfied when
\[ E_{n_r,l=0} = \frac{\hbar^2 k^2}{2\mu} = \frac{\hbar^2}{2\mu} \left( \frac{n_r \pi}{a} \right)^2, \qquad n_r = 1, 2, 3, \ldots, \]
which agrees with the finite well in the limit $V_0 \to \infty$.
For the higher-order Bessel functions we can look up the zeroes in a table.
The most stable nuclei are the ones for which the protons and neutrons fill full energy levels $n_r$ (as we will later see, we can place at most two protons and two neutrons in the same state).

\section{Solutions in the Isotropic Harmonic Oscillator}
As our final example, we consider the three-dimensional isotropic (i.e., symmetric in all directions) harmonic oscillator:
\[ V(r) = \frac{1}{2} \mu \omega^2 r^2. \]
This problem is easily solved in Cartesian coordinates by leveraging our previous results---the Hamiltonian is simply the sum of three (commuting) one-dimensional Hamiltonians in the $x$, $y$, and $z$ directions, so the eigenstates are $\ket{E} = \ket{E_x, E_y, E_z}$ and the energies are
\[ E = \left( n + \frac{3}{2} \right) \hbar \omega, \qquad n = 0, 1, 2, \ldots, \]
where $n = n_x + n_y + n_z$.
The eigenfunctions are found by performing separation of variables on the Schrödinger equation; the result is a set of functions of the form $X_{n_x}(x) Y_{n_y}(y) Z_{n_z}(z)$, where each piece is an eigenfunction of the one-dimensional harmonic oscillator.

Alternatively, we can use spherical coordinates to write the radial equation
\[ -\frac{\hbar^2}{2\mu} \frac{d^2u}{dr^2} + \frac{l(l+1)\hbar^2}{2\mu r^2} u + \frac{1}{2} \mu \omega^2 r^2 u = Eu. \]
Alternatively,
\[ \frac{d^2 u}{d\rho^2} - \frac{l(l+1)}{\rho^2} u - \rho^2 u = -\lambda u, \qquad \rho = \sqrt{\frac{\mu \omega}{\hbar}} \,r, \quad \lambda = \frac{2E}{\hbar \omega}. \]
We once again search for a solution of the form $u(r) \rho^{l+1} e^{-\rho^2 / 2} f(\rho)$ using power series methods, arriving at a solution with exponential with limiting behavior unless the series terminates.
The quantization condition
\[ E = \left( 2n_r + l + \frac{3}{2} \right) \hbar \omega, \qquad n_r = 0, 1, 2, \ldots, \]
which matches with our previous energies provided we take $n = 2n_r + l$.

The high degree of degeneracy for the harmonic oscillator suggests that there is a ``hidden'' symmetry at play that has not yet been accounted for, just like we saw with the hydrogen atom.
The relevant constant of motion can be seen from classical mechanics---the $1 / r$ and $r^2$ potentials turn out to be the only ones for which orbits close on themselves and don't precess, so the direction of the vector pointing from the orbit's apocenter to its pericenter is preserved.

\end{document}
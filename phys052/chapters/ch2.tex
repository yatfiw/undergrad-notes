\documentclass[../p052main.tex]{subfiles}
\graphicspath{{\subfix{../figures/}}}

\begin{document}

\chapter{The Time-Independent Schrödinger Equation}
\section{Separation of Variables}
With a solid foundation in wave mechanics, we can now move into actually producing solutions to the Schrödinger equation.

We begin with an ansatz, a guess as to what our solution might look like.
Specifically, suppose a function of the form $\Psi(x,t) = f(t) \psi(x)$ solves the Schrödinger equation.
(Not every solution is of this form, but it'll give us the building blocks we need to create other ones!)

What follows from this assumption?
Substituting our ansatz into Schrödinger's equation gives
\[ -\frac{\hbar^2}{2m} \frac{d^2 \psi(x)}{dx^2} f(t) + V(x)f(t)\psi(x) = i\hbar \psi(x) \frac{df(t)}{dt}. \]
Obviously this holds when $f(t) = \psi(x) = 0$, but this is a pretty boring solution.
To find some others, we can separate variables: dividing by $f(t)\psi(x)$ gives
\[ -\frac{1}{\psi(x)}\frac{\hbar^2}{2m} \frac{d^2 \psi(x)}{dx^2} + V(x) = \frac{i\hbar}{f(t)} \frac{df(t)}{dt}, \]
an equation in which the left side is in $x$ and the right side is in $t$.
Now, the only way for these two sides to be equal for all $x,t$ is for both of them to be constant; call this constant $E$.
(As we'll see later, this is the energy in our system.)
This generates two, entirely disjoint ordinary differential equations:
\begin{align*}
    \frac{i\hbar}{f(t)}\frac{df}{dt} &= E & -\frac{1}{\psi(x)}\frac{\hbar^2}{2m} + V(x) &= E \\
    \frac{df(t)}{dt} &= -\frac{iE}{\hbar} f(t) & -\frac{\hbar^2}{2m} \frac{d^2 \psi(x)}{dx^2} + V(x) \psi(x) &= E \psi (x)
\end{align*}
The time equation is easy enough to solve.
Clearly,
\[ f(t) = f(0) e^{-\frac{iE}{\hbar}t} = f(0) e^{-i \omega t}, \quad \omega = \frac{E}{\hbar}. \]
(Usually we ignore the coefficient $f(0)$ because it'll probably change when we go to normalize the wave function anyway.)
The bottom right equation, on the other hand, is much more difficult.
It is known as the time-independent Schrödinger equation, and it requires a specific choice of $V(x)$ to solve.
Once we have this, though, we may be able to solve for $\psi(t)$ to get the wave function
\[ \Psi(x,t) = e^{-i \omega t} \psi(x). \]
This function solves the Schrödinger equation!
It is often referred to as a stationary state since its associated probability density is time-independent.
$E = \hbar \omega$ is the energy of this state.

% \begin{summary}
%     Using separation of variables on the Schrödinger equation gives $\Psi(x,t) = e^{-i\omega t} \psi(x)$, where $\psi$ is a solution to the time-independent Schrödinger equation
%     \[ -\frac{\hbar^2}{2m} \frac{d^2 \psi(x)}{dx^2} + V(x) \psi(x) = E \psi (x). \]
%     These $\Psi$ are not the only solutions to the Schrödinger equation, but they are the building blocks for all others.
%     Since their $|\Psi|^2$ are time-independent, these solutions are often called stationary states. 
% \end{summary}

\section{The Infinite Square Well}
To begin our study of the time-independent Schrödinger equation, let's pick the most ideal potential energy function we can: the infinite square well (also known as the particle in a box).
It is defined as follows:
\[ V(x) = \begin{cases} 0 & 0 \leq x \leq L, \\ \infty & \text{elsewhere}. \end{cases} \]
We can imagine our particle being ``trapped'' in the region in which $V$ is finite.
Infinite energy and momentum is bad.
Let's solve the time-independent Schrödinger equation under these conditions!

Outside the bounds of the box we have $V(x) = \infty$, so here $\psi(x) = 0$.
Inside the box, however, $V(x) = 0$, which gives the equation
\[ -\frac{\hbar^2}{2m} \frac{d^2 \psi(x)}{dx^2} = E \psi (x). \]
Define $k^2 = 2mE / \hbar^2$, so this turns into
\[ \frac{d^2 \psi}{dx^2} = -k^2 \psi, \]
which has the general solution
\[ \psi(x) = A \sin (kx) + B \cos (kx). \]
Now, when we go to pick our boundary conditions, note that the function must be continuous for its second derivative to exist in the first place.
So we must have $\psi(0) = 0$ and $\psi(L) = 0$.
The former immediately gives $B = 0$; the latter,
\[ 0 = A \sin kL. \]
The only way we get interesting solutions from this is to take
\[ kL = n \pi, \quad n = 1, 2, \ldots \]
Negative values of $n$ don't lead to new (linearly independent) solutions, and $n = 0$ is just boring.
The allowed values of $k$ can be labeled based on what integers they use:
\[ k_n = \frac{n \pi}{L}. \]
From how we defined $k$, though, this also means we only have certain allowed energies:
\[ E_n = \frac{\hbar^2 k_n^2}{2m} = \frac{n^2 \hbar^2 \pi^2}{2mL^2}. \]
The wave functions corresponding to these energies are
\[ \psi_n(x) = A_n \sin \frac{n \pi x}{L}, \quad 0 \leq x \leq L, \]
where $A = \sqrt{2/L}$ is found via normalization.
So, in summary,
\[ \psi_n(x) = \begin{cases} \sqrt{\frac{2}{L}} \sin \frac{n \pi x}{L} & 0 \leq x \leq L, \\ 0 & \text{elsewhere}, \end{cases} \qquad n = 1, 2, \ldots \]
Notice some things about this solution.
\begin{itemize}
    \item The associated wave function
    \[ \Psi_n(x,t) = e^{-i \omega t} \psi_n(x), \]
    appears to evolve in precisely the same way as, say, a guitar string would---all of the $n$ sinusoidal extrema periodically wriggle up and down with a set amplitude.
    \item Each wave function $\Psi_n$ is associated with a different amount of energy $E_n$; since these wave functions are ``discrete'', only supporting a half-integer number of wavelengths, their associated energy levels are also discrete!
    \item There is no $E = 0$ state.
    Even in the ground state $n=1$, the wave function still needs to oscillate so that it's connected to each wall of the well.
    (We don't consider $n=0$ because it isn't normalizable.)
\end{itemize}

% \begin{summary}
%     The infinite square well is a very simple choice of $V(x)$.
%     Applying continuity boundary conditions gives the quantized stationary states and energies
%     \[ \psi_n(x) = \begin{cases} \sqrt{\frac{2}{L}} \sin \frac{n \pi x}{L} & 0 \leq x \leq L, \\ 0 & \text{elsewhere}, \end{cases} \qquad E_n = \frac{n^2 \hbar^2 \pi^2}{2mL^2}, \]
%     for positive integers $n$.
% \end{summary}

\section{Time Evolution and Measurement Properties}
In general, the time-independent Schrödinger equation gives a discrete set of solutions $\psi_n$, each of which is associated with a quantized energy level $E_n$ and phase factor $e^{-i E_n t / \hbar}$.
Though the wave functions $\Psi_n(x,t)$ are stationary states (i.e., $|\Psi_n(x,t)|^2$ is time-independent), we can combine them together to get solutions that do evolve in time.

\begin{example}[]
    Suppose we construct a wave function using the linear combination
    \begin{align*}
        \Psi(x,t) &= \frac{1}{\sqrt{2}} \Psi_1(x,t) + \frac{1}{\sqrt{2}} \Psi_2(x,t) \\
        &= \frac{1}{\sqrt{2}} e^{-i E_1 t / \hbar} \psi_1(x) + \frac{1}{\sqrt{2}} e^{-i E_2 t / \hbar} \psi_2(x) \\
        &= \frac{1}{\sqrt{2}} e^{-i E_1 t /\hbar} \left[ \psi_1 + e^{-i (E_2 - E_1) t / \hbar} \psi_2 \right].
    \end{align*}
    The corresponding probability density function described turns out to be
    \begin{align*}
        |\Psi(x,t)|^2 &= \frac{1}{2} |\psi_1|^2 + \frac{1}{2}|\psi_2|^2 + \frac{1}{2} \psi_2^* \psi_1 e^{i (E_2 - E_1) t /\hbar} + \frac{1}{2} \psi_1^* \psi_2 e^{i (E_2 - E_1) t /\hbar} \\
        &= \frac{1}{2} \psi_1^2 + \frac{1}{2} \psi_2^2 + \psi_2 \psi_1 \cos \frac{(E_2 - E_1) t}{\hbar},
    \end{align*}
    where the last step is valid because $\psi_n$ are real-valued.
    Notice that this function is composed of two parts, one constant and one time-varying, and the time-varying portion has frequency $\omega = (E_2 - E_1) / \hbar$.
\end{example}

As it turns out, the wave functions produced by the time-independent Schrödinger equation form an orthonormal set!
The vector space spanned by these functions is called a Hilbert space.

We can use this to our advantage to determine how initial wave functions evolve through time.
First, we write the wave function as a linear combination of stationary states:
\[ \Psi(x,0) = \sum_{n=1}^{\infty} c_n(0) \psi_n(x). \]
To incorporate time-dependence, we simply multiply each stationary state by its phase factor; these factors vary the coefficients of the linear combination through time.
\begin{align*}
    \Psi(x,t) &= \sum_{n=1}^{\infty} c_n(t) \psi_n(x) = \Psi(x,t) \\
    &= \sum_{n=1}^{\infty} c_n(0) e^{-iE_nt/\hbar} \psi_n(x)
\end{align*}
Now, to determine the coefficients $c_n(0)$, we calculate the inner product
\begin{align*}
    \big< \Psi(x,0), \psi_n(x) \big> &= \int_{-\infty}^{\infty} \Psi^*(x,0)\psi_n(x) \,dx \\
    \intertext{To see how this is useful, let's decompose $\Psi$ into stationary states:}
    &= \int_{-\infty}^{\infty} \big( c_1(0) \psi_1^*(x) + c_2(0) \psi_2^*(x) + \cdots \big) \psi_n(x) \,dx \\
    &= \int_{-\infty}^{\infty} c_1(0)\psi_1^*(x)\psi_n(x) \,dx + \int_{-\infty}^{\infty} c_2(0)\psi_2^*(x)\psi_n(x) \,dx + \cdots \\
    \intertext{Notice that, by the orthogonality of stationary states, all but one of these integrals cancel:}
    &= \int_{-\infty}^{\infty} c_n(0) \psi_n^*(x) \psi_n(x) \,dx = c_n(0)
\end{align*}
As a side note, it will be useful to us to characterize cancellations like these using the Kronecker delta:
\[ \delta_{mn} = \begin{cases} 1 & m = n, \\ 0 & m \neq n. \end{cases} \]
For example, if $\psi_m$ and $\psi_n$ are normalized, then we can write
\[ \int_{-\infty}^{\infty} \psi_m^*\psi_n \,dx = \delta_{mn}. \]
Now, aside from being the coefficients in the linear combination for $\Psi$, $c_n$ have another very important interpretation: they describe the probability of measuring a particle as having a certain energy!
Specifically,
\[ P(E_n) = |c_n|^2. \]
So we can write expectation values for energy:
\[ \left< E \right> = \sum_{n=1}^{\infty} |c_n(0)|^2 E_n \qquad \left< E^2 \right> = \sum_{n=1}^{\infty} |c_n(0)|^2 E_n^2 \]

% \vspace{-5pt}
% \begin{summary}
%     In general, the time-independent Schrödinger equation gives a discrete set of mutually orthonormal solutions $\psi_n$, each with an associated energy $E_n$ and phase factor $e^{-i \omega_n t}$.
%     We can linearly combine these stationary states to get solutions that evolve in time.
%     The coefficients of such a linear combination is given by the inner product
%     \[ c_n(0) = \int_{-\infty}^{\infty} \Psi^*(x,0)\psi_n(x) \,dx. \]
%     Each $|c_n|^2$ represents the probability of finding $E_n$ upon a measurement of a particle's energy.
%     So we have
%     \[ \left< E \right> = \sum_{n=1}^{\infty} |c_n(0)|^2 E_n, \qquad \left< E^2 \right> = \sum_{n=1}^{\infty} |c_n(0)|^2 E_n^2. \]
% \end{summary}

\section{The Energy Operator}
The expression for $\left< E \right>$ above (via linear combination) is a good start, but it requires that we know \textit{all} of the $c_n$ ahead of time, which in general we won't.
To fix this, let's take a look at the expressions for the other expectation values we know:
\begin{align*}
    \left< x \right> &= \int_{-\infty}^{\infty} \Psi^*(x,t) x \Psi(x,t) \,dx & \left< p_x \right> &= \int_{-\infty}^{\infty} \Psi^*(x,t) \left( \frac{\hbar}{i} \frac{\partial}{\partial x} \right) \Psi(x,t) \,dx \\
    \left< x^2 \right> &= \int_{-\infty}^{\infty} \Psi^*(x,t) x^2 \Psi(x,t) \,dx & \left< p_x^2 \right> &= \int_{-\infty}^{\infty} \Psi^*(x,t) \left( \frac{\hbar}{i} \frac{\partial}{\partial x} \right)^2 \Psi(x,t) \,dx
\end{align*}
This notation allows us to understand these integrals in a new light.
We are not, in fact, just doing multiplications and derivatives in these integrals.
Instead, we are applying the position and momentum operators
\[ x_\textrm{op} = x \;\text{ and }\; p_\textrm{op} = \frac{\hbar}{i} \frac{\partial}{\partial x} \]
to $\Psi$, and then doing stuff with the result.
So we expect that there is some energy operator $E_\textrm{op}$ that we can use to determine $\left< E \right>$.
We call this operator the Hamiltonian $H$ and, as we might expect, it is the sum of the kinetic and potential energy operators!
\[ H = \frac{p_\textrm{op}^2}{2m} + V(x_\textrm{op}) = -\frac{\hbar^2}{2m}\frac{\partial^2}{\partial x^2} + V(x) \]
So the important expectation values for $E$ are
\[ \left< E \right> = \int_{-\infty}^{\infty} \Psi^*(x,t) H \Psi(x,t) \,dx \,\text{ and }\, \left< E^2 \right> = \int_{-\infty}^{\infty} \Psi^*(x,t) H^2 \Psi(x,t) \,dx. \]
Importantly, we can also write the time-independent Schrödinger equation as
\[ H \psi(x) = E \psi(x). \]
So the eigenfunctions of $H$ are the $\psi_n$ found via separation of variables, and its eigenvalues are their corresponding energies $E_n$!
We'll often refer to $\psi_n$ as energy eigenfunctions and $E_n$ as energy eigenvalues.

In fact, every observable has an associated operator whose eigenvalues are precisely the possible outcomes of a measurement!
Take the momentum operator, for example.
A particle's momentum is determined by its wavenumber $k$; applying $p_\textrm{op}$ to a wave function with this wavenumber gives
\[ p_\textrm{op} e^{ikx} = \frac{\hbar}{i} \frac{\partial}{\partial x} e^{ikx} = \hbar k e^{ikx}. \]
So $e^{ikx}$ is an eigenfunction of $p_\textrm{op}$, and its corresponding eigenvalue is the momentum $\hbar k$.
Since $k$ can vary continuously, this tells us that momentum is continuous (unlike energy!).

% \begin{summary}
%     We have the position, momentum, and energy operators:
%     \[ x_\textrm{op}= x \qquad p_\textrm{op} = \frac{\hbar}{i}\frac{\partial}{\partial x} \qquad H = \frac{p_\textrm{op}^2}{2m} + V(x_\textrm{op}) \]
%     The time-independent Schrödinger equation can be written as $H \psi = E \psi$, so the eigenvectors of the Hamiltonian $H$ are stationary states and the eigenvalues are their associated energies.
%     In general, the eigenvalues of an operator correspond to the possible outcomes of the measurement of some observable.
%     Here are the relevant expectation values for energy:
%     \[ \left< E \right> = \int_{-\infty}^{\infty} \Psi^*(x,t) H \Psi(x,t) \,dx \qquad \left< E^2 \right> = \int_{-\infty}^{\infty} \Psi^*(x,t) H^2 \Psi(x,t) \,dx \]
% \end{summary}

\end{document}
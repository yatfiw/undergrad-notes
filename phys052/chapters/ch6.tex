\documentclass[../p052main.tex]{subfiles}
\graphicspath{{\subfix{../figures/}}}

\begin{document}

\chapter{Identical Particles}
\section{Multiparticle Systems}
The Hamiltonian of a one-dimensional system of two distinguishable, non-interacting particles looks like
\[ \hat H = -\frac{\hbar^2}{2m_1} \frac{\partial^2}{\partial x_1^2} + V(x_1) - \frac{\hbar^2}{2m_2} \frac{\partial^2}{\partial x_2^2} + V(x_2). \]
As usual, we have the eigenvalue equation and Born rule
\[ \hat H \Psi(x_1,x_2) = E \Psi(x_1, x_2), \qquad dP = |\Psi(x_1,x_2)|^2 dx_1dx_2. \]
We could use separation of variables to get the wave function and energy
\[ \Psi(x_1,x_2) = \psi_{n_1}(x_1)\psi_{n_2}(x_2), \quad E = E_{n_1} + E_{n_2}, \]
and this wave function is normalized if it satisfies
\[ \iint_{\mathbb{R}^2} |\Psi(x_1,x_2)|^2 dx_1dx_2 = 1. \]
But in quantum mechanics identical particles are indistinguishable, and the wave function must reflect this.
To formalize this requirement we introduce the exchange operator, which essentially swaps all the ``labels'' for particle 1 with those for particle 2:
\[ \hat P_{12} \Psi(1,2) = \Psi(2,1). \]
For indistinguishable particles we require that $|\Psi(1,2)|^2 = |\Psi(2,1)|^2$, so
\[ \hat P_{12} \Psi(1,2) = e^{i\delta} \Psi(1,2). \]
We immediately get eigenfunctions---the eigenvalues can be found by noting that $\hat P_{12}^2 \Psi(1,2) = \Psi(1,2)$, so $e^{i\delta} = \pm 1$.
Taking $+1$ corresponds to a symmetric state and $-1$ corresponds to an antisymmetric state.
\begin{itemize}
    \item When a multiparticle state is symmetric under the exchange of any two identical particles, such particles are called bosons.
    Bosons have integral intrinsic spin. %<3
    
    \item When a multiparticle state is antisymmetric under the exchange of any two identical particles, such particles are called fermions.
    Fermions have half-integral intrinsic spin.
\end{itemize}
Note that, if two identical fermions were in the same quantum state, then the terms of the overall wave function would cancel and 
there would be no particle at all.
So no two identical fermions can be in the same quantum state---this is known as the Pauli exclusion principle.

This discussion has important implications for atoms with multiple electrons.
Consider a helium atom---if we ignore the kinetic energy of the nucleus and the repulsion between electrons, we have the Hamiltonian
\[ \hat H = -\frac{\hbar^2}{2m} \nabla_1^2 - \frac{Ze^2}{4\pi \epsilon_0 r_1} - \frac{\hbar^2}{2m} \nabla_2^2 - \frac{Ze^2}{4\pi \epsilon_0 r_2}. \]
Solving the corresponding Schrödinger equation via separation of variables shows that the energy eigenfunctions can each be written as a product of the single-electron hydrogenic energy eigenfunctions.
But since electrons are fermions, we must be careful that their overall state is antisymmetric.
So, the ground state must be
\[ \Psi(1,2) = \frac{1}{\sqrt{2}} \psi_{1s}(\mbf{r}_1) \psi_{1s}(\mbf{r}_2) \left[ \chi_+(1) \chi_-(2) - \chi_-(1) \chi_+(2) \right]. \]
Notice that the spatial part of the wave function is symmetric and that the spin state is antisymmetric, so the overall state is antisymmetric.
This is an example of a singlet state---that is, a state whose overall spin quantum number is $s = 0$.
We could verify this by computing $\hat{\mbf{S}}^2 \Psi = (\hat{\mbf{S}}_1^2 + \hat{\mbf{S}}_2^2) \Psi$, where $\hat{\mbf{S}}_1^2$ and $\hat{\mbf{S}}_2^2$ only act on particle-1 and particle-2 eigenstates, respectively.

The first-excited state of the helium atom also has a singlet state:
\[ \Psi(1,2) = \frac{1}{\sqrt{2}} \left[ \psi_{1s}(\mbf{r}_1)\psi_{2s}(\mbf{r}_2) + \psi_{2s}(\mbf{r}_1)\psi_{1s}(\mbf{r}_2) \right] \left[ \chi_+(1) \chi_-(2) - \chi_-(1) \chi_+(2) \right]. \]
But there are also three triplet states, which have $s = 1$:
\begin{align*}
    \Psi(1,2) &= \frac{1}{\sqrt{2}} \left[ \psi_{1s}(\mbf{r}_1)\psi_{2s}(\mbf{r}_2) - \psi_{2s}(\mbf{r}_1)\psi_{1s}(\mbf{r}_2) \right] \chi_+(1) \chi_+(2), \\
    \Psi(1,2) &= \frac{1}{\sqrt{2}} \left[ \psi_{1s}(\mbf{r}_1)\psi_{2s}(\mbf{r}_2) - \psi_{2s}(\mbf{r}_1)\psi_{1s}(\mbf{r}_2) \right] \chi_-(1) \chi_-(2), \\
    \Psi(1,2) &= \frac{1}{\sqrt{2}} \left[ \psi_{1s}(\mbf{r}_1)\psi_{2s}(\mbf{r}_2) - \psi_{2s}(\mbf{r}_1)\psi_{1s}(\mbf{r}_2) \right] \left[ \chi_+(1) \chi_-(2) + \chi_-(1) \chi_+(2) \right].
\end{align*}
It's common to label states using the spectroscopic notation $^{2S+1}L_{J}$ where $S$, $L$, and $J$ are the spin, orbital, and total angular momentum quantum numbers in the eigenvalue equations
\[ \hat{\mbf{S}}^2 \Psi = S(S+1)\hbar^2, \quad \hat{\mbf{L}}^2 \Psi = L(L+1)\hbar^2, \quad \hat{\mbf{J}}^2 \Psi = J(J+1)\hbar^2. \]
So in spectroscopic notation the above singlet states fall under $^1S_0$, while the triplet states look like $^3S_1$.
(We use the letters S, P, D, F, etc. to denote orbital angular momentum.)

When we account for Coulomb repulsion, different spectroscopic labels correspond to different energies.
For example, notice that the triplet states vanish when $\mbf{r}_1 = \mbf{r}_2$ while they don't in singlet states.
So singlet electrons are more likely to overlap than triplet electrons, meaning the Coulomb repulsion contributes more to the singlet energies than the triplet energies.

If we add another electron into the mix and try writing down a ground state of the form
\[ \Psi(1,2,3) = \psi_{1s}(\mbf{r}_1)\psi_{1s}(\mbf{r}_2)\psi_{1s}(\mbf{r}_3) X(1,2,3), \]
we notice that it isn't possible to create an overall-antisymmetric state.
Thus we are forced to push one of the electrons into the $2s$, and chemistry is born!%<3

\section{Quantum Statistics}
We'll turn our attention, now, to very large multiparticle systems.
A consequence of the Pauli exclusion principle is that the wave function for an entangled state of $N$ fermions will have $N!$ terms, so if we have, say, an Avogadro's number of particles, then writing down a wave function is hopeless.
The Fermi energy provides an alternative for ground-state systems, but for excited states we'll need to do some statistics.

Define $n(E)$ to be the average number of particles in a quantum state with energy $E$ (we allow fractional $n$ for superpositions), so
\[ N = \sum_{i}^{} n(E_i) \simeq \int_{0}^{\infty} n(E) D(E) \,dE, \]
where $D(E) \,dE$ is the number of states in an energy interval $dE$.
The integral is valid in the large-$N$ limit.
The total energy is
\[ E = \sum_{i}^{} E_i n(E_i) \simeq \int_{0}^{\infty} E \,n(E) D(E) \,dE. \]
So what, exactly, does $n(E)$ look like for excited states?
It depends on temperature!
Before continuing, though, we should clarify what we mean by ``temperature''.
It is the property of a system that determines the direction of likely energy transfer when in contact with another surface---energy flows from high to low temperatures, and there is no flow at all if both surface temperatures are the same.
The ``excess'' energy that one system can lose to another is quantified by $k_B T$, where $k_B$ is the Boltzmann constant.
(For example, a degenerate Fermi gas would be very cold because it already has the smallest allowed energy.)

Suppose we have some particles in a potential well.
If the system is in thermal equilibrium, we can now put some constraints on $n(E)$.
\begin{itemize}
    \item For bosons and fermions, $n(E) \geq 0$.
    \item For fermions, $0 \leq n(E) \leq 1$.
    \item For bosons and fermions, $N = \sum_{i}^{} n(E_i) \simeq \int_{0}^{\infty} n(E) D(E) \,dE$.
    \item $n(E)$ decreases monotonically in $E$.
\end{itemize}
In particular, we can quote two results from statistical mechanics:
\[ n_\textrm{BE}(E) = \frac{1}{e^{\alpha}e^{E / k_BT} - 1}, \quad n_\textrm{FD}(E) = \frac{1}{e^{\alpha}e^{E / k_BT} + 1}. \]
These are the Bose-Einstein distribution (for bosons) and Fermi-Dirac distribution (for fermions), respectively.
$\alpha$ is just a normalization constant; by convention we set $\alpha = -\mu / k_BT$, where $\mu$ is called the chemical potential.
Graphically, the Fermi-Dirac distribution looks something like a logistic curve while the Bose-Einstein looks more exponential.

\section{Cavity Radiation}
We'll use this idea of an energy distribution to describe why hot things glow.
In more technical terms, we'll investigate the origins of cavity (or blackbody) radiation.

Take a bunch of photons (spin-1 bosons) and plop them into a metal box with side length $L$.
The electric field in the box satisfies
\[ \nabla^2 \boldsymbol{\mathcal{E}} = \frac{1}{c} \frac{\partial^2}{\partial t^2} \boldsymbol{\mathcal{E}}. \]
We can break this into three separate equations for three individual components, each of which can be solved via separation of variables.
Imposing the boundary conditions $\nabla \cdot \boldsymbol{\mathcal{E}} = 0$ (from Gauss's law) and $\mathcal{E}_{||} = 0$ at the box boundary (since the box is conducting), we would find that, in the $x$-direction,
\[ \mathcal{E}_x = \mathcal{E}_{0x} \cos\left( \frac{n_x \pi x}{K} \right) \sin\left( \frac{n_y \pi x}{K} \right) \sin\left( \frac{n_z \pi x}{K} \right) \sin \omega t, \]
where $n_x,n_y,n_z \in \mathbb{N}$.
The same goes for $\mathcal{E}_y$ and $\mathcal{E}_z$, just with the cosine shuffled around.
Substituting this into the wave equation gives
\[ \omega^2 = c^2 \frac{\pi^2}{L^2} \left( n_x^2 + n_y^2 + n_z^2 \right), \]
and the photon energy is
\[ E = \hbar \omega = \frac{\hbar c \pi}{L} \sqrt{n_x^2 + n_y^2 + n_z^2} = \frac{\hbar c \pi}{L} r, \]
where $r$ is a distance in $n$-space.

We'll set up an integral for the total energy inside the box in order to determine the density of states for photons.
It turns out that photons cannot have spin projection 0, so for each ``box'' in $n$-space we have two polarizations.
Thus the number of states in a width-$dr$ shell in this space is $2 \cdot (1/8) \cdot 4\pi r^2dr$, so the total energy of all the states in all shells is
\begin{align*}
    E_\textrm{tot} &= \int_{0}^{\infty} E(r) \,n_\textrm{BE}(E(r)) \,\pi r^2 dr \\
    \intertext{The change of variables $E = (\hbar c\pi / L)r$ gives}
    &= \int_{0}^{\infty} E \,n_\textrm{BE}(E) \,\pi \left( \frac{L^2}{\hbar^2 c^2 \pi^2} E^2 \right) \left( \frac{L}{\hbar c \pi} dE \right) \\
    &= \int_{0}^{\infty} E n_\textrm{BE}(E) \frac{L^3 E^2}{\hbar^3 c^3 \pi^2} \,dE \\
    \intertext{From here we can pick out the density of states $D(E) = (L^3 E^2) / (\hbar^3 c^3 \pi^2)$. (We may also write this as $D(\nu) = 8\pi L^3 \nu^2 / c^3$.) But we'll go a little further and substitute $E = h\nu$ to get}
    &= \int_{0}^{\infty} \frac{L^3 \nu^2 8\pi}{c^3} n_\textrm{BE}(\nu) h\nu \,d\nu, \\
    \intertext{where $\mu = 0$ for photons. Finally substituting the Bose-Einstein distribution gives}
    &= \int_{0}^{\infty} \frac{8\pi hL^3}{c^3} \frac{\nu^3}{e^{h\nu / k_B T} - 1} d\nu,
\end{align*}
and we pick out the Planck function
\[ \rho(\nu) = \frac{8\pi h\nu^3}{c^3 \left( e^{h\nu / k_B T} - 1 \right)} \]
as an energy density per unit frequency.
Thus higher temperatures are associated with higher frequencies!
We can also write the Planck function in terms of wavelength,
\[ \rho(\lambda) = \frac{8\pi hc}{\lambda^{5} \left( e^{hc / \lambda k_BT} - 1 \right)}, \]
and solving the equation $d\rho / d\lambda = 0$ tells us that we should expect the energy density to peak at
\[ \lambda_\textrm{max} = \frac{2.9 \times 10^{-3} \textrm{ m K}}{T}. \]
This is known as Wien's law.
It'll also be useful to know the energy density inside our cavity.
A quick substitution $x = h\nu / k_BT$ gives
\[ \frac{E_\textrm{tot}}{L^3} = \int_{0}^{\infty} \frac{8\pi hL}{c^3} \frac{\nu^3}{e^{h\nu / k_B T} - 1} d\nu = \frac{8\pi k_B^{4}T^{4}}{c^3h^3} \int_{0}^{\infty} \frac{x^3}{e^{x} - 1}dx = \frac{8\pi k_B^{4}T^{4}}{c^3h^3} \frac{\pi^{4}}{15}. \]
We'll define $a$ such that $E_\textrm{tot} / L^3 = aT^{4}$.

We should note, at this point, that all this stuff we've done relies on the validity of the integral approximation, which itself hinges on $\lambda_\textrm{peak} \ll L$.
If $\lambda > 2L$, then none of this would have any hope at working.
But if we keep running with this premise, we'll arrive at a pretty neat result!

Suppose, now, that our metal box is a perfect absorber (a blackbody).
The box has temperature $T$ and the blackbody has temperature $T_{BB}$.
At equilibrium we expect to see no net energy flow, meaning
\[ R_{T,\textrm{inc}} = R_{T,\textrm{em}} \]
for the blackbody, where the radiancy $R(\nu)$ is the power per area per frequency.
We claim that the radiancy incident on the box walls is the same everywhere.

Suppose we put a little area-$A$ hole in one of the box's walls.
To determine the energy per frequency that escapes this hole, we'll integrate over a hemisphere inside the box:
\[ R_T(\nu) \cdot A \cdot \Delta t = \int \rho (\nu) f_\textrm{esc} dV, \]
where $f_\text{esc} = A \cos \theta / 4\pi r^2$ is the fraction of photons that escape and where $\theta$ corresponds to approaching the hole head-on.
So in spherical coordinates we have
\begin{align*}
    R_T A \Delta t &= \iiint \rho(\nu) \frac{A \cos \theta}{4\pi r^2} \,r^2\sin\theta \,dr d\theta d\phi \\
    &= \frac{\rho(\nu) A}{4\pi} \int_{0}^{c\Delta t} dr \int_{0}^{2\pi} d\phi \int_{0}^{\frac{\pi}{2}} \sin\theta \cos\theta \,d\theta \\
    &= \frac{\rho(\nu) A}{4\pi} \cdot c \Delta t \cdot 2\pi \cdot \frac{1}{2} \\
    R_T &= \frac{\rho(\nu)c}{4}
\end{align*}
This is the radiancy incident on the walls of the cavity!
By conservation of energy, it's also the radiancy absorbed and emitted by the blackbody.
To finish things off, we calculate the total radiancy (the flux) from a blackbody:
\[ F = \int_{0}^{\infty} R_T(\nu) d\nu = \frac{c}{4} \int_{0}^{\infty} \rho(\nu) d\nu = \frac{ca}{4} T^{4}, \]
where $a$ is the quantity defined earlier.
This gives us the Stefan-Boltzmann constant,
\[ \sigma = \frac{ca}{4} = \frac{2\pi^{5} k_B^{4}}{15c^2 h^3}, \]
and the total power (the luminosity)
\[ P = FA = A\sigma T^{4}. \]

\section{Bose-Einstein Condensation}
Now we'll take a gas of bosons confined to a length-$L$ box and determine the ``critical temperature'' below which most of the bosons are in their ground state.
This turns out to have some really interesting consequences for the matter's properties!

Our first thought might be to define $k_B T_C = \Delta E$, where $\Delta E$ is the gap between a boson's ground and first-excited states.
Any lower $T_C$ and most of the atoms won't have enough energy to be excited.
But this produces a critical temperature that's about a thousand times lower than the observed value, so we'll need to be a little more careful about things!

Let's begin with our expression for the total number of bosons in the gas:
\begin{align*}
    N &= \int_{0}^{\infty} n_\textrm{BE}(E) \cdot (2s+1) \cdot \frac{1}{8} \cdot 4\pi r^2 dr. \\
    \intertext{Taking $s=0$ for simplicity, substituting $n_\textrm{BE}(E)$, and changing variables to $E = (\hbar^2\pi^2 / 2mL^2)r^2$ gives}
    &= \frac{V(2m)^{3 / 2}}{4\hbar^3 \pi^2} \int_{0}^{\infty} \frac{E^{1 / 2} dE}{e^{(E - \mu)/k_BT} - 1}.
\end{align*}
But there's an issue here: the integrand is physically undefined for $E < \mu$, which is a problem if $\mu$ is positive.
Thankfully we can work around this by estimating the value of $\mu$ when we're close enough to the critical temperature.
For a very large number $N_0$ of bosons in the ground state, we can make the Taylor approximation
\[ N_0 = n_\textrm{BE}(E_0) = \frac{1}{e^{(E_0 - \mu) / k_BT} - 1} \simeq \frac{k_BT}{E_0 - \mu}. \]
Thus $\mu \simeq E_0 - k_BT / N_0 \simeq E_0$ when enough bosons are in the ground state, meaning writing the integral as
\begin{align*}
    N &= N_0 + \frac{V(2m)^{3 / 2}}{4\hbar^3 \pi^2} \int_{E_0}^{\infty} \frac{E^{1 / 2} dE}{e^{(E - \mu)/k_BT} - 1} \\
    \intertext{makes it valid. It evaluates to}
    &= N_0 + 2.6V \left( \frac{mk_BT}{2\pi\hbar^2} \right)^{3 / 2}.
\end{align*}
It turns out that if we define the critical temperature $T_C$ to be that at which $\mu(T_C) = 0$, then evaluating our original integral (before fixing the $\mu > 0$ issue) gives
\[ N = 2.6V \left( \frac{mk_BT_C}{2\pi\hbar^2} \right)^{3 / 2}; \]
we can thus rewrite our general result as
\[ N = N_0 + N \left( \frac{T}{T_C} \right)^{3 / 2}, \quad T_C = \frac{2\pi\hbar^2}{mk_B} \left( \frac{N}{2.6V} \right)^{2 / 3}. \]
This is the temperature below which a large fraction of bosons settle into the ground state; this fraction can be found by solving the above equation for $N_0 / N$, and is zero for $T > T_C$.

The state of matter that the bosons take on below this temperature is called a Bose-Einstein condensate since it's reminiscent of the transition that occurs when a gas cools and condenses to a liquid.
The process described here, however, is very different since it has nothing to do with attractive interactions between atoms and molecules.
The critical temperature is miniscule---for dilute gases it's on the order of nanokelvin and microkelvin, respectively.
In liquid helium, however, the separation between atoms is comparable to the de Broglie wavelength, meaning these quantum effects take place much more quickly (at around 2.2 K).

\section{Lasers and Masers}
Let's go back to working with photons.
We know that atomic transitions can absorb and emit photons at particular wavelengths, but it turns out that there are multiple ways for this emission to occur.
On one hand there's the familiar spontaneous emission, where an excited electron falls to a lower energy level.
On the other hand we have stimulated emission, where an existing photon interacts with the electron and spurs the transition to a lower energy and the emission of a new, ``coherent'' (identical) photon.

Suppose the walls of a cavity are comprised of atoms with two energy levels $E_1$ and $E_2$, and let $\rho(\nu,T)$ be the energy density of photons in the cavity with energy $E_2-E_1$.
Define $B_{12} \rho(\nu,T)$ and $B_{21} \rho(\nu)$ as the probability of absorption and stimulated emission, respectively, per atom per unit time.
Also define $A_{21}$ as that probability of spontaneous emission.
Then we have the differential equation
\[ \frac{dN_2}{dt} = -\frac{dN_1}{dt} = B_{12} \rho(\nu, T) N_1 - A_{21} N_2 - B_{21}\rho(\nu,T)N_2, \]
where $N_1$ and $N_2$ are the numbers of electrons in $E_1$ and $E_2$, respectively.
In thermal equilibrium these derivatives are zero and
\[ B_{12} \rho(\nu,T) N_1 = [A_{21} + B_{21} \rho(\nu,T)] N_2. \]
Also, from statistical mechanics
\[ \frac{N_2}{N_1} = -e^{(E_2 - E_1) / k_BT} = e^{-h\nu / k_BT} \]
in equilibrium, so together we get
\[ \rho(\nu,T) = \frac{A_{21}}{B_{12} e^{h\nu / k_BT} - B_{21}}. \]
If we map this to the known Planck function for cavity radiation we get
\begin{align*}
    A_{21} &= (8h\pi \nu^3 / c^3) B, \\
    B &= B_{21} = B_{12}.
\end{align*}
The rate of emission is thus given by
\begin{align*}
    \left[ A_{21} + B \rho(\nu,T) \right] N_2 &= N_2 \left[ A_{21} + \frac{8\pi h\nu^3 / c^3 \cdot B}{e^{h\nu / k_BT} - 1} \right] \\
    &= N_2 A_{21} \left[ 1 + N_\textrm{BE}(h\nu) \right]
\end{align*}
So the rate of emission is enhanced by the presence of photons!
To quantify this enhancement, suppose we have some photons incident at $x=0$ on a volume with some atoms in it.
If we ignore the effects of spontaneous emission, the evolution of the excited states is governed by
\begin{align*}
    \frac{dN_2}{dt} &= B \,\rho(\nu) (N_1 - N_2). \\
    \intertext{Noting that $d / dt (\rho(\nu)\Delta \nu) = -d / dt (N_2 h \nu / V)$, we can rewrite this as}
    \frac{d\rho(\nu)}{dt} &= \frac{h\nu}{V \Delta\nu} B\,\rho(\nu) (N_2-N_1). \\
    \intertext{Using $dx / dt$ in the chain rule gives}
    \frac{d\rho(\nu)}{dc} &= \frac{h\nu}{cV \Delta\nu} B\,\rho(\nu) (N_2-N_1),
    \intertext{and absorbing everything into a constant gives}
    \frac{d\rho(\nu)}{dx} &= \alpha\,\rho(\nu).
\end{align*}
So we get the exponential solution
\[ \rho(\nu,x) = \rho(\nu,0) e^{\alpha x}, \quad \alpha = B \frac{h\nu}{cV\Delta\nu}, \]
where $\alpha$ is called the gain constant.
When $\alpha > 0$, we get exponential growth in the amount of energy in our beam of light!
(In other words, the Light is Amplified via the Stimulated Emission of Radiation.)

In practice many lasers have three energy levels.
Atoms are transitioned from the ground state into a highly unstable excited state, and then they fall into an intermediate metastable state with a longer lifetime.
This metastable state is what we need to sustain $N_2 > N_1$ and thus lasing.

\end{document}
%<3
%<3
%<3
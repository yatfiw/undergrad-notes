\documentclass[../m131main.tex]{subfiles}
\graphicspath{{\subfix{../figures/}}}
%<3
\begin{document}

\chapter{Sequences and Series}
\section{Sequences}
Our attention, now, shifts to sequences of elements.
We'll begin with a rigorous definition for these objects.

\begin{definition}[Sequence]
    A sequence $\left\{ p_n \right\}$ in a metric space $X$ is a function $f : \N \to X$ which maps $n \mapsto p_n$.
\end{definition}

Oftentimes we'll encounter sequences that seem to end up arbitrarily close to a particular point, whereas others never seem to settle anywhere.
We can capture this intuition with the following definition.

\begin{definition}[Convergence]
    A sequence $\left\{ p_n \right\}$ converges if there is a $p \in X$ such that for any $\epsilon > 0$ there exists an $N \in \N$ such that $n \geq N$ implies $d(p_n, p) < \epsilon$.
\end{definition}

In this case we say ``$p_n$ converges to $p$'' or ``$p$ is the limit of the sequence'', and we write $p_n \to p$ or $\ds \lim_{n \to \infty} p_n = p$.

\vspace{-3pt}
If we're trying to prove the convergence of a given sequence, we may do so directly by determining a value for $N$ (usually as a function of $\epsilon$) for which the sequence becomes contstrained to a radius-$\epsilon$ neighborhood.
Some of the following properties might also be useful.

\begin{theorem}[Convergence properties]
    \begin{enumerate}[label=(\alph*)]
        \item If $p_n \to p$ and $p_n \to q$ then $p = q$.
        \item If $p_n$ converges then $p_n$ is bounded.
        \item If $p$ is a limit point of $E \subset X$ then there exists a sequence $\left\{ p_n \right\}$ in $E$ such that $p_n \to p$.
        \item $p_n \to p$ if and only if every neighborhood of $p$ contains all but finitely many $p_n$.
    \end{enumerate}
\end{theorem}

\begin{proof}
    We prove each part in turn.
    \begin{enumerate}[label=(\alph*)]
        \item Suppose $p_n \to p$ and $p_n \to q$ and let $\epsilon = d(p,q)$.
        Assume, for contradiciton, that $p \neq q$, so $\epsilon > 0$.
        By the definition of convergence, there exist $N_p, N_q$ such that $n \geq N_p \implies d(p_n, p) < \epsilon / 2$ and $n \geq N_q \implies d(p_n, q) < \epsilon / 2$.
        Let $N = \max \left\{ N_p, N_q \right\}$; by the triangle inequality,
        \[ \epsilon = d(p,q) \leq d(p_n, p) + d(p_n, q) < \epsilon, \]
        a contradiction.

        \item Let $\epsilon = 1$.
        Then there exists an $N$ such that $n > N \implies d(p_n, p) < 1$.
        All $p_n$ are contained in the ball $B_{R+1}(p)$, where $R = \max \left\{ 1, d(p_1, p), \ldots, d(p_N, p) \right\}$, so $p_n$ is bounded.

        \item If $p$ is a limit point of $E$ then every neighborhood contains a point $p_n \neq p$ such that $p_n \in E$.
        So we can construct a convergent sequence $\left\{ p_n \right\}$ where $p_n \in B_{1 / n}(p)$.

        \item ($\Rightarrow$)
        Consider an $\epsilon > 0$.
        Then there exists an $N$ such that $n > N \implies d(p_n, p) < \epsilon$.
        There are finitely many natural numbers satisfying $n \leq N$, so there are finitely many $p_n$ outside this neighborhood.

        ($\Leftarrow$)
        Consider a neighborhood $N_\epsilon (p)$ that contains all but $M$ points.
        Then $d(p_n, p) < \epsilon$ for all $n \geq M + 1$.
    \end{enumerate}
    This completes the proof.
\end{proof}

Notably, bounded sequences do not necessarily converge, and points of convergence are not necessarily limit points of a sequence's range.

Convergent sequences in $\C$ also play nicely with the field operations, giving us a few more useful rules.

\begin{theorem}[Convergence properties in $\C$]
    Suppose we have two sequences $\left\{ s_n \right\}, \left\{ t_n \right\} \in \C$, where $s_n \to s$ and $t_n \to t$.
    Then:
    \begin{enumerate}[label=(\alph*)]
        \item $\lim_{n \to \infty} (s_n + t_n) = s + t$.
        \item $\lim_{n \to \infty} (cs_n) = cs$ for scalar $c$. 
        \item $\lim_{n \to \infty} (c + s_n) = c + s$ for scalar $c$.
        \item $\lim_{n \to \infty} (s_n t_n) = st$.
    \end{enumerate}
\end{theorem}

\begin{proof}
    For each of the proofs below we consider $\left\{ s_n \right\} \to s$ and $\left\{ t_n \right\} \to t$.
    Suppose $c$ is a scalar.
    \begin{enumerate}[label=(\alph*)]
        \item Fix $\epsilon > 0$.
        There exist $N_1$ such that $n > N_1 \implies |s_n - s| < \epsilon / 2$ and $N_2$ such that $n > N_2 \implies |t_n - t| < \epsilon / 2$.
        Let $N = \max \left\{ N_1, N_2 \right\}$; then for $n > N$ we have, by the triangle inequality,
        \[ |(s_n + t_n) - (s + t)| \leq |s_n - s| + |t_n - t|, \]
        which is less than $\epsilon$, as desired.

        \item Fix $\epsilon > 0$.
        There exists an $N$ such that $n > N \implies |s_n - s| < \epsilon / c$.
        So
        \[ |cs_n - cs| = |c| |s_n - s|, \]
        which is less than $\epsilon$, as desired.

        \item This is a simple combination of the previous two arguments.

        \item Fix $\epsilon > 0$ and let $k = \max \left\{ s, t, 1, \epsilon \right\}$.
        There exist $N_1$ such that $n > N_1 \implies |s_n - s| < \epsilon / 2$ and $N_2$ such that $n > N_2 \implies |t_n - t| < \epsilon / 2$.
        Let $N = \max \left\{ N_1, N_2 \right\}$; then for $n > N$ we have
        \begin{align*}
            |s_n t_n - st| &= |(s_n - s)(t_n - t) + s(t_n - t) + t(s_n - s)|,
            \intertext{and by the triangle inequality,}
            &\leq |(s_n - s)(t_n - t)| + |s(t_n - t) + t(s_n - s)| \\
            &< \frac{\epsilon}{3k} \cdot \frac{\epsilon}{3k} + |s| \cdot \frac{\epsilon}{3k} + |t| \cdot \frac{\epsilon}{3k} \\
            \intertext{Since $k \geq 1$, replacing it with a 1 will not make the expression smaller, meaning}
            &\leq \frac{\epsilon^2}{9} + \frac{\epsilon}{3} + \frac{\epsilon}{3},
        \end{align*}
        which is less than $\epsilon$, as desired.
    \end{enumerate}
    This completes the proof.
\end{proof}

\subsection*{Subsequences, Bounded Sequences, and Cauchy Sequences}
Just as we can have subsets embedded within potentially larger sets, we can have subsequences embedded within potentially larger sequences.

\begin{definition}[Subsequence]
    Consider a sequence $\left\{ p_n \right\}$.
    If there is an increasing sequence $n_1 < n_2 < \cdots$ in $\N$, then $\left\{ p_{n_i} \right\}$ is a subsequence.
\end{definition}

\begin{theorem}[Subsequences of convergent sequences]
    If $p_n \to p$ then every subsequence $p_{n_i} \to p$.
\end{theorem}

\begin{proof}
    Every neighborhood of $p$ contains all but finitely many points of $p_n$, so it also contains all but finitely many points of $p_{n_i}$.
\end{proof}

Of course, if a sequence does not converge, then its subsequences may not all converge to the same value.
We'll introduce some vocabulary that describes this.

\begin{definition}[Subsequential limit]
    If a subsequence of $\left\{ p_n \right\}$ converges to $p$ then $p$ is a subsequential limit.
\end{definition}

\begin{definition}[Upper and lower limits]
    Let $E$ be the set of all subsequential limits of $\left\{ s_n \right\}$, including possibly $\pm \infty$.
    Then the upper and lower limits are, respectively,
    \[ S^* = \sup E = \limsup_{n \to \infty} s_n \quad\text{ and }\quad S_* = \inf E = \liminf_{n \to \infty} s_n. \]
\end{definition}

We can also talk about bounded sequences, that is, sequences whose ranges are bounded.
In particular, we can make a nice statement about bounded sequences that are everywhere-increasing or -decreasing.

\begin{definition}[Monotonic sequence]
    A sequence $\left\{ s_n \right\}$ is monotonoically increasing if $s_n \leq s_{n+1}$ for all $n$.
    It is monotonically decreasing if $s_n \geq s_{n+1}$.
    Either could be called monotonic.
\end{definition}

\begin{theorem}[]
    A monotonic sequence in $\R$ is bounded if and only if it converges.
\end{theorem}

\begin{proof}
    ($\Rightarrow$) Suppose $\left\{ s_n \right\}$ is bounded and monotonically increasing.
    Let $s = \sup (\pfn{range} \left\{ s_n \right\})$.
    (Note that $s$ exists because $\left\{ s_n \right\}$ is a bounded, nonempty set in the reals.)

    By definition, $s_N < s$ for any $N$ because $s$ is the supremum.
    Also, for any $\epsilon > 0$ there exists an $N$ such that $s - \epsilon < S_N$.
    Since the sequence is increasing, $s_n \geq s_N$ for all $n > N$, so $\left\{ s_n \right\}$ converges so its supremum $s$.
    (We could make the same for the infimum of a monotonically decreasing sequence.)

    ($\Leftarrow$) Suppose $\left\{ s_n \right\}$ converges.
    Then it is bounded.
\end{proof}

If a sequence is not bounded, we say it is unbounded.
We can be a little more specific with the behavior of this unboundedness, though.

\begin{definition}[Unbounded sequence]
    If for all $M \in \R$ there exists an $N \in \N$ such that $n > N$ implies...
    \begin{itemize}
        \item ... $s_n > M$, then $s \to \infty$.
        \item ... $s_n < M$, then $s \to -\infty$.
    \end{itemize}
\end{definition}

Finally, our intuition for convergence seems to imply that, after a point, a sequence's terms are all really ``close'' to each other.
Unfortunately this is not the same as convergence, but the two ideas are related.

\begin{definition}[Cauchy sequence]
    A sequence $\left\{ p_n \right\}$ is Cauchy if for all $\epsilon > 0$ there exists an $N$ such that $m,n \geq N$ implies $d(p_n, p_m) < \epsilon$.
\end{definition}

\begin{theorem}[]
    If $\left\{ p_n \right\}$ converges then $\left\{ p_n \right\}$ is Cauchy.
\end{theorem}

\begin{proof}
    Let $\left\{ p_n \right\}$ be a convergent sequence.
    Given $\epsilon > 0$, there exists $N$ such that $n > N$ implies $d(p_n, p) , \epsilon / 2$.
    In this case $n,m > N$ implies
    \[ d(p_n, p_m) \leq d(p_n, p) + d(p_m, p) \]
    by the triangle inequality.
    So $d(p_m, p_n) < \epsilon$, as desired.
\end{proof}

\section{Compactness and Completeness}
We can draw a surprisingly tight connection between subsequences and compact sets, starting with a theorem that we may use to restate something familiar,

\begin{theorem}[]
    Every sequence in a compact metric space $X$ has a subsequence converging to a point in $X$.
\end{theorem}

\begin{proof}
    Let $R = \pfn{range} \left\{ p_n \right\}$.
    If $R$ is finite, then some $p \in \left\{ p_n \right\}$ appears infinitely many times and there is a subsequence that converges to $p$.

    If $R$ is infinite, then it is an infinite subset of a compact set and thus has a limit point $p$ in the broader metric space $X$.
    Take a point $p_i$ within a neighborhood $N_r(p)$; then, reduce $r < d(p_i, p)$ and pick another point with index greater than $i$.
    Continuing this process indefinitely constructs a sequence converging to $p$.
\end{proof}

\begin{corollary}[Bolzano-Weierstrass theorem]
    Every bounded sequence in $\R^k$ contains a convergent subsequence.
\end{corollary}

But we can do better!
Without proof, we'll use an equivalent notion of compactness to generalize further.

\begin{definition}[Sequentially compact]
    A metric space $X$ is sequentially compact if every sequence has a convergent subsequence in $X$.
\end{definition}

\begin{theorem}[]
    A metric space is compact if and only if it is sequentially compact.
\end{theorem}

\begin{corollary}[Bolzano-Weierstrass theorem]
    A subset of $\R^k$ is sequentially compact if and only if it is closed and bounded.
\end{corollary}

But this is just the Heine-Borel theorem!
We can use this development to list a few equivalent properties of subsets of $\R^k$.

\begin{theorem}[Big exciting theorem about $\R$]
    Let $E \subset \R^k$.
    The following are equivalent.
    \begin{itemize}[topsep=0pt]
        \item $E$ is closed and bounded.
        \item $E$ is compact.
        \item $E$ is sequentially compact.
        \item Every sequence in $E$ has a convergent subsequence.
    \end{itemize}
\end{theorem}

We've seen that another important property of $\R$ involves completion, which we can define using sequences.

\begin{definition}[Complete]
    A metric space $X$ is complete if every Cauchy sequence in $X$ converges to a point in $X$.
\end{definition}

\begin{theorem}[Completion] \label{thmCompletion}
    Every metric space $(X, d)$ has a completion $(X^*, d)$.
    That is, there is a way to define $X^* \supseteq X$ such that $X^*$ is complete and has the same metric as $X$.
\end{theorem}

We'll start with the particular case of $\Q$ to outline how we might construct such a completion.
This will give us a new way to think about the completeness of the reals: $\R$ is the completion of $\Q$, meaning it's a metric space in which every Cauchy sequence converges!

\section{Constructing the Reals}
Up to this point we've essentially been treating the set of real numbers as given.
But we're now in a position to rigorously construct $\R$ from the ground up as the completion of the rationals!
First, let's make a slight revision to our definitions of Cauchy sequences and convergence.

\begin{definition}[Cauchy and convergence in $\Q$]
    A sequence $\left\{ a_n \right\}$ is a Cauchy sequence if for any positive $\omega \in \Q$ there exists an $n \in \N$ such that $n,m > N$ implies $|a_n - a_m| < \omega$.

    A sequence $\left\{ a_n \right\}$ converges to $a \in \Q$ if for any positive $\omega \in \Q$ there exists an $n \in \N$ such that $n > N$ implies $|a_n - a| < \omega$.
\end{definition}

The only significant change we've made is ensuring that the $\epsilon$ we've given (now called $\omega$) is rational.

The idea is to define a real number as a Cauchy sequence of rational numbers.
But for any Cauchy sequence, there are infinitely many others that ``converge'' to the same number.
To avoid this ambiguity, we instead define each real number as an equivalence class of Cauchy sequences.

\begin{definition}[Real numbers]
    Let $\left\{ a_n \right\}$ and $\left\{ b_n \right\}$ be Cauchy sequences in $\Q$.
    We say they are equivalent if $a_n - b_n \to 0$.

    The real numbers $\R$ is the set of all equivalence classes of these Cauchy sequences.
\end{definition}

Showing that the above relation is an equivalence relation involves a relatively straightforward application of the definition of convergence and the triangle inequality.
Now let's add some structure to this set we've created!

\begin{definition}[Algebraic operations of $\R$]
    Let $a,b \in \R$, so there are Cauchy sequences of rational numbers $\left\{ a_n \right\}, \left\{ b_n \right\}$ that represent the equivalence classes for $a$ and $b$.
    \begin{itemize}
        \item (Addition)
        Define $a+b$ to be the equivalence class of $\left\{ a_n + b_n \right\}$.
        \item (Multiplication)
        Define $a \cdot b$ to be the equivalence class of $\left\{ a_n \cdot b_n \right\}$.
    \end{itemize}
\end{definition}

We have to make sure these operations are well-defined---that is, the result of performing them shouldn't depend on which representatives of the equivalence classes we choose.
We'll prove that our definition for addition is well-defined; we'll skip multiplication for brevity's sake.

\begin{theorem}[Addition is well-defined]
    Addition in $\R$ is well-defined.
\end{theorem}

\begin{proof}
    Choose two members $\left\{ a_n \right\}, \left\{ \alpha_n \right\}$ of the same equivalence class, so $a_n - \alpha_n \to 0$.
    Also choose two members $\left\{ b_n \right\}, \left\{ \beta_n \right\}$ of a different equivalence class, so $b_n - \beta_n \to 0$.
    We want to show that $\left\{ a_n + b_n \right\}$ is equivalent to $\left\{ \alpha_n + \beta_n \right\}$.

    Consider $(a_n + b_n) - (\alpha_n + \beta_n)$.
    We can rewrite ths as $(a_n - \alpha_n) + (b_n - \beta_n)$.
    By our definition of Cauchy equivalence, for rational $\omega > \Q$ there exist $N_1, N_2$ such that $n > N_1$ implies $|(a_n - \alpha_n) - 0| < \omega / 2$ and $n > N_2$ implies $|(b_n - \beta_n) - 0| < \omega / 2$.
    So for $N = \max \left\{ N_1, N_2 \right\}$ we have, for $n > N$,
    \[ |(a_n - \alpha_n) + (b_n - \beta_n)| \leq \frac{\omega}{2} + \frac{\omega}{2} = \omega. \]
    So $(a_n + b_n) - (\alpha_n + \beta_n) \to 0$, as desired.
\end{proof}

The next step would be to show that $\R$ with these operations is a field, but this is a relatively painful and unenlightening process so we'll omit that, too.
Now let's add an order!

\begin{definition}[Order of $\R$]
    A number $a \in \R$ is positive if $a \neq 0$ and it is represented by a Cauchy sequence such that for some $N$, $n > N$ implies $a_n > 0$.
    We say that $a > b$ (for $b \in \R$) if $a-b$ is positive.
\end{definition}

To prove that this is an order we'd need to show that it satisfies trichotomy and transitivity.
We must also show that the order is well-defined!
Again, we'll omit the details of these arguments.

Lastly, we must show that we've created an ordered field.
Again, we'll only show this for the additive axiom.

\begin{theorem}[Additive ordered field axiom]
    Let $a,b,c \in \R$ with $b < c$.
    Then $a + b < a + c$.
\end{theorem}

\begin{proof}
    Let $a = [\left\{ a_n \right\}]$, $b = [\left\{ b_n \right\}]$, $c = [\left\{ c_n \right\}]$, where the brackets denote equivalence classes.
    Suppose $c > b$, so by the definition of order there exists an $N$ so that $n > N$ implies $c_n - b_n > 0$.
    Then $c_n > b_n$, and we can add another rational number $a_n$ to both sides of the inequality and rearrange to get
    \[ (a_n + c_n) - (a_n + b_n) > 0 \]
    for any $n > N$.
    Thus $(a + c) - (a + b) > 0$ and $a + c > a + b$, as desired.
\end{proof}

From here, we could prove all the other cool stuff we know to be true about real numbers!
We'll list three big properties here, leaving the proofs as exercises.

\begin{theorem}[Archimedean property]
    $\R$ has the Archimedean property.
\end{theorem}

\begin{theorem}[Density of $\Q$ in $\R$]
    $\Q$ is dense in $\R$.
\end{theorem}

\begin{theorem}[Completeness]
    $\R$ has the least upper bound property.
\end{theorem}

We could use this construction as a template for proving Theorem 2.\ref{thmCompletion}.
The distance between two elements in a completion $X^*$, which are equivalence classes of Cauchy sequences in $X$, is given by the limit of the distance between their terms.
This ensures that $X$ is truly embedded within $X^*$---in particular, there exists a distance-preserving mapping $\varphi : X \to X^*$; we call such a mapping an isometry.
(We may interpret $\varphi(X)$ as the set $X$ equipped with the new metric.)
It turns out that $X^*$ with this metric is complete!
Two important corollaries are that $\varphi(X)$ is dense in $X^*$ and $\varphi(X) = X^*$ for complete $X$.

% \section{Completion of Metric Spaces (14b)} % FINISH! (TODO)
% Now we'll provide a general proof for Theorem 2.\ref{thmCompletion}, following our construction for $\R$.

% Consider a metric space $(X,d)$.
% We say that two Cauchy sequences $\left\{ p_n \right\}, \left\{ q_n \right\}$ in $X$ are equivalent if
% \[ \lim_{n \to \infty} d(p_n, q_n) = 0. \]
% It is, again, relatively straightforward to show that this is an equivalence relation.
% We'll define $X^\star$ as the set of the corresponding equivalence classes; now we just need to put a metric on this set.

% Let $P,Q \in X^*$ have representative sequences $\left\{ p_n \right\}, \left\{ q_n \right\}$, and define the metric
% \[ \Delta (P, Q) = \lim_{n \to \infty} d(p_n, q_n). \]
% We could prove that $\Delta$ is both well-defined and a metric on $X^*$, but those proofs would clutter the flow here so we'll omit them.
% Now, in order for this metric to be sensible, we must ensure that $X$ is truly embedded within $X^*$.
% In particular, the distance between two elements in $X$ must be the same as between the two equivalent elements in $X^*$.
% Let $p,q \in X$ have equivalents $P,Q \in X^*$ with representatives $\left\{ p_n \right\}, \left\{ q_n \right\}$.
% So
% \[ \Delta (P,Q)  = \lim_{n \to \infty} d(p_n,q_n) = \lim_{n \to \infty} d(p,q) = d(p,q), \]
% as desired.
% Thus there exists a distance-preserving mapping $\varphi : X \to X^*$; we call such a mapping an isometry.
% We may interpret $\varphi(X)$ as the set $X$ equipped with the new metric $\Delta$.

% Finally, we'll show in class that $X^*$ is complete. % TODO
% As a corollary, we'll also show that a complete metric space is its own completion.

% \begin{corollary}[]
%     If a metric space $X$ is complete and there exists an isometry $\varphi : X \to X^*$, where $X^*$ is the completion of $X$, then $\varphi(X) = X^*$.
% \end{corollary}

% \begin{proof}
%     Coming soon... % TODO
% \end{proof}

% We have one last result.

% \begin{corollary}[]
%     $\varphi(X)$ is dense in $X^*$.
% \end{corollary}

% \begin{proof}
%     We'll show that every point $P \in X^*$ is in $\varphi(X)$ or is a limit point of $\varphi(X)$. 

%     It suffices to show that every $P \not\in X^*$ is a limit point of $\varphi(X)$.
%     Let $\left\{ p_n \right\}$ be a representative of the equivalence class $P$; since this sequence is Cauchy, for any $N_\epsilon (P)$, $\epsilon > 0$, there exists $N \in \N$ such that $m,n > N$ implies $d(p_m,p_n) < \epsilon$.
%     Now, let $Q = \varphi(p_m)$, so
%     \[ \Delta (P,Q) = \lim_{n \to \infty} d(p_n, p_m) < \epsilon. \]
%     So $Q \in \varphi(X)$ and $P \not\in \varphi(X)$, meaning we have $Q \neq P$ where $Q \in N_\epsilon (P)$ and $Q \in X^*$ for any $\epsilon > 0$.
%     Thus $P$ is a limit point of $\varphi(X)$ and $\varphi(X)$ is dense in $X^*$, as desired.
% \end{proof}

\section{Series}
We'll shift our attention, now, to adding up successive terms in a sequence.

\begin{definition}[Series]
    Suppose $q > p$.
    Given $\left\{ a_n \right\}$, the sum
    \[ \sum_{n=p}^{q} a_n = a_p + a_{p+1} + \cdots + a_q. \]
    is called a series, and each $a_n$ is called a term.
    We may have an infinite number of terms, in which case the sum is called an infinite series.
\end{definition}

But we can frame series in such a way that they just become sequences in disguise!
To make this more clear, we'll define the terms in such a sequence.

\begin{definition}[Partial sum]
    The $n$th partial sum of a series is the sum of the first $n$ terms in the series.
\end{definition}

So determining the behavior of a series is equivalent to determining the behavior of the sequence of its partial sums.
We're particularly interested in the convergence of infinite series; we'll spend some time developing criteria to determine this convergence.
Let's start by taking advantage of the completeness of $\R$.

\pagebreak

\begin{theorem}[Cauchy criterion]
    Suppose $a_n \in \R$ and $m > n \in \N$.
    Then $\sum_{n=1}^{\infty} a_n$ converges if and only if for all $\epsilon > 0$ there exists an $N$ such that $n,m > N$ implies $\left| \sum_{k=n}^{m} a_k \right| < \epsilon$.
\end{theorem}

\begin{proof}
    Consider the sequence of partial sums $\left\{ s_n \right\}$ where $s_k = \sum_{n=1}^{k} a_n$ and $a_n \in \R$.
    The distance between two terms in this sequence is given by
    \[ d(s_{k-1}, s_m) = \left| \sum_{n=1}^{m} a_n - \sum_{n=1}^{k-1} a_n \right| = \left| \sum_{n=k}^{m} a_n \right|. \]

    ($\Rightarrow$) Suppose $\sum_{n=1}^{\infty} a_n$ converges.
    Then the corresponding sequence of partial sums converges and thus is Cauchy.
    So for all $\epsilon > 0$ there exists an $N$ such that $n,m > N$ implies $\left| \sum_{n=k}^{m} a_n \right| < \epsilon$, as desired.

    ($\Leftarrow$) Suppose there exists an $N$ such that $m,n > N$ implies $\left| \sum_{n=k}^{m} a_n \right| < \epsilon$ for any $\epsilon > 0$.
    Then $\left\{ s_n \right\}$ is Cauchy, and because the terms in the sequence are real it also converges.
    Thus $\sum_{n=1}^{\infty} a_n$ converges.
\end{proof} % TODO: change indices?

In the special case $m=n$, then we get a corollary about the terms of a convergent series.

\begin{corollary}[Term test]
    If $\sum_{n=1}^{\infty} a_n$ converges then $a_n \to 0$.
\end{corollary}

The converse of this statement is false, but the contrapositive provides a useful sufficient condition for divergence.
There's another theorem we can prove that relates a series' terms to its convergence.

\begin{theorem}[Nonnegative test]
    Let $a_n \in \R$.
    If $a_n \geq 0$ for all $n$ then $\sum_{n=1}^{\infty} a_n$ converges if and only if its partial sums are bounded.
\end{theorem}

\begin{proof}
    Consider an infinite series with nonnegative terms.
    Then the corresponding sequence of partial sums is monotonic, meaning it is bounded if and only if it converges.
\end{proof}

We're now in a position to make a statement about the convergence of a particular class of series that appears in many applications.

\begin{definition}[Geometric series]
    Let $x \in \R$.
    The series of the form
    \[ \sum_{n=0}^{\infty} x^{n} = 1 + x + x^2 + \cdots \]
    is called a geometric series.
\end{definition}

\begin{theorem}[Geometric series convergence]
    The sum $\sum_{n=0}^{\infty} x^{n}$ converges to $1 / (1-x)$ if and only if $|x| < 1$.
\end{theorem}

\begin{proof}
    If $|x| \geq 1$ then the terms do not approach zero and the series diverges.
    If $|x| < 1$ then the $n$th partial sum $s_n$ of the series obeys
    \begin{align*}
        s_n &= 1 + x + \cdots + x^{n}, \\
        (1-x)s_n &= (1 + x + \cdots + x^{n}) - (x + x^2 + \cdots + x^{n+1}) \\
        &= 1 - x^{n+1}.
    \end{align*}
    Since $x \neq 1$ we have $s_n = (1 - x^{n+1}) / (1-x)$, which we can show converges to $1 / (1 - x)$.
\end{proof}

Once we know about the convergence of relatively simple series, we can use them to determine the behavior of somewhat more complex series.

\begin{theorem}[Comparison test]
    \begin{enumerate}[label=(\alph*)]
        \item If $|a_n| \leq c_n$ for sufficiently large $n$ and $\sum_{n=1}^{\infty}c_n$ converges, then $\sum_{n=1}^{\infty}a_n$ converges.
        \item If $a \geq d_n \geq 0$ for sufficiently large $n$ and $\sum_{n=1}^{\infty}d_n$ diverges, then $\sum_{n=1}^{\infty}a_n$ diverges.
    \end{enumerate}
\end{theorem}

\begin{proof}
    Coming soon... % TODO
\end{proof}

We also have two more closely-related tests that once again concern the behavior of the individual terms in their limit.

\begin{theorem}[Root test]
    Given $\sum_{n=1}^{\infty} a_n$, let $\alpha = \limsup_{n \to \infty} \sqrt[n]{|a_n|}$.
    Then:
    \begin{itemize}
        \item If $\alpha < 1$, the series converges.
        \item If $\alpha > 1$, the series diverges.
        \item If $\alpha = 1$, the test is inconclusive.
    \end{itemize}
\end{theorem}

\begin{proof}
    We have three things to prove.
    \begin{itemize}
        \item ($\alpha = 1$)
        Consider the divergent $\sum_{n=1}^{\infty} \frac{1}{n}$.
        By Rudin 3.20 $\sqrt[n]{n} \to 1$, so $\alpha = \limsup_{n \to \infty} \sqrt[n]{1 / n} = 1$.

        Now consider the convergent (by Rudin 3.28) $\sum_{n=1}^{\infty} \frac{1}{n^2}$.
        We have $\alpha = \limsup_{n \to \infty} \sqrt[n]{1 / n^2} = 1$.
        So $\alpha = 1$ cannot give us any information about the convergence of a series!

        \item ($\alpha < 1$)
        There exists a $\beta$ such that $\alpha < \beta < 1$ and, by the definition of limsup, an $N$ such that $n > N$ implies $\sqrt[n]{|a_n|} < \beta$ for all $n$.
        So $|a_n| < \beta^{n}$.
        By the comparison test, $\sum_{n=1}^{\infty} a_n$ converges.
        
        \item ($\alpha > 1$)
        We have $\limsup_{n \to \infty} \sqrt[n]{|a_n|} > 1$, so $\limsup_{n \to \infty} a_n > 1$.
        Now consider the subsequence $\sqrt[n]{|a_{n_k}|} \to \alpha$; there exists an $N$ such that $k > N$ implies $|a_{n_k}| > 1$.
        Thus $\sum_{n=1}^{\infty} a_n$ diverges by the term test.
    \end{itemize}
    This completes the proof.
\end{proof}

\begin{theorem}[Ratio test]
    The sum $\sum_{n=1}^{\infty} a_n$ converges if $\limsup_{n \to \infty} |a_{n+1} / a_n| < 1$ and diverges if there is an $N$ such that $n \geq N$ implies $|a_{n+1} / a_n| \geq 1$.
\end{theorem}

The proof of the ratio test is very similar to that of the root test, so we omit it.

\subsection*{Algebraic Manipulations of Series}
Sums of convergent series are relatively simple.
If we have $\sum_{n=0}^{\infty} a_n \to A$ and $\sum_{n=0}^{\infty} b_n \to B$, then
\[ \sum_{n=0}^{\infty} a_n + \sum_{n=0}^{\infty} b_n = \sum_{n=0}^{\infty} (a_n + b_n) = A + B. \]
Products are a little more interesting!
The product of the above series is
\[ \sum_{n=0}^{\infty} c_n = (a_0b_0) + (a_1b_0 + b_0a_1) + (a_2b_0 + a_1b_1 + a_0b_2) + \cdots, \]
with terms grouped based on the sums of their indices.
So the $n$th term in the series is the sum of all the products whose terms add to $n$:
\[ c_n = \sum_{k=0}^{n} a_k b_{n-k}. \]
This is called the Cauchy product (and is the discrete analog of convolution).
We'll come back to this---first, let's see what we can say about products of terms.

\begin{theorem}[Summation by parts]
    Let $A_n = \sum_{k=1}^{n} a_k$ and define $A_0 = 0$.
    Then
    \[ \sum_{n=p}^{q} a_n b_n = A_qb_q - A_{p-1}b_p + \sum_{n=p}^{q-1} A_n (b_n - b_{n+1}). \]
\end{theorem}

There's a straightforward algebraic proof for this, but a geometric picture is much more enlightening.
Construct a rectangle with length $a_1 + \cdots + a_q$ and height $b_q$, and inside it draw smaller $a_n \times b_n$ rectangles.
The area covered by the rectangles from $p$ onward is equal to the total area $A_q b_q$ of the big rectangle minus the area not included in the $p$-$q$ rectangles!
This comes with a nice couple of corollaries.

\begin{theorem}[]
    If $A_n = \sum_{k=1}^{n} a_k$ is bounded for all $n$, $b_n > 0$ is decreasing, and $b_n \to 0$, then $\sum_{n=1}^{\infty} a_n b_n$ converges.
\end{theorem}

\begin{proof}
    Choose $M$ such that $|A_n| \leq M$ for all $n$.
    Given $\epsilon > 0$, there is an integer $N$ such that $b_N \leq (\epsilon / 2M)$.
    So for $N \leq p \leq q$ we have
    \begin{align*}
        \left| \sum_{n=p}^{q}a_nb_n \right| &= \left| \sum_{n=p}^{q-1} A_n (b_n - b_{n+1}) + A_q b_q - A_{p-1} b_p \right| \\
        &\leq M \left| \sum_{n=p}^{q-1} (b_n - b_{n+1}) + b_q - b_p \right| \\
        &= 2Mb_p \leq 2Mb_N \leq \epsilon.
    \end{align*}
    Convergence now follows from the Cauchy criterion.
\end{proof}

\begin{corollary}[Alternating term test]
    $\sum_{n=1}^{\infty} c_n$ converges if the $|c_i|$ are decreasing, the $c_i$ are alternating in sign, and $c_i \to 0$.
\end{corollary}

\begin{proof}
    In the previous theorem let $a_n = (-1)^{n+1}$ and $b_n = |c_n|$.
\end{proof}

Sometimes we'll have a diverging series that becomes convergent if we allow its terms to alternate signs.
To formalize this behavior, we'll define a slightly more restrictive type of convergence.

\begin{definition}[Absolute convergence]
    A series $\sum_{n=0}^{\infty} a_n$ converges absolutely if $\sum_{n=0}^{\infty} |a_n|$ converges.
\end{definition}

\begin{theorem}[]
    If $\sum_{n=0}^{\infty}$ converges absolutely, then $\sum_{n=0}^{\infty} a_n$ converges.
\end{theorem}

\begin{proof}
    We know that $|a_n| \leq |a_n|$ and $\sum_{n=1}^{\infty} |a_n|$ converges, so by comparison $\sum_{n=1}^{\infty} a_n$ converges.
\end{proof}

As it turns out, this stronger notion of absolute convergence is exactly what we need to guarantee that the product of two series converges!
We'll omit the proof here.

\begin{theorem}[]
    If $\sum_{n=0}^{\infty} a_n \to A$ and $\sum_{n=0}^{\infty} b_n \to B$, both converging absolutely, then
    \[ AB = \sum_{n=0}^{\infty} \left( \sum_{k=0}^{n} a_k b_{n-k} \right). \]
\end{theorem}

We'll finish off with a shocking result about rearrangements of series.
We'll omit this proof too, but it's in Rudin.

\begin{theorem}[Riemann rearrangement theorem]
    Consider $a_n, \alpha, \beta \in \R$ with $\alpha \leq \beta$.
    If $\sum_{n=0}^{\infty} a_n$ converges, but not absolutely, then there exists a rearrangement of the series such that $\alpha$ and $\beta$ are the lower and upper limits of the series, meaning the series can converge to any desired real number.

    However, if the series converges absolutely then all rearrangements converge to the same limit.
\end{theorem}

\end{document}
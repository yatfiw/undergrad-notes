\documentclass[../m131main.tex]{subfiles}
\graphicspath{{\subfix{../figures/}}}
%<3
\begin{document}

\chapter{Basic Topology} \label{ch:topology}
\section{Cardinality}
In this first part of the course we'll talk about sets, structure for sets, and the consequences of this structure.
Let's start with the natural numbers $\N = \left\{ 1, 2, 3, \ldots \right\}$; the rules that precisely define the natural numbers are specified by the Peano axioms.

\begin{definition}[Peano axioms]
    Let $\N$ be a set containing the element 1.
    Define a successor function $S : \N \to \N$, with the following:
    \begin{itemize}
        \item $S(x) \neq 1$ for all $x \in \N$
        \item If $S(x) = S(y)$, then $x = y$ for all $x,y \in \N$.
        \item Let $A$ be any subset of $\N$ which contains 1 and is closed under $S$, so $S(x) \in A$ for all $x \in A$.
    \end{itemize}
    Then $A = \N$.
\end{definition} %<3

These axioms together ensure that there is a first element in $\N$, and that for each element in $\N$ there is a ``next'' element.
This implies all of the basic properties of the natural numbers!
We give one without proof.

\begin{theorem}[Well-ordering principle]
    Every nonempty subset of $\N$ has a smallest element.
\end{theorem}

The Peano axioms also form the basis of mathematical induction.
Specifically, let $P(n)$ be a set of statements indexed by $n \in \N$, and define $S = \left\{ n : P(n) \text{ is true} \right\}$.
To show that $P(n)$ is true for all $n$, it suffices to show that $S$ satisfies the Peano axioms!
Specifically, we must show that
\begin{itemize}
    \item $P(1)$ is true, meaning the first element of $S$ is 1 (the base case); and that
    \item $P(k) \implies P(k+1)$, so each element in $S$ has a successor (the inductive step).
\end{itemize}
If we can do both of these things, then $S = \N$ by the Peano axioms, and $P(n)$ is true for all $n$.
One common variant on this principle is called strong induction, in which the inductive step assumes the truth of all $P(1), P(2), \ldots, P(k)$ rather than just $P(k)$.

We can leverage our definition of the natural numbers to rigorously define the size of a set.

\begin{definition}[Cardinality]
    Suppose $J_n = \left\{ 1, 2, \ldots, n \right\} \subset \N$.
    If $A$ is in one-to-one correspondence with $J_n$, then $A$ has cardinality (or size) $n$ and we write $|A| = n$.
    If $A = \emptyset$, we define $|A| = 0$.
\end{definition}

Most interesting sets, however, have far too many elements to have such a correspondence.
This is what we usually mean by ``infinity''.

\begin{definition}[Finite and infinite]
    We say $A$ is finite if there is a bijection between $A$ and $J_n$.
    Otherwise, $A$ is infinite.
\end{definition}

To begin our characterization of infinity, we stick with what we know: sets that are structurally similar to the natural numbers.

\begin{definition}[Countable]
    A set $A$ is countable if $A$ is in one-to-one correspondence with $\N$.
\end{definition}

We have two powerful theorems that we can use to determine the cardinalities of sets built using countable sets.
They're both pretty unintuitive at first glance.

\begin{theorem}[]
    Every infinite subset of a countable set is countable.
\end{theorem}

\begin{proof}
    Suppose $A$ is a countable set, so $A = \left\{ x_1, x_2, \ldots \right\}$.

    Consider a subset $E \subseteq A$.
    By the well-ordering principle, $E$ has a smallest element; index this element as $n_1$.
    The next element receives index $n_2$, then $n_3$, and so on, so $E = \left\{ x_{n_1}, x_{n_2}, \ldots \right\}$.
    This process creates the bijection $f(k) = x_{n_k}$ between $E$ and $\N$, so $E$ is countable.
\end{proof}

\begin{theorem}[Hilbert's hotel]
    The union of a countable set of countable sets is countable.
\end{theorem}

\begin{proof}
    Suppose $E_n = \left\{ x_{n_1}, x_{n_2}, \ldots \right\}$ is an uncountable set, so
    \[ S = \bigcup_{i=1}^\infty E_i \]
    is a union of a countable set of countable sets.
    We can list out the elements in $S$ as an infinite grid:
    \[ \begin{matrix}  
        x_{11} & x_{12} & x_{13} & \cdots \\
        x_{21} & x_{22} & x_{23} & \cdots \\
        \vdots & \vdots & \vdots & \ddots
    \end{matrix} \]
    We can create a bijection with $\N$ just by slithering across the diagonals of this grid!
    For example, the first six instances of this bijection are
    \begin{alignat*}{2}
        x_{11} &\mapsto 1 &\qquad x_{31} &\mapsto 4 \\
        x_{12} &\mapsto 2 &\qquad x_{22} &\mapsto 5 \\
        x_{21} &\mapsto 3 &\qquad x_{13} &\mapsto 6
    \end{alignat*}
    Therefore, there is a bijection between $S$ and $\N$, and $S$ is countable.
\end{proof}

We can use these to show, for example, that the rational numbers are countable, even if it seems like there should be many more of them.
This might suggest that we can't get any bigger than countable, but we can!

\begin{theorem}[Cantor's diagonalization argument]
    The set $(0,1) \subset \R$ is uncountable.
\end{theorem}

\begin{proof}
    Construct an arbitrary mapping $f : \N \to (0,1)$:
    \begin{align*}
        1 &\mapsto x_1, \quad x_1 = 0.x_{11}x_{12}x_{13} \cdots \\
        2 &\mapsto x_2, \quad x_2 = 0.x_{21}x_{22}x_{23} \cdots \\
        &\;\;\vdots
    \end{align*}
    Now construct the real number $r = 0.r_1r_2r_3 \cdots$ by assigning each decimal place as follows:
    \[ r_i = \begin{cases} 1 & \text{if } x_{ii} \neq 1 \\ 2 & \text{if } x_{ii} = 1 \end{cases} \]
    This construction guarantees that $r$ differs from $x_1$ in its first decimal place, from $x_2$ in its second decimal place, and so on.
    So $r$ is not in the range of $f$ for any $f : \N \to (0,1)$, meaning there is no bijection between $(0,1)$ and $\N$.
    Therefore, $(0,1)$ is uncountable.
\end{proof}

This shows that the set of real numbers between 0 and 1 is somehow larger than the set of all natural numbers.
We can go even larger than this!

\begin{definition}[Power set]
    The power set $P(A)$ of a set $A$ is the set of all subsets of $A$.
\end{definition}

\begin{theorem}[Cantor's theorem]
    There is a function $f : A \to P(A)$ that is injective, but there is no such surjective function.
\end{theorem}

\begin{proof}
    It is easy to show that there is an injection $f: A \to P(A)$---one example is $f(x) = \left\{ x \right\}$.

    Now, for contradiction, suppose that there is a surjection $g : A \to P(A)$.
    So by the definition of $P(A)$, for any $B \subseteq A$, there exists a $y \in A$ such that $g(y) = B$.
    In this case, $y \in B$ if and only if $y \in g(y)$.

    But we can construct a $B$ that is not in the range of $g$!
    Specifically, consider the subset of all elements in $A$ that, under $g$, do not map to themselves; that is,
    \[ B = \left\{ x \in A : x \not\in g(x) \right\}. \]
    By definition, $y \in B$ if and only if $y \not\in g(y)$, but this contradicts our previous conclusion about $B$.
    Therefore, there is no such surjection $g$.
\end{proof}

This proof is really just a slicker version of Cantor's diagonal argument.
Imagine we're constructing $B$ by going element by element in $A$---we include $x \in A$ if it maps to a subset that does not contain itself, and we otherwise exclude it.
This new subset $B$ differs from all others in $\pfn{range}(g)$.
Any $x \in A$ satisfying $x \not\in g(x)$ mustn't map to $B$ because $x \in B$.
We can make the reverse argument for $y \in A$ satisfying $y \in g(y)$.

Cantor's theorem provides a way to construct arbitrarily large infinities by constructing arbitrarily large power sets, so there is no ``biggest'' infinity.
Another natural question is whether or not there's is a set whose cardinality falls between these infinities.
This is called the continuum hypothesis, and it has been shown to be undecidable under today's standard axioms for set theory!

\section{Order, Fields, and Bounds}
Right now our sets are nothing more than collections of things.
A reasonable first step in giving them some structure is to assign an order to the elements.

\begin{definition}[Order]
    An order on a set $S$ is a relation $<$ satisfying the following.
    \begin{itemize}
        \item (Trichotomy) If $x,y \in S$, then exactly one is true: $x < y$, $x = y$, or $y < x$.
        \item (Transitivity) If $x,y,z \in S$ and if $x < y$ and $y < z$, then $x < z$.
    \end{itemize}

    \medskip
    \textit{Note: Writing $x > y$ is equivalent to writing $y < x$.
    Also, $x \leq y$ is equivalent to ``$x < y$ or $x = y$'', and $x \geq y$ is defined similarly.}
\end{definition}

We can also slap on a couple of reasonable operations so we can start doing some algebra.

\begin{definition}[Field]
    A field $F$ is a set with two operations $+,\times$ that each satisfies the following axioms.
    \begin{itemize}
        \item (Closure) $x + y \in F$ and $xy \in F$.
        \item (Commutativity) $x + y = y + x$ and $xy = yx$.
        \item (Associativity) $(x + y) + z = x + (y + z)$ and $(xy)z = x(yz)$
        \item (Identity) $F$ contains 0 such that $0 + x = x$ and contains 1 such that $1x = x$.
        \item (Inverse) For every $x \in F$ there is a $-x \in F$ such that $x + (-x) = 0$, and if $x \neq 0$ then there is a $x^{-1} \in F$ such that $x x^{-1} = 1$.
    \end{itemize}
    The operations also satisfy the distributive law, $x(y + z) = xy + xz$.
\end{definition}

Both of these ideas fit together in such a way that they complement each other very nicely.

\begin{definition}[Ordered field]
    Suppose $F$ is a field and has an order.
    Let $x,y,z \in F$;
    we say $F$ is an ordered field if the following hold.
    \begin{itemize}
        \item If $y < z$, then $y + x < z + x$.
        \item If $y < z$ and $x > 0$, then $xy < xz$.   
    \end{itemize}
\end{definition}

With this, we can talk about what it means for a set to be restricted in the context of some larger set.

\begin{definition}[Upper and lower bound]
    Let $E \subseteq S$ be an ordered set.
    If there is an element $\beta \in S$ such that for all $x \in E$ we have $x \leq \beta$, then $\beta$ is an upper bound for $E$.
    In this case we say $E$ is bounded above.
    We can similarly define lower bound, in which case $E$ is bounded below.
\end{definition}

\begin{definition}[Supremum and infimum]
    Again consider $E \subseteq S$.
    If there exists an $\alpha \in S$ that satisfies
    \begin{itemize}
        \item $\alpha$ is an upper bound of $E$, and
        \item if $\gamma < \alpha$ then $\gamma$ is not an upper bound of $E$,
    \end{itemize}
    then $\alpha$ is the least upper bound or supremum for $E$.
    We write $\alpha = \sup E$.
    We can similarly define the greatest lower bound or infimum of $E$.
\end{definition}

Note that the supremum and the maximum are not the same thing---maxima have the additional requirement that they must actually be in the set they're describing.
The same goes for infima and minima.
We'll focus on suprema from here on, but all of the statements we make for suprema have analogs for infima.
The supremum has some intuitive properties, many of which we can state without proof.

\begin{theorem}[Properties of suprema] \label{thm:sup_props}
    \begin{enumerate}[label=(\alph*)]
        \item $\gamma$ is an upper bound for $A$ if and only if $\sup A \leq \gamma$.
        \item $a \leq \gamma$ for all $a \in A$ if and only if $\sup A \leq \gamma$.
        \item $a < \gamma$ for all $a \in A$ if and only if $\sup A \leq \gamma$.
        \item If $\gamma < \sup A$, then there exists $a \in A$ such that $\gamma < a \leq \sup A$.
        \item If $\sup A \leq \sup B$ and $\sup B \leq \sup A$, then $\sup A = \sup B$.
    \end{enumerate}
\end{theorem}

\begin{theorem}[]
    If $A \subseteq B$ and both suprema exist, then $\sup A \leq \sup B$.
\end{theorem}

\begin{proof}
    Because $A \subseteq B$, all $a \in A$ satisfy $a \in B$.
    So by definition, $a \leq \sup B$.
    From Theorem \ref{ch:topology}.\ref{thm:sup_props}(b), we get $\sup A \leq \sup B$.
\end{proof}

Not every set that has an upper bound has a least upper bound.
It is, however, very nice when they do!

\begin{definition}[Axiom of completeness]
    A set $S$ satisfies the axiom of completeness (or the least upper bound property) if every nonempty subset of $S$ that has an upper bound has a supremum in $S$.
\end{definition}

As it turns out, this property is the defining feature of $\R$.
We state this without proof for now, but we'll come back to it later.

\begin{theorem}[Completeness of $\R$]
    There exists an ordered field $\R$ which both satisfies the completeness axiom and contains $\Q$ as a subfield.
\end{theorem}

Quite a few interesting facts follow from this!
For example, it is the reason why every real number has a decimal expansion, and why every real number has an $n$th root.
There are a couple of other consequences which we'll find useful.

\begin{theorem}[Archimedean property]
    If $x,y \in \R$ and $x > 0$, then there exists a positive integer $n$ such that $nx > y$.
    (Equivalently, if $x > 0$, then there exists $n \in \N$ such that $1 / n < x$.)
\end{theorem}

\begin{proof}
    Consider a set $A = \left\{ nx : n \in N \right\}$.
    For contradiction, suppose $A$ were bounded above by $y$, that is, $nx < y$ for all $n$.
    But we know $A \subset \R$ has a supremum by the axiom of completeness.

    Call this least upper bound $\sup A = \alpha$.
    Then $\alpha - x$ is not an upper bound for $A$.
    But then there exists an $m \in \N$ such that $\alpha - x < mx$, which implies that $\alpha < mx + x = (m+1)x$.
    Thus, $\alpha$ is not an upper bound for $A$, a contradiction.
    Therefore, $nx > y$, as desired.
\end{proof}

\begin{theorem}[$\Q$ is dense in $\R$]
    Suppose we have $x,y \in \R$ with $x < y$. Then there exists at $q \in \Q$ such that $x < q < y$.
\end{theorem}

\begin{proof}
    Choose $n \in \N$ satisfying $1 / n < y - x$; such an $n$ exists by the Archimedean property.
    Consider multiples of $1 / n$; again, by the Archimedean property, these multiples are unbounded so we can choose the first multiple such that $m / n > x$.

    Now we show that $m / n < y$.
    Suppose not: then $(m-1) / n < x$ and $m / n > y$.
    But combining these implies that $1 / n > y - x$, a contradiction.
\end{proof}

The first theorem here roughly states that there are no ``infinitely large'' or ``infinitely small'' real numbers to the point of being inaccessible.
This second theorem hints at one of three equivalent characteristics of what we call dense sets, which we'll look more into later.

\section{Metric Spaces, Open Sets, and Closed Sets}
Let's keep adding structure to our sets by introducing distance.

\begin{definition}[Metric space]
    A set $X$ is a metric space if there exists a metric $d : X \times X \to \R$ such that the following are satisfied for all $p,q,r \in X$.
    \begin{itemize}
        \item (Nonnegativity) $d(p,q) \geq 0$, with $d = 0 \iff p = q$.
        \item (Symmetry) $d(p,q) = d(q,p)$.
        \item (Triangle inequality) $d(p,q) \leq d(p,r) + d(r,q)$.
    \end{itemize}
\end{definition}

We can use this to define the notion of a point's ``closeness'' to another point!

\begin{definition}[Neighborhood]
    A neighborhood of a point $x$ with radius $r$ is the set $N_r(x) = \left\{ y : d(x,y) < r \right\}$.
    This is also known as an open ball of radius $r$; if we allow $d = r$ then we have a closed ball, denoted by $\overline{N_r(x)}$.
\end{definition}

Neighborhoods are useful in classifying the different types of points that arise when we construct subsets.

\begin{definition}[Limit, interior, and isolated point]
    Let $X$ be a metric space and $E \subseteq X$.
    Consider a point $p \in X$.
    \begin{itemize}
        \item $p$ is a limit point of $E$ if every neighborhood of $p$ contains a point $p \in E$ where $p \neq q$.
        \item $p$ is an interior point of $E$ if there exists a neighborhood $N(p)$ such that $N \subseteq E$.
        \item $p$ is an isolated point of $E$ if $p \in E$ and $p$ is not a limit point of $E$.
    \end{itemize}
\end{definition}

Note that a limit point need not be in the subset it is characterized by!
It just provides a definite ``edge'' to the subset.
Interestingly, the neighborhoods around these limit points are infinite.

\begin{theorem}[Neighborhoods of limit points]
    If $p$ is a limit point of $E$, then every neighborhood of $p$ contains infinitely many points of $E$.
\end{theorem}

\begin{proof}
    Consider a limit point $p$ of a set $E$.
    For contradiction, suppose there exists a neighborhood $N$ of $p$ that contains only finitely many points $e_1, e_2, \ldots, e_N$.

    Let $r$ be the minimum nonzero distance between $p$ and one of the points in the neighborhood.
    We know this minimum exists because the set of distances is finite.
    We can then construct a new neighborhood of size $r$, which contains no points except $p$, which is a contradiction since $p$ is a limit point.
\end{proof}

A key motivator behind classifying points like we have is the definition of two very important types of sets.

\begin{definition}[Open and closed sets ]
    Suppose $X$ is a metric space and $E \subseteq X$.
    \begin{itemize}
        \item $E$ is open if every point in $E$ is an interior point of $E$.
        \item $E$ is closed if $E$ contains all of its limit points.
    \end{itemize}
\end{definition}

Importantly, open and closed sets are not opposites!
It is possible to have a set that falls into neither category, or even both of them.
Not all is lost though---there is still a nice relationship between open and closed sets.

\begin{lemma}[Neighborhoods are open sets]
    A neighborhood $N(p)$ of a point $p$ is an open set.
\end{lemma}

\begin{proof}
    Recall that $N(p)$ is defined by some radius $r$.
    Therefore, if a point $q \in N(p)$, then $d(p,q) < r$.

    Let $r' - d(p,q)$ be the distance from $q$ to the edge of the neighborhood, and construct a smaller neighborhood $N(q)$ with this radius.
    By the triangle inequality, any point $x \in N(q)$ must satisfy
    \[ d(x,y) \leq d(x,q) + d(q,p) < r' + d(q,p) = r. \]
    So $d(x,p) < r$ and $q$ is an interior point of $N(p)$.
    Therefore, $N(p)$ is open.
\end{proof}

\begin{theorem}[Complement of an open set is closed]
    $E$ is open if and only if its complement $E^c$ is closed.
\end{theorem}

\begin{proof}
    We provide a chain of equivalent statements.
    \begin{align*}
        \text{$E$ is open} &\iff \text{any $x \in E$ is an interior point} \\
        &\iff \text{for all $x \in E$ there exists a neighborhood of $x$ that is disjoint from $E^c$} \\
        &\iff \text{$x \in E$ is not a limit point of $E^c$} \\
        &\iff \text{$E^c$ contains all of its limit points} \\
        &\iff \text{$E^c$ is closed},
    \end{align*}
    as desired.
\end{proof}

We'll consider what happens when we take unions and intersections of these sets.
Finite unions and intersections lead to fairly intuitive results; infinite sets, as usual, are a little strange!

\begin{lemma}[De Morgan's Law]
    Suppose we have a (potentially uncountable) collection of sets $\left\{ E_\alpha \right\}$ (where $\alpha$ labels the sets).
    Then
    \vspace{-2pt}
    \[ \left( \bigcup_\alpha E_\alpha \right)^c = \bigcap_\alpha E_\alpha^c. \]
\end{lemma}

\begin{proof}
    Consider an element $y \in \left( \bigcup_\alpha E_\alpha \right)^{c}$, so $y \not\in \bigcup_\alpha E_\alpha$.
    Then $y$ is not in $E_\alpha$ for any $a$, meaning $y \in E_\alpha$ for all $a$ and $y \in \bigcap_\alpha E_\alpha^c$.
    This argument holds in reverse.
\end{proof}

\begin{theorem}[Unions and intersections of open and closed sets]
    \begin{enumerate}[label=(\alph*)]
        \item Arbitrary unions of open sets are open.
        \item Finite intersections of open sets are open.
        \item Finite unions of closed sets are closed.
        \item Arbitrary intersections of closed sets are closed.
    \end{enumerate}
\end{theorem}

\begin{proof}
    We'll prove each part in turn.
    \begin{enumerate}[label=(\alph*)]
        \item Suppose $x \in \bigcup_{i=1}^\infty A_i$, where all $A_i$ are open.
        Then $x \in A_i$ for some $i$ and $x$ has a neighborhood $N(x) \subset A_i$.
        But $A_i \subseteq \bigcup_{i=1}^\infty A_i$, meaning $x$ is an interior point of the union and the union is open.

        \item Suppose $x \in \bigcup_{i=1}^n A_i$, where all $A_i$ are open.
        For each $A_i$, there exists a neighborhood $N_{r_i}(x) \subseteq A_i$.
        Let $r$ be the minimum radius of these neighborhoods; then $N_r(x)$ is contained in the intersection and the intersection is open.

        \item Consider a collection of closed sets $B_1, \ldots, B_n$; by (b), the intersection of their complements is open.
        The complement of this intersection is closed, so by De Morgan's Law, $\bigcup_{i_=1}^n B_i$ is closed.

        \item Consider a collection of closed sets $B_1, B_2 \ldots$; by (a), the union of their complements is open.
        The complement of this union is closed, so by De Morgan's law, $\bigcap_{i=1}^\infty$ is closed.
    \end{enumerate}
    This completes the proof.
\end{proof}

Let's look a little more closely at closed sets by defining the closure of a set.

\begin{definition}[Closure]
    The closure of a set $A$ is $\overline{A} \equiv A \cup A'$, where $A'$ is the set of limit points of $A$.
\end{definition}

\begin{theorem}[]
    A set is closed if and only if it is equal to its closure.
\end{theorem}

\begin{proof}
    Consider a set $E$ and the set $E'$ of its limit points.
    $\overline{E} = E \cup E'$ is the closure of $E$.

    ($\Rightarrow$)
    Assume $E$ is closed.
    By definition, $E' \subset E$, so $\overline{E} \subset E$.
    Also, since $E \subset \overline{E}$, we conclude $\overline{E} = E$.
    \smallskip

    ($\Leftarrow$)
    Assume $E = \overline{E}$.
    Then $E = E \cup E'$, and so it contains all of its limit points and is closed.
\end{proof}

Another natural property of the closure is that, when a ``parent set'' is known to be closed, then any of its subsets remains in the parent set after being closed.

\begin{theorem}[]
    If $E \subseteq F$, where $F$ is closed, then $\overline{E} \subseteq F$.
\end{theorem}

\begin{proof}
    If $F$ is closed then it contains all of its limit points.
    In particular, it must contain all limit points of a subset of itself.
    So $F \subseteq E'$, meaning $\overline{E} \subseteq F$.
\end{proof}

A nice corollary of this theorem is that the smallest closed superset of $E$ is its closure $\overline{E}$.
Finally, we can put all this together to get a pretty unintuitive theorem---our first result related to sequences.

\begin{theorem}[Nested interval property of $\R$]
    For each $n \in \N$, consider $I_n = \left\{ x \in \R : a_n \leq x \leq b_n \right\}$.
    If each $I_{n+1} \subseteq I_n$, then $\bigcap_{i=1}^n I_i$ is nonempty.
\end{theorem}

\begin{proof}
    Let $A = \left\{ a_1, a_2, \ldots \right\}$ be the left-endpoints of $\left\{ I_n \right\}$.
    $A \in \R$ is bounded above by $b_1$, so $x = \sup A$ exists.
    So $a_n \leq x \leq b_n$ for all $n$.
    Therefore, $x \in \bigcup_{n=1}^\infty$, and the intersection is nonempty.
\end{proof}

This can, in fact, be generalized to closed sets in $\R^n$.

\section{Dense Sets}
We are now ready to formally define density, as we alluded to before, and to explore its properties.

\begin{definition}[Dense set]
    Consider a metric space $X$.
    A set $E \subseteq X$ is dense in $X$ if
    \begin{itemize}
        \item every point of $X$ is in $E$ or is a limit point of $E$,
        \item $\overline{E} = X$, or
        \item every open set of $X$ contains a point of $E$.
    \end{itemize}
\end{definition}

We'll kick things off with a direct application of the nested interval theorem.

\begin{lemma}[]
    A countable intersection of dense, open sets in $\R$ is not empty.
\end{lemma}

\begin{proof}
    Let $\left\{ G_1, G_2, \ldots \right\}$ be a collection of dense, open sets in $\R$.
    First, by definition, every open set in $\R$ contains a point in $G_1$.
    Consider an arbitrary nonempty closed interval $I_1= [a_1, b_1]$; this is the closure of the open interval $A_1 = (a_1, b_1)$, which must contain a point in $G_1$.

    We continue on to construct a closed interval $I_2 \subseteq I_1$, which is the closure of the open interval $A_2$.
    $A_2$ must contain a point in $G_2$, and also one in $G_1$.
    Continuing on, we get a countable sequence of nested, closed intervals, all of which we know to be nonempty by the nested interval property of $\R$.

    Each $I_n$ is the closure of the open interval $A_n$, so we have also constructed a countable nested sequence of nonempty open intervals.
    Thus the intersection of these open sets is nonempty and contains a point in every $G_n$, so the intersection of all $G_n$ is nonempty.
\end{proof}

Now consider, for a moment, the set of positive rationals.
This set is obviously not dense in $\R$, but we might say that it's at least dense \textit{somewhere} in $\R$, particularly in the positive reals.
The set of integers, on the other hand, isn't dense anywhere in $\R$.
We'll find this characterization useful in proving our next theorem.

\begin{definition}[Nowhere-dense set]
    A set $E$ is nowhere-dense in a metric space $X$ if $\overline{E}$ contains no nonempty open intervals.
\end{definition}

We might re-interpret this definition to mean that there are no open sets of $X$ contained in $\overline{E}$, or that $E$ is not dense in any nonempty open subset of $X$.
Intuitively, nowhere-dense sets are very ``thin'', so countable unions of nowhere-dense subsets represent a very small type of infinite set (called meager sets).
This allows for another perspective on the cardinality of $\R$---given what we know already, we expect that $\R$ is not a meager set, and does turn out to be the case!

\begin{theorem}[Baire's theorem]
    $\R$ cannot be written as the countable union of nowhere-dense sets.
\end{theorem}

\begin{proof}
    Let $S = \left\{ S_1, S_2, \ldots \right\}$ be a countable set of nowhere-dense (in $\R$) sets, so each $\overline{S_n}$ contains no nonempty open intervals.
    For contradiction, suppose $\bigcup_{n=1}^\infty \overline{S_n} = \R$.
    By de Morgan's law, $\bigcap_{n=1}^\infty \overline{S_n}^c = \emptyset$.
    Now, each $\overline{S_n}^c$ is dense and open, and their countable intersection is empty---which contradicts the previous lemma.
    So $\bigcup_{n=1}^\infty \overline{S_n} \neq \R$, and since $S_n \subset \overline{S_n}$, we also have $\bigcup_{n=1}^\infty S_n \neq \R$, as desired.
\end{proof}

\section{Compact Sets}
Now we'll turn our attention to a different kind of set, one that has much in common with finite sets.
We need to make a couple of other definitions first.

\begin{definition}[Cover]
    Let $X$ be a metric space.
    An open cover (or, simply, cover) of $E$ in $X$ is a collection of open sets $\left\{ G_\alpha \right\}$ whose union contains (``covers'') $E$.
    If there is a subcollection of $\left\{ G_\alpha \right\}$ that still covers $E$, we call this subcollection a subcover.
\end{definition}

\begin{definition}[Compact set]
    We say a set $K$ is compact in $X$ if every open cover of $K$ contains a finite subcover.
\end{definition}

\begin{theorem}[]
    Finite sets are compact.
\end{theorem}

\begin{proof}
    Consider a finite set $\left\{ x_1, \ldots, x_n \right\}$ and some arbitrary open cover $\left\{ G_\alpha \right\}$.
    For each $x_i$, choose a $G_{\alpha_i}$ that contains $x_i$.
    Then these $G_{\alpha_i}$ cover the set and comprise a finite subcover of $\left\{ G_\alpha \right\}$.
\end{proof}

The definition we've provided for compact sets isn't exactly abstract, but it isn't super well-motivated either.
We'll dedicate the remainder of this section to characterizing these sets, starting with a couple of theorems.

\begin{definition}[Bounded set]
    Let $K$ be a set in a metric space $X$.
    $K$ is bounded if $K \subset N_r(x)$ for some $x \in X$.
\end{definition}

\begin{theorem}[Compact sets are bounded]
    If $K$ is compact then $K$ is bounded.
\end{theorem}

\begin{proof}
    Let $K$ be a compact set and $B(x)$ a unit ball about a point $x$.
    Then $\left\{ B(x) \right\}_{x \in K}$ is a cover of $K$, and there exists a finite subcover $\left\{ B(x_i) \right\}_{i=1}^N$.
    Take one of these unit balls, say $B(x_1)$, and define $M = \max \left\{ d(x_1, x_i) \right\}_{i=1}^N$ to be the maximum distance to the center of another unit ball.
    Then the neighborhood $N_{M+2}(x_1)$ contains the subcover $\left\{ B(x_i) \right\}_{i=1}^N$ and thus contains $K$.
    So $K$ is bounded.
\end{proof}

\begin{theorem}[Compact sets are closed]
    If $K$ is compact then $K$ is closed.
\end{theorem}

\begin{proof}
    Let $K$ be compact and consider $p \not\in K$; we will show that there exists a $N(p) \subseteq K^c$.

    For all $q \in K$ let $V_q = N_r(q)$ and $U_q = N_r(p)$, where $r = \frac{1}{2} d(p,q)$ so that $V_q \cap U_q = \emptyset$.
    Notice that $\left\{ V_q \right\}$ is a cover of $K$, so there exists a finite subcover $\left\{ V_{q_1} , \ldots, V_{q_n}\right\}$.
    Also, the intersection of all $U_{q_i}$ is open because it's the intersection of finitely many open sets; in fact, $W$ is a ball of radius $\hat r = \min \left\{ r_1, \ldots, r_n \right\}$.

    Now, $W \cap V_{q_i} = \emptyset$ for all $i$ since $W \subset U_{q_i}$.
    Thus $K^c$ contains $W$, $K^c$ is open, and $K$ is closed.
\end{proof}

\subsection*{Sufficient Conditions for Compactness}
Now we'll move on from necessary conditions for compactness to find sufficient conditions.

\begin{theorem}[]
    A closed subset of a compact set is compact.
\end{theorem}

\begin{proof}
    Let $K$ be a compact set and $B \subseteq K$ be a closed subset.
    Consider an open cover $\left\{ U_\alpha \right\}$ of $B$.
    The set $B^c$ is open, so $\left\{ U_\alpha \right\}$ is also open.
    This union is a cover of $K$.

    By the compactness of $K$, there exists a finite subcover $\left\{ U_{\alpha_1}, \ldots, U_{\alpha_n}, B^c \right\}$ of $K$.
    But $B^c$ is not necessary to cover $B$, so $\left\{ U_{\alpha_1}, \ldots, U_{\alpha_n} \right\}$ is a finite subcover for any cover and $B$ is compact.
\end{proof}

\begin{corollary}[]
    Suppose $F$ is closed and $K$ is compact.
    Then $F \cap K$ is compact.
\end{corollary}

\begin{proof}
    $K$ is compact, so it is closed.
    Hence the intersection $F \cap K \subseteq K$ is closed and thus compact.
\end{proof}

Continuing in our quest for sufficient conditions, we prove compactness for a particular type of set: closed intervals in $\R$.
This will prove to be very useful in our final theorems of this section.

\begin{theorem}[]
    The closed interval $[a,b]$ is compact in $\R$.
\end{theorem}

\begin{proof}
    Suppose, for contradiction, that $[a,b]$ is not compact, so there exists an open cover $\left\{ G_\alpha \right\}$ that has no finite subcover.
    Divide this interval at the halfway point; then $\left\{ G_\alpha \right\}$ covers both subintervals and at least one of them, say $I_1$ contains no finite subcover.

    Then divide $I_1$ in half in a similar manner.
    One of the resulting halves $I_2$ has no finite subcover.
    Continue doing this to obtain a sequence of nested close intervals where each interval is smaller than the last and has no finite subcover.

    By the nested closed interval theorem, there is an $x \in I_n$ for all $n$.
    So $x$ is in a set $G_{\hat \alpha} \in \left\{ G_\alpha \right\}$ and there is a neighborhood $N(x) \subseteq G_{\hat \alpha}$.
    Since the intervals can get arbitrarily small, there is an $I_n \subseteq N(x)$.
    Hence $G_{\hat \alpha}$ is a finite subcover for $I_n$, a contradiction.
\end{proof}

Now for what is perhaps the best characterization of compact sets in $\R$!

\begin{theorem}[Heine-Borel theorem]
    A set $K \subset \R$ is compact if and only if $K$ is closed and bounded.
\end{theorem}

\begin{proof}
    ($\Rightarrow$)
    We already have two individual theorems showing that, if $K$ is compact, then it is closed and bounded.
    \smallskip

    ($\Leftarrow$)
    Let $K \subset \R$ be closed and bounded.
    So there exists an $r > 0$ such that $K \subseteq [-r, r]$.
    Hence $K$ is a closed subset of a compact set and $K$ is compact.
\end{proof}

In fact, this result applies to any Euclidean space $\R^n$.
The more general proof has a very similar structure to the one above.
We have another theorem that applies to all metric spaces, but it is weaker.

\begin{theorem}[]
    $K$ is compact if and only if every infinite subset $E$ of $K$ has a limit point in $K$.
\end{theorem}

\begin{proof}
    ($\Rightarrow$)
    Assume $K$ is compact.
    Suppose, for contradiction, that no point of $K$ is a limit point of some infinite $E \subseteq K$.
    Then each $q \in K$ has a neighborhood $N(q)$ containing at most one point of $E$ (namely, $q$, if $q \in E$).
    But then, $\left\{ N(q) \right\}$ is a cover of $E$ with no finite subcover, a contradiction.

    ($\Leftarrow$) Exercise.
\end{proof}

Finally, we can use this to prove something that we probably already knew intuitively, but it's nice to get some closure.

\begin{corollary}[Bolzano-Weierstrass theorem]
    Every bounded infinite subset of $\R$ has a limit point in $\R$.
\end{corollary}

\begin{proof}
    Consider an infinite, bounded set $E \subset \R$.
    $E$ is contained in some compact interval $[a,b]$, so it has a limit point in $[a,b]$ and thus in $\R$.
\end{proof}

This theorem also generalizes to $\R^k$.

    %<3
\end{document}
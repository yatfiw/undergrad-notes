\documentclass[../m55main.tex]{chapters}
\graphicspath{{\subfix{../figures/}}}

\begin{document}

\section{Introduction}
Just as the central objects of previous chapters were combinations and integers, the central object of this chapter is the graph.

\begin{definition}[Graph]
    A graph $G = (V,E)$ consists of a finite vertex set $V$ and an edge set $E$ where $E$ contains size-2 subsets of $V$.
\end{definition}

This definition is precise, but it isn't very nice to work with.
In practice, we usually think of a graph using its picture.
For example, we associate the graph
\begin{align*}
    V &= \{ 1, 2, 3, 4, 5, 6, 7, 8 \} \\
    E &= \Big\{ \{ 1,4 \}, \{ 1,5 \}, \{ 2,3 \}, \{ 3,5 \}, \{ 4,5 \}, \{ 6,7 \} \Big\}
\end{align*}
with the picture below.

\begin{center}
    \begin{tikzpicture}[spring layout, iterations=500, node distance=25pt]
            \foreach \i in {1,...,8}
                \node (node \i) [fill=black!70, text=white, circle, font=\scriptsize, inner sep=1pt, minimum size=8pt] {\i};
            \draw 
                (node 1) edge (node 4)
                edge (node 5)
                (node 4) edge (node 5)
                (node 5) edge (node 2)
                edge (node 3)
                (node 2) edge (node 3)
                (node 6) edge (node 7);
    \end{tikzpicture} % TODO: find a way to vertically align (tikz -> baseline?)
\end{center}

Note that, since graphs are defined in terms of sets, they cannot contain loops or ``multi-edges''.
(If we were to allow multi-edges, we would have a multi-graph.)

We say that two vertices are adjacent if there is an edge between them.
The degree $d(v)$ of a vertex $v$ is the number of vertices that are adjacent to $v$.

Now we can get into some basic properties of graphs.

\begin{lemma}[Handshake lemma]
    For any graph, the sum of the degrees of the vertices is twice the number of edges.
    That is, if $G = (V,E)$, then
    \[ \sum_{v \in V} d(v) = 2 |E|. \]
\end{lemma}

\begin{proof}
    If we count the edges leaving each vertex, then every edge is counted exactly twice. 
\end{proof}

This result's namesake is its application to gatherings of people.
If partygoers go around and shake hands with each other, each handshake increases the partygoers' ``degree sum'' by two.
Therefore, this degree sum must always be even.

\begin{corollary}[Oddballs]
    In any graph, the number of vertices with odd degree must be even.
\end{corollary}

\begin{proof}
    If the number of vertices with odd degree is odd, then the sum of the degrees is
    \[ \text{even} + \cdots + \text{even} + \text{odd} + \cdots + \text{odd} = \text{even} + \text{odd} = \text{odd}, \]
    which can't happen (by the handshake lemma).
\end{proof}

Graphs are considered to be equal if they have the same vertex and edge sets.
We can push the vertices around as much as we want and, so long as no edges are created or severed, we will still have the same graph.

Two graphs that have the same fundamental structure (though not necessarily the same vertex labels) are called isomorphic.
Specifically, two graphs $G,H$ are isomorphic if $x \sim y \iff f(x) \sim f(y)$ for some function (isomorphism) $f$, where $x,y \in V_G$ and $\sim$ is an ``adjacency'' relation.

The following graphs are not equal, but they are isomorphic.

\medskip
\begin{center}
    \begin{tikzpicture}[spring layout, iterations=500, node distance=25pt]
            \foreach \i in {1,...,4}
                \node (node \i) [fill=black!70, text=white, circle, font=\scriptsize, inner sep=1pt, minimum size=8pt] {\i};
                \node (node A) [fill=black!70, text=white, circle, font=\scriptsize, inner sep=1pt, minimum size=8pt] {A};
                \node (node B) [fill=black!70, text=white, circle, font=\scriptsize, inner sep=1pt, minimum size=8pt] {B};
                \node (node C) [fill=black!70, text=white, circle, font=\scriptsize, inner sep=1pt, minimum size=8pt] {C};
                \node (node D) [fill=black!70, text=white, circle, font=\scriptsize, inner sep=1pt, minimum size=8pt] {D};
            \draw 
                (node 1) edge (node 2)
                edge (node 3)
                edge (node 4)
                (node A) edge (node B)
                edge (node C)
                edge (node D);
    \end{tikzpicture}
\end{center}

There are certain types of graphs that are particularly interesting.
One class of these is defined below.

\begin{definition}[Complete graph]
    A complete graph $K_n$ consists of $n$ vertices, where every vertex is adjacent to every other vertex.
\end{definition}

One well-known problem in graph theory has to do with coloring the edges of a complete graph (say, red and blue) and seeing what monochromatic subgraphs emerge, if any.
A useful analogy for these is in the context of friends and strangers: if an edge between two vertices is colored red, we say the vertices are ``friends'', whereas if the edge is blue the vertices are strangers.
We'll start simple with $K_6$.

\begin{theorem}[Ramsey's theorem, (3, 3)]
    In any collection of six people, there must exist three mutual friends or three mutual strangers.

    Equivalently, if the edges of $K_6$ are colored red and blue, then it must contain a red $K_3$ or a blue $K_3$.
\end{theorem}

\begin{proof}
    After coloring the edges of $K_6$, vertex 6 must have
    \begin{enumerate}[label=(\alph*)]
        \item at least three red edges or
        \item at least three blue edges.
    \end{enumerate}
    We'll begin with case (a).
    Consider the subgraph of $K_6$ which includes vertex 6 along with three of its red-adjacent nodes.

    \begin{center}
        \begin{tikzpicture}[spring layout, iterations=500, node distance=25pt]
                \node (node 6) [fill=black!70, text=white, circle, font=\scriptsize, inner sep=1pt, minimum size=10pt] {6};
                \node (node A) [fill=black!70, text=white, circle, font=\scriptsize, inner sep=1pt, minimum size=8pt] {A};
                \node (node B) [fill=black!70, text=white, circle, font=\scriptsize, inner sep=1pt, minimum size=8pt] {B};
                \node (node C) [fill=black!70, text=white, circle, font=\scriptsize, inner sep=1pt, minimum size=8pt] {C};
                \draw 
                    (node 6) edge[color=red] (node A)
                    edge[color=red] (node B)
                    edge[color=red] (node C)
                    (node A) edge[dashed, color=gray] (node B)
                    edge[dashed, color=gray] (node C)
                    (node B) edge[dashed, color=gray] (node C);
        \end{tikzpicture}
    \end{center}

    If any of the edges among $A,B,C$ are red, then we have a red triangle.
    Otherwise, all of the edges are blue, meaning we have a blue triangle!

    Case (b) can be proved in the same way.
\end{proof}

A follow-up question that we could ask in response to this is whether we can state a similar theorem for, say, $K_5$.
It turns out that we can't; finding a counterexample is relatively straightforward.
Thus, 6 is known as the third Ramsey number---it describes the smallest complete graph that contains either a red $K_3$ or a blue $K_3$ after edge coloring.

We'll now discover a similar number for a red $K_3$ or blue $K_4$.

\begin{corollary}[Ramsey's theorem, (3, 4)]
    Any group of ten people must contain three mutual friends or four mutual strangers.

    Equivalently, if the edges of $K_{10}$ are colored red and blue, then it must contain a red $K_3$ or a blue $K_4$.
\end{corollary}

\begin{proof}
    After coloring the edges of $K_{10}$, vertex 10 must have
    \begin{enumerate}[label=(\alph*)]
        \item at least four red edges or
        \item at least six blue edges.
    \end{enumerate}
    We'll begin with case (a).
    Consider the subgraph of $K_{10}$ which includes vertex 10 along with four of its red-adjacent vertices.

    \begin{center}
        \begin{tikzpicture}[spring layout, iterations=500, node distance=25pt]
                \node (node 10) [fill=black!70, text=white, circle, font=\scriptsize, inner sep=1pt, minimum size=8pt] {10};
                \node (node A) [fill=black!70, text=white, circle, font=\scriptsize, inner sep=1pt, minimum size=11pt] {A};
                \node (node B) [fill=black!70, text=white, circle, font=\scriptsize, inner sep=1pt, minimum size=11pt] {B};
                \node (node C) [fill=black!70, text=white, circle, font=\scriptsize, inner sep=1pt, minimum size=11pt] {C};
                \node (node D) [fill=black!70, text=white, circle, font=\scriptsize, inner sep=1pt, minimum size=11pt] {D};
                \draw 
                    (node 10) edge[color=red] (node A)
                    edge[color=red] (node B)
                    edge[color=red] (node C)
                    edge[color=red] (node D)
                    (node A) edge[dashed, color=gray] (node B)
                    edge[dashed, color=gray] (node C)
                    edge[dashed, color=gray] (node D)
                    (node B) edge[dashed, color=gray] (node C)
                    edge[dashed, color=gray] (node D)
                    (node C) edge[dashed, color=gray] (node D);
        \end{tikzpicture}
    \end{center}

    If any of the edges among $A,B,C,D$ are red, then we have a red $K_3$.
    Otherwise, all of the edges are blue, meaning we have a blue $K_4$.

    As for case (b), consider the graph of $K_{10}$ which includes vertex 10 along with six of its blue-adjacent vertices (similar to the above subgraph).
    If we remove vertex 10 and consider only vertices $A$-$F$, we get a $K_6$; by the previous theorem, this must contain a red $K_3$ or a blue $K_3$.
    If it contains a red $K_3$, then we're done; if it contains a blue $K_3$, then since each vertex is also connected to vertex 10 via a blue edge, this blue $K_3$ is part of a larger blue $K_4$.
\end{proof}

We can go a bit lower with $K_9$.

\begin{corollary}[Ramsey's theorem, (3, 4)]
    The above holds for nine people or, equivalently, $K_9$.
\end{corollary}

\begin{proof}
    Suppose we color the edges of $K_9$.
    By the previous corollary, if any vertex has
    \begin{enumerate}[label=(\alph*)]
        \item at least four red edges or
        \item at least six blue edges,
    \end{enumerate}
    then we're done.
    Otherwise, since each vertex has eight edges, every vertex has exactly three red edges and five blue edges.
    But this is impossible by the oddball corollary since the red subgraph has nine vertices, each with degree 3.
\end{proof}

This is as far as we can go with (3, 4); there is a counterexample for $K_8$.
The corresponding Ramsey number, then, is $R(3,4) = 9$.

\section{Walking on Graphs}
Now we'll introduce some more vocabulary so that we can more effectively describe other classes of graphs.

\begin{definition}[Walk]
    A walk on a graph is a sequence of adjacent vertices, with repetition.
    The length of such a walk is the number of vertices in the sequence, minus one.
\end{definition}

\begin{definition}[Path]
    A path is a walk with no repeated vertices.
\end{definition}

We can state a seemingly trivial relationship between these two objects, but it reveals a new technique of proof that we will take advantage of throughout this chapter.

\begin{theorem}[]
    For vertices $x$ and $y$, a path from $x$ to $y$ exists if and only if a walk from $x$ to $y$ exists.
\end{theorem}

\begin{proof}
    ($\Rightarrow$) Let $P$ be a path from $x$ to $y$.
    By definition, $P$ is also a walk from $x$ to $y$.
    \smallskip

    ($\Leftarrow$) We give an extremal argument.
    If there is a walk from $x$ to $y$, then there is a walk $W$ of minimum length.
    We claim that $W$ is a path.

    If there are repeated vertices in $W$, then there is a shorter walk that can be created by removing the vertices between the repetitions, so $W$ is not a minimal walk.
    Therefore, if $W$ is a minimal walk, then there are not repeated vertices in $W$.
\end{proof}

We have a couple more definitions along the same lines as the previous ones.

\begin{definition}[Trail]
    A trail is a walk with no repeated edges.
    A trail is closed if it begins and ends at the same vertex.
\end{definition}

\begin{definition}[Cycle]
    A cycle is a closed trail such that removing the last vertex also yields a path.
\end{definition}

We can use this vocabulary to formalize an characteristic of graphs that is immediately obvious when looking at one.

\begin{definition}[Connected graph]
    A graph is connected if, for all vertices $x$ and $y$, a path exists from $x$ to $y$.
\end{definition}

Another immediate application of all this vocabulary comes in defining a couple other important classes of graphs.

\begin{definition}[Eulerian graph]
    An Eulerian trail is one that visits all edges of a graph at least once.

    A connected graph is Eulerian if it can be drawn as a closed trail; such a trail is called an Eulerian circuit.
\end{definition}

The degrees of the vertices of a graph with Eulerian characteristics have nice relationships!
These are given below; the proof of the second theorem is very similar to that of the first, so we omit it for clarity.

\begin{theorem}[]
    If $G$ is a connected graph that can be drawn as an Eulerian trail from $x$ to $y$, where $x \neq y$, then
    \begin{enumerate}[label=(\alph*)]
        \item vertices $x$ and $y$ each have odd degree and
        \item all other vertices have even degree.
    \end{enumerate}
\end{theorem}

\begin{proof}
    Consider any Eulerian trail from $x$ to $y$.
    For any vertex $v$ other than $x$ and $y$, every time we enter $v$ we must also immediately leave $v$, so $d(v)$ is even.
    The same is true for $x$ after the initial exit and for $y$ after the final entry, so $d(x)$ and $d(y)$ are both odd.
\end{proof}

\begin{theorem}[]
    If a connected graph $G$ is Eulerian, then all vertices have even degree.
\end{theorem}

It turns out that both of these are actually if and only if theorems---their converses are also true!

\begin{theorem}[Eulerian graph theorem]
    If $G$ is a connected graph and every vertex has even degree, then $G$ is Eulerian.
\end{theorem}

\begin{proof}
    We do strong induction on the number of edges in $G$.
    The base case, $e = 0$, is obviously true.

    Now let $G$ be a connected graph with $e > 0$ edges.
    Suppose as our inductive hypothesis that the theorem holds for all connected graphs with fewer than $e$ edges.
    Since every vertex has even degree, every vertex has a degree of at least 2.
    So $G$ must contain a cycle $C$.

    If $G = C$, then we're done.
    Otherwise, remove the edges of $C$ from $G$, creating the graph $G - C$, which has fewer edges than $G$; moreover, each vertex of $G - C$ still has even degree.
    From here, we have two cases.
    \begin{enumerate}[label=(\alph*)]
        \item If $G - C$ is connected, then by the inductive hypothesis, $G - C$ is Eulerian.
        Thus $G - C$ can be drawn as an Eulerian circuit, beginning and ending at a vertex $v$ on $C$; from here we can draw $C$.
        Thus all of $G$ can be drawn as an Eulerian circuit, meaning $G$ is Eulerian.

        \item If $G - C$ is not connected, then $G - C$ has connected components, each of which is Eulerian (by the inductive hypothesis).
        We can draw $G$ using the same idea as before.
        We trace out $C$, traversing each new component of $G$ when we first visit it; we are guaranteed to visit each component at least once.
    \end{enumerate}
    Either way, we conclude that $G$ is Eulerian.
\end{proof}

\begin{corollary}
    If $G$ is connected and all vertices have even degree, except for vertices $x$ and $y$, then $G$ has an Eulerian trail from $x$ to $y$.
\end{corollary}

\begin{proof}
    Insert a new vertex $z$ that is adjacent to only $x$ and $y$.
    This new graph $G + z$ is still connected, and all vertices have even degree, so by the previous theorem it is Eulerian.
    So $G + z$ has an Eulerian circuit beginning and ending at $z$, thus removing $z$ gives an Eulerian trail from $x$ to $y$.
\end{proof}

When we generalize these Eulerian notions slightly, we get a new class of graphs, defined below.

\begin{definition}[Hamiltonian graph]
    Given a graph $G$:
    \begin{itemize}
        \item a Hamiltonian path is one that visits every vertex of $G$.
        \item a Hamiltonian cycle is one that visits every vertex of $G$.
        \item $G$ is Hamiltonian if it contains a Hamiltonian cycle.
    \end{itemize}
\end{definition}

Unfortunately, unlike Eulerian graphs, it is unknown if there is an efficient way to determine if a large graph has a Hamiltonian cycle.
But there are sometimes efficient tests!
(For example, if $G$ has $n$ vertices and each vertex has a degree of at least $n/2$, then $G$ is Hamiltonian.)

\section{Trees and Planar Graphs}
Now we'll describe a couple of other types of graphs.

\begin{definition}[Tree]
    A tree is a connected graph with no cycles.
    A disconnected graph whose components are trees is called a forest.
\end{definition}

A quick aside: Cayley's theorem states that, for $n \geq 1$,the number of distinct (unlabeled) trees with $n$ vertices is $n^{n-2}$.
The proof of this statement is beyond the scope of this course.

\begin{definition}[Leaf]
    In a tree, a vertex with degree 1 is called a leaf.
\end{definition}

We'll show, now, that all trees have leaves!

\begin{theorem}[]
    Any tree with at least two vertices has at least two leaves.
\end{theorem}

\begin{proof}
    We give an extermal argument.
    Consider the longest path in $T$, which has $v_1$ as its first vertex.
    We claim that $v_1$ must be a leaf.

    Suppose, to the contrary, that $v_1$ is adjacent to a vertex $v \neq v_2$.
    If $v$ is on $T$, then $T$ contains a cycle; if $v$ is not on $T$, then $v_1$ is adjacent to both $v$ and $v_2$, giving a path that is longer than $T$.
    Either way, we get a contradiction.

    By the same logic, the final vertex in $T$ is also a leaf.
    Therefore, $T$ contains two leaves, meaning its graph also contains two leaves.
\end{proof}

Notice, now, that when we remove a leaf from a tree, we are still left with a tree.
This leads to nice induction proofs, like the one below!

\begin{theorem}[]
    Every tree with $n$ vertices has $n-1$ edges.
\end{theorem}

\begin{proof}
    We'll do induction on $n$, the number of vertices on the graph.
    The base cases $n=1$ and $n=2$ are obviously true.

    Now suppose as our inductive hypothesis that the statement holds for any tree with $n$ vertices.
    If we take a tree $T$ with $n+1$ vertices and remove one of its leaves, we are left with a tree that has $n$ vertices; by the inductive hypothesis, the ``reduced'' tree has $n-1$ edges, so $T$ has $n$ edges, as desired.
\end{proof}

Finally, we have a theorem which is useful for ``communicating'' uniquely between vertices.

\begin{theorem}[]
    Any two vertices on a tree are connected by a unique path.
\end{theorem}

\begin{proof}
    Since the tree $T$ is connected, we know that a path $P$ exists from $x$ to $y$.
    Suppose there was also a different path $Q$ that connected these vertices.

    Let $P$ and $Q$ be identical up until vertex $a$, and let $b$ be the next point on $P$ that is also on $Q$.
    (We may have $a = x$ and $b = y$.)
    Then, starting at $a$, we can reach $b$ via the path $P$, and we can get back to $a$ by backtracking $Q$.
    This describes a cycle in $T$, which is not allowed by the definition of a tree.
\end{proof}

Trees are a special type of another class of graphs, defined below.

\begin{definition}[Planar graph]
    A planar graph is a graph that can be drawn in such a way that no edges cross.
\end{definition}

Notice that planar graphs divide the plane into regions, called faces.
(The region outside of the graph is also a face.)
This brings us to one last theorem from Euler.

\begin{theorem}[]
    If $G$ is a connected plane graph with $n$ vertices, $\mathcal{E}$ edges, and $f$ faces, then
    \[ n - \mathcal{E} + f = 2. \]
\end{theorem}

\begin{proof}
    We do induction on $\mathcal{E}$, the number of edges.
    The base cases $\mathcal{E} = 0$ and $\mathcal{E} = 1$ are trivial.

    Suppose, as our inductive hypothesis, that the statement holds when $\mathcal{E} = K$.
    Now let $G$ be a connected plane graph with $\mathcal{E} = k+1$.
    We have two cases.
    \begin{enumerate}[label=(\alph*)]
        \item If $G$ is a tree with $n$ vertices, then $\mathcal{E} = n-1$ and $f = 1$, which satisfies $n - \mathcal{E} + f = 2$.
        (No induction necessary here.)
        \item If $G$ is not a tree, then it must contain a cycle.
        After removing an edge from the cycle, the new graph remains connected, but has one fewer face; that is, it has $n$ vertices, $k$ edges, and $f-1$ faces.
        By the inductive hypothesis:
        \begin{align*}
            n - k + (f-1) &= 2 \\
            n - (k + 1) + f &= 2 \\
            n - \mathcal{E} + f &= 2,
        \end{align*}
        as desired.
    \end{enumerate}
    Either way, we get $n - \mathcal{E} + f = 2$.
\end{proof}

This actually explains why we can't construct Venn diagrams with more than three circles!
One with four circles would have $n = 12$, $f = 16$, and $\mathcal{E} = 24$, which violates the above theorem.

Usually, graphs are made nonplanar when they have ``too many edges''.
The next theorem gives an edge-related condition that a graph must satisfy to even possibly be planar.

\begin{theorem}[]
    If $G$ is planar with $n \geq 3$ vertices and $\mathcal{E}$ edges, then $\mathcal{E} \leq 3n - 6$.
\end{theorem}

\begin{proof}
    Suppose $G$ has $n$ vertices, $\mathcal{E}$ edges, and $f$ faces.
    We have two cases.
    \begin{enumerate}[label=(\alph*)]
        \item $G$ is connected.
        Construct the ``edge-face'' matrix $M$ with $\mathcal{E}$ rows and $f$ columns; the $(i,j)$ entry is 1 if edge $i$ borders face $j$, and the entry is otherwise 0.

        Let $x$ be number of 1s in $M$.
        Since every edge borders at most two faces, $x \leq 2\mathcal{E}$; also, since each face has at least three edges, $x \geq 3f$.
        Thus $3f \leq  2\mathcal{E}$.
        By Euler's theorem, $f = 2 - n + \mathcal{E}$, so
        \begin{align*}
            3(2 - n + \mathcal{E}) &\leq 2\mathcal{E} \\
            \mathcal{E} &\leq 3n - 6,
        \end{align*}
        as desired.

        \item $G$ is disconnected.
        Insert $k-1$ edges at arbitrary locations such that they create the connected graph $G^+$.
        (Note that this graph is still planar.)
        Applying Case (a) to $G^+$ gives
        \[ \mathcal{E} < \mathcal{E} + (k - 1) \leq 3n - 6, \]
        as desired.
    \end{enumerate}
    either way, we get $\mathcal{E} \leq 3n - 6$.
\end{proof}

We can use this to quickly show that the complete graph $K_5$ is nonplanar.

Another important graph that turns out to be nonplanar is $K_{3,3}$, which is drawn below.

% TODO: make this look better
\begin{center}
    \begin{tikzpicture}[spring layout, iterations=500, node distance=25pt]
            \foreach \i in {1,...,3}
                \node (node \i) [fill=red!70, text=white, circle, font=\scriptsize, inner sep=1pt, minimum size=8pt] {};
            \foreach \i in {4,...,6}
                \node (node \i) [fill=blue!70, text=white, circle, font=\scriptsize, inner sep=1pt, minimum size=8pt] {};
            \draw 
                (node 1) edge (node 4)
                edge (node 5)
                edge (node 6)
                (node 2) edge (node 4)
                edge (node 5)
                edge (node 6)
                (node 3) edge (node 4)
                edge (node 5)
                edge (node 6);
    \end{tikzpicture}
\end{center}

A proof of this graph's nonplanarity is below.

\begin{theorem}[]
    $K_{3,3}$ is nonplanar.
\end{theorem}

\begin{proof}
    Suppose, to the contrary, that $K_{3,3}$ is planar.
    Then by Euler's theorem, $f = 5$, so its edge-face matrix has 9 rows and 5 columns.
    Let $x$ be the number of 1s in this matrix, so $x \leq 18$.
    But for $K_{3,3}$ every face must have at least four edges (notice that it has no triangles), meaning $x \geq 4f = 20$, a contradiction.
\end{proof}

As it turns out, by Kuratowski's theorem, every nonplanar graph contains $K_5$ or $K_{3,3}$, or a subdivision of $K_5$ or $K_{3,3}$.
(To subdivide a graph is to add degree-2 nodes along the existing edges of the graph, so as to not drastically change its overall structure.)

\section{Graph Coloring}
One well-known class of problems in graph theory has to do with coloring graphs---assigning each vertex a color, with certain restrictions.
For our purposes, we'll restrict ourselves to the following.

\begin{definition}[Properly colored graph]
    A graph is properly colored by giving each vertex a color in such a way that no adjacent vertices have the same color.
\end{definition}

We introduce some related vocabulary that more specifically describes graphs that are properly colorable.

\begin{definition}[$k$-colorable]
    A graph is $k$-colorable if it can be properly colored with $k$ colors or less.
    (A graph that is 2-colorable is also called bipartite.)
\end{definition}

\begin{definition}[Chromatic number]
    The chromatic number of a graph, denoted $\chi (G)$, is the smallest $k$ for which $G$ is $k$-colorable.
\end{definition}

A graph is 1-colorable only when $G$ has no edges.
Creating a condition for 2-colorability is slightly more interesting!

\begin{theorem}[2-colorable graphs]
    $G$ is 2-colorable if and only if it has no odd cycles (cycles with an odd number of vertices).
\end{theorem}

\begin{proof}
    ($\Rightarrow$) We prove the contrapositive.
    Suppose $G$ has an odd cycle consisting of vertices $v_1, \ldots, v_{2k+1}$.
    If we color this graph using two colors, then all of the odd vertices must be one color, and all the even vertices another.
    But $v_1$ and $v_{2k+1}$ are both odd, so they must have the same color, meaning the graph is not 2-colorable.
    \smallskip

    ($\Leftarrow$) Suppose $G$ has no odd cycles.
    It suffices to prove this for connected $G$ because, if each component is 2-colorable, then so is $G$.

    We have two cases.
    \begin{enumerate}[label=(\alph*)]
        \item Suppose $G$ is a tree.
        We'll prove this case by induction on the number $n$ of vertices.
        The statement is clearly true for the base case $n=1$.

        Suppose as our inductive hypothesis that any tree with $n$ vertices is 2-colorable.
        (The no-odd-cycles condition is built into the tree by definition.)
        Consider a tree with $n+1$ vertices and remove a leaf; we now have a tree with $n$ vertices, which we assume to be colorable.
        Now add back the leaf, assign it the color opposite that of the vertex it's adjacent to, and we're done.

        \item Suppose $G$ is not a tree.
        Temporarily remove edges from $G$ until we do have a tree (called a spanning tree); from case (a), this tree is 2-colorable.
        Now put all the edges back---we claim that no edge connects vertices of the same color.

        When we add an edge from, say, $x$ to $y$, we get an even cycle $C$.
        This means the number of steps from $x$ to $y$ is odd, so $x$ and $y$ have opposite colors.
        Thus, when we insert all of the deleted edges, we still have a 2-coloring.
    \end{enumerate}
    Either way, if $G$ has no odd cycles, then it is 2-colorable.
\end{proof}

We'll now consider colorings of planar graphs in particular.
This has an important application in mapmaking!
Every map (in the colloquial sense) can be represented equivalently using using its dual graph.
This is the graph whose vertices are regions on the map, and whose edges represent boundaires between regions.

The four-color theorem states that any map can be ``properly colored'' using only four colors.
That is, using four colors, one can color a map such that no adjacent regions have the same color.

We will not prove the four-color theorem here, but we will prove its analogs for 6-colorings and 5-colorings.

\begin{theorem}[Six-color theorem]
    Every planar graph is 6-colorable.
\end{theorem}

\begin{proof}
    We do induction on the number of vertices $n$.
    The base cases $n \leq 6$ are obvious.

    Suppose, as our inductive hypothesis, that any planar graph with $n$ vertices is 6-colorable.

    Consider a graph with $n+1$ vertices.
    This graph has at least one vertex $v$ of degree 5 or less; if we remove $v$, we are left with a graph that has $n$ vertices, which (by the IHOP) is 6-colorable.
    Now if we add $v$ back into the graph, it is adjacent to at most five vertices, so there is a color we can assign to $v$ that has not been used by any adjacent vertex.
\end{proof}

\begin{theorem}[Five-color theorem]
    Every planar graph is 5-colorable.
\end{theorem}

\begin{proof}
    We do induction on the number of vertices $n$.
    The base cases $n \leq 5$ are obvious.

    Suppose, as our inductive hypothesis, that any planar graph with $n$ vertices is 5-colorable.

    Consider a graph with $n+1$ vertices.
    This graph has at least one vertex $v$ of degree 5 or less; if we remove $v$, we are left with a graph that has $n$ vertices, which (by the IHOP) is 5-colorable.
    Now if we add $v$ back into the graph and it has less than five adjacent colors, then we're done; otherwise, $v$ has five adjacent colors.
    We claim that, in this case, one vertex can be re-colored, which would free up a color for $v$.

    Label the vertices adjacent to $v$ as $v_1, \ldots, v_5$ in clockwise-ascending order.
    (This order is not necessarily unique.)
    Suppose these vertices are respectively colored red, yellow, green, blue, and some fifth color.

    Change the color of $v_1$ from red to green.
    Then take all green vertices adjacent to $v_1$ and color them red.
    Then color all red vertices adjacent to those green.
    If we continue this, we get one of two cases.
    \begin{enumerate}[label=(\alph*)]
        \item The process eventually terminates and we can immediately color $v$ red.
        \item We come back to $v_3$ and color it red, putting us back at square one.
        But now we can do the same process with $v_2$ and $v_4$: change $v_4$ from blue to yellow, and continue the chain until it terminates.
        (Here, the chain will never reach $v_2$ since it's being guarded by an entirely red-green chain!)
        We can now safely color $v$ blue.
    \end{enumerate}
    Either way, we can re-color the graph in such a way that allows us to safely color $v$, as desired.
\end{proof}

FORGOT THE LEMMA THAT ALL PLANAR GRAPHS HAVE A VERTEX WITH DEGREE FIVE OR LESS!

\section{Tournaments (Lec. 23)}
TOURNAMENTS (Directed graphs)
\begin{itemize}
    \item DEF. A tournament is a complete directed graph.
    \item THM. Every graph has a directed Hamiltonian path. (Pf. Induction on vertices.)
    \item DEF. For a tournament, player $x$ is a king (or king chicken) if, for every other player $y$, either $x \to y$ or there exists a player $z$ such that $x \to z \to y$.
    \item THM. Every tournament has a king. (Pf. The person who won the most games is a king.)
\end{itemize}
MINIMUM SPANNING TREES (Weighted graphs)
\begin{itemize}
    \item come back later...
\end{itemize}

\end{document}
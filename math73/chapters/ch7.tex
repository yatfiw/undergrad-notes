\documentclass[../m73main.tex]{chapters}
\graphicspath{{\subfix{../figures/}}}

\begin{document}

\section{Inner Product Spaces}
We begin this chapter by continuing our generalization of Euclidean vector spaces.
Here, we define the inner product as a generalization of the dot product.

\begin{definition}[Inner product]
	An inner product on a vector space $V$ is an operation that assigns to every pair of vectors $\mbf{u}, \mbf{v} \in V$ a real number $\left< \mbf{u}, \mbf{v} \right>$ such that the following properties hold for all vectors $\mbf{u}$, $\mbf{v}$, and $\mbf{w}$ in $W$ and all scalars $c$:
	\begin{enumerate}[label=(\alph*)]
		\item $\left< \mbf{u}, \mbf{v} \right> = \left< \mbf{v}, \mbf{u} \right>$.
		\item $\left< \mbf{u}, \mbf{v} + \mbf{w} \right> = \left< \mbf{u}, \mbf{v} \right> + \left< \mbf{u}, \mbf{w} \right>$.
		\item $\left< c \mbf{u}, \mbf{v} \right> = c \left< \mbf{u}, \mbf{v} \right>$.
		\item $\left< \mbf{u}, \mbf{u} \right> \geq 0$ and $\left< \mbf{u}, \mbf{u} \right> = 0$ if and only if $\mbf{u} = \mbf{0}$.
	\end{enumerate}
	A vector space with an inner product is called an inner product space.
\end{definition}

\begin{theorem}[Properties of the inner product]
	Let $\mbf{u}$, $\mbf{v}$, and $\mbf{w}$ be vectors in an inner product space $V$ and let $c$ be a scalar.
	\begin{enumerate}[label=(\alph*)]
		\item $\left< \mbf{u} + \mbf{v}, \mbf{w} \right> = \left< \mbf{u}, \mbf{w} \right> + \left< \mbf{v}, \mbf{w} \right>$.
		\item $\left< \mbf{u}, c \mbf{v} \right> = c \left< \mbf{u}, \mbf{v} \right>$.
		\item $\left< \mbf{u}, \mbf{0} \right> = \left< \mbf{0}, \mbf{v} \right> = 0$.
	\end{enumerate}
\end{theorem}

We can now define such notions as length, distance, and orthogonality in abstract vector spaces.
The remainder of this section will be dedicated to rehashing many of the related results from Euclidean spaces.

\begin{definition}[Geometry in inner product spaces]
	Let $\mbf{u}$ and $\mbf{v}$ be vectors in an inner product space $V$.
	\begin{enumerate}
		\item The length (or norm) of $\mbf{v}$ is $\|\mbf{v}\| = \sqrt{\left< \mbf{v}, \mbf{v} \right>}$.
		(A unit vector is a vector of length 1.)
		\item The distance between $\mbf{u}$ and $\mbf{v}$ is $\pfn{d}(\mbf{u}, \mbf{v}) = \|\mbf{u} - \mbf{v}\|$.
		\item $\mbf{u}$ and $\mbf{v}$ are orthogonal if $\left< \mbf{u}, \mbf{v} \right> = 0$.
	\end{enumerate}
\end{definition}

\begin{theorem}[Pythagorean theorem]
	Let $\mbf{u}$ and $\mbf{v}$ be vectors in an inner product space $V$.
	Then $\mbf{u}$ and $\mbf{v}$ are orthogonal if and only if
	\[ \|\mbf{u} + \mbf{v}\|^2 = \|\mbf{u}\|^2 + \|\mbf{v}\|^2. \]
\end{theorem}

\begin{definition}[Orthogonal projection]
	Let $\{\mbf{u}_1, \ldots, \mbf{u}_k\}$ be an orthogonal basis for a subspace $W$ of an inner product space $V$, and let $\mbf{v}$ be a vector in $V$.
	Then the orthogonal projection $\pfn{proj}_W(\mbf{v})$ of $\mbf{v}$ onto $W$ is
	\[ \pfn{proj}_W(\mbf{v}) = \frac{\left< \mbf{u}_1, \mbf{v} \right>}{\left< \mbf{u}_1, \mbf{u}_1 \right>} \mbf{u}_1 + \cdots + \frac{\left< \mbf{u}_k, \mbf{v} \right>}{\left< \mbf{u}_k, \mbf{u}_k \right>}. \]
	The component of $\mbf{v}$ orthogonal to $W$ is the vector
	\[ \pfn{perp}_W(\mbf{v}) = \mbf{v} - \pfn{proj}_W(\mbf{v}) \]
\end{definition}

\begin{theorem}[Cauchy-Schwarz inequality]
	Let $\mbf{u}$ and $\mbf{v}$ be vectors in an inner product space $v$.
	Then
	\[ |\left< \mbf{u}, \mbf{v} \right>| \leq \|\mbf{u}\| \|\mbf{v}\|, \]
	with equality holding if and only if $\mbf{u}$ and $\mbf{v}$ are scalar multiples of each other.
\end{theorem}

\begin{theorem}[Triangle inequality]
	Let $\mbf{u}$ and $\mbf{v}$ be vectors in an inner product space $V$.
	Then
	\[ \|\mbf{u} + \mbf{v}\| \leq \|\mbf{u}\| + \|\mbf{v}\|. \]
\end{theorem}

\section{Least Squares Approximation}
This section is motivated by the search for a ``line of best fit'' for a given set of data.
We begin by speaking strictly in terms of vectors.

\begin{definition}[Best approximation of a vector]
	If $W$ is a subspace of a normed vector space $V$ and if $\mbf{v}$ is a vector in $V$, then the best approximation to $\mbf{v}$ in $W$ is the vector $\overline{\mbf{v}}$ in $W$ such that
	\[ \|\mbf{v} - \overline{\mbf{v}}\| < \|\mbf{v} - \mbf{w}\| \]
	for every vector $\mbf{w}$ in $W$ different from $\overline{\mbf{v}}$.
\end{definition}

Provided that $V$ is equipped with an inner product, our problem has a simple solution.

\begin{theorem}[Best approximation theorem]
	If $W$ is a finite-dimensional subspace of an inner product space $V$ and if $\mbf{v}$ is a vector in $V$, then $\pfn{proj}_W(\mbf{v})$ is the best approximation to $\mbf{v}$ in $W$.
\end{theorem}

We can use this pair of ideas to determine approximate solutions to inconsistent linear systems.

\begin{definition}[Least squares solution]
	If $A$ is an $m \times n$ matrix and $\mbf{b}$ is in $\R^m$, a least squares soution of $A \mbf{x} = \mbf{b}$ is a vector $\overline{\mbf{x}}$ in $\R^n$ such that
	\[ \|\mbf{b} - A \overline{\mbf{x}}\| \leq \|\mbf{b} - A \mbf{x}\| \]
	for all $\mbf{x}$ in $\R^n$.
\end{definition}

Note that $\mbf{b} - A \overline{\mbf{x}} = \pfn{perp}_{\pfn{col}(A)}(\mbf{b})$.
Therefore, $\mbf{b} - A \overline{\mbf{x}}$ is in $(\pfn{col}(A))^\perp = \pfn{null}(A^T)$, and we have the following.

\begin{theorem}[Least squares theorem]
	Let $A$ be an $m \times n$ matrix and let $\mbf{b}$ be in $\R^m$.
	Then $A \mbf{x} = \mbf{b}$ always has at least one least squares solution $\overline{\mbf{x}}$.
	Moreover:
	\begin{enumerate}[label=(\alph*)]
		\item $\overline{\mbf{x}}$ is a least squares solution of $A \mbf{x} = \mbf{b}$ if and only if $\overline{\mbf{x}}$ is a solution of the normal equations $A^TA \overline{\mbf{x}} = A^T \mbf{b}$.
		\item $A$ has linearly independent columns if and only if $A^TA$ is invertible.
			  In this case, the least squares solution of $A \mbf{x} = \mbf{b}$ is unique and is given by
			  \[ \overline{\mbf{x}} = (A^TA)^{-1}A^T \mbf{b}. \]
	\end{enumerate}
\end{theorem}

Suppose we have a set of points of the form $(x_i, y_i)$, and we want to find a line of best fit $y = a + bx$ through them.
To do this, we use the above theorem to find a least-squares solution to the linear system
\[ \begin{bmatrix} 1 & x_1 \\ 1 & x_2 \\ \vdots & \vdots \\ 1 & x_n \end{bmatrix} \begin{bmatrix} a \\ b \end{bmatrix} = \begin{bmatrix} y_1 \\ y_2 \\ \vdots \\ y_n \end{bmatrix}. \]

Computing $A^T A$ may take some time, and solving the resulting system of equations may take even longer.
However, if we have a $QR$ factorization of $A$, our job is made much easier.

\begin{theorem}[Least squares via $QR$ factorization]
	Let $A$ be an $m \times n$ matrix with linearly independent columns and let $\mbf{b}$ be in $\R^m$.
	If $A = QR$ is a $QR$ factorization of $A$, then the unique least square ssolution $\overline{\mbf{x}}$ of $A \mbf{x} = \mbf{v}$ is
	\[ \overline{\mbf{x}} = R^{-1} Q^T \mbf{b}. \]
\end{theorem}

\section{The Singular Value Decomposition}
In this final section, we will briefly investigate one last matrix decomposition.
These have to do with the singular values of a matrix, deifned below.

\begin{definition}[Singular values]
	If $A$ is an $m \times n$ matrix, the singular values of $A$ are the square roots of the eigenvalues of $A^T A$ and are denoted by $\sigma_1, \ldots, \sigma_n$. It is conventional to arrange the singular values so that $\sigma_1 \geq \sigma_2 \geq \cdots \geq \sigma_n$.
\end{definition}

If the eigenvalues of a matrix have to do with a lack of motion under a linear transformation, the singular values of a matrix quantifies the maximum and minimum motions of space under a linear transformation.
We can use these singular values, along with the matrix's eigenvectors, to construct a relatively straightforward decomposition.

\begin{theorem}[Singular value decomposition]
	Let $A$ be an $m \times n$ matrix with singular values $\sigma_1 \geq \sigma_2 \geq \cdots \geq \sigma_r > 0$ and $\sigma_{r+1} = \sigma_{r+2} = \cdots = \sigma_n = 0$.
	Then there eixst an $m \times m$ orthogonal matrix $U$, an $n \times n$ orthogonal matrix $V$, and an $m \times n$ matrix $\Sigma$ such that
	\[ A = U \Sigma V^T. \]
	The matrix $\Sigma$ contains the ordered singular values of $A$ along its main diagonal.
	The matrices other $V$ and $U$ are of the following forms:
	\begin{gather*}
		V = \begin{bmatrix} \mbf{v}_1 & \cdots & \mbf{v}_n \end{bmatrix} \\
		U = \begin{bmatrix} \mbf{u}_1 & \cdots & \mbf{u}_m \end{bmatrix}
	\end{gather*}
	where $\mbf{v}_1, \ldots, \mbf{v}_n$ are the eigenvectors of $A^T A$, $\mbf{u}_i = \frac{1}{\sigma_i} A \mbf{v}_i$ for $i \leq r$, and $\mbf{u}_{r+1}, \ldots, \mbf{u}_m$ are vectors added to the set $\{ \mbf{u}_1, \ldots, \mbf{u}_r \}$ to make it a basis for $\R^m$.
\end{theorem}

We can now make one last addition to the Fundamental Theorem.

\begin{theorem}[Additions to the FTLA]
	Let $A$ be an $n\times n$ matrix and let $T : V \to V$ be a linear transformation whose matrix $[T]_{\mathcal{C} \from \mathcal{B}}$ with respect to bases $\mathcal{B}$ and $\mathcal{C}$ of $V$ and $W$, respectively, is $A$.
	The following statements are equivalent:
	\begin{enumerate}[label=(\alph*)]
		\item $A$ is invertible. \\
		\phantom{~}\hspace{-19.5pt} \vdots
		\setcounter{enumi}{20}
		\item 0 is not a singular value of $A$.
	\end{enumerate}
	For the full theorem, see Appendix A.
\end{theorem}

\end{document}
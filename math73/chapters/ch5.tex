\documentclass[../m73main.tex]{subfiles}
\graphicspath{{\subfix{../figures/}}}

\begin{document}

\chapter{Orthogonality}

\section{Orthogonality in $\R^n$}
Here, we extend the notion of orthogonality from pairs of vectors to sets of vectors.

\begin{definition}[Orthogonal set of vectors]
	A set of $\mbf{v}_1, \mbf{v}_2, \ldots, \mbf{v}_k$ in $\R^n$ is called an orthogonal set if all pairs of distinct vectors in the set are orthogonal. 
\end{definition}

There are many advantages to working with orthogonal sets.
One of these is that orthogonal sets are necessarily linearly independent, meaning they make for convenient bases for subspaces.

\begin{theorem}[Orthogonal sets are linearly independent]
	If $\{\mbf{v}_1, \mbf{v}_2, \ldots, \mbf{v}_k\}$ is an orthogonal set of nonzero vectors in $\R^n$, then these vectors are linearly independent.
\end{theorem}

For this reason, orthogonal sets often make a convenient choice of basis for subspaces.

\begin{definition}[Orthogonal basis]
	An orthogonal basis for a subspace $W$ of $\R^n$ is a basis of $W$ that is an orthogonal set.
\end{definition}

Of course, any vector in $W$ can be written as a linear combniation of these orthognal vectors.
Another convenience that comes with orthogonal sets is that the coefficients of this linear combination are easy to find.

\begin{theorem}[Coefficients of an orthogonal linear combination]
	Let $\{\mbf{v}_1, \mbf{v}_2, \ldots, \mbf{v}_k\}$ be an orthogonal basis for a subspace $W$ of $\R^n$ and let $\mbf{w}$ be any vector in $W$.
	Then the unique scalars $c_1, c_2, \ldots, c_k$ such that
	\[ \mbf{w} = c_1 \mbf{v}_1 + c_2 \mbf{v}_2 + \cdots + c_k \mbf{v}_k \]
	are given by
	\[ c_1 = \frac{\mbf{w} \cdot \mbf{v}_i}{\mbf{v}_i \cdot \mbf{v}_i} \text{ for } i = 1, \ldots, k. \]
\end{theorem}

Notice that, in the case of an orthogonal basis, we simply project the vector onto each of the basis vectors to find the coefficients of the linear combination.
Things are even more convenient if our orthogonal set consists entirely of unit vectors.

\begin{definition}[Orthonormal set]
	A set of vectors in $\R^n$ is an orthonormal set if it is an orthogonal set of unit vectors.
	An orthonormal basis for a subspace $W$ of $\R^n$ is a basis of $W$ that is an orthonormal set.
\end{definition}

\begin{theorem}[Coefficients of an orthonormal linear combination]
	Let $\{\mbf{q}_1, \mbf{q}_2, \ldots, \mbf{q}_k\}$ be an orthonormal basis for a subspace $W$ of $\R^n$ and let $\mbf{w}$ be any vector in $W$.
	Then
	\[ \mbf{w} = (\mbf{w} \cdot \mbf{q}_1) \mbf{q}_1 + (\mbf{w} \cdot \mbf{q}_2) \mbf{q}_2 + \cdots + (\mbf{w} \cdot \mbf{q}_k) \mbf{q}_k,\]
	and this representation is unique.
\end{theorem}

When a matrix has columns that form an orthonormal set, they turn out to have many very nice properties.
The rest of this section is dedicated to examining these.

Notice that we can think of matrix multiplication as a grid of dot products between row and column.
When the row and column are orthogonal to each other, they'll ``cancel out.''
It turns out that when we multiply a matrix with its transpose, it happens in such a way that it leaves behind only the identity matrix.

\begin{theorem}[Transpose of a matrix with orthonormal columns]
	The columns of an $m \times n$ matrix $Q$ for an orthonormal set if and only if $Q^T Q = I_n$.
\end{theorem}

Clearly, for square choices of $Q$, the transpose $Q^T$ is the inverse of $Q$.
This is a powerful result, rewritten below below.

\begin{definition}[Orthogonal matrix]
	A square matrix $Q$ whose columns form an orthonormal set is called an orthogonal matrix.
\end{definition}

\begin{theorem}[Inverse of an orthogonal matrix]
	A square matrix $Q$ is orthogonal if and only if $Q^{-1} = Q^T$.
\end{theorem}

Perhaps expectedly, orthogonal matrices have some convenient properties.
One of the most important ones is that they preserve the lengths of the vectors they transform---that is, they are isometric.

\begin{theorem}[Orthogonal matrices are isometric]
	Let $Q$ be a $n \times n$ matrix.
	The following statements are equivalent:
	\begin{enumerate}[label=(\alph*)]
		\item $Q$ is orthogonal.
		\item $\|Q \mbf{x}\| = \|\mbf{x}\|$ for every $\mbf{x}$ in $\R^n$.
		\item $Q \mbf{x} \cdot Q \mbf{y} = \mbf{x} \cdot \mbf{y}$ for every $\mbf{x}$ and $\mbf{y}$ in $\R^n$.
	\end{enumerate}
\end{theorem}

We'll finish off with a couple of theorems elaborating on some properties of orthogonal matrices.

\begin{theorem}[Orthogonal matrices have orthonormal rows]
	If $Q$ is an orthogonal matrix, then its rows form an orthonormal set.
\end{theorem}

\begin{theorem}[Properties of orthogonal matrices]
	Let $Q$ be an orthogonal matrix.
	\begin{enumerate}[label=(\alph*)]
		\item $Q^{-1}$ is orthogonal.
		\item $\det Q = \pm 1$.
		\item If $\lambda$ is an eigenvalue of $Q$, then $|\lambda| = 1$.
		\item If $Q_1$ and $Q_2$ are orthogonal matrices of the same size, then so is $Q_1 Q_2$.
	\end{enumerate}
\end{theorem}

\section{Orthogonal Complements and Projections}
Now we generalize two more ideas: the vector normal to a plane, and the projection of one vector onto another.           
We begin with the first of these.

\begin{definition}[Orthogonal complement]
	Let $W$ be a subspace of $\R^n$.
	We say that a vector $\mbf{v}$ in $\R^n$ is orthogonal to $W$ if $\mbf{v}$ is orthogonal to every vector in $W$.
	The set of all vectors that are orthogonal to $W$ is called the orthogonal complement of $W$, denoted $W^\perp$.
\end{definition}

The orthogonal complement of a subspace has some fairly intuitive properties.

\begin{theorem}[Properties of orthogonal complements]
	Let $W$ be a subspace of $\R^n$.
	\begin{enumerate}[label=(\alph*)]
		\item $W^\perp$ is a subspace of $\R^n$.
		\item $\left( W^\perp \right)^\perp = W$.
		\item $W \cap W^\perp = \{\mbf{0}\}$.
		\item If $\pfn{span}(\mbf{w}_1, \mbf{w}_2, \ldots, \mbf{w}_k)$, then $\mbf{v}$ is in $W^\perp$ if and only if $\mbf{v} \cdot \mbf{w}_i = 0$ for all $i = 1, \ldots, k$.
	\end{enumerate}
\end{theorem}

Using orthogonal complements, we can make describe some fundamental relationships between the subspaces associated with a matrix.
To relate the row space to the null space, notice that for $A \mbf{x} = \mbf{0}$ to be true, we must have $\mbf{r} \cdot \mbf{x}$ for every row $\mbf{r}$ in $A$; therefore, all rows in $A$ are orthogonal to vectors in $\pfn{null}(A)$.
A simmilar line of reasoning relates $\pfn{col}(A)$ to $\pfn{null}(A^T)$. 

\begin{theorem}[Fundamental spaces and orthogonality]
	Let $A$ be a matrix.
	Then $(\pfn{row}(A))^\perp = \pfn{null}(A)$ and $(\pfn{col}(A))^\perp = \pfn{null}(A^T)$.
\end{theorem}

Now, we'll move on to generalizing projections.

\begin{definition}[Orthogonal projection onto a space]
	Let $W$ be a subspace of $\R^n$ and let $\{\mbf{u}_1, \mbf{u}_2, \ldots, \mbf{u}_k\}$ be an orthogonal basis for $W$.
	For any vector $\mbf{v}$ in $\R^n$, the orthogonal projection of $\mbf{v}$ onto $W$ is defined as
	\[ \pfn{proj}_W(\mbf{v}) = \pfn{proj}_{\mbf{u}_1}(\mbf{v}) + \pfn{proj}_{\mbf{u}_2}(\mbf{v}) + \cdots + \pfn{proj}_{\mbf{u}_k}(\mbf{v}). \]
	The component of $\mbf{v}$ orthogonal to $W$ is the vector
	\[ \pfn{perp}_W(\mbf{v}) = \mbf{v} - \pfn{proj}_W(\mbf{v}). \]
\end{definition}

We can use orthogonal projections to decompose a vector into orthogonal components.

\begin{theorem}[Orthogonal decomposition]
	Let $W$ be a subspace of $\R^n$ and let $\mbf{v}$ be a vector in $\R^n$.
	Then there are unique vectors $\mbf{w}$ in $W$ and $\mbf{w}^\perp$ in $W^\perp$ such that
	\[ \mbf{v} = \mbf{w} + \mbf{w}^\perp. \]
\end{theorem}

Lastly, we have a relationship between the dimensions of a subspace and its orthogonal complement.

\begin{theorem}[Dimension of orthogonal subspace]
	If $W$ is a subspace of $\R^n$, then
	\[ \pfn{dim } W + \pfn{dim } W^\perp = n. \]
\end{theorem}

Note that the rank theorem is a corollary to this theorem.

\section{The Gram-Schmidt Process and the $QR$ Factorization}
Given how useful orthogonal bases have proven to be, it might also be useful to know how to construct one from a non-orthogonal basis.
We do this using a series of projections.

\begin{theorem}[Gram-Schmidt process]
	Let $\{\mbf{x}_1, \mbf{x}_2, \ldots, \mbf{x}_k\}$ be a basis for a subspace $W$ of $\R^n$, let $W_i = \pfn{span}(\mbf{x}_1, \mbf{x}_2, \ldots, \mbf{x}_i)$, and define the following:
	\begin{alignat*}{2}
		\mbf{v}_1 &= \mbf{x}_1 \\
		\mbf{v}_2 &= \pfn{perp}_{W_1}(\mbf{x}_2) &&= \mbf{x}_k - \pfn{proj}_{\mbf{v}_1}\mbf{x}_k \\
		&\phantom{~!}\vdots \\
		\mbf{v}_k &= \pfn{perp}_{W_{k-1}}(\mbf{x}_k) &&= \mbf{x}_k - \pfn{proj}_{\mbf{v}_1}\mbf{x}_k - \pfn{proj}_{\mbf{v}_2}\mbf{x}_k - \cdots - \pfn{proj}_{\mbf{v}_{k-1}}\mbf{x}_k
	\end{alignat*}
	Then for each $i = 1, \ldots, k$, $\{\mbf{v}_1, \mbf{v}_2, \ldots, \mbf{v}_i\}$ is an orthogonal basis for $W_i$.
	In particular, $\{\mbf{v}_1, \mbf{v}_2, \ldots, \mbf{v}_k\}$ is an orthogonal basis for $W$.
\end{theorem}

We can use this to decompose any invertible square matrix into an orthogonal matrix and a triangular matrix.

\begin{theorem}[$QR$ factorization]
	Let $A$ be an $m \times n$ matrix with linearly independent columns.
	Then $A$ can be factored as $A = QR$, where $Q$ is an $m \times n$ matrix with orthonormal columns and $R$ is an invertible upper triangular matrix.
\end{theorem}

\end{document}
\documentclass[../p111main.tex]{subfiles}
\graphicspath{{\subfix{../figures/}}}

\begin{document}

\chapter{Oscillators}
\section{Oscillations About Equilibria}
Now we'll spend some time developing the theory of small oscillations about equilibria.
It turns out to be incredibly powerful and far-reaching, as far as physics goes!

The general form of a one-dimensional Lagrangian is
\[ L = \frac{1}{2} M(q) \dot q^2 - U(q). \]
We know, by now, that every equilibrium point $q_0$ looks like a harmonic oscillator when we look closely enough; when we define the displacement $\delta \equiv q - q_0$ from equilibrium and do a second-order (to preserve the kinetic energy) Taylor expansion our Lagrangian becomes
\begin{align*}
    L &= \frac{1}{2} M(q_0 + \delta) \dot \delta^2 - U(q_0 + \delta) \\
    &\approx \frac{1}{2} M(q_0) \dot \delta^2 - \frac{1}{2}U''(q_0) \delta^2,
\end{align*}
plus a constant that vanishes upon finding the Euler-Lagrange equation
\[ M(q_0) \ddot \delta = -U''(q_0) \delta \;\implies\; \ddot \delta = \frac{U''(q_0)}{M(q_0)} \delta. \]
For $U''(q_0) < 0$ we get exponential solutions and our implicit $|\delta| \ll q_0$ assumption quickly goes invalid.
But $U''(q_0) > 0$ gives oscillations with
\[ \omega^2 = \frac{U''(q_0)}{M(q_0)}. \]
This idea of determining the behavior of the system will reappear in our discussion of more complex oscillators.

\section{Two Coupled Oscillators}
Consider two masses $m$ in a space between two walls.
Each mass is connected to its respective wall with a spring $k$, and the masses are connected to one another with a spring $k$'.
The masses' rightward displacements from equilibrium are given by $x_1, x_2$.
The Lagrangian of this setup is
\[ L = \frac{1}{2} m \left( \dot x_1^2 + \dot x_2^2 \right) - \frac{1}{2} k \left( x_1^2 + x_2^2 \right) - \frac{1}{2} k' \left( x_1 - x_2 \right)^2, \]
and the Euler-Lagrange equations are
\begin{align*}
    m \ddot x_1 &= -k x_1 - k' x_1 + k' x_2, \\
    m \ddot x_2 &= -k x_2 - k' x_2 + k' x_1.
\end{align*}
If we define $x_+ \equiv x_1 + x_2$ and $x_- \equiv x_1 - x_2$ then we can decouple these equations:
\[ m \ddot x_+ = -kx_+, \qquad m\ddot x_- = -(k + 2k') x_-. \]
So our solutions are
\begin{align*}
    x_+(t) &= A_+ \cos (\omega_+ t + \delta_+), \quad \omega_+^2 \equiv k / m, \\
    x_-(t) &= A_- \cos (\omega_- t + \delta_-), \quad \omega_-^2 \equiv (k + 2k') / m.
\end{align*}
We can write this in a way that's a bit more suggestive of what this is hinting at.
If we define a vector $\mbf{X}(t)$ whose components are $x_1(t)$ and $x_2(t)$ then we have
\[ \mbf{X}(t) = \begin{bmatrix} (x_+ + x_-) / 2 \\ (x _+ - x_-) / 2 \end{bmatrix} = \frac{A_+}{2} \begin{bmatrix} 1 \\ 1 \end{bmatrix} \cos (\omega_+ t + \delta_+) + \frac{A_-}{2} \begin{bmatrix} 1 \\ -1 \end{bmatrix} \cos (\omega_- t + \delta_-). \]
Each term here corresponds to an eigenmode of our system: the first represents the ``together'' motion of the two masses, while the second corresponds to the ``apart'' motion.

It will soon become useful for us to have this solution in complex form.
We can write
\begin{align*}
    \mbf{X}(t) &= \pfn{Re} \left( \frac{A_+}{2} \begin{bmatrix} 1 \\ 1 \end{bmatrix} e^{i \omega_+ t} e^{i \delta_+} + \frac{A_-}{2} \begin{bmatrix} 1 \\ -1 \end{bmatrix} e^{i \omega_- t} e^{i \delta_-} \right), \\
    \intertext{and if we define $C_+ = (A_+ / 2) e^{i\delta_+}$ with the first normal mode $\mbf{Z}_+$ (along with analogous quantities $C_-, \mbf{Z}_-$),}
    &= \pfn{Re} \Big( C_+ \mbf{Z}_+ e^{i\omega_+ t} + C_- \mbf{Z}_- e^{i\omega_- t} \Big).
\end{align*}
We've gone from four free parameters to just two!
(This is just taking advantage of the fact that one complex number encodes the same amount of information as two real numbers.)

\subsection*{An Eigenvalue Problem}
Now, we could've come to all the same conclusions in a much more streamlined way using some linear algebra.
Notice that we could've written our original (coupled) Euler-Lagrange equations as
\[ \ddot{\mbf{X}} = -\frac{1}{m} \mbf{K} \mbf{X}, \qquad \mbf{K} = \begin{bmatrix} k + k' & -k' \\ -k' & k + k' \end{bmatrix}. \]
Rather than solve this problem directly, we'll say that $\mbf{X}(t) = \pfn{Re}(\mbf{Z(t)})$ for some complex-valued function $\mbf{Z}$ and focus on $\ddot{\mbf{Z}} = -(1 / m) \mbf{K} \mbf{Z}$.
Substituting the ansatz $\mbf{Z}(t) = \mbf{Z}_0 e^{i \omega t}$ for some constant vector $\mbf{Z}_0$ gives
\begin{align*}
    (i\omega)^2 \mbf{Z}_0 e^{i\omega t} &= -\frac{1}{m} \mbf{K} \mbf{Z}_0 e^{i\omega t}, \\
    m \omega^2 \mbf{Z}_0 &= \mbf{K} \mbf{Z}_0.
\end{align*}
This is a classic eigenvalue problem!
Going through the motions tells us that we have eigenvalues when
\[ \omega^2 = \frac{k}{m} \;\text{ or }\; \omega^2 = \frac{k + 2k'}{m}, \]
just like we've already found.
The corresponding eigenvectors are the $\mbf{Z}_+$ and $\mbf{Z}_-$ from before.

\subsection*{Weak Coupling}
Now let's quickly look at the weak coupling limit $k' \ll k$.
Here we have
\[ \omega_2 = \sqrt{\frac{k}{m}} \sqrt{1 + \frac{2k'}{k}} \approx \sqrt{\frac{k}{m}} \left( 1 + \frac{1}{2} \frac{2k'}{k} \right) = \omega_1 + \frac{k'}{\sqrt{mk}}. \]
If $\omega_0 = (\omega_+ + \omega_-) / 2$ and $\Delta \omega = \omega_- - \omega_+$ then our solution from before becomes
\[ \mbf{Z}(t) = e^{i\omega_0 t} \left( C_1 \begin{bmatrix} 1 \\ 1 \end{bmatrix} e^{-i \Delta \omega \,t} + C_2 \begin{bmatrix} 1 \\ -1 \end{bmatrix} e^{i \Delta \omega \,t} \right). \]
If mass 2 initially has zero displacement then substituting $x_2(0) = 0$ gives $C_1 = C_2$ and
\[ \mbf{Z}(t) = C_1 e^{i \omega_0 t} \begin{bmatrix} e^{-i \Delta \omega \,t} + e^{i \Delta \omega \,t} \\ e^{-i \Delta \omega \,t} - e^{i \Delta \omega \,t} \end{bmatrix} = 2 C_1 e^{i\omega_0 t} \begin{bmatrix} \cos \Delta \omega \,t \\ -i \sin \Delta \omega \,t \end{bmatrix}. \]
Defining $C_1 = D_1 e^{i \delta_1}$ for real $D_1, \delta_1$ gives
\[ \mbf{X}(t) = \pfn{Re}(\mbf{Z}(t)) = D_1 \begin{bmatrix} \cos(\omega_0 t + \delta_1) \cos (\Delta \omega \,t) \\ \sin(\omega_0 t + \delta_1) \sin (\Delta \omega \,t) \end{bmatrix}. \]
So $x_1(t)$ and $x_2(t)$ both look like beats with out-of-phase envelopes!

\section{Several Coupled Oscillators}
Now let's look at a system of $N$ coupled oscillators near equilibrium (at $q_1 = \cdots = 0$).
To second order,
\[ U(q) \approx U(0) + \frac{1}{2} \sum_{i,j}^{} q_i q_j \frac{\partial^2 U(0)}{\partial q_i \partial q_j}, \]
where the linear terms have gone to zero because $U(0)$ is a minimum.
We'll define a symmetric matrix $\mbf{K}$ with entries
\[ \mbf{K}_{ij} = \frac{\partial^2 U(0)}{\partial q_i \partial q_j} \]
so we can write
\[ U(q) = \frac{1}{2} \sum_{i,j}^{} q_I \mbf{K}_{ij} q_j = \frac{1}{2} \mbf{X}^T \mbf{K} \mbf{X}, \]
where $\mbf{X}$ is a coordinate vector with components $q_1, \ldots, q_N$.
We can do a similar thing with the kinetic energy which, including only the second-order terms, looks like
\begin{align*}
    T &\approx \frac{1}{2} \mbf{M}(q)_{11} \dot q_1^2 + \frac{1}{2} \mbf{M}(q)_{12} \dot q_1 \dot q_2 + \frac{1}{2} \mbf{M}(q)_{13} \dot q_1 \dot q_3 + \cdots \\
    &= \frac{1}{2} \sum_{i,j}^{} \dot q_i \mbf{M}_{ij}(0) \dot q_j \\
    &= \frac{1}{2} \dot{\mbf{X}}^T \mbf{M}(0) \dot{\mbf{X}}, \qquad \mbf{M}_{ij} = \frac{\partial T}{\partial \dot q_i \partial \dot q_j}.
\end{align*}
So the Lagrangian is
\[ L = \frac{1}{2} \sum_{i,j}^{} \dot q_i \mbf{M}_{ij} \dot q_j - \frac{1}{2} \sum_{i,j}^{} q_i \mbf{K}_{ij} q_j. \]
Let's now derive the Euler-Lagrange equation for $q_k$.
We have
\begin{align*}
    \frac{\partial L}{\partial q_k} &= -\frac{1}{2} \sum_{i,j}^{} \mbf{K}_{ij} \frac{\partial}{\partial q_k} (q_i q_j) \\
    &= -\frac{1}{2} \sum_{i,j}^{} K_{ij} \left( q_j \delta_{ik} + q_i \delta_{jk} \right) \\
    &= -\frac{1}{2} \sum_{j}^{} K_{kj} q_j - \frac{1}{2} \sum_{i}^{} K_{ik} q_i \\
    &= -(\mbf{K} \mbf{X})_k
\end{align*}
We could similarly show that $\partial L / \partial \dot q_k = (\mbf{M} \dot{\mbf{X}})_k$, meaning $(\mbf{M} \ddot{\mbf{X}})_k = -(\mbf{K} \mbf{X})_k$ and, in all, the Euler-Lagrange equations are
\[ \mbf{M} \ddot{\mbf{X}} = -\mbf{K} \mbf{X}. \]
When we go to solve these, we once again define a complex-valued $\mbf{Z}$ satisfying $\mbf{X}(t) = \pfn{Re}(\mbf{Z}(t))$ to write $\mbf{M} \ddot{\mbf{Z}} = -\mbf{K} \mbf{Z}$.
Substituting the ansatz $\mbf{Z}(t) = \mbf{Z}_0 e^{i \omega t}$ gives
\[ \omega^2 \mbf{M} \mbf{Z}_0 = \mbf{K} \mbf{Z}_0 .\]
This is what we call a generalized eigenvalue problem!
We'd proceed by solving $\det (\mbf{K} - \omega^2 \mbf{M}) = 0$ and them finding the null space of $\mbf{K} - \omega^2 \mbf{M}$ for each of the resulting eigenvalues.
Note that each of the eigenvalues $\omega^2$ will turn out to be real and positive, and that all of the eigenvectors $\mbf{Z}_i$ satisfy $\mbf{Z}_i^T \mbf{M} \mbf{Z}_j = 0$.
(This is what we'll mean by orthogonality in this context.)

\pagebreak

\section{Linear Resonance}
Consider, now, an isolated damped oscillator with mass $m$, spring constant $k$, damping coefficient $b$, and driving force $F_d$.
By Newton's second law we have
\begin{align*}
    m \ddot x + b \dot x + k x &= F_d, \\
    \ddot x + 2\beta \dot x + \omega_0^2 x &= \frac{F_d}{m},
\end{align*}
where $2\beta \equiv b / m$ and $\omega_0^2 \equiv k / m$.
We could show that the homogeneous solutions to this differential equation decay exponentially in time, so we'll ignore them in this discussion.
As for the particular solution, take $F_d(t) = \cos (\omega t)$ and define $x(t) = \pfn{Re}(z(t))$, turning the equation into
\[ \ddot z + 2\beta \dot z + \omega_0^2 z = f e^{i\omega t}. \]
Substituting the ansatz $z = Ae^{i \omega t}$ yields
\[ A = \frac{f}{\omega_0^2 - \omega^2 + 2i \beta \omega} = \frac{f(\omega_0^2 - \omega^2)}{(\omega_0^2 - \omega^2)^2 + 4\beta^2 \omega^2} - i \frac{2\beta\omega f}{(\omega_0^2 - \omega^2)^2 + 4\beta^2 \omega^2}, \]
where in the last expression we'e multiplied by the conjugate of the first denominator.
So if $A = a + bi$ then we could show that our solution looks like
\begin{align*}
    x_p(t) &= \pfn{Re} \big[ (a + bi)(\cos \omega t + i \sin \omega t) \big] \\
    &= C \cos (\omega t + \delta),
\end{align*}
where
\[ C = \frac{f}{\big[ (\omega_0^2 - \omega^2)^2 + 4\beta^2 \omega^2 \big]^{1 / 2}}, \qquad \tan \delta = \frac{2\beta \omega}{\omega_0^2 - \omega^2}. \]
Note that $C$ is maximized at $\omega = \omega_0$, and that the maximum value is $C_\textrm{max} = f / 2\beta \omega_0$.
But $\omega_0$ is not the only frequency that gets amplified---to communicate the ``characteristic width'' of amplified frequencies, we'll say we want $(\omega_0^2 - \omega^2)^2 \sim 4 \beta^2 \omega^2$ and so
\[ (\omega_0 + \omega)(\omega_0 - \omega) \sim 2\beta \omega \;\implies\; (\omega_0 - \omega) \sim \beta \]
if we take $\omega_0 + \omega \sim 2\omega$.
So $2\beta$, in a sense, communicates the width of frequencies that get amplified!
The output phase changes with different input frequencies, too---in certain limiting cases we have
\[ \delta = \begin{cases} 0 & \omega \ll \omega_0, \\ \pi & \omega \gg \omega_0, \\ \pi / 2 & \omega = \omega_0. \end{cases} \]

\end{document}
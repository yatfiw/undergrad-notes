\documentclass[../m082main.tex]{subfiles}
\graphicspath{{\subfix{../figures/}}}

\begin{document}

\chapter{Introduction}

\begin{definition}[Ordinary differential equation]
    An ordinary differential equation (ODE) is an equation involving a function of one independent variable and its derivatives.
    A function $y$ is a solution to an ordinary differential equation if, when substituted into the ODE, both sides of the equation are equal for all values of the independent variable.
\end{definition}

This is in contrast with partial differential equations (PDEs), which involve functions with more than one independent variable and their partial derivatives.
We will not discuss PDEs in any detail here, so the initialisms "ODE" and "DE" will be interchangeable in these notes.

There are many different ways we can classify differential equations.
We list four of them below.

\begin{definition}[Classifications of an ODE]
    \begin{itemize}
        \item The order of a DE is that of the highest-order derivative it contains.

        \item An autonomous DE is one in which the independent variable does not appear explicitly in the DE.
        A DE that is not autonomous is called nonautonomous.

        \item A linear DE in $y$ is one of the form
        \[ a_n(x)y^{(n)} + \cdots + a_1(x)y' + a_0(x)y = b(x), \]
        where $a_0, a_1, \ldots, a_n, b$ are differentiable functions of the independent variable $x$.
        A DE that is not linear is called nonlinear.

        \item A driven (or forced, non-homogeneous) DE includes a nonzero term that does not contain the dependent variable.
        A DE that is not driven is called undriven (or unforced, homogeneous).
    
        \medskip
        \textit{Note: The terms "driven" and "undriven" are usually reserved for linear DEs.}
    \end{itemize}
\end{definition}


% \begin{definition}[Order of an ODE]
%     The order of a DE is that of the highest-order derivative it contains.
% \end{definition}

% \begin{definition}[Autonomous ODE]
%     An autonomous DE is one in which the independent variable does not appear explicitly in the DE.
%     A DE that is not autonomous is called nonautonomous.
% \end{definition}

% \begin{definition}[Linear ODE]
%     A linear DE in $y$ is one of the form
%     \[ a_n(x)y^{(n)} + \cdots + a_1(x)y' + a_0(x)y = b(x), \]
%     where $a_0, a_1, \ldots, a_n, b$ are differentiable functions of the independent variable $x$.
%     A DE that is not linear is called nonlinear.
% \end{definition}

% \begin{definition}[Driven ODE]
%     A driven (or forced, non-homogeneous) DE includes a nonzero term that does not contain the dependent variable.
%     A DE that is not driven is called undriven (or unforced, homogeneous).

%     \medskip
%     \textit{Note: The terms "driven" and "undriven" are usually reserved for linear DEs.}
% \end{definition}

When we solve a differential equation, there are two ways we can express the solution.
Which we use depends on the complexity of the solution.

\begin{definition}[Explicit or implicit solution to an ODE]
    An explicit solution to an ODE consists of an isolated dependent variable written as a function of the independent variable.
    An implicit solution is not solved for the independent variable.
\end{definition}

The solution to an ODE is usually a class of related equations rather than one particular equation.
If we want to pinpoint a specific solution with specific parameters, we might impose some conditions on what points the solution crosses.

\begin{definition}[Initial- or boundary-value problem]
    An initial-value problem (IVP) is a pairing of a differential equation with conditions all specified at some value of the independent variable.
    A boundary-value problem (BVP) is a pairing of a differential equation with conditions specified at the extremes of the independent variable.
\end{definition}

In general, after we solve a DE and obtain a family of solutions, we can impose on the solutions whatever condition we are given to yield a specific solution to the DE.
Sometimes, however, we aren't concerned with any particular initial condition or boundary condition and just want a general solution.

\end{document}
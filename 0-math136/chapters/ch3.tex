\documentclass[../m136main.tex]{subfiles}
\graphicspath{{\subfix{../figures/}}}

\begin{document}

\chapter{Integration}
\section{Contour Integrals}
The integral in $\C$ is defined in pretty much the same way as it is in $\R$, and evaluating them is reminiscent of line integrals in multivariable calculus.

\begin{definition}[Integral]
    Let $\gamma$ be a directed smooth curve in $\C$ that is split into several pieces $\gamma_k$, each with endpoints $z_k, z_{k+1}$ and a point $c_k$ between them.
    Then
    \[ \int_\gamma f(z) \,dz = \lim_{\substack{N \to \infty \\ \Delta z_k \to 0}} \sum_{k=0}^{N-1} f(c_k) \Delta z_k, \]
    provided the limit exists and is independent of the choice of partition.
\end{definition}

\begin{theorem}[]
    Let $f$ be a function continuous on the directed smooth curve $\gamma$.
    Then if $z = z(t)$, $t \in [a,b]$ is a parametrization of $\gamma$, we have
    \[ \int_\gamma f(z) \,dz = \int_a^b f(z(t)) \, z'(t) \,dt. \]
\end{theorem}

\begin{proof}
    Suppose $\gamma$ has a parametrization $z(t) = x(t) + i \,y(t)$, $t \in [a,b]$; the path is traversed once and $z'(t) \neq 0$ for all $t$.
    Defining $\Delta t_k = t_{k+1} - t_k$, we can write
    \begin{align*}
        \int_\gamma f(z) \,dz &= \lim_{\substack{N \to \infty \\ \Delta t_k \to 0}} \sum_{k=0}^{N-1} f(z(c_k)) \frac{z(t_{k+1}) - z(t_k)}{\Delta t_k} \Delta t_k, \\
        \intertext{where $c_k \in t_k, t_{k+1}$. There is no mean value theorem in $\C$, but we can split this difference quotient into real and imaginary parts to apply the MVT in $\R$: for some $\alpha_k, \beta_k \in (t_k, t_{k+1})$, we can write}
        &= \lim_{\substack{N \to \infty \\ \Delta t_k \to 0}} \sum_{k=0}^{N-1} f(z(c_k)) \big( x'(\alpha_k) + i \, y'(\beta_k) \big) \Delta t_k \\
        &= \int_a^b f(z(t)) \, z'(t) \,dt,
    \end{align*}
    as desired.
\end{proof}

We could use this to show that
\[ \oint_\gamma (z - z_0)^{n} \, dz = \begin{cases} 0 & n \neq -1, \\ 2\pi i & n = -1, \end{cases} \] %<3
where $\gamma$ is a once-traversed circle centered at $z_0$, oriented counterclockwise; this particular integral will prove to be very useful later.
We should expect that this result, like any other integral, is independent of how we choose to parametrize $\gamma$, and this is the case!
Proving it is a simple application of the above theorem.

\pagebreak

\begin{corollary}[]
    Let $\tilde \gamma$ be a reparametrization of $\gamma$---that is, suppose there is a $\phi(t)$ satisfying $\phi'(t) > 0$ such that $\gamma(t) = \tilde \gamma(\phi(t))$.
    Then
    \[ \int_\gamma f(z) \,dz = \int_{\tilde \gamma} f(z) \,dz. \]
\end{corollary}

Given a path $\gamma$, we will use $-\gamma$ to denote $\gamma$ as traversed in the opposite direction, and we could show that
\[ \int_{-\gamma} f(z) \,dz = -\int_\gamma f(z) \,dz. \]
In general we will integrate over contours, which are finite sequences of directed smooth curves that connect one after another.
We will not always need to evaluate such contour integrals however; an upper bound is often good enough, and we can get one pretty easily.

\begin{lemma}[ML lemma]
    Let $\gamma$ be a contour with length $\ell(\lambda)$.
    If $f$ is continuous on $\gamma$ and if $|f(z)| \leq M$ for all $z$ on $\gamma$, then
    \[ \left| \int_\gamma f(z) \,dz \right| \leq M \cdot \ell(\gamma). \]
\end{lemma}

\begin{proof}
    Suppose $\int_\gamma f(z) \,dz = r_0 e^{i \theta_0}$.
    Then
    \begin{align*}
        \left| \int_\gamma f(z) \,dz \right| &= e^{-i\theta_0} \int_a^b f(z(t)) \,z'(t) \,dt \\
        &= \int_a^b \Re \left[ e^{-i\theta_0} f(z(t)) \, z'(t) \right] dt \\
        &\leq \int_a^b M \left| z'(t) \right| dt = M \cdot \ell(\gamma),
    \end{align*}
    as desired.
\end{proof}

Note that the ``lower triangle inequality'' $\big| |a| - |b| \big| \leq |a + b|$ will often be useful in using this inequality.

\section{Path Independence}
Now we're ready for some of the bigger theorems in integral calculus, starting with an old friend.

\begin{theorem}[Fundamental theorem]
    Suppose $f$ is continuous on a domain $\Omega$ and has a primitive (antiderivative) $F(z)$ throughout $\Omega$.
    Then for any contour $\gamma$ in $\Omega$ with endpoints $z$ and $w$ we have
    \[ \int_\gamma f(z) \,dz = F(w) - F(z). \]
\end{theorem}

\begin{proof}
    Let $\gamma : [a,b] \to \C$ be a smooth contour.
    Then
    \[ \int_\gamma f(z) \,dz = \int_a^b f(\gamma(t)) \gamma'(t) \,dt = F(\gamma(b)) - F(\gamma(a)). \]
    For any other contour, we can break it up into smooth contours and sum them to get the same.
\end{proof}

Note that this implies that $F$ is analytic, and so also continuous, in $\Omega$.
Next we have a less immediately applicable set of equivalent statements, but one whose use will soon become clearer.

\begin{theorem}[ABC theorem]
    Let $f$ be continuous in a domain $\Omega$.
    Then the following statements are equivalent.
    \begin{enumerate}[label=(\alph*),topsep=0pt]
        \item $f$ has an antiderivative in $\Omega$.
        \item If $\gamma$ is a closed loop in $\Omega$ then $\int_\gamma f(z) \,dz = 0$.
        \item The contour integrals of $f$ are path-independent in $\Omega$.
    \end{enumerate}
\end{theorem}

\begin{proof}
    It is easy to see that (a) implies both (b) by the fundamental theorem.
    Now let $\gamma_1, \gamma_2$ be any two contours in $\Omega$ with the same endpoints and define $\Gamma = \gamma_1 - \gamma_2$.
    If we assume (b), then
    \[ 0 = \int_{\gamma_1} f(z) \,dz + \int_{-\gamma_2} f(z) \,dz = \int_{\gamma_1} f(z) \,dz - \int_{\gamma_2} f(z) \,dz \]
    and we get (c).
    Now, assuming (c), fix $z_0 \in \Omega$ and let $\gamma(z)$ be a path from $z_0$ to some $z \in \Omega$.
    Define
    \[ F(z) = \int_{\gamma(z)} f(w) \,dw, \quad z \neq z_0. \]
    This function is well-defined because $F(z)$ is independent of the specific choice of path $\gamma$, and we could apply the definition of the derivative to show that $F' = f$, so we get (a).
\end{proof}

\section{Cauchy's Theorem}
The ABC theorem might seem difficult to apply at present, but we're about to derive another result that gives a sufficient condition for (b).
We need to lay a bit of topological groundwork first, though.

\begin{definition}[Homotopy]
    Let $\gamma_0, \gamma_1 : [0,1] \to \Omega$ be closed curves in a domain $\Omega$.
    $\gamma_1$ is homotopic to $\gamma_0$ if there exists a continuous homotopy $H : [0,1] \times [0,1] \to \Omega$ such that
    \[ H(0,t) = \gamma_0(t), \qquad H(1,t) = \gamma_1(t), \qquad H(s,0) = H(s,1). \]
    (We include the third condition because we're interested in closed curves.)
\end{definition}

\begin{definition}[Simply connected domain]
    A domain $\Omega$ is simply connected if every closed path in $\Omega$ is homotopic to a point in $\Omega$---in other words, the domain has no ``holes''.
\end{definition}

We could do some straight forward calculus to show that if
\[ I(s) = \int_{\gamma_s} f(z) \,dz = \int_0^1 f(H(s,t)) \frac{\partial H}{\partial t} (s,t) \,dt, \]
then $I'(s) = 0$ so long as $f \in C^1$ (to bring the $\partial / \partial s$ into the integral) and $H \in C^2$ (for the equality of mixed partials).
Thus if $f$ has a continuous derivative, its integral is the same over any pair of homotopic curves.

A couple more clarifications.
A simple closed curve is one that does not self-intersect, and it is positively oriented if the enclosed region is on the left of the curve.
(This is how we generalize counterclockwise-ness.)

\pagebreak

Now we'll give a couple of proofs for the same theorem---one using this idea of contour deformation we just arrived at, and another using vector calculus.

\begin{theorem}[Cauchy's theorem (preliminary)]
    If $f$ is analytic with a continuous derivative on a simply connected domain $\Omega$ and $\gamma$ is a positively oriented simple closed curve in $\Omega$, then
    \[ \oint_\gamma f(z) \,dz = 0. \]
\end{theorem}

\begin{proof}
    On a simply connected domain $\Omega$, a closed loop $\gamma$ in $\Omega$ is homotopic to a point $z_0 \in \Omega$.
    Thus
    \[ \oint_\gamma f(z) \,dz = \int_{\left\{ z_0 \right\}} f(z) \,dz = 0 \]
    as desired. 
\end{proof}

\begin{proof}
    Let $\gamma$ be parametrized by $z(t)$, $t \in [a,b]$.
    If $z = x + iy$ and $f = u + iv$, then we can write
    \begin{align*}
        \oint_\gamma f \,dz &= \int_a^b f(z(t)) z'(t) \,dt \\
        &= \int_a^b (ux' - vy') + i(vx' + uy') \,dt \\
        &= \oint_\gamma u \,dx - v \,dy + i \oint_\gamma v \,dx + u\,dy, \\
        \intertext{and by Green's theorem}
        &= \iint_\Omega (-v_x - u_y) \,dA + i \iint_\Omega (u_x = v_y) \,dA;
    \end{align*}
    by the Cauchy-Riemann equations, this goes to zero.
\end{proof}

These proofs are simple, but we'd like to remove the condition of having a continuous derivative.
We begin with a special case.

\begin{lemma}[]
    Let $R = [a,b] \times [c,d]$ be a rectangle in $\Omega$.
    If $f$ is analytic on $\Omega$ then
    \[ \oint_{\partial R} f \,dz = 0. \]
\end{lemma}

\begin{proof}
    Suppose a rectangle $R$ has diagonal and perimeter lengths $D$ and $P$ respectively.
    Divide this rectangle into four sub-rectangles and suppose, without loss of generality, that $R_{11}$ is the sub-rectangle for which $\left| \oint_{\partial R_{i1}} f \,dz \right|$ is maximized.
    Now split this maximal rectangle into four and let $R_{12}$ be the piece that maximizes the line integral.
    Continuing on this way for $k$ subdivisions reveals that
    \[ \left| \oint_{\partial R} f\,dz \right| \leq 4^{k} \left| \oint_{\partial R_{1k}} f\,dz \right|. \]
    From analysis we know that there exists a $z_0 \in \bigcap_{k=1}^\infty R_{1k}$.
    $f$ is analytic at this point, meaning
    \[ f(z) = f(z_0) + f'(z_0) (z - z_0) + o(z - z_0), \qquad \lim_{z \to z_0} \frac{o(z - z_0)}{z - z_0} = 0. \]
    Thus for every $\epsilon > 0$ there is a $k > 1$ such that $|o(h) / (z - z_0)| < \epsilon$ for all $z \in R_{1k}$.
    By the ABC theorem and ML lemma,
    \[ \left| \oint_{\partial R_{1k}} f(z) \,dz \right| = \left| \oint_{\partial R_{1k}} o(h) \,dz \right| \leq (\epsilon \cdot 2^{-k} D) \cdot 2^{-k} P. \]
    The original integral therefore becomes
    \[ \left| \oint_{\partial R} f\,dz \right| \leq 4^{k} \left| \oint_{\partial R_{1k}} f\,dz \right| \leq 4^{k} \cdot (\epsilon \cdot 2^{-k} D \cdot 2^{-k} P) = \epsilon DP. \]
    Since our $\epsilon > 0$ was arbitrary, this integral is zero.
\end{proof}

Now we'll take another step by proving Cauchy's theorem, but only in a disk.

\begin{lemma}[Cauchy's theorem (in a disk)]
    Let $\Omega$ be an open disk with radius $R$ centered at $z_0$.
    If $f$ is analytic in $\Omega$ and $\gamma$ is a closed contour in $\Omega$ then
    \[ \oint_\gamma f(z) \,dz = 0. \]
\end{lemma}

\begin{proof}
    Given $w \in \Omega$ define $\gamma_w$ as the path from $z_0$ to $w$ traversed in two purely horizontal and vertical components $\gamma_1, \gamma_2$.
    The function
    \[ F(w) = \int_{\gamma_w} f(z) \,dz \]
    is well-defined for all $w \in \Omega$, and we'd like to prove that $F'(w) = f(w)$.

    We first determine $F(w+h) - F(w)$---given $\epsilon > 0$, there exists a $\delta > 0$ not only such that $|z-w| < \delta$ implies $|f(z) - f(w)| < \epsilon$, but also such that $\left\{ |z-w| < \delta \right\} \subseteq \Omega$.
    For $|h| < \delta$ we similarly define $\gamma_{w+h}$ in terms of horizontal and vertical components and notice that
    \[ \int_{\gamma_{w+h}} f \,dz = \int_{\gamma_w} f \,dz \pm \oint_{\partial R} f \,dz + \int_{w \to w+h} \hspace{-6pt} f \,dz, \]
    where $R$ is the rectangle formed by connecting $\gamma_w$ and $\gamma_{w+h}$ and the $\pm$ is whichever sign is needed to make things cancel nicely.
    The integral over $\partial R$ is zero by the previous lemma, meaning
    \[ F(w+h) - F(w) = \int_{w \to w+h} \hspace{-6pt} f \,dz. \]
    Now by the ML lemma,
    \[ \left| \frac{F(w+h) - F(w)}{h} - f(w) \right| = \frac{1}{|h|} \left| \int_{w \to w+h} \hspace{-6pt} (f(z) - f(w)) \,dz \right| < \frac{1}{|h|} (\epsilon \cdot 2|h|) = 2\epsilon, \]
    and since $\epsilon > 0$ was arbitrary we arrive at $F'(w) = f(w)$ for all $w \in \Omega$.
    Thus by the ABC theorem $\oint_\gamma f \,dz = 0$, as desired.
\end{proof}

Finally, we can make a contour deformation argument to get what we want!

\pagebreak

\begin{theorem}[Cauchy's theorem]
    If $f$ is analytic in a simply connected domain $\Omega$ and $\gamma$ is any closed contour in $\Omega$, then
    \[ \oint_\gamma f(z) \,dz = 0. \]
\end{theorem}

\begin{proof}
    If $\gamma_0$ is a closed contour in $\Omega$ then it is homotopic to a circle $\gamma_1$ interior to $\gamma_0$.
    If we break the domain of the homotopy $H$ into a bunch of very small chunks $R_{ij}$ and define $\Gamma_{ij} = H(\partial R_{ij})$ then by the previous theorem
    \[ \int_{\Gamma} f(z) \,dz = \sum_{i,j}^{} \int_{\Gamma_{ij}} f(z) \,dz = \sum_{i,j}^{} 0 = 0, \]
    where $\Gamma$ is the contour produced by traversing $\gamma_0$, moving inward to traverse $\gamma_1$, and then back out again.
    If $\gamma_0 = H(0,t)$ and $\gamma_1 = H(1,t)$ then we can write
    \[ 0 = \int_\Gamma f(z) \,dz = \int_{-H(0,t)} \hspace{-6pt} f(z) \,dz + \int_{H(1,t)} \hspace{-6pt} f(z) \,dz + \int_{H(s,0)} \hspace{-6pt} f(z) \,dz + \int_{-H(s,1)} \hspace{-6pt} f(z) \,dz, \]
    meaning $\oint_{\gamma_0} f(z) \,dz = \oint_{\gamma_1} f(z) \,dz$.
    Thus integrals in $\Omega$ are invariant under homotopy and for some $z_0 \in \Omega$
    \[ \oint_{\gamma_0} f(z) \,dz = \int_{\left\{ z_0 \right\}} f(z) \,dz = 0, \]
    as desired.
\end{proof}

\section{Cauchy's Integral Formula}
Contour deformation also provides motivation for our next key result.

\begin{theorem}[Cauchy's integral formula]
    Let $\gamma$ be a simple closed positively-oriented curve.
    If $f$ is analytic in a simply connected domain $\Omega$ containing $\gamma$ and $z_0$ is interior to $\gamma$, then
    \[ f(z_0) = \frac{1}{2\pi i} \oint_\gamma \frac{f(z)}{z - z_0} \,dz. \]
\end{theorem}

\begin{proof}
    $f$ is continuous at $z_0$, so for every $\epsilon > 0$ there is a $\delta > 0$ such that $|z - z_0| < \delta$ implies $|f(z) - f(z_0)| < \epsilon$.
    Let $r \in (0, \delta)$ be the radius of a positively-oriented circle $\gamma_r$ centered at $z_0$; then, because $2\pi i = \oint_{\gamma_r} dz / (z - z_0)$,
    \[ \left| \oint_{\gamma_r} \frac{f(z)}{z - z_0} \,dz - 2\pi i \,f(z_0) \right| = \left| \oint_{\gamma_r} \frac{f(z) - f(z_0)}{z - z_0} \right| \leq \oint_{\gamma_r} \frac{|f(z) - f(z_0)|}{|z - z_0|} \,dz < \frac{\epsilon}{r} \cdot 2\pi r. \]
    Since $\epsilon > 0$ was arbitrary we conclude, by contour deformation, that
    \[ \oint_\gamma \frac{f(z)}{z - z_0} \,dz = \oint_{\gamma_r} \frac{f(z)}{z - z_0} = 2\pi i \,f(z_0), \]
    as desired.
\end{proof}

Now, if we treat this integral as a function (rather than evaluating it only at $z_0$), its derivatives turn out to behave nicely.

\pagebreak

\begin{theorem}[]
    Let $g$ be continuous on the contour $\gamma$, and for each $z \notin \gamma$ set
    \[ G(z) = \int_\gamma \frac{g(s)}{s - z} \,ds. \]
    Then $G$ is analytic and
    \[ G'(z) = \int_\gamma \frac{g(s)}{(s-z)^2} \,ds. \]
\end{theorem}

\begin{proof}
    We could do some algebra and then apply the ML lemma to get
    \begin{align*}
        \left| \frac{G(z+h) - G(z)}{h} - \int_\gamma \frac{g(s)}{(s-z)^2} \,dz \right| &= \left| \int_\gamma g(s) \left[ \frac{h}{(s-z-h)(s-z)^2} \right] \right| \\
        &= |h| M \int_\gamma \frac{1}{|s-z-h| |s-z|^2} \,ds, \\
        \intertext{where $M = \max_{s \in \gamma} |g(s)|$. Now let $d = \min_{s \in \gamma} |s-z|$; then $|s-z - h| \geq |s-z| - |h| \geq d - d / 2$, so}
        &\leq |h| M \int_\gamma \frac{1}{(d / 2) d^2} \,ds, 
    \end{align*}
    which goes to zero as $h \to 0$.
\end{proof}

In showing this, we've also shown that the expression in Cauchy's formula is analytic with
\[ f'(z) = \frac{1}{2\pi i} \oint_\gamma \frac{f(s)}{(s-z)^2} \,ds. \]
It turns out that we can go through a similar line of reasoning to show that this $f'$ is analytic with a similar integral expression; a simple proof by induction gives the following.

\begin{theorem}[Cauchy's integral formula (for the $k$th derivative)]
    If $f$ is analytic on and inside a positively-oriented simple closed curve $\gamma$ and $z$ lies inside $\gamma$, then
    \[ f^{(k)}(z) = \frac{k!}{2\pi i} \oint_\gamma \frac{f(s)}{(s-z)^{k+1}} \,ds, \qquad k \in \N. \]
\end{theorem}

\begin{corollary}[]
    If $f$ is analytic in a domain $\Omega$ then all of its derivatives exist and are analytic on $\Omega$.
\end{corollary}

Because differentiability implies continuity, $f \in C^\infty$ and the continuous-derivative assumption we made earlier on was true all along!      % TODO: something about C^\infty versus C^\omega, continuous vs. analytic?
Now we have a couple more theorems that tie some loose ends.

\begin{corollary}[]
    If $f = u + iv$ is analytic then $u$ and $v$ are harmonic functions.
\end{corollary}

\begin{proof}
    Since $f \in C^\infty$, we have $u,v \in C^\infty$.
    The Cauchy-Riemann equations imply $u_{xx} = v_{yx}$ and $v_{yy} = -v_{xy}$, and since $v \in C^2$ its mixed partials are equal and so $\Delta u = \Delta v = 0$.
\end{proof}

\pagebreak

\begin{theorem}[Morera's theorem]
    If $f$ is continuous on $\Omega$ and $\oint_\gamma f \,dz = 0$ for all closed contours $\gamma$ in $\Omega$, then $f$ is analytic.
\end{theorem}

\begin{proof}
    Suppose $\oint_\gamma f \,dz = 0$ for all $\gamma$.
    Then by the ABC theorem there exists an $F$ such that $F = f'$.
    Thus $F$ is analytic, and so is its derivative $f$.
\end{proof}

\section{Bounds for Analytic Functions}
Now we''ll use these results to place restrictions on the behavior of analytic functions.

\begin{theorem}[Cauchy's estimate]
    Let $f$ be analytic on a closed ball $B(z_0, R)$.
    If $|f(z)| \leq M$ for all $z \in B$, then
    \[ \left| f^{(k)}(z_0) \right| \leq \frac{k! M}{R^k}, \qquad k \in \N. \]
\end{theorem}

\begin{proof}
    The integrand in the Cauchy integral formula (for the $k$th derivative) is bounded by $M / R^{n+1}$, so by the ML lemma
    \[ \left| f^{(k)}(z_0) \right| \leq \frac{k!}{2\pi} \left( \frac{M}{R^{k+1}} \cdot 2\pi R \right), \]
    as desired.
\end{proof}

Notice that if a function is entire and bounded over all of $\C$, then it satisfies the conditions of the previous theorem for any ball $B(z_0, R)$.
Thus if we take $R \to \infty$ its derivatives must vanish and we get the following.

\begin{theorem}[Liouville's theorem]
    If $f(z)$ is bounded and entire, then $f$ is constant.
\end{theorem}

We can use Liouville's theorem to prove a familiar statement, but an unexpected one for this context!

\begin{theorem}[Fundamental theorem of algebra]
    Every non-constant polynomial with complex coefficients has at least one zero.
\end{theorem}

\begin{proof}
    Suppose $P(z) = a_n z^n + \cdots + a_0$, $a_n \neq 0$, has no zeroes, so $H(z) = 1 / P(z)$ is entire.
    For sufficiently large $|z|$ we have $|P(z) / z^n| \geq |a_n| / 2$, meaning there exists an $R \gg 1$ such that
    \[ |H(z)| = \frac{1}{|P(z)|} \leq \frac{2}{R^n |a_n|}, \qquad |z| \geq R. \]
    Also, by continuity (and the extreme value theorem) there exists a finite $M$ such that $|H(z)| \leq M$ for all $|z| \leq R$, meaning
    \[ |H(z)| \leq \max \left\{ M, \frac{2}{R^n |a_n|} \right\}, \qquad z \in \C. \]
    Thus $H(z)$ is constant, and so is $P(z)$---a contradiction.
    Thus $P(z)$ has at least one root.
\end{proof}

In fact, all of these polynomials' roots lie in $\C$---any polynomial can be factored ito $P(z) = (z - z_1) Q_1(z)$, where $Q_1$ is another polynomial that can be factored in the same way.
(This process has a finite number of iteration.)
We therefore call $\C$ ``algebraically closed''.
Now for one last theorem!

\begin{theorem}[Maximum-modulus principle]
    If $f$ is analytic on a domain $\Omega$ and $|f(z)|$ has a local maximum at $z_0 \in \Omega$, then $f$ is constant on a neighborhood of $z_0$.
\end{theorem}

\begin{proof}
    Suppose $z_0 \in \Omega$ with $|f(z_0)| \geq f(z)$ for all $z \in \Omega$', where the domain $\Omega' \subset \Omega$.
    Choose $R > 0$ such that $B(z_0, R) \subset \Omega'$; by Cauchy's integral formula
    \[ f(z_0) = \frac{1}{2\pi i} \oint_{\partial B} \frac{f(z)}{z - z_0} \,dz = \frac{1}{2\pi} \int_{0}^{2\pi} f(z_0 + Re^{it}) \,dt, \]
    and so
    \[ |f(z_0)| \leq \frac{1}{2\pi} \int_{0}^{2\pi} |f(z + Re^{it})| \,dt \leq \frac{1}{2\pi} \int_{0}^{2\pi} |f(z_0)| \,dt = |f(z_0)|. \]
    It follows that the two intermediate expressions are equal, so
    \[ \frac{1}{2\pi} \int_{0}^{2\pi} \Big( |f(z_0)| - |f(z_0 + Re^{it})| \Big) dt = 0 \]
    and $|f(z_0)| = |f(z_0 + Re^{it})|$.
    Thus $|f|$ is constant on a $\partial B(z_0, R)$ of any radius $0 < r < R$, and so $|f|$ is constant on $B(z_0, R)$.

    Now, if $f = u + iv$ then $u^2 + v^2$ is constant, meaning both of its partial derivatives are equal to zero.
    An application of the Cauchy-Riemann equations gives the matrix equation
    \[ \begin{bmatrix} u & -v \\ v & u \end{bmatrix} \begin{bmatrix} u_x \\ u_y \end{bmatrix} = \begin{bmatrix} 0 \\ 0 \end{bmatrix}, \quad (x_0, y_0) \in B. \]
    If this coefficient matrix is singular somewhere in $B$ then $u = v = 0$ there, and we're done.
    Otherwise the vector above must be the zero vector, meaning $u_x = v_y = 0$ and $v_x = -u_y = 0$ on $B$.
    Thus $u$ and $v$ are constant on $B$, and so is $f$.
\end{proof}

\end{document}